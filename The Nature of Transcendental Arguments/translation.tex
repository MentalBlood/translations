\documentclass{article}

\usepackage{polyglossia}
\setdefaultlanguage[indentfirst=true,forceheadingpunctuation=false]{russian}
\setotherlanguages{english}

\RequirePackage[a4paper, headsep = 0.5 \headsep, left=2.5cm, right=2.1cm, top=2cm, bottom=2.1cm]{geometry}

\setmainfont{Times New Roman}
\newfontfamily\cyrillicfont{Times New Roman}[Script=Cyrillic]

\title{Природа Трансцендентальных Аргументов}
\author{Марк Сакс}
\date{2005}
\begin{document}

\maketitle

\begin{abstract}
Это статья про вид доказательства присущий трансцендентальным аргументам. Похоже что трудности с ними возникают потому что их понимают как формулировки отношений между понятиями или содержаниями высказываний. Чтобы показать силу трансцендентальных аргументов я предлагаю понимать их феноменологически. Так получится лучше понять некоторые трансцендентальные доказательства Канта. Феноменологический подход поможет также понять связь между трансцендентальными аргументами и трансцендентальным идеализмом.
\end{abstract}

Далее рассмотрим что следует понимать под ТА. Речь о виде доказательства которое в основных из них используется. Конечно возможен не один ответ на этот вопрос и адекватный ответ не обязан соответствовать всем возможным пониманиям ТА. ТА уже давно существуют и потому не следует ожидать подобных прорывов. Поэтому попытаемся дать ответ который только основным ТА соответствует, не вдаваясь в долгую и запутанную историю ТА со времен Канта.

\section{Концептуальное устройство трансцендентальных аргументов}

В общих чертах, трансцендентальные аргументы таковы: начинаются с посылок которые [даже] скептик примет и затем через ряд аргументов выводят то в чем скептик сомнивался. То есть в итоге скептик должен признать несоместимость скептической возможности и того что он не может не принять.

То как это в ТА достигается, конечно, не столь ясно. Похоже что они раскрывают пресуппозициональные отношения. Часто те кто говорит о пресуппозициях также говорят о необходимых условиях. Что же понимается под ''пресуппозицией'' и ''необходимым условием''?

Обычно считается, что это отношения между понятиями или пропозициями. Это отчасти согласуется с формулировками Канта.В рамках своего трансцендентального идеализма он использует такие термины как ''репрезентация'' двояко, и не понимая этой двоякости можно подумать что он только о предпосылках между эмпирическими представлениями говорит. Лингвистический поворот и преобладание концептуальных схем еще больше провоцировали понимать ТА как связанные с языковыми репрезентациями. И с этим есть проблемы. Концентрация на концептуальных схемах при понимании трансцендентальных аргументов провоцирует считать что ТА связаны с концептуальными структурами, выделенными из объектов опыта, а не с самими этими объектами. Из-за этого ТА часто толкуют как если бы они были об отношениях посылок именно концептуальных.

Я в этой статье покажу что такая замена переживаемых объектов концепциями упускает ключевое в понимании ТА. Правильное понимание природы ТА покажет что современное отношение к ним некорректно.

\section{Проблема}

Следует четко понимать проблемное отношение между двумя объяснениями: того, что делают ТА и того как они это делают. Возникает оно если отношение ''пресуппозиция'' или ''необходимое условие'' понимать через логическое следствие или дедуктивный вывод. Такое понимание не схватывает действительное увеличение знаний которое ТА должны бы давать. Заключение ТА существенно и, можно сказать, синтетичское, если только скептик не представляет никакого интереса. Ведь дедуктивный вывод сам по себе не привносит чего-то существенного. Конечно экспликация логического следствия может, как сказал бы Кант, увеличить наше экспликативное знание, но мы то ищем знания амплиативное. Значит где-то дедуктивный вывод должен прерываться и делаться синтетический/амплиативный ход.

Этот ход может быть переходом от одной линии аргументации к другой или может быть внутренним по отношению к исходной линии. Эти два описания можно понимать как разные способы сказать об одном и том же. Для простоты однако далее буду придерживаться второй.

Есть конечно вопрос: как понимать истинность пропозиции выражающей этот амлиативный ход. Эмпирически она, какзалось бы, не может быть установлена. Посылки ТА конечно контингентны, например: ''у меня есть другой опыт'', ''я испытываю изменения'', ''я самосознателен''. Эти посылки контингентны и установлены эмпирически, но столь фундаментальны, что отрицать их неразумно. Но какова природа менее очевидных шагов на пути к существенному выводу? Чтобы быть действительно информативным, пропозиция выражающая этот ход должна быть установлена чем-то отличным от просто логического следования из посылок. С другой стороны, если это будет эмпирическое утверждение, то непонятно как можно в нем не сомниваться. То есть нужно чтобы эти пропозиции были и информативными, и не контингентными.

Какого же рода истины содержат эти шаги ТА, если истины эти основаны не на одних лишь значениях но и не на контингентном наблюдении? Этот вопрос - руководящий для Канта в том как возможны синтетические априорные суждения. Очевидно что ТА так или иначе именно такое знание должны давать.

\section{Общая форма трансцендентальных аргументов}

Итак, похоже что в любом успешном ТА должен хотя бы один амплиативный/синтетический шаг и шаг этот ни контингентен, ни дедуктивен. И однако шаг этот должен быть априорным и значит несомненным.

Пусть этот шаг это шаг от \text{q} к \text{r} в следующей схеме:

\[p \to q \to r \to s\]

Здесь \text{p} - посылка, приемлимая для скептика, \text{s} - утверждение, которое он не хочет принимать. Для простоты схема показывает ТА как расширенный modus ponens, но, конечно, форма может быть и другой.

Беглый взгляд на некоторые примеры ТА поможет понять о каких ходах речь:

\begin{itemize}
	\item Чтобы был возможен опыт последовательности, должно быть что-то неизменное.
	\item Чтобы был возможен опыт изменений, опытный мир должен включать каузальность.
	\item Если возможен опыт различных предметов, то предметы эти находятся в едином пространстве-времени.
	\item Если у меня есть самосознание, то и у других оно тоже есть.
\end{itemize}

Ходы эти можно понимать как материальные импликации. Но как их посылки подтверждаются? Какой ход можно здесь задействовать чтобы показать эти ключевые пропозиции синтетическими и в то же время априорными?

Предлагаемый мной ответ подразумевает что в рамках дедуктивного аргумента их следует понимать как предполагаемые. Их истинность недоказуема пока они понимаются как формулирующие концептуальные отношения или отношения между содержаниями пропозиций. Ограничение понимания лишь концептуальным и пропозициональным конечно позволит схватить сказанное, но не его априорность. Чтобы же обеспечить априорность, нужен переход с концептуального уровня на феноменологический. Только так можно обнаружить основания истинности трансцендентального утверждения. Потому что на этом уровне можно понять амплиативный ход - первичный, но затененный материальной импликацией.

Конечно простое указание на феноменологичность взамен концептуальности еще не дает прогресса. В конце концов, это, возможно, лишь возврат к апостериорному. Надо значит понять, к чему этот спуск с концептуального уровня приводит.

\section{Ситуированные мысли}

На этот амплиативный шаг обычно указывают говоря об идентификации ''условий возможности'': можно сказать что в приведенной выше схеме \text{r} является условием возможности \text{q}. Но разговор об условиях возможности слишком расплывчат. Его можно понимать и как не более чем семантическое условие возможности осмысленности данного понятия, а это возвращает к концептуальному пониманию посылки. Думаю, к нужному понятию приближает указание Эллисона на эпистемическое состояние - условие возможности знания или опыта. Эллисон однако не связывает это понятие с ТА и вообще мало говорит о нем: ''Хотя понятие это центрально для трансцендентального предприятия Канта, он его никогда открыто не рассматривает, и потому трудно, если вообще возможно, дать ему точное определение''. Эллисон утверждает, что эпистемическое состояние ''необходимо для репрезентации объекта / объективного положения дел'' Предполагается конечно что состояние это отличается от психологического или онтологического. Идея в том, что есть набор эпистемических условий возможности нашего опыта - категории и формы пространства и времени - и они определяют структурирование любого нашего объективного опыта.

Но насколько эпистемические условия отличны от психологических состояний? Эллисон вскольз говорит об эпистемических условиях что они являются ''условиями, определяющими то, что может считаться объектом нашего разума'', и это, кажется, согласуется с их возникновением в совокупности когнитивных стуктур которые необходимы для нашего разума. Конечно одно лишь настаивание на неконтингентности когнитивного аппарата не поможет отличить эпистемические состояния от психических. И, наверное, лучше говорить об эпистемическом состоянии как только о не опирающимся на конкретный когнитивный аппарат, которому есть альтернативы. Нам значит нужно такое понимание этого состояния, которое не основано на конкретной доктрине о когнитивных или психологических структурах.

Для обоснования синтетических переходов и лучшего понимания что такое эпистемическое состояния, введем понятие ситуатизированной мысли. Ситуированная мысль - это которая возможна из определенной точки фреймворка, и содержание ее восприятием с этой точки и определяется. Это не голое содержание, рассматриваемое из ниоткуда - а информативное, феноменологически интегрированное. Называя такое мыслью, я имею ввиду его отличие от просто феноменологического или перцептивного опыта. Это отличие двояко. Во-первых, любой опыт внутренне структурирован/артикулирован, иначе тогда это не опыт - потому что нет возможности отличить один опыт от другого. Но артикулированность опыта не обязательно лингвистична, и не обязательно он рассматриваемым субъектом лингвистически артикулирован, и даже не обязательно артикуляция эта полностью когнитивна. Отчасти смысл в сосредоточении на этой артикуляции и придании ей когнитивной значимости отличной от таковой в грубом опыте. Во-вторых, ситуативная мысль отлична от соответствующего опыта отсутствием требования ситуативности субъекта - достаточно лишь чтобы он приближался в мыслях к тому, что имел бы, будь ситуирован. Иначе говоря, ситуированная мысль феноменологически информирована, но не является феноменологическим опытом.

Такая мысль, значит, нечто среднее между чистым пропозициональным содержанием и пропозиционально артикулируемым опытом. Это то есть не совсем настоящий опыт - скорее репрезентация в мышлении ситуированной конструкции пропозиционального содержания, размышление о том, каково быть так ситуированным, чтобы был опыт такого вот содержания. Важно, что для обоснования ТА требуется только наличие у нас такой ситуированной мысли, опыт же, воплощающий эту мысль, не обязателен. То есть для трансцендентального доказательства достаточно мысленного перехода от чисто пропозициональной артикуляции к феноменологическому описанию, а спускаться на уровень ситуированного опыта не обязательно.

Это конечно нечеткая характеристика ситуированной мысли, но, надеюсь, ее достаточно. Идея в том, что, хотя голое пропозициональное понимание \text{q} толкает к амплиативному переходу от \text{q} к \text{r} как нечто могущее быть предложено только в ходе аргументации ,- несмотря на это - ситуированной мысль о содержании, выраженном в \text{q} достаточно для априорности этого амплиативного хода.

\section{Пример ситуированной мысли}

Полезно рассмотреть пример такой мысли. Можно представить субъекта, расположенного перед деревом в саду и перцептивно с ним связанного так, чтобы возникала мысль что в саду есть дерево. Пропозициональное содержание такой мысли - ''Дерево в саду''. Ситуированность мысли значит принятие перспективы субъекта на сцену восприятия, перспектива эта соответствует голому пропозициональному содержанию: что задействованно для субъекта в этом его позиционировании, как он позиционирован в своем восприятии. Ситуированная мысль здесь позволяет субъекту уловить не только ''Дерево в саду'', но и тут же знать ''Передо мной дерево'' или ''Между мной и горизонтом - дерево''.

Благодаря чему же возможен переход от ''Дерево в саду'' (\text{q}) к ''Дерево напротив меня'' (\text{r})? Очевидно ведь что дедуктивно это не следует: может, я смотрю на экран с изображениями дерева, получаемыми с камеры, и камера эта в саду - это не значит что я там же, где камера. Дело в том, что в случае с камерой нет ситуированности \text{q}. Если бы она была, то мысль схватывалась бы с моей точки зрения и ею была бы информирована, и одно только наличие мысли было бы достаточно для перехода к \text{r}. То есть переход от \text{q} к \text{r} синтетичен, ведь в \text{r} содержится больше чем в \text{q}, и все же этот переход может быть априорным - не требуется никаких дополнительных наблюдений. И значит шаг от \text{q} к \text{r} обосновывается не концептуальными отношениями между предложениями, а чем-то другим - не содержанием \text{q}, а тем, что привнесено ситуированным ее мыслящим. (Здесь важно отметить что ситуированность \text{q} нельзя понимать как часть ее содержания. К этому я еще вернусь.)

Пример этот, конечно, тривиален, но эта тривиальность помогает понять, что значит ситуированность мысли. А связь между ситуированной мыслью и трансцендентальным доказательством видна, если рассмотреть например содержание высказывания ''Марк сейчас не существует''. Будучи неситуированно оно, конечно, не представляет особого интереса: сказано просто что Марк не существует, ну и конечно он может не существовать. Но если это мысль самого Марка, то она проинформирована фактом ее мышления Марком. Именно так исходно безобидное утверждение становится доказательством cogito. Ведь действительно, cogito - яркий пример основанного на ситуированности мысли доказательства. Обычно, впрочем, в cogito используется местоимение от первого лица - ''Я не существую'' - но ситуированность мысли здесь сохраняется, то есть только для того, чья мысль информирована фактом им мыслимости, ''Я не существую'' самопротиворечива. Аналогично работает и привычная форма ''Я мыслю, следовательно я существую''. Если толковать этот аргумент как просто пропозицию, понимать ее несетуированно, то в нем не будет такой силы. Потому что то, что только и может произвести доказательство - что может обосновать ''следовательно'' - это ситуированная мысль, то есть сам акт мышления пропозиции ''Я мыслю'' - именно он приводит к существованию того, кто мыслит.

Здесь стоит еще раз указать на ключевую особенность ситуированного мышления: неситуированная мысль может рассматриваться как чистое пропозициональное/концептуальное ее содержание, содержание это можно рассматривать как бы из ниоткуда. Но ситуированную мысль так рассмотреть не получится, ведь в ситуированном мышлении содержание мысли определяется эпистемической позиционированностью субъекта относительно фактов, делающих мыслимую пропозицию истинной. То есть содержание ситуированной мысли не ограничивается пропозициональным. И, значит, в приведенных выше примерах речь не о чисто пропозициональном содержании ''В саду дерево'', а результат действительного или вооброжаемого опыта дерева в саду - ситуированная мысль; и аналогично речь не о пропозиции ''Я мыслю'', а о том, что порожденно действительным или вооброжаемым опытом моего мышления. Это все демонстрирует, как ситуированная мысль переходит с чисто пропозиционального или семантического уровня на уровень чисто феноменологического описания.

Итак, похоже что связь между cogito и сутированной мыслью характерна для ТД: основные ТА обычно как раз ситуированные мысли и используют для обоснования синтетических априорных утверждений, заключенных в них. Ранее было указано, что в большинстве ТА есть хотя бы одна центральная пропозиция, скрывающая под собой этот синтетический ход, так вот, истинность этой пропозиции можно установить априорно, но только с помощью ситуированного мышления.

\section{Два тематических исследования: Кант о субстанции и причинности}

В качестве примера ТА за авторством Канта, рассмотрим центральный ход Первой Аналогии:

Всякое изменение - лишь перестройка неизменного.

Рассматривая его лишь как пропозицию, трудно понять, почему вообще эта пропозиция должна быть истинной - она звучит как просто метафизическая догма. Она совершенно точно не аналитична, но и непонятно, что за факт мира мог бы эмпирически ее подтвердить или опровергнуть. Признать ее априорной можно только если она выражает ситуированную мысль.

Чтобы продемонстрировать это, а также то, насколько близки ситуированная мысль и трансцендентальное постижение, стоит вернуться к примеру с деревом. Преполагаемое переживание этого дерева можно приукрасить: пусть его ударяет и совершенно испепеляет молния. Иметь ситуированную мысль ''В саду было дерево, а теперь в саду нет ни одного дерева'' - значит иметь или представлять переживание себя, не наблюдающего дерево сейчас, но наблюдавшего его ранее. Но чтобы эта мысль была связной, нужно думать, что всюду передо мной есть нечто - некий перцептуальный фон, обеспечивающий изменение и, значит, событийность. Как выражается Кант: ''Появление или прекращение существования не мыслимо вне постоянного''. Чтобы наличие дерева в момент времени 1 и его отсутствие в момент времени 2 можно было связать в одной мысли, нужно предположить, хотя бы в рамках этой мысли, неизменный пространственно-временной мир. Более того, поскольку пространство и время сами по себе воспринять нельзя, нужно что-то, что заполнит эту пространственно-временную область опыта - что-то, что сохраняется в рамках связанных с деревом переживаний и что, значит, позволит обеспечить единство ситуированной мысли.

Итак, теперь ясно, почему рассматриваемое утверждение - не просто догма. Не важно, можно ли обосновать некоторую неизменную субстанцию, которая конституировала бы возможность всякого изменения - достаточно гораздо более скромного вывода о необходимости неизменного для восприятия изменения. Этот переход - от изменения вещей к тому, что за этими изменениями должен стоять некий инвариантный объект - ключевой для аргумента, и априорная его истинность выявляется в рассмотрении пропозиции ''вещи меняются'' как ситуированной мысли, то есть в рассмотрении того, каковы вещи для ситуированного субъекта - субъекта, переживающего изменения. Ведь именно только ситуированное понимание мысли способно выявить истинность априорно ей подразумеваемого - ничто другое нам здесь не поможет, но, что важнее, ничто другое и не нужно - и это несмотря на неаналитичность рассматриваемого хода, несмотря на его информативность касательно мира - касательно того, как он должен сохранять целостность в нашем опыте.

Здесь можно высказать возражение: почему рассматриваемое изменение нельзя воспринять вне внешнего мира, вне этой основы, закрывающей разрыв между ситуированными содержаниями мысли и тем, что находится в саду в моменты времени 1 и 2? Можно ответить, что для того, чтобы можно было откалибровать эти две вещи, должно быть что-то, что оставалось бы неизменным, но почему же это что-то должно быть именно внешним? Если мы примем взгляд Канта о последовательности и атомистичности субъективного восприятия, то, похоже, единственное, что могло бы сделать эти два восприятия соизмеримыми - это восприятие их как изменений единого внешнего мира, лежащего в их основе. (Без этого мы не смогли бы даже осмыслить эти две ситуированные мысли как соответствующие одному субъекту.) Но это противоречит попытке понять ТА в терминах ситуированного мышления. Ведь суть этого понимания в том, что обоснование ТА априорно - на основе только ситуированности соответствующих мыслей, без, значит, зависимости от психологического склада мыслящего. А утверждение последовательности и атомистичности опыта - это уже приверженность определенной форме трансцендентальной или эмпирической психологии. Не то что бы дело в том, что нет психологических моделей, которые отвергали бы одну или обе этих концепции.

На это возражение можно отвечать по-разному. Наиболее очевидный ответ - согласиться с ним, но показать его безвредность для понимания ТА с помощью ситуированных мыслей: мол, это показывает лишь, что этот конкретный аргумент Канта не слишком хорош как ТА, ведь он, похоже, использует не только ситуированную мысль. Или же можно ответить что Кант все-таки веское доказательство привел. И тогда будет несколько способов разделить этот аргумент и психологическую модель, и один из них особенно интересен в контексте этой статьи.

Если кратко: аргумент этот можно освободить от опоры на психологическую модель, если понимать рассматриваемое изменение не как изменение между просто психологическими состояниями, но как изменение восприятия предположительно внешних объектов, так что структурированными во времени считаются не только субъективные переживания, но и сами по себе воспринимаемые положения дел. Чтобы были возможны их представления как изменяющихся по отношению друг к другу или как постоянных, нужен постоянный внешний мир, на фоне которого и осмысленно утверждать что там, где был один объект, теперь находится другой, или что тот же объект там и остается. То есть для субъекта единственным способом понять объективное изменение - иметь перед собой фоном единый неизменный мир. Потому как опыту для репрезентации своих объектов - исчезновения дерева и появления облака дыма, например - находящимися в объективных отношениях необходимо толковать их как модификации этого постоянного мира. И это следствие безразлично к атомичности и последовательности опыта - он мог бы быть дан и органически единым потоком. В этом последнем случае тогда, пожалуй, и не потребуется постоянного внешнего мира для объяснения единства сознания, ведь оба опыта - дерева и облака дыма - будут тривиально связаны, будучи содержанием единого субъективного порядка. Но будучи истолкованы объективно они были бы несоизмеримы подобно содержанию до дискретности отдельных сновидений. То есть для ситуирования мыслителя относительно сразу обеих объектных областей и осмысления им отношений между их обитателями (пусть даже только временной последовательности), нужен общий мир в котором оба этих объектных содержания проявятся.

Итак, ответ на указанное выше возражение в том, что, хотя и возможно толкование аргумента как опирающегося на конкретную психологическую теорию и, значит, не способного дать действительно трансцендентального доказательства, - несмотря на это - предложенная здесь линия аргументации свободна от такого рода ограничений и, значит, опирается только на априорные пресуппозиции ситуированной мысли.

Перейду теперь ко второму примеру ТА, он тоже взят у Канта. Центральное положение Второй Аналогии заключается в существовании причины у всякого события. Будучи рассмотренно как пропозиция оно, конечно, тоже похоже на догму. Кант, впрочем, ясно указывает на невозможность его понимания через одни лишь понятия, и что оно все же известно ''с аподиктической точностью''. Можно понять, что имеет ввиду Кант, если рассмотреть предполагаемое ТД как вопрос раскрытия пресуппозиций наличия ситуированной мысли, которая и лежит в основе центральной пропозиции. Достаточно понять значение переживания нечта как события. Очевидно что восприятие события исключает возможность нескольких сосуществующих положений дел, и значит порядок таких восприятий необратим. Такая необратимость и есть верный признак восприятия событий. Что же в опыте ее обеспечивает, что в обстоятельствах опыта способно делать представлять восприятия последовательными? Кант считает, что необратимость нескольких восприятий для субъекта возможна только если одно положение дел является следствием другого. Иначе говоря, только так субъект может \textit{через ресурсы, данные ему в опыте}, понять эти положения дел как последовательные, и, значит, объяснить необратимость своих их восприятий. Но это лишь подтверждение закономерного порядка - причинности - действующего всюду где можно говорить о переживании событий.

Стоит отметить, что вывод аргумента - не только лишь необходимость \textit{веры} в причинность, но необходимость самого эмпирического мира этой причинности подчиняться. Причинность переживаемого мира не значит переживание его причинности. Действительно, содержимого перцептивного опыта может быть и недостаточно для обнаружения причинности мира. Аргумент же Канта призван показать другой способ обнаружения этой причинности. И наша вера в такой причинный порядок обязательно будет соответствовать фактам переживаемого мира (потому что они необходимы его структуре), даже если факты эти нельзя непосредственно вычитать из нашего опыта. Это все важно для демонстрации неограниченности привносимого ситуированностью мыслей феноменологически данным содержанием чувственных состояний - демонстрации того, что ситуированность может говорить и о неперцептивных описаниях необходимости связности мира в нашем опыте.

Такая интерпретация Второй Аналогии показывает, как ситуированное понимание импликации в ''Все происходящее имеет свою причину'' эту импликацию обосновывает. А именно: рассмотрение необходимого для ситуированности пред необратимостью воспринимаемых положений дел и, значит, пред событиями, показывает необходимость причинных законов, управляющих этими положениями.

Основное утверждение здесь - присутствие в не только рассмотренных, но и в прочих ТА, хотя бы одного такого вот предполагающего ситуированное понимание утверждения. Иначе говоря, решающие синтетические ходы в ТА априорны только через отношение содержания высказывания к \textit{возможному опыту}.

\section{Размышления}

С таким подходом к ТА некоторые их особенности требуют пояснений.

Во-первых, понимание центральных их ходов через ситуированные мысли проливает некоторый свет на вопрос, являются ли ТД (с такими ходами) формальными дедуктивными умозаключениями. Кое-что об этом уже говорилось, но сейчас еще раз вернемся. Несколько лет назад вокруг этого вопроса - вопроса формальной обоснованности ТА - шла оживленная дискуссия, но она затихла до сколь-нибудь четкого вывода. Я утверждаю, что понимание ТА как формально обоснованных выводов некорректно. Ведь формальные дедуктивные аргументы касаются отношений между пропозициями, а полное содержание ситуированной мысли не схватывается как только пропозициональное - привносимое ситуированностью выходит за рамки всякого пропозиционального. И, значит, хотя общая форма ТА может показаться дедуктивным аргументом (расширенным \textit{modus ponens}, например) - несмотря на это - в центральных ходах ТА невозможно говорить о необходимых условиях, ограничиваясь таким только пропозициональным пониманием. Именно и только признав это, смогли мы делать эти шаги априорно. Понятие ''аргумент'', похоже, легко ассоциируется с отношениями между пропозициями вообще и составляющими дедуктивные формами в частности, так что лучше говорить о в первую очередь ТД, а не о ТА - так получится избежать путаницы. ТД подразумеваются априорными, и необходимые условия, фигурирующие в их решающих ходах производятся ситуированными мыслями, а не через дедуктивными пропозициональными связями.

Иначе говоря: общее доказательство, обоснованиями центральных ходов которого выступают ситуированные мысли, может быть понято как дедуктивный вывод только если эти амплиативные ходы предполагаются, а проэксплицировать их можно только в рамках требующего перехода от чисто пропозиционального к ситуированному мышлению ТД. ТА как дедуктивные аргументы, значит, вынужденны паразитировать на ТД ради сохранения сколь-нибудь значимого эмистемологического интереса.

Замечательно, что сам Кант говорит обычно как раз о ТД, а не о ТА. Быть может, это несущественный терминологический факт, и тем не менее, говоря о ТД, Кант говорит вещи, легко наводящие на мысль о том, на что я указывал, говоря о различии между пропозициональным содержанием и ситуированными мыслями:

\textit{В рамках трансцендентального познания и покуда мы интересуемся только концепциями понимания, нами может руководить только возможность опыта. Доказательство такое, значит, показывает не непосредственное следствие от одного понятия к другому (от происходящего к его причине, например) - то был бы неоправданный скачок. Доказательство это продолжается демонстрацией невозможности опыта и, значит, его объектов, без доказываемой связи. Значит, доказательство это должно показать возможнсоти синтетического априорного перехода к некоторому не содержащемуся в одних только понятиях знаниях о рассматриваемых вещах.}

Понимание ТД в терминах необходимых условий ситуированных мыслей не только выделяет их среди других доказательств - оно также объясняет корректность указания на это их уникальное как на называемое Кантом трансцендентальным доказательством в рамках его понимания ТД как выявляющего предпосылки к опыту. Именно поэтому они всегда начинаются с базовой ситуированной мысли о наличии у меня опыта и, значит, соотнесенности переживаний со мной. То есть дело не в том, что есть вот некие трансцендентальные аргументы, наиболее интересные из которых начинаются с приемлимой даже для скептика посылки — скорее само понятие ТА подразумевает такую посылку, а успешное применение ТА для ответа скептику является уже побочным продуктом.

Вторая особенность ТА касается статуса необходимых условий. Согласно, например, аргументу в пользу постоянства, из Первой Аналогии, постоянного мира может быть достаточно, но что при этом делает его также необходимым для опыта изменений или, вообще говоря, временных отношений в объектном мире? То есть, почему нет другого способа получить тот же опыт? В достаточно широком смысле, мы не можем совершенно исключить альтернативы. Но в настолько широком смысле это касается уже и основных математических теорем и законов логики, так что, пожалуй, не стоит обращать внимание на такое ''ограничение''. Оставляя, значит, в стороне такую радикальную альтернативность, стоит указать на непосредственность связи ситуированной мысли с эпистемическими условиями — настолько очевидна эта непосредственность, что нельзя даже помыслить чего-то подолнительного что было бы промежуточным, посреднеченским звеном в этой связи, и которое, значит, могло бы быть заменено для получения упомянутой альтернативы. Связи эти видятся необходимыми именно от непосредственности — от отсутствия логического пространства для размещения сколь-нибудь изменяемого и, значит, могущего показать альтернативные возможности возникновения рассматриваемых ситуированных мыслей. Непосредстенная эта связь скрывается чисто пропозициональным пониманием ТА — на таком уровне, конечно, есть повод для сомнения, ведь на нем и создается впечатление возможности существования содержания посылки без выводимого. Учитывая что необходимость эта не психологического или метафизического рода и столь непосредственна, можно говорить о ней как о sui generis — необходимых условиях для ситуированности как таковой. Это уже, значит, трансцендентальная необходимость — связанная с предпосылкой к опыту.

Третья особенность ТА, стоящая упоминания, это связь ТА с трансцендентальным идеализмом в свете понимания первого через ситуированные мысли. Рассмотрим, например, следующий отрывок из Первой Аналогии:

\textit{Нам следует сначала доказать наличие постоянного в явлениях и что преходящее ничем иным как постоянным и определяется. Но такое доказательство не может быть догматическим — из понятий — потому как касается синтетического априорного. Но раз никто еще не рассматривал такие положения как действительные только по отношению к возможному опыту, то и не удивительно, что, хотя указанный принцип всегда постулируется как основа опыта, сам он никогда не доказывался.}

Говоря, что рассматриваемые положения действительны только по отношению к возможному опыту, Кант указывает на свою стратегию ограничения сферы действия метафизических заявлений для наконец установления их априорности. Но в контексте разговор о таких положениях наводит на мысль о более скромном утверждении: что решающий шаг в аргументе возможен успешным только через понятие ситуированной мысли (= возможного опыта), а никак не через голую пропозициональность. Легкость такой интерпретации отрывка, основной задачей ставящего формулирование трансцендентальной идеалистической установки аргументации, недвусмысленно указывает, что ситуированность выражает ТИ. Выражение можно сохранить вне остального в этой позиции.

Связь ситуированности и ТИ можно выявить — даже не прибегая к каким-либо трансцендентальным различиям — прояснив необходимость и достаточность ситуированности для ТД. Если сказанное верно, то невозможность истолкования мыслей как ситуированных влечет недоступность выводов ТД как представляющих априорное знание. Здесь стоит рассмотреть случай человека столь отстраненного, что совершенно схожего мыслями с наблюдающим за садом через монитор и не знающим, где он сам относительно камеры в нем находится. Интересно, что для работы ТД необходимо и достаточно хотя бы даже виртуальной ситуированности мысли — ситуированной относительно объектной области, которая воспринимается как независимая от наших представлений — и не важно, такова ли ситуация в действительности, как она представлена мыслителем. То есть, во всяком случае, в контексте, в котором хотя бы даже виртуально субъект находится, априорно — то есть касательно всякого, кто находится в таком контексте — будут известны некоторые вещи. Это, впрочем, допускает сомнение в действительности конкретной кажимой ситуированности. В таком случае, значит, есть область, в которой ситуированная мысль порождает ТД, допуская этих доказательств действительность только в этой области. Область, в которой здесь ситуирована мысль, соответствует \textit{эмпирической реальности} Канта, а признание такими областями достоверности наших трансцендентальных доказательств ограниченности — его \textit{трансцендентальному идеализму}. Вкратце, связь с трансцендентальным идеализмом в следующем: в контексте ситуированного мышления устанавливается необходимость сделать рассматриваемый ход, и необходимость эта устанавливается одинаково для всякого прочего контекста, независимо от верности полученных выводов в потенциально более широком контексте и даже независимо от этого вмещающего рассматриваемый контекста существования.

Наконец, уместен комментарий касательно разницы между условиямим ситуированного мышления и условиями просто опыта. Как уже было сказано, аппарат ситуированности служит для фокусирования на соответствующем измерении опыта и артикулированного его осмысления. Но важно, конечно, и то, как именно ситуированность дает это осмысление. Говоря об условиях опыта вообще, мы говорим о том, что верно не только для нашего опыта, но и для любого прочего. Кант, однако, разумно поступает, не защищая подобные заявления, и указывает, например, что опыт Бога не будет подчиняться тем эпистемическим условиям, что применимы к нам. Объясняет он это тем, что у нас есть способности ума, не свойственные всем формам опыта. Эта ремарка, значит, освобождает трансцендентальные аргументы от неудобного бремени законодательного регулирования всех возможных форм опыта вообще. Впрочем, за это приходится платить довольно сомнительной трансцендентальной психологией. Это также подразумевает проведение значимой линии в неверном месте, ведь не только Бог, но и другие формы опыта — например, животные — исключаются из рассмотренных трансцендентальных аргументов юрисдикции. Распаковка трансцендентальных аргументов через ситуирование, однако, проводит эту линию правильно и обоснованно, и не прибегает к трансцендентальной психологии. Как было показано, речь в случае ситуированности об условиях любого субъектного опыта, и, значит, подразумевается верность не только для человеческого опыта, но все еще не верно для опыта Бога, так как он не предполагает какую-либо точку зрения. Связь трансцендентальных аргументов с точками зрения — перспективами — давно уже подтверждена. В этой же статье указывается, что перспектива наша не ''испорчена'' некой когнитивной структурой нами привносимой — нет — сама ситуированность, перспективность, мысли является такой структурой.

\section{Заключение}

Здесь, пожалуй, стоит остановится на, так сказать, историческом диагнозе. Интерес к трансцендентальным аргументам был достаточно силен со времен Канта как в аналитической, так и в прочей философии. Но теперь мы можем провести одно существенное различие. Как было ранее сказано, стандартное понимания трансцендентальных аргументов — в терминах концептуального и пропозиционального — упускает ключевой источник их силы, и такая интерпретация особенно распространена в аналитической среде. Отказ же от восхождения в лингвистическое в пользу спуска в феноменальное представляет скорее неаналитический подход.

В частности, представленная форма доказательства через понимание посылок как ситуированных, близка к тому, что можно назвать трансцендентальной или чистой феноменологией. И если представленное верно, то ясно эпистемическое в противовес просто описательному этой феноменологии значение, оно как раз и схватывает и природу трансцендентальных доказательств Канта, и ограничение этих доказательств эмпирической реальностью, уровнем возможного опыта.

Наилучший путь развития этой темы, таким образом — сочетание двух подходов. Формально дедуктивные аргументы нужны как оболочка, тогда как центральные ходы, соответствующие посылкам в аргументации, устанавливаются независимо, посредством трансцендентального доказательства через ситуирование. И тогда получится вывести синтетические априорные пропозиции.

