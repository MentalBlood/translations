\documentclass[11pt]{book}

\usepackage{polyglossia}
\setdefaultlanguage[indentfirst=false,forceheadingpunctuation=false]{russian}
\setotherlanguages{english}

\RequirePackage[a4paper, headsep = 0.5 \headsep, left=2.5cm, right=2.1cm, top=2cm, bottom=2.1cm]{geometry}

\setmainfont{Times New Roman}
\newfontfamily\cyrillicfont{Times New Roman}[Script=Cyrillic, Ligatures=TeX]

\title{Парадокс следования правилам и его последствия для метафизики}
\author{Джоди Аззуни}
\date{2017}
\begin{document}

\maketitle

\tableofcontents

\chapter{Введение}

\qquad

\textbf{Аннотация} \quad Здесь я дам краткое содержание всей книги и проясню несколько методологических моментов, а также укажу причины избегания таких терминов как ''понимание'', ''значение'' и ''факт''. Я опишу некоторые существенные различия моего понимания соблюдения правил от такового Крипке, например --- и в этом разница между моим подходом и подходом большинства занимающихся этой темой философов --- я уделяю большое внимание ''следованию правилам'' в его применении к задачам в ходе взаимодействия с миром, и этого его отличия от, например, разного рода арифметических упражнений вроде сложения чисел.

\qquad

Центральным теме этой книги --- ''парадоксу следования правилам'' --- является термин ''понимание'': как в выражениях ''она понимает, как складывать'', ''он понимает, как продолжать считать числа больше десяти'' и ''она понимает эти концепции''. Уместно предупреждение касательно такой методологии: ''понимание'' здесь обозначает поразительно сложную человеческую способность, включающуюю субличностные, сознательные/феноменологические, социологические и нормативные элементы. Понятие это слишком сложное для \textit{использования} в философском анализе --- должно скорее быть его \textit{объектом}. Это, впрочем, едва ли единственное такое опасное в своей сложности слово: еще одной ловушкой могут стать ''объяснение'' и ''смысл''. Причем даже повторяющиеся и очевидные неудачи, похоже, не мешают философам продолжать попытки строить сколь-нибудь точный анализ на таком песчаном фундаменте.

Куайн редко комментирует свой подход к занятию философией --- даже его признания натурализма и экстенсионализма довольно специфичны по содержанию и применению --- и тем не менее, я нашел у него редкую методологическую медитацию, стоющую \textit{запоминания}. Куайн (1981, 184) пишет:

\smallskip

\textit{Согласно описанию моих взглядов Шуденфреем: ''предложения заменили мысли, а склонность к согласию заменила веру''. Имеет ли он ввиду отсутствие для меня большего чем вера и мысли? Читая дальше, понимаю, что так и есть. Значит он неверно меня понял.}

\smallskip

Куайн продолжает:

\smallskip

\textit{Позиция моя в том, что понятия мысли и веры --- достойные объекты философского и научного анализа --- и столь же плохи как этого анализа инструменты. Если кто-то решается их таковыми использовать, то загадкой для меня остается что же, как он счел, более нуждается в анализе чем они сами.}

\smallskip

Но давайте все же придерживаться термина ''понимание'', раз уж именно ему эта книга непосредственно посвящена. Общим для нескольких философских традиций является предположение основанности всякого понимания набора концепций на понимании правил, этими концепциями управляющих. Это, похоже, отражает понимание значений слов и, например, концепций из таких простых математических практик, как счет. Хороший пример здесь --- шахматы: как только человек усвоит правила игры, он поймет, как в играть вне зависимости от обстоятельств.

Некоторые сторонники такого подхода говорят о нас как об обладающих ''предрассположенностью'' к пониманию этих правил и даже этих правил (будучих воплощенных в предрассположенностях) обладателях и что предрасположенности эти позволяют нам вести себя соответствующим образом в ситуациях при проявлении такого рода ''понимания'' --- например, при подсчете апельсинов в корзине.

Огромное количество литературы посвящено так называемому ''парадоксу следования правилам'' --- противоречию аспектов нашей практики естественной картине нашего понимания правил. Крипке (1982), например, развил масштабную философскую индустрию вокруг проблем, которые, по его мнению, Витгенштейн поднял для всякого подхода, обосновывающего понимание простых математических правил применением набора диспозиций. (Крипке, впрочем, не претендует на однозначную присущесть этой точки зрения Витгенштейну; и он, и другие, понимают что она ему, возможно, не пренадлежит, откуда и пошел термин ''Крипкенштейн''. Я избегаю этого термина, предпочитая использовать такие фразы как ''Витгенштейн Крипке'' или контекст.)

Проблема следования правилам, вкратце, такова: мы понимаем себя как следующих правилам и, пройдя соответствующую подготовку, действительно на это способных. Но средства, с помощью которых мы понимаем себя способными, согласно Крипке, не подходят для этой задачи. Кандидаты на развитие такой вот способности следовать правилам это, например, интроспективное их понимание, разные привычки и склонности их выполнять в рамках соответствующих задач и т.д. --- кандидаты эти показаны Крипке как неспособные гарантировать все необходимые элементы нашей компетентности в этим правилам следовании. Он отвечает на парадокс ''скептическим решением'' --- допускающим неудачу ''смыслового скептика'' и меняющим понимание нашей практики следования правилам несколькими важными способами. Среди этих способов отказ от понимания значения в терминах ''условий истинности'' и замена их ''условиями утверждаемости'' --- условиями с точки зрения соответствия индивидуальных практик следования правилам таковыми сообщества.

Версия парадокса Витгенштейна о следовании правилам, предложенная Крипке, оказала большое влияние. Один из вопросов которые я буду рассматривать --- это как она была понята как подрывающая индивидуалистскую картину математической практики --- точку зрения, согласно которой отдельные люди (буду называть их ''\textit{Робинзонами Крузо}'') независимо от сообщества могут осмысленно заниматься математикой и, значит, иметь ''приватные языки''. Парадокс отрицает применимость фраз вроде ''правильный подсчет'' к таким людям, потому что эти нормативные понятия осмысленно применимы только относительно стандартов сообщества.

Одним из исходных элементов моего альтернативного ''скептического решения'' будет отрицание этого шокирующего следствия: оно не следует даже если витгенштейнианские возражения Крипке против диспозиционных подходов к следованию правилам по большей части верны. Это потому, что мое решение не будет ставить стандарты сообщества выше индивидуальных. Более того, оно не не будет заменять условия истинности условиями утверждаемости, что важно для решения Крипке, отдающего предпочтение сообществу. Вместо этого мой подход фокусируется на том, как склонность к следованию правилам позволяет нам более или менее успешно взаимодействовать с миром. Предположим, что их практики демонстрируют склонности, которые меняются, постепенно оптимизируя успешные события (например, расчет неободимого на несколько дней количества еды). Тогда соответствие между индивидуальными следованиями правилам возникнет даже без явной связи этой практики с сообществом\footnote{Чтобы не перегружать объяснение и сделать его более ясным, я здесь пока опускаю детальное описание того, как это работает. В последующих главах эти детали будут прояснены.}.

Льюис (1983) в статье почти столь же влиятельной, как таковая Крипке (1982), связывает рассматриваемую версию парадокса с тогдашним антиреализмом Патнэма (1981) и предлагает общее решение. Он настаивает на подходе почти полностью встраивающем в ссылки на термины естественного вида (а также на математические термины вроде ''сумма'') \textit{предпосылку} метафизических ограничений на возможные их расширения и утверждает, что она \textit{необходима} для ответа и на рассматриваемый парадокс, \textit{и} на антиреализм Патнэма. Последующее поколение аналитических метафизиков назвало такой ответ ''референциальным магнетизмом''. В главе 4 я покажу, что семейство таких Льюизианских подходов к проблеме (далее подходы ''референциального магнетизма'') --- не работает и даже не требуется.

Льюис не использовал термин ''референциальный магнетизм'', фраза эта, видимо, пошла от Ходса (1984, 135), вопрошающего, как так получается, что

\smallskip

\textit{мы все в конечном итоге говорим на языках, в которых из всех возможных нумераторов (то есть функций \begin{math}F\end{math} 2ого типа, переносящих концепты 1ого типа на объекты, где для любых из этих концептов \begin{math}X\end{math} и \begin{math}Y\end{math} верно \begin{math}F(X)=F(Y)\equiv (Q_{E}x)(Xx, Yx)\end{math}) словосочетание ''количество ... '' обозначает стандартный нумератор? Почему стандартный нумератор является ''референциальным магнитом'', ''подтягивающим'' ссылку с помощью этой фразы (чего не могут сделать его нестандартные конкуренты)?}

\smallskip

''Референциальный магнетизм'' --- вещь тонкая, так что неудивительно, что последующее поколение философов превратно его истолковало.

Вот схема глав книги.

В главе 2 я возвращаюсь к обсуждению Крипке парадокса следования правилам для диспозиционных подходов к числовой компетентности, а также к альтернативному решению парадокса приписываемого им Витгенштейну. Соображения этой главы --- в значительной степени исходные соображения Крипке. Меня при этом не беспокоят вопросы витгенштейновской экзегезы --- насколько корректно Крипке присваивает парадокс Витгенштейну\footnote{По этому вопросу, впрочем, есть литература, например: Блэкбор (1984), Гольдфарб (1985, 1992), Макдауэлл (1984), Тейт (1986)} --- его формулировка проблем, стоящих перед смыслодиспозиционалистом, важна вне зависимости от присущести Витгенштейну. Есть, впрочем, два изменения. Во-первых, в рассуждениях Крипке центральным примером выступает субъект, практикующий арифметическое сложение, и скептический вызов о согласованности всего известного и испытанного им ранее с возможностью им практики на самом деле чего-то от сложения отличного, я же заменяю сложение более элементарной задачей --- задачей счета --- и соответствующей возможностью практиковать на самом деле что-то от него отличное. Это, конечно, незначительное изменение, и Крипке (1982, 17) сам мимоходом поднимал вопрос о счете.

Второе же изменение уже существенное и ключевое для моего решения парадокса. В моем случае центральный пример о субъекте, который считает \textit{вещи}, то есть в его задаче используются \textit{не только} числа. Дети так и учатся считать: они \textit{применяют} числа к задачам распознавания количеств предметов, и на приобретение таких знаний уходит несколько лет\footnote{Тщательное изучение этого вопроса с цитированием соответствующей литературы см. в Кэри (2009) и Баттерворт (1999), глава 3, разделы 1-4. Классическим исследованием является Гельман и Галлистел (1986)}. Это важно из-за рассмотрения многими философами ''социальных решений'' парадокса, приоритизирующих сообщество над индивидуумом, как единственных успешных. Одна из причин такого рассмотрения --- пренебрежение соображениями о применении математических концепций \textit{в мире}. Точно так же применение математики к миру находится по большей части за кулисами дискуссий Крипке и --- как побочный эффект --- за кулисами дискуссий многих этой литературы комментаторов.

В творчестве же самого Витгенштейна это не столь закулисно --- он часто приводит примеры про людей, считающих предметы. Да и Крипке тоже обращает внимание на ''числовые размеры'' наборов подсчитываемых предметов, когда мимоходом обсуждает ''зчет'' в рамках текста Витгенштейна. Но если подсчет \textit{предметов} занимает центральное место в примерах которым скептики бросают вызов, то становятся заметны вполне последовательные формы приватно-языковых практик. В частности, \textit{понятие диспозиций приводящих к когерентным таким практикам}, которое я ввожу в главе 5, не опирается ни на что за пределами контекста применения концепций к вещам в мире. Поэтому, конечно, мое скептическое решение парадокса не годится для бестелесных картезианских сущностей, от скуки вечно считающих числа в уме.

Оставляя в стороне эти разногласия с Витгенштейном Крипке, добавлю, что я частично согласен с одной важной леммой, которую Крипке выводит и парадокса и которую он (1982, 78-79) изредка подтверждает:

\smallskip

\textit{Надо отказаться от ''естественной предпосылки'' что осмысленные повествовательные предложения должны соответствовать фактам.}

\smallskip

Добавляя:

\smallskip

\textit{Прежде чем мы сможем приступить к скептической проблеме, надо прояснить картину этого соответствия фактам.}

\smallskip

Я \textit{частично} с ним согласен --- соображения следования правилам указывают на необходимость признания несоответствия \textit{некоторых} значимых повествовательных предложений фактам. Впрочем, не думаю, что это установлено для \textit{всех} таких предложений. В главе 6 я покажу, почему соображения следования правилам требуют лишь \textit{частичного} отказа от корреспондентской метафизики. Обсуждение это в то же время подкрепит мою особую форму дефляционизма относительно истины\footnote{См. Аззуни (2006) и гл. 6 этой книги. Интерпретация приведенной выше формулировки Крипке затруднена его фразов ''должен подразумеваться''. Я считаю, что осмысленные повествовательные предложение могут и даже ''должны'' претендовать на соответствие фактам --- и это совместимо даже с некоторыми из тех, что не столь уж соответствуют. (Убедительность этого толкования, конечно, зависит от значения термина ''смысл'' --- оно, все же, не совсем ясно, не ясно также, исключает ли ''уборка'' Крипке ''корреспондентность'' полностью или лишь частично.) См. обсуждение использования идиом знания/неведения в разд. 2, а именно сноску 15 гл. о моих способах маневрирования среди всего этого.}.

Здесь стоит отметить еще вот что. Диспозиционалистская надежда на посвященность значительной части анализа Крипке (1982) дроблению, находит \textit{основания} отношения соответствия между значимыми истинными предложениями и фактами в ''разуме'' (в широком смысле этого слова) человека, следующего правилам. Есть факты \textit{о человеке} и его \textit{возможностях}, которые и определяют отношения соответствия значимых истинных предложений фактам в мире. Но, думаю, соображения следования правилам указывают на неудачу этого диспозиционного проекта: диспозиции \textit{бессильны} в том, чего большинство диспозиционалистов от них требует, и получается так во многом по причинам, приведенным Крипке от имени Витгенштейна.

Я отвергаю, однако, главный тезис, который многие отсюда выводят. Крипке упорно говорил о себе как о всего лишь толкователе, но множество философов утверждают, будто соображения вроде представленных в главе 2 необходимо приводят к соотнесенности или воплощенности стандартов следования правилам вообще и математической практики в частности в \textit{обществе}, в котором человек эти правила изучает. Как говорит сам Крипке (1982, 109):

\smallskip

\textit{Что действительно отрицается, так это то, что можно было бы назвать ''частной моделью'' следования правилам --- моделью, в которой понятие следующего данному правилу человека надо анализировать просто с точки зрения фактов об этому правилу следователе и только о нем, то есть не отсылая к сообществу.}

\smallskip

Одна из целей этой книги --- показать, что несмотря на правоту Крипке относительно природы парадокса и последовавших на него ответов, модель приватного следования правилам остается этим парадоксом не затронутой. Контуры логического пространства в этой проблемной области более запутанны и тонки чем мыслители конца прошлого века осмеливались полагать.

В общих чертах обрисовав изложение Крипке его способа решения парадокса --- его, значит, интерпретацию Витгенштейна о замене условий истинности условиями утверждаемости, --- я в последующих главах приведу доводы в пользу уже другого --- не приоритезирующего общественные стандарты над индивидуальными --- решения. Также в этой главе я покажу неуспешность диспозиционного подхода, расширяющего круг необходимых диспозиций до обладаемых всеми в обществе.

В главе 3 я рассмотрю диспозиционно-смысловые языки --- действительно приватные языки. Их носители (называю их ''Крузовцами'') --- в отличие от нас --- относят каждое слово к тому, к чему их предрасположенности склоняют их его применять. Я далее начинаю исследование сферы применения таких языков и показываю как понятия вроде ''ошибка'', ''обоснование'' и ''концепция'' в них если и остаются, то лишь в некой скромной форме, впрочем эти результаты не столь существенны для приватного следования правилам. Во всяком случае, в этой и в 5ой главе покажу как они или в них необходимость нивелируются в таких приватных языках. Еще в этой главе я рассмотрю содержание фраз вроде ''согласованность с миром'' и ''разрезание мира по стыкам'' для носителей этих языков. Неожиданным выводом будет возможность использования таких языков для взаимодействия с миром при широком диапазоне благоприятных эмпирических обстоятельств, и более того --- носители этих языков могут объективно сравнивать разные диспозиционно-смысловые языки чтобы определять, какой из них лучше взаимодействует с миром. Они, впрочем, не будут использовать понятия вроде ''лучшего соответствия'' языка миру или ''правильности'' применения слова. Здесь следует указать, что ''благоприятные эмпирические обстоятельства'' значит нечто большее чем просто характеристики внешнего мира потому что они также охватывает эмпирические факты о, например, механизмах \textit{развития}\footnote{Есть забавное следствие моего решения парадокса, на которое укажу лишь в рамках этой сноски: возникает проблема с пониманием искусственных языков как неких полезных инструментов, которые, значит, можно принимать и отвергать в зависимости от целей (такое понимание описано у Карнапа (1956)). Ведь как можно сравнивать достоинства и недостатки таких языков не прибегая к некоторому метаязыку? Именно здесь эта загадка и решается.} диспозиций таких говорящих.

В главе 4 я сделаю паузу в анализе диспозиционно-смысловых языков, чтобы контраргументировать решения поставленных мной перед такими языками проблем касаемо отсылок к наборам вещей в мире, основанные на референциальном магнетизме. Здесь же я рассмотрю три его версии. Первая понимает структуру мира метафизически обеспечивающей ресурсы для дополнения того, что люди в сообществе привносят для определения отсылок. Наши слова (концепты) имеют определенное значение, выходящее за пределы психологических и нейрофизиологических ресурсов любого человека и даже за пределы того, с чем любое таких людей сообщество могло бы справиться. Вторая понимает интерпретаторов естественных языков так, как того требует --- наряду с базовой научной практикой --- семантическая теория для наложения определенной референции на этих языков термины и на естественные виды как на корреляты видовых терминов этих языков. Третья понимает \textit{априорное} конститутивное навязывание естественных видов как требуемое в качестве ''единственной игры в городе'' муровскими фактами определенной референции и предполагающими определенную референцию семантическими теориям. Разумеется, ничто из этого не защищает от референциального магнетизма, что я и показываю.

В главе 5 я представляю окончательную версию моего Крузо (Крузо 5) --- он психологически близок нам тем, что не понимает свой язык \textit{как} диспозиционно-смысловой, ведь его диспозиции к использованию терминов ему (как и нам) неведомы. Я показываю, однако, как его осознание возможности использования терминов для улучшения своего благосостояния позволяет ему навязывать последовательные стандарты этой своей приватной языковой практике.

Более подробно язык Крузо-5 я изучу в главе 6, показав способ применить к нему стандартную семантику функциональной истинности и то, как Крузо-5, подобно нам, естественным образом может использовать идиому истинности. Это должно продемонстрировать, что ''скептическое решение'' Крипке через замену условий истинности условиями утверждаемости --- не единственное такое решение парадокса. Замена эта, впрочем, нужна для вывода о невозможности приватных языков.

В главе 7 я завершу то, что не завершил до этого: разберу методологическую роль точки зрения Бога в моем подходе к оценке последователей приватных правил. Для этого я рассмотрю два возможных взгляда на языки Крузо (и наши языки): первый понимает их как постоянно меняющиеся в отношении того, к чему относятся их термины, а второй считает референцию фиксированной. Противопоставив их, я покажу, в чем смысл точки зрения Бога\footnote{Впервые я предложил эти различные способы рассмотрения работы языка в Аззуни (2000), Часть IV.}.

Далее я рассмотрю \textit{следствия} моего решения парадокса для корреспондентской метафизики. Если кратко: эмпирически обоснованная метафизика \textit{возможна}. Что же невозможно, так это пресуппозициональная роль такой метафизики в философских объяснениях эмпирического успеха или в семантике.

Ну и наконец, я концептуально свяжу парадокс с проблемой индукции Юма.

\chapter{Версия Крипке парадокса Витгенштейна и его решение}

\qquad

\textbf{Аннотация} \quad В этой главе рассматривается описание Крипке парадокса Витгенштейна и его решение. Интерпретация Крипке вызвала довольно много критических комментариев, но меня парадокс представленный Крипке интересует в отрыве от интерпретационной корректности. Здесь я рассмотрю две дистинкции: прямые и скептические решения и обосновывающие факты и факты соответствия, а так же три ключевых момента. Во-первых --- три требования Крипке для прямого решения парадокса: бесконечности, обоснования и ошибки. Во-вторых --- почему прямое социологическое решение не работает. И в третьих --- некоторые критические комментарии касательно следования правилам, возникшие после публикации соответствующей книги Крипке.

\qquad

\section{Постановка проблемы: три ограничения любого решения}

О том, кто успешно посчитал разные наборы объектов, мы можем сказать, что он научился считать. Это значит предположить, что он при следующих случаях будет продолжать считать ''так же'', несмотря на возможно новые виды придметов и большее их количество.\footnote{Некоторое неявное знание, нужное для успешного этой задачи выполнения, рассмотрено Гельманом и Галлистелом (1986, 77-82) как принципы подсчета: ''количественное слово'', ''нерелевантность порядка'' и ''абстрактность'': последнее слово в подсчете и есть количество подсчитанных предметов, потому порядок и вид считаемых объектов не важен. Описание процесса подсчета и трудностей преобретения соответствующего неявного знания см. в Кейси (2009, стр.241-244).} Пусть его научили системе счисления --- правилу образования новых нумералов из предыдущих. Системы нумералов разительно отличаются этим от наборов словесных названий чисел в большинстве языков\footnote{Рассмотрение типов числовой лексики в разных языках см. в Баттерворт (1999, особенно стр. 52-62).}, потому что вторые обычно относятся к финитной записи чисел, требующей явного создания нового словаря для все больших чисел. Однако в системы счисления создание обозначений для все большего количества цифр уже встроено. Итак, как только субъект обрел способность считать и освоил определенную систему счисления, мы говорим что он \textit{понимает}, как считать\footnote{Дети, кстати, способны схватить неопределенную природу чисел, не выучив настоящую систему счисления. Например, Кэри (2009, 252) цитирует слова одной пятилетней девочки, представляющие спонтанное изобретение аргумента в пользу бесконечности числового ряда: ''Предположим, вы считаете тысячу самым большим числом, но вы же можете сказать тысяча и один, тысяча и два и так далее''. Я в своем изложении предполагаю, что испытуемый все же освоил систему счисления, потому что это помогает избежать сложностей с числовыми языками, превосходящими знания испытуемого о самом числе. Да и в нашей культуре дети в любом случае со временем осваивают системы счисления. Это достижение трудно еще и из-за возможной путаницы с сопоставлением терминов системы счисления со словами-числами естественного языка. (См., например, многочисленные по этому поводу результаты Деэна и его коллег. Некоторые из полученных ими данных указывают на потерю определенных способностей с в то же время сохранением других после инсультов и других специфически локализованных травм головного мозга.)}.

Пусть например теперь он пытается сосчитать предметы набора большего чем всякий до этого встречавшейся --- набора из 57 предметов --- и получает правильный ответ. Крипкианский скептик оспаривает, что этот ответ соответствует предыдущему пониманию задачи счета нашим испытуемым: в прошлом, он говорит, испытуемый никогда не считал --- он зчитал\footnote{См. также Гельман и Галлистел (1986, 51), где рассматривается схожий интересный мысленный эксперимент.

Стратегия Крипке, заключающаяся в подрыве намерений субъекта через подрыв его намерений в предыдущих случаях, подвергалась широкой критике и, как мне кажется, часто неверно понималась. Форбса (1983-1984, 225-226), например, в своей интерпретации пишет: ''утверждение об отсутствии факта о намереваемом субъектом в прошлом двусмысленно: либо у него не было определенного намерения в прошлом, либо намерение было, но неясно, такое же ли, что и сейчас''. Но контекст то явно указывает на верность первого варианта прочтения. Вызов субъекту со стороны скептика ведь в том, что, возможно, он зчитал, а не считал --- вызов этот, значит, подрывает темпоральную идентичность намерений через подрыв всяких фактов о прошлом как определяющих соответствующее в настоящем. И факты о настоящем в противовес прошлому аналогичны таковым в прошлом --- они так же неспособны отличить счет от зчета. С другой стороны, Богосян (1989, 515) понимает эту загадку Крипке как вопрос о том, что вообще определяет условия содержания и смысла --- ''[обладание] условием корректности'', тогда как загадка Крипке лишь косвенно, через вопрос о намерениях субъекта в прошлом, касается этого. Это, конечно, явно неправильное прочтение. Как верно отмечает Форбс (1983, 226), это Райта (1980) (а не Крипке) интересует определяющее условия содержания или значения. (Форбс считает утверждение Райта ''более прямым и сложным'', чем таковое Крипке, и, думаю, он так считает из-за непонимания стратегии Крипке.)}.

Для ответа скептику что испытуемый субъект и в этот раз зчитал и, значит, его ответ про 57 согласуется с предыдущими касательно меньших наборов предметов, нужны факты о субъекте, лежащие в основе ее предыдущего и нынешнего понимания счета. Здесь для описания фактов --- касающихся, например, диспозиций субъекта --- фактов, которые объясняют, к чему приходит его понимание, я ввожу термин ''обосновывающие факты''.

\textit{Некоторая терминология} \quad Я отличаю ''факты соответствия'' от ''обосновывающих фактов'': первые --- это предполагаемые факты внешнего мира, которым соответствуют истинные осмысленные предложения, а вторые --- психологические/диспозиционные факты о субъекте или окружающей его среде, обеспечивающие его способность понимать истинные осмысленные предложения, и, значит, определяющие эти предложения как понимаемые соответствующими фактам внешнего мира. Как вскоре будет ясно, вызов смыслового скептика напрямую оказывает давление на предполагаемые обосновывающие факты, и это, в свою очередь, подрывает факты соответствия.

Слово ''факт'' похоже на те коварные слова, с указания на которые я начал общее введение. Я не собираюсь использовать его в философском смысле, в частности, говорить о сущностях-\textit{фактах}, которым соответствуют предложения (полагаю, что в метафизическом смысле \textit{нет ничего} чему бы они соответствовали). Просто в некоторых из них есть термины, которые отсылают друг к другу, и то, что эти термины обозначают, называется ''фактами''. (См., однако, главу 6 с уточнениям этой идеи для работы с распространенными случаями предложений с несоотнесенными терминами).\footnote{См. также Аззуни (2012c).} Я также не имею в виду метафизически нагруженное понятие ''граундинга'' или что там еще обсуждают во всякой сложной литературе, которая расцвела в последнее время.\footnote{См., например, Коррейя и Шнайдер (2012).} ''Факт'', ''обоснование/граундинг'' и ''обосновывающий факт'' в моем случае обычные дотеоретические термины.

Еще одна терминология требует описания: Крипке различает ''скептические'' и ''прямые'' решения парадокса следования правилам. Прямые направлены на поиск фактов, обосновывающих то или иное понимание субъектом счета. Скептические же признают, что обосновывающие факты не существуют и объясняют понимание субъектом счета как-то иначе.

\textit{Назад к диалектике} \quad Как уже отмечалось, именно предполагаемые обосновывающие факты (или система таких фактов) объясняют истинность того, что в прошлом субъект понимал, как считать, а не как зчитать. Эта модель обосновывающих фактов \textit{связана} с субъектом --- либо с его психическими состояниями, либо с субличностными событями или структурами, лежащими в основе психических состояний, позволивших ему научиться считать. Обосновывающие факты о понимании субъекта должны давать ответ на вопрос: как психологические состояния субъекта --- что он думал, столкнувшись с задачей --- позволили ему понять, что он считает, или: как нейрофизиологические и прочие субличностные репрезентации активируются для счета.\footnote{Кэри (2009), например, предлагает «онтогенетическое» описание того, как ребенок — уже обладающий врожденными субличностными когнитивными системами (параллельное выхватывание небольших наборов и квантификаторы естественного языка), применимыми к конкретным числовым задачам — может в течение полутора лет с помощью индукции и аналогии (так называемая «Квайнианская самонастройка») понять некоторые важные свойства чисел, например бесконечность их ряда, и научиться применять их в подсчете предметов. ПОразительная особенность парадокса следования правилам, как мы увидим, в очевидной нерелевантности эмпирической гипотезы Кэри и ей подомных, предложенных учеными-когнитивистами, для его решения.}

Крипке, я считаю, налагает три ограничения на объяснение структуры обосновывающих фактов --- диспозиционных или других --- которые составляют понимание субъектом счета, и, значит, ответ смысловому скептику.\footnote{Гинсборг (2011, 228) приводит три возражения, которые она (и другие) выдвинула Крипке против диспозиционной теории, возражения эти напоминают (хотя и есть существенные отличия) требования, которые, согласно моему прочтению, он предъявляет к обосновывающим фактам.} \textit{Требование бесконечности.} Любой субъект насчитал лишь конечное число наборов предметов, так что предполагаемая его способность считать всякие другие наборы провоцирует следующие размышления. Во-первых --- он, возможно, считал только, например, яблоки и груши, во-вторых --- он, возможно, считал только наборы из не более чем 57 предметов. Однако ясно, что его навыки счета подразумевают способность считать наборы отличные от прошлых и количественно и качественно, а если испытуемый не осознал эту нейтральность счета и почему-то чувствует, что не может сосчитать, скажем, красные предметы или предметы в определенных коробках или больший чем всякий предыдущий набор предметов, то, очевидно, что-то не так. Возможно он систематически пропускает числа, начинает заново при достижении какого-то числа или даже фиксируется на определенном числе как на ответе и говорит что остальные объекты не в счет.\footnote{Конечно, некоторые из этих ''ошибок'' происходят во время обучения счету, но не большинство. Дети проходят определенные стадии по мере приобретения навыков счете, и ученые-когнитивисты говорят о них как ''знающих один'', ''знающих два'', ''знающих три'', ''знающих четыре'', ''знающих подмножество'' и, наконец, ''знающих количественный принцип''. ''Знающие одно'' умеют отличать один объект от многих, но не могут пока различать наборы предметов по количеству предметов в них, аналогично ''знающие два'' умеют еще различать наборы из не более чем двух предметов по эти предметов в них количествах, и т.д.. Также на определенном этапе освоения навыков счета дети пропускают числа, но, научившись считать, уже никогда этого не делают. Подробное описание этих процессов и ссылки на соответствующую литературу см. в Кэри (2009) и Баттерворт (1999).} Под ''пониманием'' мы ведь подразумеваем способность решить всякую задачу по счету, сколь бы она не отличалась от уже до этого решенных.

Второе ограничение --- это \textit{требование обоснования}: когда испытуемый посчитал предметы и так узнал их количество, его ответ оправдан. Ему \textit{следовало} продолжать так же, учитывая его \textit{осмысление}, не случайно его способ счета дает ожидаемый ответ.

Не менее важно и то, что взрослые и дети убежденны в своей оправданности в таких ответах --- они ведь \textit{считают}, что научились считать, что поняли, что \textit{значит} ''счет'', и если кого-то спросить, \textit{почему} он считает именно так, он ответит ''Потому что \textit{так} это и делается'' или ''Потому что \textit{это и есть} счет'' или ''Потому что \textit{так} принято''. Если ребенок или взрослый считает каким-то странным, необычным образом, он будет оправдываться указанием на одинаковость ответов полученных его способом и обычным.\footnote{''Я сначала группирую их по пять штук, потому что такие группы проще распознать, затем считаю их и умножаю на пять'' --- такое описание может дать ребенок, и дает он его ровно потому что уверен в одинаковости результатов этого его нового метода и обычного (строгого перечисления).}

Обратите внимание, что предложенные обоснования --- что это \textit{так} и делается и т.д. --- за исключением редких случаев, феноменологически достоверны, т.е. субъект, изучив процедуру счета, не будет уже колебаться относительно ее корректности и сразу даст объяснение: ''Это \textit{и есть} счет''. И мы также принимаем это его оправдание от третьего лица и говорим ''Он \textit{понимает}, что значит ''счет'''' или ''Он умеет считать''.

И здесь есть одна тонкость. Согласно моему описанию ''обоснования'', у нас обычно есть \textit{два} ключевых ожидания от субъекта в отношении счета или какой-нибудь другой задачи. Во-первых --- если он умеет считать, то все что он при этом делает, согласуется с его \textit{намерением} (с тем, что он \textit{имеет в виду}). Иногда ведь мы терпим неудачу в выполнении задачи именно из-за \textit{несоответствия} между \textit{делаемым} и \textit{намереваемым}: не делаем того, что ''должны'' были сделать. (''Посмотри что ты наделал! Разве ты \textit{этого} хотел?''.) Второе же ожидание в том, что то, что субъект делает --- действительно \textit{счет}. Могут ведь произойти и другие ошибки: субъект может неправильно решить поставленную задачу, решив, что ему ''нужно'' делать что-то отличное от того, что действительно нужно. ''Нужность'' здесь иная: дело не в том, что он должен был действовать так, как он поступил, учитывая, что и как он понимал, а в том, что он намеревался выполнить одну задачу, но следовало то ему намереваться выполнить другую. Феноменологическая уверенность, значит, ощущаемая при выполнении поставленной задачи, включает оба ожидания и оттого двумя способами и может быть подорвана.

Не так уж много литературы есть об этой ''нормативности'' касательно соблюдения правил и того, насколько нагруженной она должна быть.\footnote{Дебаты о ''нормативности содержания'' или ''нормативности значения'' --- т.е., как нормативность участвует в следовании правилам и определении значения,- впрочем, продолжаются. Некоторый взгляд на нормативность значение приписывается и Крипке, потому что, например, согласно Гинзборг (2011, 228), он утверждал, что ''человек, \textit{склонный} или бывший \textit{склонен} определенным образом реагировать в определенной ситуации, не обязательно \textit{должен} так реагировать''. См., например, Богосян (1989, 2003), Глюер (1999), Викфорсс (2001). Как я уже говорил (и еще буду повторять), ''нормативность'', ''рациональность'', ''нормы'', ''правильность'' или ''уместность'' касаются этой темы только в смысле предполагания ими успешного следования правилам, а не в смысле приполагания самим следованием правилам какого-то из них. Частично дело в том, что используемое здесь ''должен'' лишь гипотетично и может быть понято аналогично таковому в, например, ''Если хотите жить, вы должны не прыгать с моста''. (''Если хотите жить, я бы советовал ....'') Или ''если действительно хотите жить, было бы неуместно прыгать с моста, не так ли?'' (Представьте, что это говорит ангел-хранитель в фильме, потенциальному самоубийце.) Гиббард (2003, 85) пишет: ''... стоит ли нам гулять --- может зависеть от погоды, но это ведь не делает погоду нормативной в каком-то особом философском смысле''. Итак, здесь я на стороне отвергающих ''нормативность значения''. Подробнее об этом я говорю в Разделе 2.3.} Я буду понимать ее в ее тонком, облегченном варианте. ''Обоснование'' же буду понимать как \textit{согласованность} того, что \textit{имеется в виду} субъектом, и того, что им действительно делается. То есть требование обоснования значит достаточность ''встроенного'' в то, что субъект имел в виду, для подтверждения соответствия того, что он делает --- тому, что он имел в виду. Это и лежит в основе его феноменологической уверенности --- он может ''объяснить'', почему им делаемое и есть им намереваемое и другими от него требуемое. То есть здесь фигурирует как многообразие того, что имеется в виду --- оно определяет соответствующее поведение, --- так и \textit{наше} понимание этого значения и им подразумеваемого.

Требование обоснования тесно связано с \textit{требованием ошибки}: наша уверенность в способе и результатах счета справедлива с точностью до случайной ошибки. Любой ведь может ошибиться при подсчете предметов и так получить неверный ответ --- в этом нет ничего необычного. Мы, однако, четко отличаем такую ошибку от непонимания концепции счета, и --- что особенно важно --- даже систематические ошибки от случаев недостаточного усвоения счета и случаев когда \textit{с субъектом} просто что-то не так.

Отвлечься и проглядеть какой-нибудь предмет или дважды его посчитать --- значит ошибиться. Другое дело, когда испытуемый из раза в раз намеренно считает, например, все красные объекты дважды. Если он при этом действительно думает, что считает, то, похоже, он не понимает что значит считать, а иногда даже можно сказать, что и неспособен этому научиться. Важный признак ошибки --- возможность ее \textit{признания}. (''Вы пропустили вот этот'' --- ''Упс, сейчас исправлю''.)

Как бы мы ни описывали способность конкретного субъекта считать и как бы мы его диспозицию, преобретенную в ходе обучения счету, ни характиризовали, надо оставить место для способности давать \textit{неправильные} ответы и этих ответов неправильность \textit{признавать}. Странно, конечно, называть это способностью, но именно таково требование. Мы часто получаем неправильные ответы, даже прекрасно разбираясь в счете, и любое описание схемы обосновывающих фактов нами используемых для объяснения способности считать --- способности, раз уж на то пошло, выполнять любую данную задачу --- должно оставлять место для возможности ошибок и возможности этих ошибок распознания.

Следует еще раз подчеркнуть, насколько ошибки --- даже систематические к ним \textit{тенденции} --- совместимы с нашим пониманием таких понятий как счет, сложение, вычитание, умножение и деление. Некоторые люди просто потрясающе хорошо считают --- быстро и точно и даже большие числа --- но врядли можно сказать, что они как-то более глубоко и точно \textit{понимают} саму концепцию счета. Они, наверное, знают больше соответствующих трюков и обходных путей и \textit{запомнили} больше фактов о числах, да и просто быстрее оперируют ими. Даже кто-то вроде меня, кто почти гарантированно допустит какую-то элементарную ошибку, считая в уме или даже на бумаге, не считается плохо разбирающимся в \textit{понятиях} счета, сложения и т.д.. Что же \textit{действительно} требуется, так это способность \textit{распознать} допущенную ошибку --- без этого субъект не может считаться полностью усвоившим рассматриваемые концепции.\footnote{Как я укажу далее, это требование распознавание хотя и применимо к счету и прочим арифметическим операциям, все же не является обязательным для всякой концепции, которую мы пытаемся понять --- даже наоборот: часто смысл термина считается понятным даже тогда когда не понятно, когда некоторые (или даже все) его использования неверны. (Например, я не буду считать кого-то не понимающим концепцию Бога, даже если я не согласен с ним во всем или почти во всем, что он о Боге утверждает.)}

\section{Почему интроспективных ресурсов недостаточно}

Я начну с предложения, которое приписываю Витгенштейну Крипке. Дело в том, что ресурсы для понимания счета имеют лишь два возможных источника. Первый --- интроспективное понимание, то есть осознание чего-то, равносильное пониманию счета. Второй --- склонности к моделям поведения, сознательному или бессознательному. Лишь после признания интроспективного варианта безуспешным, сможем мы обратиться ко второму.

Вспомним смысл вызова скептика: почему субъект всегда считал а не зчитал, что на это указывает? Вполне естественно сначала ответить исходя из первого --- интроспективного --- варианта. Субъект, например, обычно считается \textit{распознающим} нужную для подсчета закономерность потому что видел некоторое количество примеров и теперь, значит, \textit{видит} как действовать аналогично. Но (Крипке (1982, 18)) ведь никакое конечное число примеров не определяет закономерность полностью. И точно так же любая попытка вменить субъекту осведомленность о правиле или алгоритме, совместимом с предыдущими случаями его применения своих способностей к счету и которое, значит, определяет ответы на последующие задачи --- любая такая попытка безуспешна от множественности интерпретаций конечного набора предыдущих задач и корректных на них ответов. Ведь любое такое правило можно понимать по-новому хотя и все еще совместимо с тем, что ''имеет в виду'' субъект, и со всеми прошлыми его подсчетами.\footnote{Крипке ставит проблему смыслового скептика от первого лица, то есть как проблему совместимости моей прошлой практики и нынешнего мышления с ''квожением'' вместо предполагавшегося сложения. Я же поставил проблему эту от третьего лица и касательно счета и ''зчета''. Учитывая, что мы обычно позволяем себе описывать феноменологию третьих лиц (во многом так же, как я это и сделал выше), это, думаю, не вызовет каких-то сложностей. Если же читателя использование такой обыденной практики все же не устраивает, он может легко переформулировать мои тезисы и рассуждения в терминах первого лица --- это не на что существенно не повлияет.}

Из этого теперь видно, как требование бесконечности Крипке исключает естественное описание необходимой модели обоснования фактов в терминах \textit{интроспекции}. Как и отмечали многие философы, проблема не в приобритении способности давать правильные ответы для \textit{бесконечного} числа новых случев --- нет --- нужна способность давать правильные ответы для \textit{новых} случаев вообще. Даже если испытуемый во второй раз сталкивается с тем, что кажется нам точно таким же заданием на счет, он все же может сделать что-то другое, но при этом описать это как ''то же самое''. (Ну ведь и правда --- второе задание, может, он решает уже во вторник, а не в среду, или в полнолуние, а не в новолуние, во всяком случае он \textit{позже} это делает.)\footnote{Крипке (1982, 52, сноска 34) неявно признает такую точку зрения, когда цитирует Витгенштейна: ''Если я знаю это заранее, какой от этого знания мне прок позже? То есть: как мне знать, что делать с этим более ранним знанием, когда шаг уже сделан?'' (Витгенштейн (1956, I, §3))}

Проблема в том, что возможные ментальные состояния, охватывающие понимание правил посредством воплощения их в лингвистических формах --- даже если с использованием кванторов --- или во что-то другое, например, в мысленных образах --- визуальных, кинестетических, или как воспоминаний о ранее выполненных задачах --- эти возможные ментальные состояния, тем не менее, надо \textit{применить} к текущей, новой, задаче. И то, как они применяются, указывает на то, \textit{как они интерпретируются} --- это не что-то заранее фиксированное тем что сейчас можно интроспектировать. Многие думают, что это урок Витгенштейна (1958) из параграфа 139 и близлежащих. Патнэм (1981, 20) излагает его так: ''Феноменологи не видят, что хотя описываемое ими является внутренним \textit{выражением} мысли, \textit{понимание} этого выражения --- понимание собственных мыслей --- это, тем не менее, не \textit{являние}, а \textit{способность}''.\footnote{Райт (1984, 771) же более осторожен, говоря: «Интуитивно, понимание выражения скорее способность, чем склонность».}

Здесь следует выделить несколько моментов. Как широко отмечалось (например, Райт (1989, 109)), для оценки достаточности интроспективных ресурсов для определения соответствия того, что имеет в виду субъект сейчас, тому, что он имел в виду раньше, Крипке идеализирует сознательный доступ к своим прошлым психическим состояниям, как бы давая субъекту знать \textit{все} о своей прежней психической жизни и поведении, так что \textit{воспоминания о своих намерениях} становятся частью интроспективных ресурсов, и тогда, конечно, можно не сомневаться, что субъект именно считать намеревался, а не зчитать. Он, возможно, явно думал о слове ''считать'', например, или само понятие ''счет'' как-то иначе задействовалось в ее мышлении --- через образы или понимание примитивной целесообразности делаемого.\footnote{См., например, Гинсборг (2011).} Почему же подобных воспоминаний может быть недостаточно для ответа скептику, почему нельзя сказать ''Это и есть то, что я \textit{имею в виду сейчас}, ведь это ровно то, что я textit{имел в виду раньше}''?

Крипке (1982, 51) действительно отвергает версию, согласно которой, состояние ''состояние сложения'' является примитивным или \textit{sui generis}, потому что природа состояния \textit{sui generis} остается ''совершенно загадочной''. Из-за этого его ответа (и его обоснования) некоторые философы обвинили Крипке в предвзятости относительно антиредукционистских ответов смысловому скептику.\footnote{Богосян (1989, 527), Фодор (1990, 135-136), Макдауэлл (1984) и другие; дополнительные примеры см. в Гинсборг (2011, 229), сноска 5. Следует добавить, что термин ''проблема следования правилам'' может вводить в заблуждение. Речь ведь о том, что \textit{всякое} в субъекте --- ресурс для продолжения того же, что он делал раньше, и ресурс этот не обязательно буквально ''правило'' или имеющая ''значение'' часть публичного языка. Это важно, потому что, похоже, некоторые комментаторы понимают Крипке как работающего в рамках какого-то из такого рода ограничений.} Гинсборг (2011, 229), например, пишет, что поддерживающие антиредукционизм касательно значения критики возражали, говоря, что ''факт намерения или следования правилу следует определять в чисто натуралистических терминах''.\footnote{Действительно, Гольдфарб (1985 например 476) понимает Крипке как бросающего вызов ''любой предполагаемой физикалистской редукции значения''.} Это, однако, недооценка мысштабы и методы смыслового скептика Крипке, ведь предположение просто таки интроспективного схватывания своих знаний субъектом и такого \textit{схватывания} достаточности для ответа скептику, действительно делает эти интроспективные ресурсы совершенно загадочными. Как, в конце концов, содержимому сознания субъекта (словам публичного языка, словам мысленного языка, ментальным состояниям, образам и чему угодно еще) удается существовать связанно и репрезентировать \textit{именно эту}, а не другую, вещь? Один из даваемых ответов --- некоторые ментальные состояния (или слова, или что-то еще) просто фактически \textit{способны} на это, в частности, есть \textit{внутренние} смысловые состояния, в которых мы иногда и находимся, \textit{отдавая себе в этом отчет}. Увы, но к чему уж интроспекция явно не имеет доступа, так это к механизму устранения \textit{неоднозначности}. Интуитивность описания Крипке смыслового скептицизма сама  по себе показывает, что \textit{все} воспоминания о выполнении определенных задач --- воспоминания, \textit{включающие} использования слов, концепций и образов, различные впечатления касательно \text{намерений} и правильности делаемого --- все они вполне совместимы как со счетом, так и со зчетом. Проблема для субъекта здесь в отсутствии у его интроспективного понимания счета ''содержания, достаточного для исключения всех нежелательных интерпретаций его прежнего понимания'' (Райт (1984, 765)).

А значит обвинение в предвзятости относительно антиредукционизма некорректно. Скорее уж вариант примитивного (нередуцируемого) смыслового состояния используется в качестве ответа смысловому скептику дважды, и оба раза неуспешно: первый раз он безуспешен в рамках исследования доступных субъекту интроспективных источников, второй --- при рассмотрении диспозиционных подходов (это будет обсуждаться в Разделе 2.3).

Прежде чем перейти к диспозициям, дам еще два замечания. Во-первых, некоторые философы обращаются к стандартным нашим способам обобщения конечного числа случаев чтобы применить их к нашему доступу к собственному ментальному содержанию. Например, на основании некоторых наблюдений, мы можем сделать вывод, что все вороны черные. Почему подобное же схватывание общности не может возникнуть от созерцания воспоминаний о предыдущих задачах и их выполнении?\footnote{См., например, Форбс (1983-1984, 234-235).} Крипке противопоставляет этому важный факт касательно нашего опыта намерения делать что-то конкретное вообще и опыта намерения считать в частности: \textit{это не гипотеза}. (Момент этот, кстати, возникнет еще в качестве возражения против диспозиционных подходов.) Мы просто \textit{знаем} свое намерение --- знаем, что имеем в виду --- а не выдвигаем индуктивно или как-то еще касательно этого гипотезы. Второе же наблюдения касается схожего возражения смысловому скептику, выдвигаемого Райтом (1984, 776-777):

\smallskip

\textit{Особенность нашего интуитивного понятия намерения такова, что скептический аргумент бессилен против него. Намерение --- наряду с мыслью, настроением, желанием и ощущением --- таково, что субъект имеет к его содержанию непосредственный, авторитетный доступ, и что содержание его может быть открытым и обобщенным так, что относиться ко всем ситуациям определенного рода.}

\smallskip

Это, несомненно, коректно характеризует ''феноменологию значения''\footnote{См. также Гольдфарб (1985) и Гинсборг (2011).}, но смысловой скептик \textit{успешно} оспаривает такого опыта достоверность лишь задавшись вопросом о в этом интроспективном опыте \textit{конкретном} источнике необходимого объема и общности того, что имеется в виду --- о том, \textit{что}, значит, определяло, что имеется в виду счет, а не зчет.\footnote{Гольдфарб (1985, 474) предполагает, что фрегевский ''непосредственный доступ к сфере чувств'' --- вполне убедительный ответ смысловому скептику. Тейт (1986), возможно, предлагает тот же ответ (здесь я не уверен). Во всяком случае, кажется уместным замечание Рассела о преимуществах воровства перед честным трудом. Предполагается, что нужно указать, что такого \textit{в субъекте}, что позволяет схватить ему именно счет а не зчет. Указание на определенные умственные его способности в качестве такого признака --- это, конечно, не ответ, как и утверждение, что определенные объекты (функции) --- просто вещи, которые субъекты \textit{могут} схватить.} Почему вообще наше якобы непосредственное \textit{ощущение} обобщенности и определенности того, что имеем в виду, пригодно в качестве ответа смысловому скептику? Почему тогда убедительность смыслового скептицизма не является само по себе доказательством возможности признания нами опыта намеревания лишь своеобразным интроспективным \textit{долговым обязательством}, исполнение которого зависит намереваемой и реализуемой функции?

\section{Как три ограничения Крипке блокируют диспозиционные подходы}

Именно здесь берутся за тот или иной диспозиционный подход. (Частично для того, чтобы найти ресурсы (\textit{осознаваемые или нет}) \textit{у} субъекта, которые указали бы на \textit{принужденность} субъекта к определенному намерению --- намерению считать, например.) Начну с простой формулировки этого подхода:

\smallskip

  \textbf{ДИС 1} \quad Понять ''счет'' --- значит быть готовым дать [правильный] ответ на вопрос о количестве предметов в наборе.

\smallskip

Это, впрочем, не работает, потому что совместимо с умением давать правильные ответы в действительности \textit{не} считая. Например, мы (взрослые люди, младенцы и многие животные) не считаем группы из одного, двух или трех предметов, потому что \textit{распознаем} их сразу таковыми.\footnote{Это так называемая ''субитизация'' и, похоже, она ограничена четырьмя предметамв случае взрослых и тремя --- в случае детей и некоторых животных. См. Мандлер и Шебо (1982).} А можно ведь представить и того, кто мог бы так \textit{распознавать} наборы любого размера (по, например, заполнению пространства объектами разных форм и размеров). Такой человек мог бы даже не осознавать --- и даже не быть способным осознать --- возможности линейно упорядочить наборы по количеству предметов в них. Так что следует еще зафиксировать конкретный способ подсчета:

\smallskip

\textbf{ДИС 2} \quad Если некто освоил ''счет'' и его спросили о количестве предметов в наборе, он склонен указать на каждый предмет один за другим, называя числа начиная с ''1'' при указании на первый, число следующее за ''2'' --- при указании на второй, и назвать ответом число, названное при указании на последний.\footnote{ДИС 2 неправомерно выделяет лишь \textit{один} метод. Как уже отмечалось в Разделе 2.1, субъект может овладеть другими подходящими (возможно, упрощенными) методами. Я оставил попытки модифицировать ДИС 2 так, чтобы избежать этой проблемы, потому что модификации эти не повлияли бы на общую диалектическую этой главы траекторию.}

\smallskip

Что ж, давайте посмотрим, насколько ДИС 2 соответствует требованиям Крипке. Начнем с требования бесконечности. И проблема в том, что мы обычно \textit{не} склонны действовать так, как в нем описано --- а он ведь требует такого поведения при любых или почти любых обстоятельствах. Если, например, набор достаточно большой и субъект устал считать, то он уже не сможет или откажется выполнять предписания ДИС 2. И это, похоже, верно в отношении всех людей. Вообще, эта их склонность чувствительна не только к размерам наборов, но и к свойствам составляющих их предметов и к общему состоянию субъекта. Если, например, предметы сильно блестят, или расположены слишком близко друг к другу, или очень маленькие, или ползают друг по другу, то субъект будет склонен вовсе отказаться от такого задания (и бежать в попытках спастись), чем прилежно заняться их подсчетом. То есть, если верить ДИС 2, \textit{никто вообще} не освоил счет.

Как отмечает сам Крипке (1982, 30–32), лучше охарактеризовать эту склонность как то, что человек \textit{сделал бы}, будь, например, его мозг больше, а глаза не столь чувствительны... Впрочем и так модифицированный ДИС 2 не работает: возможно все эти изменения \textit{в нашем строении} заставят нас как раз таки отказаться от правильного выполнения задачи. Мы же не можем гарантировать, что изменения, направленные на, казалось бы, устранение избегания или неправильного выполнения задачи счета, не повлияют на людей каким-то еще другим, мешающим выполнению задач, образом.\footnote{Факты нашей биологии (например, строение мозга), физические законы и т.д. --- все это имеет значение. Популярное рассмотрение этой проблемы с картинками и диаграммами см. в Фокс (2011).}

А что насчет третьего ограничения? Можно подумать, что ошибка ДИС 2 в неучтенности того простого эмпирического факта, что мы допускаем ошибки. Мы ведь обычно не требуем от тех, кто еще только учится считать (да и от самих себя), стабильно верного счета при любых обстоятельствах --- мы по разным причинам делаем исключения, например когда субъект устал, болен или пьян, или предметы такие, что их неудобно считать, и даже допускаем казалось бы \textit{необъяснимые} неудачи при даже привальном выполнении задания. (''Ой, точно, \textit{о чем} я вообще думал?'') То есть несмотря на такие вот отклонения от предписываемого ДИС 2, мы все же обычно готовы признать субъекта умеющим считать. Но почему бы тогда не исключить такие случаи, позволив в них все же считать субъекта умеющим считать?

Потому что это предвосхищает основание против альтернатив. Рассмотрим, например, последовательность* --- нечто, отличающееся от последовательности лишь в небольшом количестве случаев. Точно так же можно описать субъекта как понимающего счет*, исключив \textit{задачи, в которых он склонен отклоняться от счета*}. И что же тогда укажет нам на использование им счета а не счета*?

Это исправление могло быть предложено и ранее --- для удовлетворения требования бесконечности ДИС 2: исключения можно использовать для учета бесконечного количества случаев, в которых субъекты склонны не выполнять задачу.\footnote{Эти модификации (с использованием контрфактуалов и с использованием исключений) на самом деле представляют одну и ту же стратегию, грамматически по-разному реализованную: в первом случае нарушения ДИС 2 упаковываются в контрфактические события, а во втором исключения содержатся в антецедентах изъявительных кондиционалов.} Но это также предвосхищает основание против альтернативных подходов --- ДИС 2* например --- где ''следующее'' заменяется на ''следующее*'' и где различия проявляются только в тех ответах, что субъект не дает.

Проблема ДИС 2 в том, что он пытается определить склонность субъекта при столкновении с задачей счета через \textit{корректные} его процедуры, но --- как мы только что видели --- даже рассмотрение склонностей этих как имеющихся у реальных субъектов и использование исключений для случаев этих склонностей непроявления не помогает. Сторонники понимания испытуемым некой нестандартной формы счета могут использовать этот же ход.

Рассмотрим, наконец, требование обоснования. Здесь проблема в том, что диспозиционная точка зрения предлагает в действительности не используемые субъектами, для обоснования счета в прошлом, ресурсы. Крипке (1982, 23) пишет:

\smallskip

\textit{Должен ли я обосновать свое нынешнее убеждение, что я имел в виду счет, а не зчет с точки зрения гипотезы о прошлых моих склонностях?}

\smallskip

Я понимаю его так: \textit{только лишь} описание склонности как обоснования для \textit{счета} не сочетается с нашим феноменологическим впечатлением касательно понимания другими людьми их схватывания таких концепций, как счет. Только посмотрите, насколько странно это бы звучало:

\smallskip

\textit{Я умею считать, ведь склонен отвечать на вопрос ''Сколько здесь предметов?'' указывая на каждый из них один за другим и произнося ''1'' на первом из них, следующее за ''1'' число --- ''2'' --- на втором и т.д. до последнего, произнесенное на котором число и называю ответом.}

\smallskip

Скажи кто-нибудь так, он скорее подумает о себе как о выучившемся какой-то бессмысленной, нелепой процедуре \textit{вместо}, собственно, счета, или как об описывающем только \textit{побуждения}, но не образ действий, им предпринимаемый согласно выученной концепции.\footnote{Крипке (1982, 17) пишет: ''Мы, обыкновенно, рассматривая математичекое правило вроде сложения, понимаем себя руководствующимися им в каждом новом случае''. И (Крипке (1982, 10)) ''Я следую указаниям ...''.} Мы заподозрим это, потому что мелочное это его обоснование вообще-то ничего не говорит о том, почему же описание его склонностей \textit{имеет хоть какое-то отношение} к умению считать.

Ранее я уже упоминал, что мы часто доказываем свое умение демонстрацией соответствующего поведения, вроде ''Думаешь, я не умею считать? Ну, смотри! Раз, два, ...''. Но это же не значит, что счет в такой демонстрации и \textit{заключается}. Скорее, демонстрирующий предполагает знание собеседником счета и \textit{это свое предположение} и свое знание счета ему и показывает через демонстрацию своих навыков и этих навыков осознание.

Обратите внимание, что (согласно моей интерпретации) акцент Крипке на подрыве диспозиционных подходов требованием обоснования заключается не в этого обоснования ''нормативности'' в противовес диспозициям --- нет --- диспозиционные подходы попросту не соответствуют пониманию своей обоснованности субъектами, нарушают очевидные об \textit{опыте} такого себя осознания факты.

\section{Попытка защиты диспозиционных подходов}

Многие комментаторы Крипке защищают диспозиционный подход аналогией между использованием контрфактуалов для описания поведения стекла, соли или газов в воображаемых обстоятельствах и таковым их использованием для описания действий субъекта,\footnote{Блэкберн (1984, 289-290), Форбс (1983-1984, 229-230), Фодор (1990, 94-95). Среди одобряющих этот ответ, например, Богосян (1989), Гинсборг (2011). Райт (1984, 771), напротив, категорически отвергает его, указывая на интуитивно большую схожесть понимания со способностью, нежели со склонностью.} предполагая включенность в первые как исключений (''\textit{ceteris paribus}''), так и идеализаций, которые, согласно Крипке, бессильны подкрепить ответ смысловому скептику, хотя и, сталкиваясь с требованиями бесконечности, предположительно диспозиционно их преодолевают. В бесчисленном множестве разных обстоятельств структурные факты, определяющие диспозиции стекла и соли, определяют также и бесчисленное множество исключений: ''соль \textit{не} растворяется в холодной соленой воде'', ''соль \textit{не} растворяется в воде, если Бог вмешивается и регулирует \textit{соответвенно} движение молекул'' и т.д..

Но есть существенное различие между стеклом, солью и пр. и считающим субъектом.\footnote{В этом и следующих пяти абзацах приводится защита неприятия Крипке диспозиционных подходов, им самим, и, насколько могу судить, кем либо другим, однако, не данная.} Различие это в использовании лежащих в основе идеализированных описаний стекла и пр. структурных фактов для определения этих описаний ложными, а не истинными. Дело ведь не в том, что соль всегда расстворяется или что газы ведут себя подобно скоплениям маленьких твердых шариков --- дело в \textit{склонности} соли вести себя именно так, \textit{как она себя и ведет} исходя из своих структурных свойств. То есть \textit{строго говоря}, соль \textit{не} ''водорастворима'', а лишь растворяется в воде в широком \textit{таких-то} диапазоне обстоятельств (но не растворяется в \textit{других}), однако первые настолько особенные, что ярлык ''водорастворимая'' все же оказывается полезен. Ну и аналогично газы --- в широком диапазоне обстоятельств --- ведут себя \textit{примерно} как скопления маленьких твердых шариков.

Другое дело --- счет. Изучая нейрофизиологию способностей к счету, \textit{можно} обнаружить, что испытуемые склонны делать нечто отличное от просто счета --- хотя и \textit{нельзя} обнаружить, что они не понимают, что такое счет. Эмпирически, конечно, допускается обнаружение верности характиристики человеческих склонностей как функции зчета; но ведь это \textit{не} значило бы, что испытуемые зчитают, а не считают --- не было бы доказательством, что они не умеет считать, а умеют зчитать, скорее лишь указало бы на слонность совершать определенные систематические ошибки при счете.

Не буду, впрочем, давать прогнозов на основе мысленного эксперимента, а лишь опишу текущие исследования касательно математические способностей и что они предполагают об наших математических склонностях \textit{прояснить}. Первое, что следует отметить --- открытия касательно математических склонностей \textit{на самом деле} показывают их отклонение от представляемого нами правильными способами решения. Философы предположили, что нужные для ответа скептику Крипке диспозиции можно ''стратифицировать'', отделив первичные, нужные для правильного счета --- от вторичных, правильному счету мешающих.\footnote{См., например, Блэкберн (1984, 290) и Форбс (1983-1984). Гольдфарб (1985, 477) пишет: ''Редукционист может, например, заявить, что в будущем физиологическая психология раскроет два различных на науных основаниях механизма. Состояния первого уровня будут соответствовать языковой компетентности и, будучи несдерживаемы, приводят к лишь правильным ответам, состояния же второго уровня соответствуют признакам мешающим и так объясняющим ошибки''. Но, хотя генерирующие \textit{правильные} функции бесконечные диспозиции и \textit{возможны нейрофизиологически} --- касательно арифметических способностей это было исключено еще к началу 1990-х годов.} Пусть это различие и в чем-то соответствует нашему опыту решения математических задач --- мы действительно совершаем ошибки, которые, как \textit{иногда} затем признаем, вызванны были мешающими факторами --- это не свойство задействующихся в нашей арифметической компетенции предрасположенностей.\footnote{Вот некоторая литература по этой теме: Кэри (2009), Деэн (1997). Ее, конечно, гораздо больше --- это ведь очень активная сейчас область исследований, потому что ''диспозиции'' крайне сложны и запутаны и о них не получается говорить как о лишь механически следуемых нами (покуда ничто не мешает и не вызывает грубые ошибки) воплощенных арифметических правилах.} Интересно, что наш интуитивный числовой ряд подчиняется закону Вебера: числа расположены неравномерно, группируясь по мере возрастания, также у нас есть нечеткое, аналоговое ''чувство'' количества объектов в группе.

Определяй наши диспозиции наше понимание чисел, у нас, возможно, было бы и более одной используемой при счете системы чисел: одна --- нечеткая, но открытая, другая --- строго конечная, а третья --- подчиняющаяся закону Вебера. И сторонник диспозиционного подхода должен, конечно, объяснить, как этот ошеломляюще сложный клубок нейрофизиологичесвких способностей обеспечивает такое простое, стратифицированное понимание стандартных арифметический функций вкупе с условиями \textit{ceteris paribus}, исключающими поведенческие отклонения от этих функций.

Это ведет нас еще к одному разрушительному для эмирически обнаруживаемых диспозиций моменту. Даже если описанные сложности с ними не были обнаружены текущими исследованиями, наше признание эмпирической возможности отклонения субличностных арифметических способностей, от необходимых для корректного решения соответствующих задач, указывает на отличие нашего понимания усвоенности счета от \textit{только лишь} склонности к выполнению задач на него. Именно \textit{в этом смысле} следует проводить различие между нашими склонностями и тем, что мы \textit{должны} делать --- \textit{правильными} действиями.

Это, однако, поднимает следующую проблему: учитывая исчерпаемость ресурсов (к пониманию счета и прочих математических концепций) субъекта диспозициями, остается вовсе неясным определяющее субъекта как намеревающегося считать. Иначе говоря, что фиксирует смысл --- содержание --- его концепции \textit{счета}? Что, в конце концов, \textit{могло бы} помочь нам здесь, если не его диспозиции?\footnote{Фиксация содержания --- проблема, по мнению Богосяна (1989), Крипке и поднимает, и именно ее непосредственно поднимает Райт (1980) (см. примечание 4). Но соображения, им при этом используемые, отличаются от приведенных мной: он (2003, 496) пишет ''Раньше я недооценивал силу [требования бесконечности]''.}

Здесь уместны два замечания. Во-первых --- проблема фиксации содержания возникает здесь в диалектическом поиске в субъекте \textit{чего-то} определяющего усвоение им в прошлом счета, а не зчета, и, значит, счета им сейчас и намеревание. Мы могли бы, конечно, оспорить его намерение, исходя из несостоятельности диспозиционных подходов, но это потребует дополнительных предположений, которые, думаю, Крипке не делает. Аргумент против диспозиционных подходов смысловой скептик сопроваждает аргументами против интроспективных ресурсов.

Во-вторых --- ''нормативность'' арифметических законов или нашего таких законов понимания. Проблема диспозиционных подходов, в конечном итоге, в отличии склонности реагировать определенным образом в определенных ситуациях от должности так реагировать (Гинзборг (2011, 228)). Будет лучше --- во всяком случае, не столь запутывающе --- сказать, что это все значит, а именно: лишь негативная характеристика описания диспозиций как недостаточного для каких-либо утверждений касательно его умения считать. Слова ''правильно '' и ''должен'' \textit{лишь} вбивают клин между описанием возможного поведения субъекта и реальной функцией (например, функцией последовательности), рассматриваемой через ее этим субъектом понимание --- и ''редукция'' не удается, вот и \textit{все}. Аналогично, сказать ''правильность --- нормативный вопрос'' (Богосян (2003, 36)) --- значит не что иное, как отличие фактической функции от той, которую субъект \textit{склонен} выполнять.

Заманчиво, наверное, предположить неизменность или исчезновение различия Хомского между знанием и умением вместе с различием первичным и вторичным в диспозициях --- но это не так. Обе стороны --- знание и умение --- эмпирически чувствительны: ''эффективность'' --- не просто свалка всякого неграмматического поведения субъекта. Синтаксис языка субъекта можно пересматривать на основании эмпирических результатов, и, кроме того, неумение тоже эмпирически ограниченно и проверяемо. Нарушение памяти, например, нельзя сходу считать причиной ошибок в синтаксисе --- факт этого нарушения открыт эмпирической проверке. Вторая причина различия --- отсутствие в случае синтаксиса ключевого арифметического фактора --- правильности. Как я уже указывал, универсальные диспозиции к выполнению счета с помощью необычных (конечных, нечетких и т.д.) систем счисления --- показатель не не-счета, а лишь отличия нарушенного синтаксиса от предполагаемого лингвистом.

\section{Почему прямое социологическое решение не работает}

Как мы увидели, диапазон индивидуальных диспозиций очень мал и они не соответствуют ограничениям Крипке. Иначе говоря, они не дают психологические/индивидуальные обосновывающие факты для каждой возможной задачи счета (где обосновывающие факты --- достаточные для определения фактов соответствия вроде ''\textit{Каждая совокупность объектов имеет одно и только одно число, называемое количеством}''). Потому и возникает соблазн дополнить диспозиции контрфактуалами. Это, конечно, очень естественный ответ на проблему, ведь мы часто оправдываемся, указывая на усталость, которая и помешала выполнить задачу --- мы так абстрагируем характеристики нашего понимания счета и способности к нему от прочих факторов, ограничивающих наши к счету возможности (почти так же действия машины Тьюринга абстрагированы от физических ограничений). Именно поэтому кажется, что ограничение бесконечности удовлетворимо дополнением диспозиций контрфактуалами.

Однако, стратегия эта не удовлетворяет ограничению обоснования. Действительно: сам факт сомнения в способности таких дополненных диспозиций выполнить требуемую от них работу указывает, что диспозиции --- \textit{какие бы они не были и как бы не дополнялись} --- не являются \textit{стандартами} правильного счета. Будь они таковыми, не было бы и речи об их правильности --- \textit{какой бы ответ они не давали}. Но как бы слонности субъекта к счету не выглядели, кажется возможным сомневаться в их ответов правильности и более того, дают ли наши диспозиции такие ответы. Было бы странно услышать от субъекта после подсчета им что-то вроде ''Видите ли, это правильный ответ, потому что именно его я был расположен дать''.\footnote{Это чрезвычайно важный аспект нашего понимания следования правилам, и обусловлен он глубокими фактами нашей психологии, например тем, как мы учитываем (или не учитываем) субличностные способности касательно сознательно решаемых задач. Это интересное взаимодействие между личностным и субличностным, впрочем, не имеет значения для первоначальных моих моделях приватного следования правилам (см. Главу 3), однако включается в пятой модели в Главе 5.}

Это побуждает искать стандарты правильного счета где-то еще. Более того, современная научная практика предполагает незначительность склонности \textit{людей} к правильному счету --- на них в этом редко полагаются, в большинстве своем используя теории и инструменты (буду называть их ''калькуляторы''). Итак, похоже, учитывая широкую распространенность калькуляторов, единственно значимой для нас диспозицией становится диспозиция \textit{принимать} результаты, полученные с эти калькуляторов помощью. Исторически, использование счетов, счетных досок, веревок с узлами и прочего для счета --- указывает, что значимость лишь этой диспозиции касательно счета не нова. Наша способность считать точно и приблизительно не опирается на что-либо разумно характеризуемое как диспозиция к счету --- мы же не рассматриваем в качестве этой диспозиции таковую к принятию результатов, выдаваемых [изобретенными другими людьми] инструментами.

Обсуждая в Разделе 1 требование обоснования, я подчеркивал наше чувство оправданности в выборе методов счета, и это справедливо даже касательно [уверенного в себе] ребенка --- он будет озадачен заявлением, что все им делаемое как счет --- на самом деле зчет. Крипке весьма успешно использует этот феноменологический факт против диспозиционной теории: наша вера в обоснованность корректности нами делаемого не удовлетворяется одним лишь признанием диспозиций делать это определенным образом. Факт этот (в сочетании с некоторыми другими) может означать необходимость оценивать правильность поведения согласно социальным стандартам.

Учтите ведь, что испытуемый обыкновенно не слишком уверен в своих способностях к счету. Но даже уверенный в своих способностях может в них усомниться в определенных обстоятельствах.\footnote{Здесь часто приводится неуверенная реакция в некоторой ситуации --- например, в классе, где были даны [очень простые] инструкции ''поднимайте руку, когда учительница держит зеленую карточку, и опускайте --- когда она держит красную'', и где однако учительница и большинство учеников согласованно эти инструкции нарушают.} Это потому, что ключевым для понимания любой задачи является ''способность'' на ошибки и возможность инструментов или других субъектов на эти ошибки указать, и потому, что масштабы этих ошибок не имеют предела. Поэтому субъект может разувериться в \textit{чем угодно}, ранее, казалось бы, несомненном. А вот если бы стандартом поведения были \textit{лишь} его склонности, такая неуверенность была бессмысленна. И \textit{это}, очевидно, показывает, что рассматриваемые стандарты --- внешние к кому-либо конкретному и локализованы в тех, кто его окружает --- в сообществе.

Можно предположить, что, упоминаемые Крипке как часть феноменологии и как обычно плохо сказывающиеся на диспозиционном ответе, ''нормативные'' факты, а именно (как указано в сноске 26) --- ''Обычно, рассматривая математическое правило --- например, сложение --- мы думаем о себе как о руководимых им в каждом новом случае'' и ''Я следую указаниям'' --- недвусмысленно указывают на наше такое относительно следуемых правил впечатление как вызванное чувством их усвоенности \textit{посредством научения другими}. То есть очевидной частью феноменологии является уверенность в наученности другими как источник уверенности в теперь согласно научениям делании --- поэтому так легко поставить эксперименты, подрывающие эту уверенность: ''Все остальные со мной не согласны, может это потому что я все же не усвоил то, что, как казалось, давно знал? Или может я вовсе психически болен?''.

Нахождение стандартов в сообществе (в его, точнее, диспозициях) может также объяснить результат мысленного эксперимента, герой которого осознает субличностный источник своих способностей к счету. Он внезапно начинает видеть ярко-красные цифры на каждом предмете, цифры эти связаны одна с другой по порядку, и последнюю его что-то \textit{побуждает} произнести ответом.\footnote{Что это за побуждение? Ну, подобное побуждению почесаться или поесть, например --- этому можно сопротивляться, но ограниченно.} Естественная реакция здесь --- страх, что эти напрашивающиеся ответы на самом деле \textit{не} верны --- из-за очевидной оторванности побуждения от нашего понимания своей способности считать: мы научились этому у других, и с какой стати будем теперь полагаться на это внезапное побуждение как нашей наученности соответствующее?

Эти соображения подсказывают прямое \textit{социологическое} решение парадокса: если и существует некоторый паттерн обосновывающих фактов, который указывает на использование именно счета в противовес зчету, то паттерн этот обнаруживается в коллективных установках сообщества. Крипке (1982, 111) затрагивает это решение и отмечает, что теория такая ''будет открыта по крайней мере некоторым из тех же критических замечаний, что и исходная (индивидуально-диспозиционная)''.

Почему же прямое социологическое решение \textit{точно так же} не удовлетворяет трем ограничениям Крипке, как и индивидуальное? \textit{Ограничение бесконечности}: диспозиция сообщества --- как бы она не определялась в терминах совокупности индивидуальных диспозиций и общественно-доступных инструментов --- все еще ограничена в своем диапазоне и подвержена отклонениям от правильного счета (и это верно даже для современного общества с его мощными вычислительными инструментами). И контрфактуальное дополнение диспозиций сообщества сталкивается с теми же препятствиями, что и в индивидуальном подходе. \textit{Ограничение бесконечности}: описание диспозиций сообщества в тот или иной момент времени не указывает на этого общества оправданность используемого им способа счета, не доказывает, то есть, что в очередной задаче используемая в силу коллективных диспозиций концепция счета --- та же, что и в предыдущих. Мы ведь \textit{не} думаем --- даже концептуализируя задачи счета как выполняемые коллективно, --- что наш способ счета по определению должен быть таким, как если бы ощественные практики были понимались такими же, как раньше. Наконец, \textit{ограничение ошибки}: коллективное решение не дает логического пространства для возможности ошибок в расчетах, причем не только отдельных субъектов, но и всего сообщества. К тому же, решение через коллективные диспозиции предвосхищает основание против альтернатив так же, как и решение через индивидуальные.

\section{Условия не истинности, но утверждаемости}

Давайте тогда обратимся к скептическому решению Витгенштейна Крипке. Оно (Крипке (1982, 66)) признает, что ''отрицательные утверждения скептика неопровержимы'', то есть признает отсутствие, подходящей для ответа на вопрос о счете/зчете, модели обосновывающих фактов. Согласно Крипке, решение Витгенштейна состоит в отказе от ''условий истинности'' в пользу предложений о понимании субъектом операторов нумерации, а также ''условий истинности'' в целом. Т.к. модели обосновывающих фактов нет, то нет ничего дающего необходимые и достаточные условия истинности предложений о понимании субъектом этих [нумерационных] предложений. Но вместо них можно использовать "условия утверждаемости": условия, при которых субъект имеет \textit{право} --- в сообществе --- утверждать что он (или кто-то другой) понимает счет, в их понятие, значит, заложена принадлежность субъекта признающему наличие таких условий обществу.\footnote{В этом разделе я пытаюсь изложить позицию Крипке (1982, 74-93) касательно условий утверждаемости. Дуглас Паттерсон (24.09.2009 --- электронная почта) указал, что, похоже, общая замена условий истинности условиями утверждаемости приводит к наличию этих условий для в ином случае бессмысленных предложений (''в обществе я имею шуметь, например''). Он далее предполагает здесь непоследовательность, пока в разрешениях нет незаконной апелляции к условиям истинности.

Но я, хотя в конечном счете и отрицаю необходимость заменить условия истинности условиями утверждаемости, думаю, это неверно. Конечно, люди имеют право использовать или не использовать предложения без необходимых и достаточных условий применения. Здесь есть два варианта: либо (1) форма предоставления таких прав является ''смыслом'', либо (2) в силу определенных условий, согласно которым "значения" должны соответствовать (например, они должны быть необходимыми и достаточными условиями применения), предложения эти не имеют "значений". Но мне кажется, что в оба варианта не приводят к бессвязности построенной на ''правах'' практике утверждения или этой практике нужде в дополнительных условиях --- условиях истинности.

Замена условий истинности условиями утверждаемости (как Крипке (1982, 86) трактует Витгенштейна) не исключает обычного использования слов ''истина'' и ''ложь''. Именно потому все еще можно использовать вместо условий утверждаемости --- условия истинности, если понимать их ''дефляционно'', а не как требующие соответствия фактам, что я и указываю в Главе 6. Иначе говоря, я предполагаю, что Крипке (а, возможно, и Витгенштейн) встроил в понятие ''условий истинности'' соответствие фактам (Крипке (1982, 72): ''Декларативное предложение получает свое значение в силу своих \textit{условий истинности} --- в силу соответствия фактам, без которых оно быть таковым не может''). Но я отрицаю необходимость такого понимания ''условий истинности''.}

Крипке предполагает, что условия утверждаемости --- условия права А утверждать, что он использует сложение --- заключаются, грубо говоря, в его уверенности в своей способности отвечать правильно на последующие вопросы, тогда как его ответы могут подлежать исправлению другими. И что он также имеет право, (Крипке 1982, 90) ''опять же, с учетом исправлений со стороны других'', оценивать чей-то ответ как правильный, если бы его он дал и сам. Отрицание понимание другим счета оправдано только если ответы его отличаются от ответов отрицающего, а отрицание собственного понимание оправдано при неуверенности в ответах или способах их нахождения.

Как выше указано, ''сообщество'' связано с условиями утверждаемости двояко. Во-первых, отсылка к нему встречается в формулировки самих этих условий: ''Мы призваны признать временную природу наших диспозиций --- признать законное право \textit{других} нас поправлять''. Во-вторых, условия утверждаемости имеют смысл только в контексте общества, ведь они полагаются на понятие права в этом обществе что-то утверждать или отрицать. Опасения по поводу правомерности своих утверждений могут испытывать от общества изолированные --- называемые мной \textit{Робинзонами Крузо}.

Согласно интерпретации Витгенштейна Крипке (1982, 86-88), непоследовательность изолированного такого Робинзона, следующего правилам именно в его невозможности функционировать вне условий утверждаемости. То есть если он берется считать, то действует ''не колеблясь, но \textit{вслепую}'' (Крипке (1982, 87)). Скептический парадокс указывает на, при ограничении такими Робинзонами, отсутствие ''каких-либо условий истинности или фактов, могущих доказать согласованность настоящих намерений с прошлыми'' (Крипке (1982, 89)). То есть для Робинзона нет разницы между уверенностью в следовании правилу и действительном ему следовании, и поэтому рушится идея правильного или неправильного следования ему. Как пишет Витгенштейн (1958, параграф 202):

\smallskip

\textit{Думать, будто подчиняешься правилу --- не значит ему подчиняться, и значит нельзя подчиняться ''в частном порядке''; иначе бы для подчинения достаточно было одной только в нем уверенности.}

\smallskip

Крипке (1982, 110), впрочем, подчеркивает, что из этого \textit{не следует} невозможность описать Робинзона как правильно или неправильно соблюдающего правила: для этого достаточно чтобы описывающий относился к Робинзону этому как к члену своего общества.

Заметьте: неспособность обосновать второе и особенно третье требования Крипке с использованием индивидуальных диспозиций прямо противоречит возможности такого описания. И Крипке, кстати, сам указывает на включение Робинзона в свое сообщество как на необходимый, для оценки его как считающего, шаг --- только так мы можем применить к нему условия утверждаемости.\footnote{Проводоится различие между одиночными языками --- языками с единственным носителем у каждого, и языками приватными (''языками ощущений'') --- о которых пишет Витгенштейн (Гольдфарб (1985)). Блэкберн, Гольдфарб и Богосян считают скептическое решение Крипке не исключающим существование одиночных языков, и здесь я с ними не согласен.}

\textit{Заключительные замечания} \quad Мастерское изложение Крипке создает у большинства уверенность в скептическом его подходе \textit{оправданности} логическими контурами разбираемой проблемы. Три ограничения возникают из простых наблюдений за нашей практикой следования правилам и ограничения эти совершенно явно исключают [прямые] диспозиционные решения (что и было показано в этой главе). Факты соответствия, в свою очередь, подрываются не метафизически --- напрямую подрывая способность наших утверждений что-либо правильно или неправильно описать --- но из-за недостатка у нас ресурсов на обеспечение такого этих утверждений соответствия. Ведь, говоря о счете, речь не об отсутствии функции следования одного за другим --- речь об отсутствии необходимых обосновывающих фактов, что подрывает наши способы мышления и разговора \textit{об} этой функции. И, получается, фактов соответствия тоже нет: ментальные элементы субъекта или его языка не могут соответствовать тому, чему мы дотеоретически считаем их соответствующими, потому что субъект, значит, неспособен подразумевать соответствие своих утверждений или мыслей этому. (Это очень важная тема, она еще будет возникать в следующих главах, особенно в Главе 4, где основное внимание будет направлено на льюизианское отрицание основанности фактов соответствия только лишь на ресурсах субъекта.)

Отсутствие соответствия (говоря в рамках естественного понимания свойств идиомы истинности) тогда подрывает анализ всяких связанных с утверждениями практик --- его придется заменить анализом условий утверждаемости. Наконец, естественные психологические факты о нашей способности устанавливать для себя стандарты (от нас самих требующие определять наше им соответствие) заставляют использовать стандарты общественные для убедительности следования правилам.

Диалектика здесь тоже очень широка по своему охвату. Она выражена в терминах арифметических функций, где наиболее естественно говорить о ''следовании правилу''. Но, как отмечает Крипке, она в равной степени убедительна как аргумент в пользу совпадения настоящего намерения и прошлого.

В последующих главах этой книги я покажу, как такие выводы отклонить.

\chapter{Две версии Робинзона Крузо}

\qquad

\textbf{Аннотация} \quad В этой главе я начинаю анализ проблемы следования правилам, используя диспозиционные языки --- в которых термины применяются именно так, как субъекты применять их склонны. Я покажу, что эмпирические обстоятельства (и склонности испытуемых), будучи достаточно благоприятны, позволяют изолированным испытуемым следовать правилам. То есть испытуемые эти могут успешно оценивать и использовать для успешной навигации в мире такие языки. Диспозиции их, впрочем, в некотором роде искусственны. Но примеры, описанные в последующих главах, будут гораздо более естественными --- будут гораздо больше походить на нас.

\qquad

\section{Сообщество идиолектов}

Один из выводов, делаемых в этой книге --- возможность приватных моделей следования правилам. Я вывожу ее поэтапно, используя несколько разных Робинзонов, каждый из которых может участвовать в разных видах такого правилам следования. Прежде чем начать эту серию мысленных экспериментов, я в этом разделе кратко представлю воображаемое сообщество, которое --- в отличие от \textit{нашего} --- понимает значения своих слов в терминах своих к их использованию склонностей. Я рассмотрю некоторые детали, преводящие к подобной языковой практике провалу и покажу способы ей все же --- как не удивительно --- существовать.

\textit{Диспозиционно-смысловые языки} \quad Представьте сообщество, в котором каждый действительно намеревается использовать всякое слово --- ''таблица'', ''счет'', ''сложить'', ''сова'' и т.д. --- как значащее ровно то, к чему он склонен в данный момент его применить. Например, под словом ''собака'' он имеет в виду то, что у него возникает желание назвать собакой, под ''столом'' --- то, что склонен называть столом и т.д.. Таким образом, один и тот же предмет в одних обстоятельствах может быть собакой, а в других --- кошкой, а большинстве случаев он вообще ничем не является. (Экстенсии терминов в этих языках --- не наборы объектов, а наборы пар объект-конекст, где контексты индивидуализированы по времени, положению в пространстве и другим стандартным факторам.)

Может показаться, что такой язык вовлекается в фатальную цикличность, ведь, например, значение слова ''собака'' задается формулировкой содержащей само это слово ''собака''. Да, замкнутость возникла \textit{бы}, понимай люди в таком обществе формулировки эти как \textit{инструкции} к использованию слов, но ведь они их так \textit{не понимают}. И диспозиции --- даже в ходе обучения приобретенные --- автоматически \textit{диктуют} им ответы, и потому формулировка значения слова ''собака'' независимо характеризует его значение как ''применимое к этим предметам тогда, когда есть побуждение его так применить''. Назовем такие языки ''диспозиционно-смысловыми'', а слова в них --- имеющими ''диспозиционный смысл''.

Требования Крипке для этих языков либо теряют силу, либо тривиально выполняются. Требование обоснования тривиально выполняется: используемые в данный момент термины языка чувствительны только к побуждениям в данный момент говорящих. (Что говорящие имеют в виду тогда-то и там-то, --- это что они чувствуют необходимость иметь в виду тогда-то и там-то.). У них нет намерений (хотя они и могут делать прогнозы) касательно того, что они будут иметь в виду в будущем, и они не понимают себя как обусловленных тем, что имели в виду в прошлом. Итак, на вопрос скептика ''Откуда вы знаете, что то, что имели в виду раньше, соответствует тому, что имеете в виду сейчас?'' они отвечают

\smallskip

\textit{Я помню, что под словом ''кошка'' подразумевалось именно то, что мне тогда хотелось подразумевать, и я помню, к чему это слово применял, так что если вы спросили, имею ли я в виду в данных момент, что чувствую необходимость иметь в виду в данный момент так же, как я делал это и раньше, то мой ответ, конечно, да, ведь я прекрасно помню, что тогда происходило. Если же вопрос ваш подразумевал не это, то я не знаю, что еще он мог подразумевать.}

\smallskip

То есть скептический вызов не может даже быть сформулирован, потому что обращение к чьим-то прошлым предрасположенностям \textit{действительно оправдывает}, что он имел в виду тогда, и что сейчас он действует согласно сейчас им подразумеваемому --- странная речь из Части 2.3 вовсе не странная в \textit{таком} обществе. Другой способ указать на провал смыслового скептика перед носителями этих языков: хотя и есть обобщение того, что они имеют в виду --- \textit{они имеют в виду то, что иметь в виду побуждаемы} --- обобщение это не часть намерения говорящего, когда он что-то имеет в виду, то есть они не \textit{намерены} всегда иметь в виду то, что им хочется иметь в виду, они просто всегда так делают.

Требование ошибки также теряет силу, ведь что применения субъектом слова всегда привальное, независимо от его характера или поведения --- нельзя поступать ''плохо''. Наконец, никто в таком обществе ни в коем случае не намерен ''складывать'' или ''умножать'', или еще какие-нибудь слова применять подобно \textit{нам}: каждый намеревается ''сложить'' или ''умножить'' только в смысле соответствия своих диспозиций этим действиям в этот момент, так что и понятие ''счет'' применимо только если диспозиции его применить позволяют.

Возможным недостатком такого диспозиционно-смыслового ''языка'' является отсутствие общих слов у не имеющих \textit{совершенно} одинаковые в совершенно одинаковых контекстах, диспозиции: если у них есть хоть какие-нибудь отклонения в диспозициях относительно многих слов, то результатом будет скорее семья идиолектов, чем что-то вроде ''общественного языка''. Более того, они признают это и признают невозможность каких-либо переводов с одного идиолекта на другой (по крайней мере подавляющего большинства слов). Такая какофония идиолектов может показаться безнадежной для общения!

На самом деле такая модель не будет бесполезной для большей части общения, по крайней мере если идиолекты отличаются незначительно --- сбои тогда будут возникать лишь изредка. Вообще успешность общения между такими субъектами зависит от величины отклонений среди их идиолектов. И даже если почти все их слова содержат существенные межличностные отклонения, успешное общение все еще возможно --- если есть переводы.

Вот пример: пусть В --- член такого общества --- имеет в своем словарном запасе слово ''диспозиция'', и пусть слово это \textit{точно} применимо к диспозициям С и к его собственным. Тогда В может осознать совершенную определенность своего понимания слов --- например ''стол'', ''апельсин'', ''бежать'', ''умножать'' --- его диспозициями к таких слов применению. Аналогичное он может осознать касательно С. Затем В осознает, что не может напрямую перевести идиолект С на свой: ''стол'', например, как его С использует, не имеет тех же условий применения, что ''стол'' В --- их диспозиции ведь различаются. И все же В может перевести ''стол'' С так: ''стол'', как его использует С, означает на идиолекте В ''''стол'', как его склонен использовать С''. То есть если слово ''диспозиция'' применимо к диспозициям других людей --- если, значит, они разделяют это слово и условия применения его соответствуют диспозициям каждого из них --- то взаимный перевод слов в таком сообществе возможен.

Конечно, можно представить подобное сообщество с отклонениями касательно \textit{каждого} из слов, то есть и слова ''диспозиция'' тоже. И пусть даже способ использования ''диспозиция'' В таков, что ''''стол'' как С склонен использовать это слово'' \textit{не} имеет тех же условий использования, что ''стол'' С (потому что ''диспозиция'' В не соответствует ''диспозиция'' С). Осознав это, В увидит, что он и С говорят на разных идиолектах, которые к тому же не поддаются взаимному переводу. Смогут ли они, несмотря на невозможность перевода, общаться --- зависит от степени различия условий применения используемых ими слов.

Итак, невозможность перевода идиолектов и, значит, несостоятельность таких языков --- во всяком случае, для общения --- не следует так сразу из определямости значений слов через диспозиции их использующих.\footnote{Я благодарю Дугласа Паттерсона (электронная почта 24.09.2009) за предупреждения меня об этих проблемах при обсуждении с ним изложенной модели.}

Тем не менее, есть поводы для беспокойства касательно ценности таких языков для выживания даже в эмпирически удачных обстоятельствах. Пусть, например, диспозиции одинаковы у всех людей в нем, могут ли тогда они \textit{последовательно} применять к объектам эти свои слова? И почему их диспозиции не могут заставлять их называть одно и то же то столом, то --- в следующий же момент --- как-то иначе? Ведь \textit{тогда} даже будучи разделяемы всеми в обществе, диспозиции эти могут привести к катастрофическим результатам. Способность таких идиолектов обеспечивать общение, \textit{помогающее пережить} их носителям взаимодействие с вещами в мире --- способность эта зависит исключительно от эмпирических фактов о диспозициях каждого из них и в том числе от этих диспозиций (а значит, и применений слов) стабильности.

Пусть единообразные такие склонности в обществе крайне ''осторожны'': распространена, например, склонность не применять слово ''стол'' до тех пор, пока целевой объект не будет внимательно изучен. Такие диспозиции могут до некоторой степени стабилизировать применение слов в разных контекстах, в том числе --- слов, касающихся вычислений --- ''сложение'', ''вычисление'' и т.д.. Пусть члены общества склонны применять слово-число как результат вычислений, но только когда вычисления эти были дважды проверены определенными способами (значит, например, торопливо вычислять они не склонны --- они даже не назвали бы такое вычислением). Если при этом каждый имеет диспозицию давать одинаковые ответы после таких вычислений завершения --- пусть даже \textit{одинаково} ошибочные --- то идиолекты остаются согласованы, применение слов происходит стабильно.

Что ж, предварительный вывод таков: \textit{практическая реализуемость} диспозиционно-смысловых языков зависит исключительно от эмпирических условий. Например, от того, какими именно диспозициями обладают субъекты в таком обществе, и --- что особенно важно --- как диспозиции эти приводят к единообразиям среди них. Конечно, я еще не рассматривал, требуется ли такой языковой практике определенное единообразие вещей в самом мире, которое могло бы как-то соответствовать единообразию в носителях языка диспозициях. И еще, я не ответил на ожидаемый от некоторых философов вызов касательно в действительности не соответствия \textit{концепций} носителей таких языков и предметов реального мира. Ведь у них, казалось бы, есть лишь склонность более или менее скоординированно испускать случайные \textit{шумы}. Первый из этих вопросов будет рассмотрен при рассмотрении других видов Робинзонов, второй же будет рассмотрен уже в следующем разделе.

\section{Робинзоны в эмпирически благоприятных обстоятельствах}

Представьте, что наши Робинзоны питаются в основном кокосами. Добыча кокосов, впрочем, трудна: приходится избегать различных крупных хищников. Чтобы выжить, Робинзоны научились считать кокосы: сколько их есть сейчас, сколько будет съедено за день и, значит, сколько дней до следующего рискованного поиска.\footnote{Я предполагаю, что Крузо-1 может изобретать слова, причем даже для сложных понятий --- например, для количеств предметов --- и делает это на основе опыта и врожденных диспозиций. Сомнительно, конечно, что, даже будучи щедро одарены природой полезными диспозициями, так изолированные люди на это способны. Однако сейчас не оспариваются эмпирические предположения по этому поводу. Если способ Крипке выводить возражения против приватных языков на основе парадокса верен --- особенно важны здесь второе и третье требования --- то не важно, насколько изощренными диспозициями наши Робинзоны одарены: парадокс всяко остается.} Он также изучил территорию, где добывает пищу, определил, где больше всего встречается кокосов, и, значит, куда ему за ними лучше лезть. В общем, навыки счета оказались жизненно ему необходимы.

В описание этого сценария включен один неоспоримый факт: наборы предметов --- кокосов в данном случае --- имеют определенный числовой размер. Так что, вообще то, не удивительно, что при их подсчете --- и при подсчете, значит, дней до следующей охоты --- важно \textit{правильное} понимание Робинзоном-1 чисел, ведь ему нужно сосчитать сколько ''действительно'' кокосов на дереве.\footnote{То есть нужно соответствие один-к-одному между применяемыми числами и кокосами, к которыми он их применяет, и соответствие это должно сохранять отношение порядка, чтобы количества можно было сравнивать.} И применение Крузо-1 прочих его слов: ''кокос'', ''тигр'' и т.д. должно оцениваться на адекватность так же. Кокосы --- еда для него, он --- еда для тигров. Его слово ''кокос'', значит, должно выделять определенный набор предметов, являющихся для него пищей, ''тигр'' --- набор предметов, которых лучше избегать.

Такое понимание ''правильности'' отлично от ранее описанного в книге, особенно в Главе 2 --- там оно применялось к имеемому в виду отдельными субъектами и этого соответствию имеемому в виду ранее. Важно, что для Крузо-1 ''правильность'' значит соответствие его ответа тому, например, что находится на дереве, чтобы действия из результатов его счета следующие позволили ему собрать нужное количество кокосов.

Думаю, следует признать, что Крузо-1 действительно \textit{понимает} придуманные им слова: ''кокос'', ''тигр'', ''1'', ''2'' и т.д.. Это потому что он \textit{придал} им диспозиционные значения: они значат то, к чему его склоняет (в любое время и в любом месте) применять их. Но, учитывая указанное только что понимание ''правильности'', можно все же вопрошать: почему придание словам диспозиционных значений не приводит к полной катастрофе? Пусть наш Крузо-1 находится в ситуации, где (1) его диспозиции к слов применениям стабильны и (2) вещи кажутся ему такими, какие они в самом деле есть. Например, если он думает, что видит кокос, то он \textit{действительно} видит кокос. Некоторые могут счесть такое изложение тенденциозным. В конце концов, что такое \textit{для Крузо-1} кокос и что тогда значит стабильность его диспозиций?\footnote{Я описывал Крузо-1 и его мир с точки зрения нашего языка, так что фразы вроде ''оценить условия Крузо-1 на адекватность'' подразумевают оценки в наших терминах. Требуемое для этого будет рассмотрено дальше, особенно в Разделе 3.3 и в Главах 4 и 7, а пока давайте смотреть на это как на предварительных способ говорить об изолированном таком Крузо и о проблемах, с ним связанных, хоть и способ этот в дальнейшем будет пересмотрен.}

Но позвольте поставить вопрос по-другому. \textit{Мы} живем в мире, где тени иногда выглядят как кокосы, а кокосы иногда похожи на свернувшихся котов. Иногда даже требуется тщательная проверка, действительно ли выглядящие одинаково вещи одинаковы. То есть почти все имеет свойства, которые в сочетании с ограничениями наших чувств приводят к множеству ошибок касательно действительной вещей природы. Из-за чувств Крузо-1, из-за небольшого количества существующих в его мире видов вещей и ограничений этих вещей возможных различий --- из-за этого все в его мире так аккуратно подпадает под тот или иной \textit{узнаваемый им вид}. Он может даже распознавать и различать объекты с первого же взгляда, он \textit{не может} ощутить, что одно и то же в разное время кажется ему разным. Он не сможет, например, понять, что то, что выглядело как тигры, кокосы или белки, теперь выглядит совсем иначе. И он, наконец, не способен допускать вычислительных ошибок: если в результате счета он \textit{уверен}, что на дереве 6 кокосов, то это потому что на нем \textit{действительно} 6 кокосов.

В таком замечательном мире с такими завидными эпистемическими способностями Крузо-1 может наделять свои слова смыслом диспозиций и быть вполне успешным. Он может, например, понять, что ''7'' значит количество предметов, о котором он склонен думать, что их 7. Он может иметь в виду ''кокос'' применительно именно к тому, к чему он склонен применять это слово. Поскольку его идиолект --- диспозиционно-смысловой, три требования Крипке либо теряют силу, либо тривиально выполнимы. И, кстати, раз требование бесконечности теряет силу, то некоторые не будут думать о Крузо-1 как о \textit{считающем} в том же смысле, что и мы. Хотя я и допускаю, что он никогда не совершает ошибок, я не думаю, что он может считать сколь угодно большие наборы предметов. Т.к. есть верхний предел размера доступного для подсчета набора, есть только конечное число числовых слов.\footnote{Поскольку его числа конечны, я не буду говорить о нем как о владеющем нашей их концепцией. Впрочем, не так уж очевидно, что это стоит отрицать --- числовые понятия то у него имеются. Я сейчас обсужу другой аспект этой проблемы.}


Крузо-1 все же имеет вразумительную языковую практику --- вразумительный идиолект --- в которой слова значат ровно то, что он склонен ими называть, и благодаря очень хорошим обстоятельствам этот идиолект полезен для него --- даже незаменим. И это так даже несмотря на невозможность неправильного применения им слов.

Кто-то, возможно, укажет на \textit{невозможность} когерентной языковой практики без возможности так ошибиться. (И то же скажет, значит, касательно общества идиолектических носителей, рассмотренном в Разделе 3.1.) Он будет также отрицать соответствие слов Крузо-1 понятиям, обладающим которыми можно было бы его считать: понятием количества, кокоса и т.д.. Крузо-1, каким я его здесь представил --- утверждал бы этот оппонент --- не имеет [реальных] \textit{концепций}, ведь его их применение не предполагает ими его \textit{руководства}. Он, вместо этого, признает, что имеет лишь потребность вербально \textit{реагировать} на вещи в мире. (Грубо говоря, у него есть только полезный набор речевых \textit{тиков}, но не язык.)

Однако три соображения подтверждают лишь то, что его концепции не совсем те же, что у нас, а не что у него вовсе их нет. Во-первых, есть сопровождающее использование им своих слов ментальное содержание. Хотя его понимание своих же слов подразумевает диспозиционное их значение, это не мешает им соответствовать наборам вещей (в конкретных обстоятельствах), наборам похожим и наборам имеющим определенные количественные свойства. Действительно, такое психичесвкое содержание иногда составляет его диспозиции (например, он склонен относиться к таким-то предметам как к одинаковым вне зависимости от своего их \textit{восприятия}). Во-вторых, и \textit{наши} некоторые концепции, подобно таковым его, имеют схоже диспозиционное значение. Рассмотрим, например, понятие \textit{боли}. Это, как известно, концепция, касательно применения которой ошибаться невозможно (по крайней мере, от первого лица).\footnote{Широко распространено мнение, что Витгенштейн всячески бросал вызов таким концепциям. Но указываемые им проблемы, похоже, исходят непосредственно из соображений соблюдения правил, и потому сомнительно настаивать на них здесь. (Здесь я, кажется, согласен с Крипке (1982, 3), когда тот пишет: ''''аргумент приватного языка'' применительно к ощущениям --- лишь частный случай гораздо более общих соображений о языке, ранее обсуждавшихся ...''. См. также его сноску 47, 60-61 и текст, которому она соответствует.) Я предлагаю в дальнейшем использовать другие способы --- уже предлагавшиеся другими философами --- для подрыва этой очевидности касательно использования понятия боли от первого лица, их я обсуждаю в Разделе 5, описывая также, как они действуют в диспозиционно-смысловых языках.} Последнее, и, возможно, самое важное: ничто не мешает Крузо-1 использовать свои концепции так же, \textit{как мы}: описывать вещи в мире, рассуждать о них и принимать на основе этих рассуждений решения о действиях. Как и мы, он может решить, например, залезть на одно дерево, а не на другое (потому что, скажем, на другом меньше кокосов), или он может бросить вызов группе опасных хищников поскольку их в ней только два. Поразительные эти факты о его способностях соответствуют аналогичным и о наших: те наши концепции, к которым ''правильность'' или ''неправильность'' не применима --- ''боль'', например --- кажутся нам все же концепциями, которые можно использовать с различаемыми по правильности для описания мира, рассуждения и принятия решений.

И последнее наблюдение: было бы ошибкой думать, будто Крузо-1 в таких обстоятельствах считает себя всеведущим. У него, конечно, нет понятия ошибки: он не понимает, как можно думать о не-кокосе --- что это кокос. И все же способен понять невежественность: это когда он еще не сосчитал кокосы на дереве --- и не знает, сколько их, или когда не знает, есть ли тигр за следующим поворотом.

Здесь есть тонкость: следует различать два случая. Первый: Крузо-1 сталкивается с набором предметов --- с кучей песка, например --- которые у нет желания считать. Он тогда не склонен применить какое-либо из счетных слов, но и не расположен отрицать их применения возможность: для его идиолекта нет ответа на вопрос, сколько здесь песчинок. Он буквально не знает ответа --- ответа нет. Но есть и другой случай: когда он чувствует, что захоти он --- мог применить числовое слово --- и склонен отрицать, например, что на дереве кокосов 1, 2 или 3. Он знает, что их больше трех, не без подсчета --- без задействования, значит, своей диспозиции для указания их количества --- он не знает, сколько их.\footnote{Сложности возникают, если попытаться это различие заострить: потребуется различение диспозиций Крузо-1 применять слово от других диспозиций, с которыми у первых возможен конфликт. Впрочем, для моей аргументации здесь это и не важно --- не важно, обоснованно различие такое или нет --- так что воздержусь от дальнейшего обсуждения.}

В удачных обстоятельствах наш первый Робинзон удачлив именно благодаря простоте мира, а также идеальному соответствию имеющемуся в нем своих чувств и установок: он ведь не нуждается даже в понятии ''правильного применения'' слов. В этом, конечно, отличие его идиолекта он нашего с вами языка, но это, тем не менее, язык --- и язык несомненно полезный. Таким образом, описанный мысленный эксперимент уже сузил масштаб витгентшейновского аргумента Крипке против приватного следования правилам. Оставляя в стороне, значит, небольшое злоупотребление термином приватности касательно этого случая, получаем неожиданный результат: при определенных эмпирических обстоятельствах все же существуют полезные приватные языки: диспозиции и мир следователя приватным правилам таковы, что нет разницы между его таким следованием и лишь убежденностью о нем.

\section{Робинзон с двумя наборами сознательных диспозиций}

Давайте теперь представим второго, немного более реалистичного Робинзона --- ''Крузо-2'', в том же самом мире --- мире совершенно различимых объектов, делящихся на узнаваемые виды. Здесь, значит, как и прежде, есть продукты питания (кокосы, креветки и т.д.) и хищники (тигры, кабаны и др.), а органы чувств Крузо-2 --- при оптимальной работе --- прекрасно отличают что угодно от чего угодно.

''Оптимальная работа'' в абзаце выше призвана указать на определенную нестабильность: иногда наш Робинзон видит вещи так, как они есть, а иногда --- иначе. Например, когда свет падает иначе, или он устал, или поспешил с принятием решеня, или слишком много выпил --- тогда ему может хотеться назвать \textit{юкосом} то, что обычно хочется --- кокосом, или тенью то, что обычно хочется назвать тигром.

Первый, грубый, способ показать разницу между первым и вторым Крузо --- указать на полезность второму учиться распознавать, когда он может доверять своим диспозициям. Крипке (от имени Витгенштейна), впрочем, отрицает такую для него возможность. Он (1982, 112, сноска 88) пишет:

\smallskip

\textit{в отсутствие скептического парадокса могло бы показаться, что человек помнит свои ''намерения'' и может одно такое воспоминание о них использовать для исправления воспоминания другого --- парадокс указывает наивность таких идей, на их бессмысленность. В конечном счете, ведь, у субъекта могут быть противоречивые грубые наклонности, и исход будет зависеть только от его воли.}

\smallskip

И далее (1982, 112, сноска 88):

\smallskip

\textit{Ситуация тут не такая же, как если бы отдельные субъекты имели разные и независимые воли и, при принятии в сообщество их нового субъекта, считали, что могут положиться на его реакции --- ничего такого нет между субъект и им же самим.}

\smallskip

Здесь я не согласен. Рассмотрим два возможных случая. Первый --- не столь реалистичный, в нем Крузо-2 может различать два внутренних своих диспозиционных состояния: первое вызывает один набор диспозиций, другое --- другой (назовем для простоты первое ''отдохнувшим'', а второе --- ''усталым''). Во втором случае, уже более реалистичном, побуждения Крузо-2 меняются в зависимости от времени и места, но он все же часто может классифицировать их и распознать, в каком диспозиционном состоянии находится сейчас, произошел ли свдвиг в его диспозициях. Второй случай я рассмотрю в Разделе 5.2.

В первом же случае у Крузо-2 есть выбор, какой из \textit{двух} возможных диспозиционно-смысловых языков\footnote{Альтернативный подход --- индивидуализация диспозиционных языков так, чтобы Крузо-2 всегда лишь на одном таком говорил. От этого, впрочем, ничего здесь не зависит: достаточно бы было лишь по-другому расставить пункты в дальнейшем рассуждении.} воспринимать всерьез. То есть, хотя он \textit{говорит} всегда на том языке, на котором говорить склонен --- несмотря на это, он все же способен решить, что следует определенному из двух таких языков доверять. Он признает, что различные диспозиционные состояния заставляют его придавать словам разные значения, и что один набор диспозиций лучше, чем другой --- один ''достоин доверия'', другой же --- нет. Пусть, например, когда он отдохнул, его диспозиции соответствуют миру так же, как диспозиции Крузо-1. Но поведение его меняется от усталости. Крузо-2, однако, замечает такое довольно легко: когда он устал, его слово ''кокос'' имеет специфичное для усталого состояния значение. Такие ''уставшие'' кокосы иногда уже не подходят ему и могут даже навредить, и в отдохнувшем состоянии он видит, что некоторые из них --- \textit{юкосы}.

Способ Крузо-2 осознавать полезность доверия ''отдохнувшему'' своему состоянию и недоверия, значит, ''уставшему'' --- его способность обнаруживать явные и неявные \textit{грубые закономерности}.\footnote{Грубые закономерности я рассматривал еще в Аззуни (2000) и в Аззуни (2010b).} Например, ''кокосы сытные, а от юкосов только тошнит'' --- две очевидные грубые закономерности. Если ему в отдохнувшем состоянии доступны слова ''есть'', ''кокосы'', ''юкосы'', ''питает'' и ''болеть'', то он может сформулировать на нем такие грубые закономерности. Есть эти слова у него и в уставшем состоянии --- некоторые юкосы он и в нем способен от кокосов отличить. Но в уставшем состоянии есть другая грубая закономерность: ''некоторые кокосы сытные, а от некоторых только тошнит''. (Она сохраняется, если Крузо-2 также сохраняется, если он видит --- наоборот --- некоторые кокосы как юкосы.) С другой стороны, усталость и отдых не влияют на осознание им своего состояния --- болен он или здоров --- а от юкосов он как раз заболевает.

Даже если у него слов ''сытый'' и ''больной'', он все равно способен распознать им соответствующее, просто нужные грубые закономерности теперь будут \textit{неявными}. У него, значит, если \textit{концепции} сытости и болезни даже вне нужных для них слов, и используя эти понятия, он также уловит соответствующие грубые закономерности --- которые пусть и не сможет выразить себе же самому.

Заметьте, что он так признает, что соответствующий отдохнувшему состоянию диспозиционный язык более надежен, чем соотвутствующий уставшему: ведь схыватываемые им грубые закономерности точнее --- успешней предсказывают результаты его действий. Такие закономерности --- лучшие в отдохнувшем состоянии, и худшие в уставшем --- можно сформулировать с использованием его концепции чисел. Крузо-2 может осознать, что перед собиранием кокосов --- когда у него они закончились --- ему надо выспаться, чтобы на следующий день точнее находить именно их. (Короче говоря, это тоже знание --- неявное --- признаваемое даже будучи явно в словах выразимым.) Он не хочет ведь насобирать юкосов и потом выбрасывать их или --- еще хуже --- съесть и заболеть.

\section{Что для Крузо-2 "лучшее соответствие миру"?}

В прошлом разделе я предлагал представить, что диспозиции Крузо-2, когда он отдохнул, соответствуют миру так же, как таковые Крузо-1.

Разговоры о миросоответствии --- метафоричны, хотя, конечно, некоторые философы предпочитают относиться к ним серьезно. Можно ли сказать, что к признанию ''отдохнувшего'' языка более полезным Крузо-2 приходит через лучшее его терминов соответствие миру?

Это именно практический вопрос о способе Крузо-2 распознавать, какой из языков лучше. Примерно, конечно, понятно, как ему это удается: один набор диспозиций доставляет ему неприятности, другой же --- нет. Может даже возникнуть соблазн сформулировать так: он \textit{думал} о чем-то как о кокосе, но когда его съел и заболел, то понял --- это был юкос! Это, конечно, неправильно. Давайте использовать обычные кавычки для указания на принадлежность слова к нашему языку, r-кавычки --- для указания на принадлежность к ''отдохнувшему'' языку и t-кавычки --- на принадлежность к ''уставшему'' языку. Нельзя сказать, будто он думал он предмете как о ''кокосе'', но затем обнаружил, что был неправ. Он, однако \textit{не} обнаружил, что предмет является \textsuperscript{t}кокосом\textsuperscript{t}, потому что это был \textsuperscript{t}кокос\textsuperscript{t}. Мы, однако, \textit{можем} сказать, что он обнаружил, что это был не \textsuperscript{r}кокос\textsuperscript{r}, но не в том смысле, что он \textit{думал}, будто это \textsuperscript{r}кокос\textsuperscript{r}. Он ведь не пользовался ''отдохнувшим'' языком, произнося соответствующие всем этим ''кокосовым'' словам звуки, он произносил \textsuperscript{t}кокос\textsuperscript{t} --- корректно его, значит, применяя. (И это верно даже когда он заболел --- главное, чтобы не был уставшим.)

Однако, проблема его может выражена на любом доступном ему языке. Судя по тому, как он говорит на своем ''усталом'' языке, проблема эта в том, что от некоторых \textsuperscript{t}кокосов\textsuperscript{t} он заболевает, а от некоторых --- нет. Но это ведь не значит, что они не \textsuperscript{t}кокосы\textsuperscript{t}. И, с другой стороны, он заболевает от любого \textsuperscript{r}юкоса\textsuperscript{r}.

Здесь \textit{нам}, возможно, хочется сказать: ''отдохнувший'' язык \textit{на самом деле} описывает кокосы и юкосы такими, \textit{какие они есть}. Ведь слово \textsuperscript{r}кокос\textsuperscript{r} для Крузо-2 соответствует своей экстенсией \textit{нашему} ''кокос'', а наше слово всегда указывает верно. Но что не так с экстенсией \textsuperscript{t}кокос\textsuperscript{t}? \textit{Мы} думаем, что некоторые \textsuperscript{t}кокосы\textsuperscript{t} --- кокосы, но некоторые \textsuperscript{t}кокосы\textsuperscript{t} --- юкосы.\footnote{Возможно, пример следует немного разъяснить. Некоторые кокосовые пальмы заражены паразитической лозой, растущей внутри и на них и имеющей свои орехи. Крузо-2 в отдохнувшем состоянии может отличать эти орехи от принадлежащих пальме. Мы, конечно, понимаем, что юкосовое дерево --- это и не дерево даже, но Крузо-2 \textit{этого} не знает.} Но что с того? Слова в каждом диспозиционно-смысловом языке Крузо-2 имеют разные области применения --- они варьируются в относительно разных наборов, и в \textit{этом} нет ничего неправильного.

\textit{Нам} может хотеться сказать в ответ: множество наборов предметов, категоризированных на ''отдохнувшем'' языке Крузо-2, лучше согласуется с реальными между ними сходствами, а множество наборов, классифицированных на ''уставшем'' --- хуже. И добавить: ''отдохнувший'' язык \textit{сопоставляет} свои слова с действительно существующими естественными видами, а ''уставший'' --- нет. То есть что множества кокосов и юкосов никак не вложены друг в друга.

Здесь, однако, две проблемы. Первая: что дает нам право такое заявлять? Ее мы рассмотрим далее --- особенно в Главе 4. Вторая: Крузо-2 в любом случае не имеет доступа к таким соображениям. У него есть \textit{две} фразы: \textsuperscript{t}настоящее\textsuperscript{t} сходство и \textsuperscript{r}настоящее\textsuperscript{r} сходство. То есть каждый из его языков по своему реальные сходства характеризует. У Крузо-2 есть и две фразы, соответствующие нашей ''естественные виды''. С каких же слов помощью должен Крузо-2 объяснять идею ''более точного'' отражения мира словами на одном языке, чем словами на другом.

Обратите внимание: это не значит (по крайней мере, \textit{пока что}) нашу невозможность понять идею превосходства одного языка над другим в смысле соответствия миру. Мы ведь это и делаем, сравнивая экстенсии его слов с \textit{нашими}. Вопрос вот в чем: как \textit{Крузо-2} должен наделить для себя это различие содержанием, выходящим за рамки превосходства одного языка над другим в смысле ориентации в мире через информацию от грубых закономерностей?

Давайте пока отложим вопрос о нашем праве говорить о соответствии слов Крузо-2 миру через воображение значений за пределами обоих диспозиционно-смысловых языков --- значений, свойственных языку Бога --- языку, каждое слово которого, благодаря Его могуществу, относится к набору, содержащему только совершенно естественных видов представителей, и фраза ''естественный вид'' в котором относится только к множеству таких наборов. (Не спрашивайте, как Бог заставляет этот язык так делать --- он же Бог, и среди его сверхъестественных способностей есть и такая.) Используя свой божественный язык, он может описать ситуацию Крузо-2 так: ''Есть действительно похожие вещи, которые, значит, действительно принадлежат к одним и тем же естественным видам''. И действительные сходства вещей \textit{влияют} на Крузо-2. Один из языков Крузо-2 --- ''отдохнувший'' --- схватывает эти действительные сходства, а другой --- нет, поэтому, выбирая ''отдохнувший'' язык, Крузо-2 может ориентироваться в мире успешней. Ведь в первом случае он классифицирует вещи согласно действительному их сходству, а во втором ему это не удается.

Такой ''божественный'' способ смотреть на вещи, как уже отмечалось, недоступен \textit{Крузо-2}. И все же, его выбор ''отдохнувшего'' языка вместо ''уставшего'' рационален, потому что подразумевает распознавание лучшего в отношении способности поддерживать свое существование в мире. Крузо-2 видит, что добьется большего успеха с одним языком, и меньшего --- с другим. Он, впрочем, не может явно уловить эту разницу, как это может Бог, ведь что бы он имел в виду, говоря о ''соответствии''?

Обратите внимание: проблема Крузо-2 в том, что наличие способности понимать превосходство одного языка над другим касательно достигаемого успеха в мире --- не необходимо и достаточно для лучшего ''соответствия'' языка миру. У него есть еще, конечно, способность воображать один из языков всячески превосходней всех прочих в отношении успешности его носителя в мире, но и это не необходимо и достаточно для ''соответствующего'' миру языка. Думаю, Крузо-2 неспособен понять, что значит ''соответствие'' языка миру. (Я также предполагаю, что эта проблема --- проблема понимания Крузо-2 соответствия совершенного языка Бога миру --- глубже, чем кажется. Возможно, и \textit{мы} это не понимаем --- вопрос этот я рассмотрю далее в Разделах 7.2 и 7.3.)

Почему же Крузо-2 не может понять ''соответствие'' миру? Его языки --- диспозиционно-смысловые, и понятия, значит --- тоже. Поэтому понимание им всякой концепции следует описывать в смысле его склонности к концепции этой применению --- это и есть то, что упускается в предположительном его взгляде на превосходящесть одним языком другого касательно их соответствия миру. Он ведь вовсе не так их сравнивает, он сравнивает с точки зрения успеха в мире --- это не то же самое, что ''соответствие'' ему.

Он может лишь официально принять один из языков и отрицать, что диспозиции ''уставшего'' языка не определяют значения его слов. Мы можем подумать, что он будет делать это на ''отдохнувшем'' языке: ''Я устал, поэтому \textit{думал}, что несу кокос, но нет, я \textit{ошибался} --- это был юкос''.

Но это не верно, потому что оба его языка --- диспозиционно-смысловые. Он не использует те термины, которые не склонен использовать, поэтому, когда устал --- и говорит на ''уставшем'' языке --- его отрицание определения ''уставшими'' диспозициями, слов на ''уставшем'' языке --- \textit{бессмысленно}. Поэтому он не скажет ''Я устал, и потому подумал, что несу кокос'', имея в виду ''отдохнувший'' кокос. Ведь когда он устал, он вообще не использует этот термин, так что как же он может ''думать'', что несет \textsuperscript{r}кокос\textsuperscript{r}, когда устал?

И это так, даже если признать его знание, к чему относятся термины на другом языке. Он может \textit{сказать} ''Если бы я отдохнул, я бы подумал, что это \textsuperscript{r}юкос\textsuperscript{r}'' и быть правым, но правота эта не делает правильной формулировку ''Я думал, что у меня есть \textsuperscript{r}кокос\textsuperscript{r}''.

Поскольку Крузо-2 знает, что для него существует два диспозиционно-смысловых языка, с одним из которых он достигает большего успеха, чем с другим, он может описать свою ситуацию металингвистически, сказав: ''Я использовал менее полезное \textit{слово} \textsuperscript{t}кокос\textsuperscript{t}'', потому что устал. Он при этом все еще считает себя совершившим ошибку, но ошибка эта --- в использованном слове, и не касается \textit{мира}. Или --- что то же самое --- он может сказать ''У меня были неправильные диспозиции, когда я произносил это слово, потому что я устал'': здесь он думает о себе --- что он устал и потому употребил неправильные диспозиции. В первом же случае он указывал на два разных слова --- соответствующие разным диспозициям --- неправильное из которых он употребил. Но для него разница между этими случаями чисто терминологическая.

Еще раз: Крузо-2 может предпочесть один язык другому только потому, что тот позволяет ему лучше ориентироваться в мире, а не потому, что он --- как Бог --- видит, что один из них ''соответствует'' миру лучше. Относительная успешность одного языка перед другим, однако, объективный факт --- но не тот, что \textit{он} мог бы выразить, говоря о ''соответствии'' очерчивания языками этими мира так же как мир очерчен сам собой.

Мы вернулись к понятию \textit{соответствия}. Некоторые философы здесь могут думать, что Крузо-2 способен понять \textit{соответствие} как \textit{объяснение} превосходства одного языка над другим: более совершенный язык лучше соответствует миру, и \textit{потому} лучше подходит для в нем навигации. Крузо-2 --- сказали бы они --- занимается выводом наилучшего обяснения своего успеха: успех его должен обусловлен быть тем \textit{фактом}, что термины в нем лучше отражают то, каков мир. Его успех, должно быть, обусловлен совпадением способов его терминов делить мир с тем, как мир действительно разделен своими стыками. Он, конечно, прямо не видит, что это так --- делает вывод только на основании своего успеха.

Однако, все еще есть старая проблема: что \textit{имеет в виду} Крузо-2, говоря нечто вроде ''подходит лучше'', ''делит мир по стыкам''? (Как он может \textit{понять} эти метафоры?) Как им под ними подразумеваемое выходит за рамки лучшего в мире ориентирования? Метафоры ведь эти \textit{не могут} значить ''при ''правильном'' описании мира наилучший язык описывает мир так же'', потому что понятие ''правильного описания'' мира Крузо-2 может быть только описанием, ведущим к успеху в этом мире. Так что же они значат?

Вот, например, карты --- некоторые из них, мы можем думать, ''совершенно'' описывают местность, которую должны описать. Но как это может значит что-то кроме ''использование этих карт исключает ошибки''? Впрочем, мы, похоже, думаем, что как-то да выходит --- мы же можем представить лист бумаги с линиями на нем, а так же местность, этим линиям совершенно соответствующую. (Линии на карте изоморфны контурам местности \textit{с учетом} масштаба.) Однако \textit{образ} такой сталкивается с той же проблемой --- он по прежнему лишь описание местности ''без языка и понятий'', абстрагированное от ''визуальных'' наших диспозиций, отчего и думаем, будто карта соответствует этой якобы бессловесной (неконцептуальной) характеристике местности. Так что, похоже, проблема Крузо-2 --- и наша проблема тоже.\footnote{Множество философов на протяжении долгих лет утверждали --- вопреки, например, корреспондентной теории истины, --- что слова нельзя сравнить с реальностью. Не всегда, впрочем, очевидно, что они, говоря такое, имеют в виду. Но некоторые, похоже, имеют в виду здесь разъясненное.}

Пусть Крузо-2 на самом деле зчитает (а не считает), тогда \textit{Бог бы сказал} ''Крузо-2 поступает плохо --- относится к некоторым разным по количеству наборам предметов как к по количеству одинаковым''.\footnote{Похоже, Бог скажет так лишь если не ''переведет'' термины Крузо-2 на свои собственные --- по той же причине, по которой отдохнувший Крузо-2 не может сказать, что, когда устал, то \textit{думал}, что несет \textsuperscript{r}кокос\textsuperscript{r}, и почему мы не можем сказать, что он \textit{думал}, будто у него есть кокос, когда у него был юкос. Бог же \textit{может} сказать, что ''уставший'' язык Крузо-2 заставляет его относиться к кокосам --- как к юкосам, а к различным количествам --- как к одинаковым, не описывая при этом его \textit{мысли}. Бог также может сказать, что слова Крузо-2 не выявляют реальный различий и сходств --- тех, что влияют на его (Крузо-2) благополучие --- и именно потому он плохо поступает. Тема эта также поднимается в секциях 3.5 и 6.4} С другой стороны, пусть на острове есть еще кто-то (\textit{Пятница}, например), и пусть они одинаково отдохнувшие/уставшие. Тогда, если Пятница (мудро) решит вести переговоры с Крузо-2 только в отдохнувшем состоянии, но Крузо-2 (по глупости) ведет переговоры с Пятницей, будучи уставшим, то Пятница однозначно будет успешней Крузо-2. (Потому что он, например, поймет, что может использовать Крузо-2 как \textit{денежный насос}.) Пусть Пятница описывает это, будучи отдохнувшим, тогда он скажет, что раз Крузо-2 \textit{думает}, что у него 5 кокосов, то он готов обменять их на 5 сардин в пятницу, когда на самом деле у Крузо-2 не 5, а 57 кокосов.\footnote{Здесь снова возникает проблема из сноски 12: Пятница ведь может сказать это, ''переведя'' термины ''уставшего'' языка Крузо-2 на свои, но это будет неверное описание мышления Крузо-2.}

Достоверность таких заявлений Пятницы лишь в том, что он успешно пользуется ими в отношении Крузо-2 --- у него по прежнему нет для них Божественных оснований (хотя, скажем, его язык и язык Бога допускают дословный перевод в смысле соответствия применения слов Пятницей --- таковому Бога). Бог знает, что знает, потому что способен заставить свои слова соответствовать миру метафизически --- возможно, кстати, потому, что сотворил мир таким вот соответствующим своим словам. Пятница же не обладает такими сверхспособностями, поэтому может знать лишь то, что у него дела идут намного лучше, чем у Крузо-2.

Можно, конечно, подумать, что вступительное описание Крузо-1 и его мира в Секции 3.2 --- его описание \textit{с точки зрения Бога}. Но наверное \textit{нам} лучше не пытаться приписывать себе такое в сколь-нибудь большей степени, чем Пятнице --- лучше сказать, что мы Крузо-1 с ''пятничной'' точки зрения описываем. И тогда идеальное соответствие диспозиций Крузо-1 его окружению значит ''Будь мы на одном острове с Крузо-1, мы не смогли бы превратить его в ''денежный насос'' --- не смогли бы его эксплуатировать''.

Но вернемся к изолированному Крузо-2, оценивающему два своих языка. Повторим: тот очевидный факт, что он успешней с одним из них, чем с другим, и позволяет ему их оценивать. Сделав выбор, он может выразить свое признание превосходства выбранного языка --- как мы видели ранее --- либо критикуя свои диспозиции, либо критикую язык. Но это не предполагает (и предполагать вовсе не может), что он знает сам факт метафизического соответствия --- что термины в ''лучшем'' языке очерчивают реальность точнее, чем в другом.

И тем не менее, при подходящих обстоятельствах Крузо-2 может и осознать уместность своего доверия \textit{некоторым} диспозициям и недоверия --- другим. Для этого ему нужно, конечно, распознать (некоторые из) своих состояний. \textit{Значит} это его предпочтение \textit{не} произвольно, ведь он опирается на объективные факты об успешности различных используемых им языков. ''Успех'' здесь, значит, включает способность Крузо-2 оценить свое благополучие, к изучению чего и перейду я в следующем разделе.

\section{Интроспекция}

Я предположил, что Крузо-2 различает свои диспозиции на основе успеха при их использовании. Я также предполагаю, что он может оценивать свое благополучие после событий, которые на него влияют, и даже осознать, как эти события влияют на него. Для этого не обязательно предполагать его в этом вопросе объективность или что он всегда прав --- достаточно предположить, что он достаточно хорошо распознает, когда, например, от съеденного ему стало плохо, или когда животное стало опасным, или когда причинило ему вред, и т.д.. Допустив эти возможности, мы можем говорить о нем, как о способном учиться на своем опыте.

Сразу возникает проблема: разные Крузо могут иметь разные ''системы ценностей''. Описанные в этой книге Крузо --- существа с простыми, понятными ценностями: они хотят ''человеческого комфорта'', выжить, избежать боли и прочих неудобств, им также нужна еда, и желательно та, от которой не тошнит. Но возможно ведь другие Крузо: презирающие земные удобства, желающие быть съеденными, принимающие боль и дискомфорт и смерть. А понятие успеха ведь с конкретными ценностями и соотносится.\footnote{Я благодарю Митча Грина (устное общение --- 08.08.2009) за то, что он поднял этот вопрос и побудил меня к написанию следующих абзацев.} Рассмотрим еще раз случай, когда Крузо говорит на двух языках: ''уставшем'' и ''отдохнувшем''. Несомненно, следующий за этим его успех касательно ценных для него ''самоубийственных'' ценностей будет достигнут принятием решений в уставшем состоянии.


Очевидно невозможно отрицать связь успеха Крузо с тем, чего он хочет. То есть ''успех'' должен быть, хотя бы частично, соотнесен с целями и желаниями конкретного рассматриваемого Крузо.\footnote{''Частично'' --- потому что есть некоторые сложности: иногда мы учитываем при оценке успеха других оценку и их ценности, считая, например, неудачником достигшего ценностей, которые считаем на самом деле не ценными вовсе. Сложности эти, впрочем, не относятся прямо к понятию ''успех'' как оно здесь используется.} По этому поводу, однако, следует сделать два замечания. Во-первых, одна из целей книги --- представить несколько видов Крузо с широким диапазоном диспозиций в различных обстоятельствах, в которых они могут следовать приватным правилам. Я намерен в конечном счете, в Главе 5 укоренить полученную убежденность в траектории изменения приводящих к объективному успеху диспозиций Крузо. Конечно, мы \textit{не} можем приписать такому Крузо \textit{какой угодно} набор диспозиций и ожидать, что он будет успешно следовать правилам, поэтому я в этой книге придерживаюсь простых случаев: тех, где Крузо хочет выжить, нуждается в комфорте и т.д., а ''нездоровых'' Крузо отложу на другую работу.\footnote{Под словом ''нездоровый'' я не имею в виду, что у таких Крузо есть ценности, которые следовало бы игнорировать или исключать, и я не утверждаю, будто Крузо эти не могут следовать частным правилам несмотря на свои такие ценности --- многие, очевидно, могут. Речь лишь о рассмотрении случаев, не слишком усложняющих обсуждение --- чтобы легче продемонстрировать возможность следования приватным правилам в самых разных обстоятельствах.}

Во-вторых, хотя ''успех'' должен быть соотнесен с ценностями Крузо, он должен быть и объективным --- объективным фактом того, что Крузо становится лучше (согласно его ценностям), когда, например, он принимает такое-то количество кокосов, а не другое и т.д., объективный факт лучшей среди прочих группировки предметов в своем мире.

Этот ответ, однако, поднимает еще две проблемы. Во-первых, такие объективные факты, кажется, должны быть выражены на Божественном языке, и в то же время --- укоренены в пропозициональных установках Крузо (чего \textit{он} хочет и чего не хочет).\footnote{Своим вниманием к этой проблеме я обязан Стивену Шифферу (устное общение --- 08.08.2009).} Но любое подобное описание пропозициональных установок Крузо должно быть --- как мне кажется --- выражено на том или ином языке. Пскольку он ясно формулирует, что он хочет съесть кокос и избежать тигра, то отношение такое должно быть выразимо на всяком языке, которы он использует, и, конечно, в языке этом должны быть \textit{его} слова ''кокос'' и ''тигр'', например. Но его язык, похоже, не позволяет описать соответствующие неудачи. Вспомните, как уставший Крузо-2 хотел \textsuperscript{t}кокос\textsuperscript{t}. Он ведь и получил \textsuperscript{t}кокос\textsuperscript{t} --- где же здесь ошибка? Описывая такие установки на предположительно Божественном языке, мы уже не выражаем пропозициональные установки Крузо-2 также, как он сам их выражал и признавал.

Значит первое, что нужно сделать, --- это отказаться от идеи, что успех Крузо-2 обясняется терминами ''кокосы'' и ''тигры''. Нужно формулировать с точки зрения удовлетворение его более примитивных желаний --- голода, страха, безопасности и т.д. --- только так, посредством неявных (или явных) содержащих их грубых закономерностей (а не только с помощью \textsuperscript{t}кокоса\textsuperscript{t} и \textsuperscript{r}кокоса\textsuperscript{r}), следует обозначать его успехи.

Но, похоже, это лишь отодвигает проблему на шаг назад. Вариант первый: у Крузо-2 есть слова и для обозначения этих психических состояний: \textsuperscript{t}голод\textsuperscript{t} и \textsuperscript{r}голод\textsuperscript{r}, \textsuperscript{t}страх\textsuperscript{t} и \textsuperscript{r}страх\textsuperscript{r}, \textsuperscript{t}комфорт\textsuperscript{t} и \textsuperscript{r}комфорт\textsuperscript{r}, и т.д., и для них возникает та же проблема, что и для слов о кокосе и тигре. Вариант второй: у Крузо-2 нет слов для описания этих психологичегских состояний. Но как это поможет уйти от вопроса касательно языка его пропозициональных установок? Он, конечно, в принципе способен \textit{осознать}, что голоден или напуган, но будучи уставшим, может думать, что голоден, когда на самом деле сыт, и напуган, когда на самом деле спокоен. То есть даже без явных слов Крузо-2 проблема сохраняется, потому что он по-прежнему обременен двумя наборами концепций.

У этой проблемы есть изящное решение: припишем Крузо-2 определенный инстроспективный доступ к своем же состояниям, чтобы он обычно не думал, например, что голоден, когда на самом деле сыт, или испытывает боль, когда на самом деле ее не испытывает. То есть свяжем такие его самовосприятия с соответствующими его психическими состояниями. Крузо-2 так будет прав относительно \textit{некоторых} своих состояний, а значит, выраженные в понятиях боли, дискомфорта, голода и т.д. его успехи решат проблему. Ведь пропозициональные установки Крузо-2 --- его желания и надежды --- объективны, когда выражены в терминах не кокосов или тигров, а боли, дискомфорта и голода, и эти же понятия тогда использовал бы Бог для описания установок Крузо-2. Более того, понятия эти негибки: они не меняются от усталости.

Это решение исключает опасность создания бессвязности в рассматриваемом мысленном эксперименте и его модификациях, а так же избегает следствия, согласно которому Крузо-2 на самом деле не способен сколь-нибудь убедительно следовать приватным правилам. Однако все же есть другие две опасности. Во-первых, постулируя обладание Крузо таким доступом к собственным внутренним состояниям, мы отклоняемся от предположения использования ими диспозиционно-смысловых языков. Например, вместо диспозиции применять слово ''голод'', мы говорим уже о применимости этого слова только в отношении действительного голода. Было бы поспешным предполагать, будто у голодного всегда возникает желание применить слово ''голодный'' в отношении себя. Вообще, находиться \textit{в} каком-то состоянии и иметь \textit{слова} для его обозначения --- разные вещи, ведь состояние само по себе может и вызывать диспозиции, но не обязательно диспозиции касательно соответствующих слов применения. Вторая же опасность в превращении Крузо-2 в настолько уж искусственное существо, что соответствующие ему случаи использования приватных языков будет трудно воспринимать всерьез. К этим проблемам я возвращаюсь в оставшейся части этого раздела.

Давайте описывать Крузо-2 как находящегося в состояниях: усталого, отдохнувшего, голодного, испытывающего боль, а так же как имеющего понятия: ''голоден'', ''испытывает боль'' и т.д.. Верно ли, что он имеет диспозицию применять эти понятия \textit{тогда и только тогда}, когда находится в соответствующих состояниях? Конечно нет, это слишком сильное и, вообще то, не нужное, утверждение. Мы, однако, предположили нечто подобное в связи с его отдохнувшим состоянием и такими вещами как кокосы, например. Но давайте быть реалистичными касательно его интроспективных концепций. Крузо-2, даже когда он отдохнул, не обязательно иметь термины \textsuperscript{r}боль\textsuperscript{r}, \textsuperscript{r}страх\textsuperscript{r} и т.д., применяемые к его же состояниям тогда и только тогда, когда и бы Бог применил \textit{Свои} термины к этим состояниям. То есть использование им этих терминов, возможно, иногда отклоняются от такового Богом. (Например, можно страдать от заболеваний, вызывающих диспозицию применить касательно себя понятие жажды, тогда как Бог бы так не сделал.\footnote{Декарт (1993, 58) говорит в этом отношении о водянке.}) И термины Крузо-2 \textsuperscript{t}боль\textsuperscript{t}, \textsuperscript{t}страх\textsuperscript{t} и т.д., возможно, еще больше отклоняются в использовании им от такового Богом. Тогда следует допустить, что у него есть пары интроспективных понятий: \textsuperscript{t}страх\textsuperscript{t} и \textsuperscript{r}страх\textsuperscript{r}, например.

Если эти парные концепции слишком сильно отклоняются друг от друга --- и, значит, от предположительно описываемых ими состояний --- как их бы видел Бог), то Крузо-2 не сможет сколь-нибудь долго выживать. Но если нет --- если \textit{по большей части} они согласуются --- то они предоставляют Крузо-2 способность достаточно точно распознавать успех и неудачу своих действий, а вместе с ними --- и лучшие в смысле достижения успеха языки. Более того, посторонние, описывая его желания, смогут довольно точно использовать свои слова ''страх'', ''голод'' и т.д. для описания его же состояний.

Стоит отметить присущесть более или менее успешных корреляций самоописаний с действительным уровнем своего комфорта --- нашей самооценочной психологии. Несмотря на насыщенность современной популярной психологической литературы результатами исследований, показывающих, сколь сильно мы способны заблуждаться касательно собственных же мотиваций и сколь плохо порой оцениваем свои способности и действия --- не смотря на это --- мы все еще достаточно хорошо осознаем боль и дискомфорт. Действительно, когда дело касается базовых удобств, мы довольно точно понимаем, когда нам, например, стало лучше, чем раньше. Еще раз: наши навыки в этой области не обязательно должны быть совершенны, чтобы была возможность следовать приватным правилам. Но конечно, эти наши навыки прямо отражаются на возможности им следовать: навыки эти более чем достаточны.

Я начал говорить о нашем понимании своих психических состояний и попытался упростить способ говорить о них, обращаясь к точке зрения Бога. Но то же, что касается терминов Крузо-2 для описания психических состояний (и терминов используемых в его отношении другими), касается и его терминов для предметов в мире. Например, мы можем заменить наш ''объективный'' способ сравнения терминов психических состояний Крузо-2 с нашими, на таковой Пятницы. Вместо того, чтобы говорить о лучших/худших корреляциях между его самоатрибуциями и самими этими состояниями, мы можем говорить об использованиями этих терминов с точки зрения приносимого \textit{успеха}. Так, испытывать боль, голод или страх --- значит находиться (по отношению к набору ценностей) в худшем положении, чем если бы их не испытывал. А наличие слов для этих негативных состояний помогает их устранять и лучше ориентироваться в мире.

Исложенных соображений, пожалуй, чтобы убедить в не столь уж исследовательско-стерильном характере описываемых случаев следования приватным правилам. Кстати, соображения эти применимы и к воспоминаниям Крузо-2, и это важно, ведь память имеет решающее значение для способности сравнивать свои языки --- ведь при этом нужно \textit{вспоминать} свои прошлые психические состояния. И для описания их нужен набор понятий. Я, конечно, принял в этом отношении неявную идеализацию, предложенную Крипке (и упомянутую в Главе 2) --- о том, что субъект прекрасно помнит все свои прошлые такие состояния. Но предположение это нельзя так уж легко принять касательно носителей диспозиционно-смысловых языков, ведь даже терминам для собственных психических состояний тогда надо придавать смысл диспозиций. Я уже предположил, что это вполне возможно --- возможно допустить достаточное соответствие значений этих терминов реальным состояниям субъекта (как описал бы их Бог). Но память может показаться нечтом в этом отношении особым, особенно учитывая недавнюю психологическую литературу о ее неточности.\footnote{Смотрите, например, учебник Баддели и др. (2015).} Опять же, совершенно точная корреляция здесь \textit{не} требуется --- Крузо-2 не обязательно помнить всякое нечто тогда и только тогда, когда оно действительно произошло. Нужна лишь достаточная корреляция между двумя языками Крузо-2 --- в отношении своих состояний и языка Бога --- чтобы он мог языки эти оценить. Бог бы мог сказать, что Крузо-2, склонный систематически заблуждаться касательно произошедшего, например, с \textsuperscript{t}кокосом\textsuperscript{t} после отдыха, долго не протянет. Насколько же именно точная корреляция нужна между его воспоминаниями и тем, что происходит --- вопрос эмпирический. В рамках рассматриваемой серии мысленных экспериментов будем считать, что она достаточна.

Крузо-2 кому-то может показаться совсем уж неестественным, учитывая его интроспективную терминологию --- он ведь знает, что говорит на диспозиционных языках. Понятно, что он думает, что кокос --- это то, что он кокосом склонен называть, но имеет ли смысл утверждение, что, например, \textsuperscript{r}воспоминание о вчерашнем поедании кокоса\textsuperscript{r} --- это то, что он склонен (прямо сейчас) им называть? Впрочем, такой образ мышления о своих состояниях и нам не чужд. В частности (и несмотря на утверждения Мура касательно памяти\footnote{Мур (1962, 210-211).}), мы часто говорим нечто вроде ''помню, как положил бумажник в пальто, но очевидно, что на самом деле этого не делал''. Мы говорим и что иногда что-то помним, или думаем, что голодны или что хорошо отдохнули, когда на самом деле совершенно ясно, что это не так, и применять эти фразы заставляют нас лишь соответствующие \textit{диспозиции}. Это указывает на диспозиционный смысл фраз вроде ''думаю, что хочу пить''. И ведь действительно --- говоря подобным образом, крайне трудно осознать свою неправоту.

\section{Последние замечания}

Убедительность некоторых понятий (например, об ''оправданном'' или ''неоправданном'' применении слова, или о ''правильном'' его применении) утрачивает силу, когда речь заходит о носителях диспозиционно-смысловых языков, что и предсказывает анализ Крипке парадокса следования правилам. И все же, диспозиционно-смысловые языки, даже в представленных простых формах --- вполне успешны как приватные языки и даже полезны для своих носителей. Более того, как я показал, такие языки могут \textit{объективно} сравниваться их носителями \textit{самостоятельно}. Носители их будут, значит, объективно определять, какой язык \textit{им} лучше подходит для ориентации в мире. И это так, несмотря даже на их непонимание ''соответствия миру''. Это, я считаю, удивительные результаты.

Несмотря на оговорки относительно памяти и доступа к своим психическим состояниям (в прошлом разделе), Крузо-1 и Крузо-2 остаются, конечно, нереалистичными в психологическом плане. Особенно нереалистично, что им известна связь их языков с диспозициями (а у нас, как правило, не так). В Главе 5 я вернусь к разработке мысленных экспериментов о Крузо, наделив их теперь более реалистичной психологией. В Главе 4 же я постараюсь подточить некоторые естественные против моих утверждений возражения, особенно из Раздела 3.4, что якобы понятия ''деление по стыкам'' и ''соответствие миру'' утратили свою актуальность в контексте Крузо-1 и Крузо-2 --- такие указания дает в своей работе Дэвид Льюис.

\chapter{Референциальный магнетизм}

\qquad

\textbf{Аннотация} \quad В этой главе я рассмотрю влиятельный ответ Дэвида Льюиса на проблему следования правилам (посмертно названный ''референциальным магнетизмом''). В основополагающих его работах можно проследить три различных подхода. Первый говорит о структуре мира как метафизически обеспечивающей ресурсы, дополняющие наши для определения референции. Дело в том, что слова (понятия) имеют определенное значение, выходящее за пределы психологических и нейрофизиологических ресурсов любого человека и общества. Второй подход рассматривает интерпретаторов естественных языков согласно требованиям семантической теории (наряду с базовой научной практикой) для наложения определенной референции на термины этих языков и естественные виды как релаты видовых этих языков терминов. Третий рассматривает \textit{априорное} конститутивное навязывание естественных видов, требуемое ''единственной игрой в городе'' --- муровскими фактами определенной референции и предполагающими определенную референцию семантическими теориями. В этой голове я постараюсь показать, что ни один из этих подходов не работает.

\qquad

\section{Прямые решения через референциальный магнетизм}

Прямые решения парадокса следования правилам пытаются сохранить классическую метафизику соответствия между нашими утверждениями и миром. Они пытаются сделать это показав, например, что скептик не заметил в действительности имеющихся у нас ресурсов для определенности ссылок на вещи в мире --- ресурсов для, например, фиксации нечта как именно счета а не зчета. Или они могут пытаться дополнить недостаточные способности субъекта тем, что за его пределами. Смысловой скептик Крипке утверждает, что необходимые ресурсы --- обосновывающие факты --- не найти в нашей психологии или нейрофизиологии. Семейство прямых решений, критикуемое мной в Разделе 2.5, пытается ответить на этот вызов дополнительными ресурсами в сообществе говорящего или его \textit{переводчика}.

Семейство решений, тесно связанное с работами Дэвида Льюиса, обнаруживает нужные ресурсы либо в самом мире, либо --- в зависимости от интерпретации подхода Льюиса --- в наилучших наших об этом мире и себе теоритезациях. Утверждается нечто в мире, следующее из корреспондентной метафизики или научной практики, способное предоставить искомые обосновывающие факты. Это второе семейство прямых решений --- подходы референциального магнетизма --- столь же популярно, как и версии социологического ответа, приписываемые Крипке Витгенштейну.

Я покажу, почему ни одни из этих подходов не работает, а в последующих главах покажу, почему они и не нужны вовсе. Литература по метафизике, серьезно относящаяся к подходу Льюиса, лишь прикоснулась к текстовой сложности его своей позиции представления. Дело в том, что стиль Льюиса --- как и прочих хороших стилистов-философов, Юма, например --- маскирует сложности и двусмысленности содержания. В соответствующих обсуждениях Льюиса (Льюис 1983, 1984) я выделяю три позиции, которые разительно отличаются своими предположениями. Его риторика и аргументы показывают некоторую двусмысленность среди все трех позиций, хотя я и отмечаю, что он придерживается одной конкретной. Ни одна из этих позиций, впрочем, не отвечает смысловому скептику Крипке, и по той же причине не рассматриваем ''парадокс Патнэма'' --- хотя на последнем я и не буду останавливаться. Все три ответа рушатся от давления определенных внутренних несостыковок.

\section{Может ли Крузо использовать реальность как стандарт для своих слов?}

Некоторые философы оспорят мое утверждение в Разделе 3.4, что Крузо-2 не может сравнить сходства своих ''уставших'' и ''отдохнувших'' слов с миром. Они скажут, что Крузо-2, конечно, сравнивает эти два языка вопрошая не какой из них лучше для ориентации в мире, а какой набор слов лучше всего схватывает реальность. Я уже указывал на \textit{невозможность} этого, обратное утверждение настолько привлекательно, что заслуживает отдельной главы.

Вполне естественно, конечно, думать, что Крузо-2 осознает свое желание соответствовать словами миру, а не просто наборам вещей, к которым он \textit{склонен} относиться как к одинаковым. Но как он может \textit{понять}, что достиг этой цели? И как к ней собирается двигаться? В Секции 3.4 я поддерживаю идею существования у Бога языка, в котором каждое слово отсылает к естественному виду, и Его фразу ''естественный вид'', а также фразы рода ''такой же вид, как ...'' соответствуют только реально существующим естественным видам. Нет нужды объяснять, почему язык Бога таков --- ведь Он всемогущ --- что очень удобно. Но нужно объяснить, как может Крузо быть способен на такое.

Итак, рассмотрим нового Крузо --- Крузо-3. Представьте, что его островной мир схож с таковым Крузо-1, но его чувства не такие же, как у Крузо-1 и Крузо-2 (даже отдохнувшего), а схожи с нашими --- достаточно хороши, хотя порой и могут обмануть. Возможность такого обмана не значит, что он будет смешивать в один класс то, что Бог бы распределил по разным. И пусть он так же слегка неуклюж в счете. (Значит Бог будет думать, что рассматривает количества предметов в определенных наборах как одинаковые, когда Бог бы рассмотрел как разные и наоборот.) Чувства и навыки счета Крузо-3 не столь плохи, чтобы он погиб, но все же \textit{достаточно плохи}, чтобы он часто ощущал негативные от этого последствия. Наконец, он \textit{намеревается} использовать свои слова не только лишь как то заставят диспозиции, но и согласно реальности.

Понадобиться объяснить, как Крузо-3 должен понимать \textit{свои} слова ''кокос'', ''тигр'', ''естественный вид'' и т.д., чтобы \textit{они} такому его намерению соответствовали. Во-первых, Крузо-3 не может понять стандарты своих слов как определяемые лишь только диспозициями --- вместо этого он должен понимать, что стандарты эти следуют из реального устройства мира. Но как \textit{он} может добиться этого? Ведь все, что у него есть --- лишь \textit{реальные} диспозиции применять определенные слова --- больших способностей у него нет. Как же перейти от его действительных диспозиций (\textit{фактических} стандартов) к вещам в мире так, чтобы последние стали стандартами? Иначе говоря, как реальный мир манифестирует себя в намерениях нашего Крузо-3? Он же не может, например, \textit{подумать} или сказать: ''Стандарт применения моего слова ''кокосы'' --- \textit{реальные} кокосы'', ведь ''реальный'' --- такое же \textit{его} слово, как и ''кокос''. Кажется, что попытки Крузо-3 использовать свои слова для обозначения чего-то вне его диспозиций --- попытки, подобно Мюнхгаузену, вытянуть себя из болота за собственные волосы.

Этот третий Крузо, как и предыдущие --- должен думать при каждом пременении слова ''кокос'', например, что \textit{полагается} на свою диспозицию группировать вещи так, как они на соотносятся самом деле. Как иначе? Значит, он применяет слово (или понятие) ''кокос'' ко всему, что кажется ему на это похожим. И все же, предполагается его \textit{понимание} значения слова ''кокос'' как относящегося не ко всему, что ему лишь \textit{кажется} похожим на кокосы, а ко все, что действительно схоже с ними. (Но, учитывая, что это \textit{его} понимание --- ''похоже'' должно быть также \textit{его} словом.)

Здесь возникают \textit{две} проблемы. Во-первых, Крузо-3 должен как-то \textit{заставить} свои слова соответствовать миру. А во-вторых --- как он должен \textit{думать} или \textit{понимать} свои слова согласно этому предположению?

Давайте пока оставим эти загадки в стороне. Одним из необходимых следствий такого образа мышления, как у Крузо-3, будет его отношение к своим характеристикам предметов --- как к лишь оправданно применимым (они ведь могут применяться только согласно его диспозициям их применять). Ему, значит, придется считать себя способным на \textit{ошибки}, но понятие ошибки здесь существенно отлично от такового у Крузо-2 в отношении своих ''усталых'' диспозиций. Крузо-2 их как ошибочные относительно ''отдохнувших'' диспозиций как стандарта, и признает ошибки как использование ''неправильных'' слов (или ''неправильных'' диспозиций с правильными словами), то есть: использование слов или диспозиций, которые \textit{менее ценны}, че те, что он использовал бы вотдохнувшем состоянии. Но ведь предполагается, что Крузо-3 имел в виду нечто иное: стандартом должен быть \textit{мир}, а не какие-то диспозиции. А значит, его ошибка не в выборе языка, а в \textit{самих предметах} --- он что-то не так понял касательно ни самих и с тем, как их следует группировать.

Вернемся к первоначальным вопросам. Во-первых, как намерения Крузо-3 позволяют его словам ''похожий'', ''кокос'', ''естественный вид'' и т.д. соответствовать ''мировым стандартам'', т.е. соответствовать реальным естественным видам в мире? \textit{Что-то} должно этому способствовать, так что естественна такая этого вопроса постановка: каковы \textit{факты} о мире и Крузо-3, определяющие его слово (концепт) ''кокос'' как соответствующее правильному реальному набору предметов, а не какому-то другому? Что определяет слово ''сходство'' группирующим вещи в наборы, соответствующие реальным видам? Вопрос этот особенно неприятен из-за того, что, следуя примеру Крузо-3, \textit{не удается} получить соответствующим слову ''кокос'' все и только то что ему кокосом действительно является.

Во-вторых, как может Крузо-3 вообще мыслить о ''вещах как они есть'', независимо от его собственной предрасположенности к о них мышлению, так, что даже \textit{допустить} стандартом эту ''реальность'' вещей?

Его предполагаемое понимание значений своих слов создает пространство между тем, к чему его слова ''кокос'', ''естественный вид'', ''настоящий'' и т.д. относятся ''на самом деле'' и его ресурсами для их с этим ''на самом деле'' соотнесения. Вопросы сосредоточены на фактах, которые бы определяли --- независимо от Крузо-3 --- что значат все эти слова. Дело в том, что диспозиции Крузо-3 исчерпывают его ресурсы для такого определения, они исчерпывают все, что он о словах своих способен думать. Что же здесь может помочь? (Что может это изменить?)

\end{document}

