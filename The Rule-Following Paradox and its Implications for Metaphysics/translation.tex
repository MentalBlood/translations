\documentclass[11pt]{book}

\usepackage{polyglossia}
\setdefaultlanguage[indentfirst=false,forceheadingpunctuation=false]{russian}
\setotherlanguages{english}

\RequirePackage[a4paper, headsep = 0.5 \headsep, left=2.5cm, right=2.1cm, top=2cm, bottom=2.1cm]{geometry}

\setmainfont{Times New Roman}
\newfontfamily\cyrillicfont{Times New Roman}[Script=Cyrillic, Ligatures=TeX]

\title{Парадокс следования правилам и его последствия для метафизики}
\author{Джоди Аззуни}
\date{2017}
\begin{document}

\maketitle

\tableofcontents

\chapter{Введение}

\qquad

\textbf{Аннотация} \quad Здесь я дам краткое содержание всей книги и проясню несколько методологических моментов, а также укажу причины избегания таких терминов как ''понимание'', ''значение'' и ''факт''. Я опишу некоторые существенные различия моего понимания соблюдения правил от такового Крипке, например --- и в этом разница между моим подходом и подходом большинства занимающихся этой темой философов --- я уделяю большое внимание ''следованию правилам'' в его применении к задачам в ходе взаимодействия с миром, и этого его отличия от, например, разного рода арифметических упражнений вроде сложения чисел.

\qquad

Центральным теме этой книги --- ''парадоксу следования правилам'' --- является термин ''понимание'': как в выражениях ''она понимает, как складывать'', ''он понимает, как продолжать считать числа больше десяти'' и ''она понимает эти концепции''. Уместно предупреждение касательно такой методологии: ''понимание'' здесь обозначает поразительно сложную человеческую способность, включающуюю субличностные, сознательные/феноменологические, социологические и нормативные элементы. Понятие это слишком сложное для \textit{использования} в философском анализе --- должно скорее быть его \textit{объектом}. Это, впрочем, едва ли единственное такое опасное в своей сложности слово: еще одной ловушкой могут стать ''объяснение'' и ''смысл''. Причем даже повторяющиеся и очевидные неудачи, похоже, не мешают философам продолжать попытки строить сколь-нибудь точный анализ на таком песчаном фундаменте.

Куайн редко комментирует свой подход к занятию философией --- даже его признания натурализма и экстенсионализма довольно специфичны по содержанию и применению --- и тем не менее, я нашел у него редкую методологическую медитацию, стоющую \textit{запоминания}. Куайн (1981, 184) пишет:

\smallskip

\textit{Согласно описанию моих взглядов Шуденфреем: ''предложения заменили мысли, а склонность к согласию заменила веру''. Имеет ли он ввиду отсутствие для меня большего чем вера и мысли? Читая дальше, понимаю, что так и есть. Значит он неверно меня понял.}

\smallskip

Куайн продолжает:

\smallskip

\textit{Позиция моя в том, что понятия мысли и веры --- достойные объекты философского и научного анализа --- и столь же плохи как этого анализа инструменты. Если кто-то решается их таковыми использовать, то загадкой для меня остается что же, как он счел, более нуждается в анализе чем они сами.}

\smallskip

Но давайте все же придерживаться термина ''понимание'', раз уж именно ему эта книга непосредственно посвящена. Общим для нескольких философских традиций является предположение основанности всякого понимания набора концепций на понимании правил, этими концепциями управляющих. Это, похоже, отражает понимание значений слов и, например, концепций из таких простых математических практик, как счет. Хороший пример здесь --- шахматы: как только человек усвоит правила игры, он поймет, как в играть вне зависимости от обстоятельств.

Некоторые сторонники такого подхода говорят о нас как об обладающих ''предрассположенностью'' к пониманию этих правил и даже этих правил (будучих воплощенных в предрассположенностях) обладателях и что предрасположенности эти позволяют нам вести себя соответствующим образом в ситуациях при проявлении такого рода ''понимания'' --- например, при подсчете апельсинов в корзине.

Огромное количество литературы посвящено так называемому ''парадоксу следования правилам'' --- противоречию аспектов нашей практики естественной картине нашего понимания правил. Крипке (1982), например, развил масштабную философскую индустрию вокруг проблем, которые, по его мнению, Витгенштейн поднял для всякого подхода, обосновывающего понимание простых математических правил применением набора диспозиций. (Крипке, впрочем, не претендует на однозначную присущесть этой точки зрения Витгенштейну; и он, и другие, понимают что она ему, возможно, не пренадлежит, откуда и пошел термин ''Крипкенштейн''. Я избегаю этого термина, предпочитая использовать такие фразы как ''Витгенштейн Крипке'' или контекст.)

Проблема следования правилам, вкратце, такова: мы понимаем себя как следующих правилам и, пройдя соответствующую подготовку, действительно на это способных. Но средства, с помощью которых мы понимаем себя способными, согласно Крипке, не подходят для этой задачи. Кандидаты на развитие такой вот способности следовать правилам это, например, интроспективное их понимание, разные привычки и склонности их выполнять в рамках соответствующих задач и т.д. --- кандидаты эти показаны Крипке как неспособные гарантировать все необходимые элементы нашей компетентности в этим правилам следовании. Он отвечает на парадокс ''скептическим решением'' --- допускающим неудачу ''смыслового скептика'' и меняющим понимание нашей практики следования правилам несколькими важными способами. Среди этих способов отказ от понимания значения в терминах ''условий истинности'' и замена их ''условиями утверждаемости'' --- условиями с точки зрения соответствия индивидуальных практик следования правилам таковыми сообщества.

Версия парадокса Витгенштейна о следовании правилам, предложенная Крипке, оказала большое влияние. Один из вопросов которые я буду рассматривать --- это как она была понята как подрывающая индивидуалистскую картину математической практики --- точку зрения, согласно которой отдельные люди (буду называть их ''\textit{Робинзонами Крузо}'') независимо от сообщества могут осмысленно заниматься математикой и, значит, иметь ''приватные языки''. Парадокс отрицает применимость фраз вроде ''правильный подсчет'' к таким людям, потому что эти нормативные понятия осмысленно применимы только относительно стандартов сообщества.

Одним из исходных элементов моего альтернативного ''скептического решения'' будет отрицание этого шокирующего следствия: оно не следует даже если витгенштейнианские возражения Крипке против диспозиционных подходов к следованию правилам по большей части верны. Это потому, что мое решение не будет ставить стандарты сообщества выше индивидуальных. Более того, оно не не будет заменять условия истинности условиями утверждаемости, что важно для решения Крипке, отдающего предпочтение сообществу. Вместо этого мой подход фокусируется на том, как склонность к следованию правилам позволяет нам более или менее успешно взаимодействовать с миром. Предположим, что их практики демонстрируют склонности, которые меняются, постепенно оптимизируя успешные события (например, расчет неободимого на несколько дней количества еды). Тогда соответствие между индивидуальными следованиями правилам возникнет даже без явной связи этой практики с сообществом\footnote{Чтобы не перегружать объяснение и сделать его более ясным, я здесь пока опускаю детальное описание того, как это работает. В последующих главах эти детали будут прояснены.}.

Льюис (1983) в статье почти столь же влиятельной, как таковая Крипке (1982), связывает рассматриваемую версию парадокса с тогдашним антиреализмом Патнэма (1981) и предлагает общее решение. Он настаивает на подходе почти полностью встраивающем в ссылки на термины естественного вида (а также на математические термины вроде ''сумма'') \textit{предпосылку} метафизических ограничений на возможные их расширения и утверждает, что она \textit{необходима} для ответа и на рассматриваемый парадокс, \textit{и} на антиреализм Патнэма. Последующее поколение аналитических метафизиков назвало такой ответ ''референциальным магнетизмом''. В главе 4 я покажу, что семейство таких Льюизианских подходов к проблеме (далее подходы ''референциального магнетизма'') --- не работает и даже не требуется.

Льюис не использовал термин ''референциальный магнетизм'', фраза эта, видимо, пошла от Ходса (1984, 135), вопрошающего, как так получается, что

\smallskip

\textit{мы все в конечном итоге говорим на языках, в которых из всех возможных нумераторов (то есть функций \begin{math}F\end{math} 2ого типа, переносящих концепты 1ого типа на объекты, где для любых из этих концептов \begin{math}X\end{math} и \begin{math}Y\end{math} верно \begin{math}F(X)=F(Y)\equiv (Q_{E}x)(Xx, Yx)\end{math}) словосочетание ''количество ... '' обозначает стандартный нумератор? Почему стандартный нумератор является ''референциальным магнитом'', ''подтягивающим'' ссылку с помощью этой фразы (чего не могут сделать его нестандартные конкуренты)?}

\smallskip

''Референциальный магнетизм'' --- вещь тонкая, так что неудивительно, что последующее поколение философов превратно его истолковало.

Вот схема глав книги.

В главе 2 я возвращаюсь к обсуждению Крипке парадокса следования правилам для диспозиционных подходов к числовой компетентности, а также к альтернативному решению парадокса приписываемого им Витгенштейну. Соображения этой главы --- в значительной степени исходные соображения Крипке. Меня при этом не беспокоят вопросы витгенштейновской экзегезы --- насколько корректно Крипке присваивает парадокс Витгенштейну\footnote{По этому вопросу, впрочем, есть литература, например: Блэкбор (1984), Гольдфарб (1985, 1992), Макдауэлл (1984), Тейт (1986)} --- его формулировка проблем, стоящих перед смыслодиспозиционалистом, важна вне зависимости от присущести Витгенштейну. Есть, впрочем, два изменения. Во-первых, в рассуждениях Крипке центральным примером выступает субъект, практикующий арифметическое сложение, и скептический вызов о согласованности всего известного и испытанного им ранее с возможностью им практики на самом деле чего-то от сложения отличного, я же заменяю сложение более элементарной задачей --- задачей счета --- и соответствующей возможностью практиковать на самом деле что-то от него отличное. Это, конечно, незначительное изменение, и Крипке (1982, 17) сам мимоходом поднимал вопрос о счете.

Второе же изменение уже существенное и ключевое для моего решения парадокса. В моем случае центральный пример о субъекте, который считает \textit{вещи}, то есть в его задаче используются \textit{не только} числа. Дети так и учатся считать: они \textit{применяют} числа к задачам распознавания количеств предметов, и на приобретение таких знаний уходит несколько лет\footnote{Тщательное изучение этого вопроса с цитированием соответствующей литературы см. в Кэри (2009) и Баттерворт (1999), глава 3, разделы 1-4. Классическим исследованием является Гельман и Галлистел (1986)}. Это важно из-за рассмотрения многими философами ''социальных решений'' парадокса, приоритизирующих сообщество над индивидуумом, как единственных успешных. Одна из причин такого рассмотрения --- пренебрежение соображениями о применении математических концепций \textit{в мире}. Точно так же применение математики к миру находится по большей части за кулисами дискуссий Крипке и --- как побочный эффект --- за кулисами дискуссий многих этой литературы комментаторов.

В творчестве же самого Витгенштейна это не столь закулисно --- он часто приводит примеры про людей, считающих предметы. Да и Крипке тоже обращает внимание на ''числовые размеры'' наборов подсчитываемых предметов, когда мимоходом обсуждает ''зчет'' в рамках текста Витгенштейна. Но если подсчет \textit{предметов} занимает центральное место в примерах которым скептики бросают вызов, то становятся заметны вполне последовательные формы приватно-языковых практик. В частности, \textit{понятие диспозиций приводящих к когерентным таким практикам}, которое я ввожу в главе 5, не опирается ни на что за пределами контекста применения концепций к вещам в мире. Поэтому, конечно, мое скептическое решение парадокса не годится для бестелесных картезианских сущностей, от скуки вечно считающих числа в уме.

Оставляя в стороне эти разногласия с Витгенштейном Крипке, добавлю, что я частично согласен с одной важной леммой, которую Крипке выводит и парадокса и которую он (1982, 78-79) изредка подтверждает:

\smallskip

\textit{Надо отказаться от ''естественной предпосылки'' что осмысленные повествовательные предложения должны соответствовать фактам.}

\smallskip

Добавляя:

\smallskip

\textit{Прежде чем мы сможем приступить к скептической проблеме, надо прояснить картину этого соответствия фактам.}

\smallskip

Я \textit{частично} с ним согласен --- соображения следования правилам указывают на необходимость признания несоответствия \textit{некоторых} значимых повествовательных предложений фактам. Впрочем, не думаю, что это установлено для \textit{всех} таких предложений. В главе 6 я покажу, почему соображения следования правилам требуют лишь \textit{частичного} отказа от корреспондентской метафизики. Обсуждение это в то же время подкрепит мою особую форму дефляционизма относительно истины\footnote{См. Аззуни (2006) и гл. 6 этой книги. Интерпретация приведенной выше формулировки Крипке затруднена его фразов ''должен подразумеваться''. Я считаю, что осмысленные повествовательные предложение могут и даже ''должны'' претендовать на соответствие фактам --- и это совместимо даже с некоторыми из тех, что не столь уж соответствуют. (Убедительность этого толкования, конечно, зависит от значения термина ''смысл'' --- оно, все же, не совсем ясно, не ясно также, исключает ли ''уборка'' Крипке ''корреспондентность'' полностью или лишь частично.) См. обсуждение использования идиом знания/неведения в разд. 2, а именно сноску 15 гл. о моих способах маневрирования среди всего этого.}.

Здесь стоит отметить еще вот что. Диспозиционалистская надежда на посвященность значительной части анализа Крипке (1982) дроблению, находит \textit{основания} отношения соответствия между значимыми истинными предложениями и фактами в ''разуме'' (в широком смысле этого слова) человека, следующего правилам. Есть факты \textit{о человеке} и его \textit{возможностях}, которые и определяют отношения соответствия значимых истинных предложений фактам в мире. Но, думаю, соображения следования правилам указывают на неудачу этого диспозиционного проекта: диспозиции \textit{бессильны} в том, чего большинство диспозиционалистов от них требует, и получается так во многом по причинам, приведенным Крипке от имени Витгенштейна.

Я отвергаю, однако, главный тезис, который многие отсюда выводят. Крипке упорно говорил о себе как о всего лишь толкователе, но множество философов утверждают, будто соображения вроде представленных в главе 2 необходимо приводят к соотнесенности или воплощенности стандартов следования правилам вообще и математической практики в частности в \textit{обществе}, в котором человек эти правила изучает. Как говорит сам Крипке (1982, 109):

\smallskip

\textit{Что действительно отрицается, так это то, что можно было бы назвать ''частной моделью'' следования правилам --- моделью, в которой понятие следующего данному правилу человека надо анализировать просто с точки зрения фактов об этому правилу следователе и только о нем, то есть не отсылая к сообществу.}

\smallskip

Одна из целей этой книги --- показать, что несмотря на правоту Крипке относительно природы парадокса и последовавших на него ответов, модель приватного следования правилам остается этим парадоксом не затронутой. Контуры логического пространства в этой проблемной области более запутанны и тонки чем мыслители конца прошлого века осмеливались полагать.

В общих чертах обрисовав изложение Крипке его способа решения парадокса --- его, значит, интерпретацию Витгенштейна о замене условий истинности условиями утверждаемости, --- я в последующих главах приведу доводы в пользу уже другого --- не приоритезирующего общественные стандарты над индивидуальными --- решения. Также в этой главе я покажу неуспешность диспозиционного подхода, расширяющего круг необходимых диспозиций до обладаемых всеми в обществе.

В главе 3 я рассмотрю диспозиционно-смысловые языки --- действительно приватные языки. Их носители (называю их ''Крузовцами'') --- в отличие от нас --- относят каждое слово к тому, к чему их предрасположенности склоняют их его применять. Я далее начинаю исследование сферы применения таких языков и показываю как понятия вроде ''ошибка'', ''обоснование'' и ''концепция'' в них если и остаются, то лишь в некой скромной форме, впрочем эти результаты не столь существенны для приватного следования правилам. Во всяком случае, в этой и в 5ой главе покажу как они или в них необходимость нивелируются в таких приватных языках. Еще в этой главе я рассмотрю содержание фраз вроде ''согласованность с миром'' и ''разрезание мира по стыкам'' для носителей этих языков. Неожиданным выводом будет возможность использования таких языков для взаимодействия с миром при широком диапазоне благоприятных эмпирических обстоятельств, и более того --- носители этих языков могут объективно сравнивать разные диспозиционно-смысловые языки чтобы определять, какой из них лучше взаимодействует с миром. Они, впрочем, не будут использовать понятия вроде ''лучшего соответствия'' языка миру или ''правильности'' применения слова. Здесь следует указать, что ''благоприятные эмпирические обстоятельства'' значит нечто большее чем просто характеристики внешнего мира потому что они также охватывает эмпирические факты о, например, механизмах \textit{развития}\footnote{Есть забавное следствие моего решения парадокса, на которое укажу лишь в рамках этой сноски: возникает проблема с пониманием искусственных языков как неких полезных инструментов, которые, значит, можно принимать и отвергать в зависимости от целей (такое понимание описано у Карнапа (1956)). Ведь как можно сравнивать достоинства и недостатки таких языков не прибегая к некоторому метаязыку? Именно здесь эта загадка и решается.} диспозиций таких говорящих.

В главе 4 я сделаю паузу в анализе диспозиционно-смысловых языков, чтобы контраргументировать решения поставленных мной перед такими языками проблем касаемо отсылок к наборам вещей в мире, основанные на референциальном магнетизме. Здесь же я рассмотрю три его версии. Первая понимает структуру мира метафизически обеспечивающей ресурсы для дополнения того, что люди в сообществе привносят для определения отсылок. Наши слова (концепты) имеют определенное значение, выходящее за пределы психологических и нейрофизиологических ресурсов любого человека и даже за пределы того, с чем любое таких людей сообщество могло бы справиться. Вторая понимает интерпретаторов естественных языков так, как того требует --- наряду с базовой научной практикой --- семантическая теория для наложения определенной референции на этих языков термины и на естественные виды как на корреляты видовых терминов этих языков. Третья понимает \textit{априорное} конститутивное навязывание естественных видов как требуемое в качестве ''единственной игры в городе'' муровскими фактами определенной референции и предполагающими определенную референцию семантическими теориям. Разумеется, ничто из этого не защищает от референциального магнетизма, что я и показываю.

В главе 5 я представляю окончательную версию моего Крузо (Крузо 5) --- он психологически близок нам тем, что не понимает свой язык \textit{как} диспозиционно-смысловой, ведь его диспозиции к использованию терминов ему (как и нам) неведомы. Я показываю, однако, как его осознание возможности использования терминов для улучшения своего благосостояния позволяет ему навязывать последовательные стандарты этой своей приватной языковой практике.

Более подробно язык Крузо-5 я изучу в главе 6, показав способ применить к нему стандартную семантику функциональной истинности и то, как Крузо-5, подобно нам, естественным образом может использовать идиому истинности. Это должно продемонстрировать, что ''скептическое решение'' Крипке через замену условий истинности условиями утверждаемости --- не единственное такое решение парадокса. Замена эта, впрочем, нужна для вывода о невозможности приватных языков.

В главе 7 я завершу то, что не завершил до этого: разберу методологическую роль точки зрения Бога в моем подходе к оценке последователей приватных правил. Для этого я рассмотрю два возможных взгляда на языки Крузо (и наши языки): первый понимает их как постоянно меняющиеся в отношении того, к чему относятся их термины, а второй считает референцию фиксированной. Противопоставив их, я покажу, в чем смысл точки зрения Бога\footnote{Впервые я предложил эти различные способы рассмотрения работы языка в Аззуни (2000), Часть IV.}.

Далее я рассмотрю \textit{следствия} моего решения парадокса для корреспондентской метафизики. Если кратко: эмпирически обоснованная метафизика \textit{возможна}. Что же невозможно, так это пресуппозициональная роль такой метафизики в философских объяснениях эмпирического успеха или в семантике.

Ну и наконец, я концептуально свяжу парадокс с проблемой индукции Юма.

\chapter{Версия Крипке парадокса Витгенштейна и его решение}

\qquad

\textbf{Аннотация} \quad В этой главе рассматривается описание Крипке парадокса Витгенштейна и его решение. Интерпретация Крипке вызвала довольно много критических комментариев, но меня парадокс представленный Крипке интересует в отрыве от интерпретационной корректности. Здесь я рассмотрю две дистинкции: прямые и скептические решения и обосновывающие факты и факты соответствия, а так же три ключевых момента. Во-первых --- три требования Крипке для прямого решения парадокса: бесконечности, обоснования и ошибки. Во-вторых --- почему прямое социологическое решение не работает. И в третьих --- некоторые критические комментарии касательно следования правилам, возникшие после публикации соответствующей книги Крипке.

\qquad

\section{Постановка проблемы: три ограничения любого решения}

О том, кто успешно посчитал разные наборы объектов, мы можем сказать, что он научился считать. Это значит предположить, что он при следующих случаях будет продолжать считать ''так же'', несмотря на возможно новые виды придметов и большее их количество.\footnote{Некоторое неявное знание, нужное для успешного этой задачи выполнения, рассмотрено Гельманом и Галлистелом (1986, 77-82) как принципы подсчета: ''количественное слово'', ''нерелевантность порядка'' и ''абстрактность'': последнее слово в подсчете и есть количество подсчитанных предметов, потому порядок и вид считаемых объектов не важен. Описание процесса подсчета и трудностей преобретения соответствующего неявного знания см. в Кейси (2009, стр.241-244).} Пусть его научили системе счисления --- правилу образования новых нумералов из предыдущих. Системы нумералов разительно отличаются этим от наборов словесных названий чисел в большинстве языков\footnote{Рассмотрение типов числовой лексики в разных языках см. в Баттерворт (1999, особенно стр. 52-62).}, потому что вторые обычно относятся к финитной записи чисел, требующей явного создания нового словаря для все больших чисел. Однако в системы счисления создание обозначений для все большего количества цифр уже встроено. Итак, как только субъект обрел способность считать и освоил определенную систему счисления, мы говорим что он \textit{понимает}, как считать\footnote{Дети, кстати, способны схватить неопределенную природу чисел, не выучив настоящую систему счисления. Например, Кэри (2009, 252) цитирует слова одной пятилетней девочки, представляющие спонтанное изобретение аргумента в пользу бесконечности числового ряда: ''Предположим, вы считаете тысячу самым большим числом, но вы же можете сказать тысяча и один, тысяча и два и так далее''. Я в своем изложении предполагаю, что испытуемый все же освоил систему счисления, потому что это помогает избежать сложностей с числовыми языками, превосходящими знания испытуемого о самом числе. Да и в нашей культуре дети в любом случае со временем осваивают системы счисления. Это достижение трудно еще и из-за возможной путаницы с сопоставлением терминов системы счисления со словами-числами естественного языка. (См., например, многочисленные по этому поводу результаты Деэна и его коллег. Некоторые из полученных ими данных указывают на потерю определенных способностей с в то же время сохранением других после инсультов и других специфически локализованных травм головного мозга.)}.

Пусть например теперь он пытается сосчитать предметы набора большего чем всякий до этого встречавшейся --- набора из 57 предметов --- и получает правильный ответ. Крипкианский скептик оспаривает, что этот ответ соответствует предыдущему пониманию задачи счета нашим испытуемым: в прошлом, он говорит, испытуемый никогда не считал --- он зчитал\footnote{См. также Гельман и Галлистел (1986, 51), где рассматривается схожий интересный мысленный эксперимент.

Стратегия Крипке, заключающаяся в подрыве намерений субъекта через подрыв его намерений в предыдущих случаях, подвергалась широкой критике и, как мне кажется, часто неверно понималась. Форбса (1983-1984, 225-226), например, в своей интерпретации пишет: ''утверждение об отсутствии факта о намереваемом субъектом в прошлом двусмысленно: либо у него не было определенного намерения в прошлом, либо намерение было, но неясно, такое же ли, что и сейчас''. Но контекст то явно указывает на верность первого варианта прочтения. Вызов субъекту со стороны скептика ведь в том, что, возможно, он зчитал, а не считал --- вызов этот, значит, подрывает темпоральную идентичность намерений через подрыв всяких фактов о прошлом как определяющих соответствующее в настоящем. И факты о настоящем в противовес прошлому аналогичны таковым в прошлом --- они так же неспособны отличить счет от зчета. С другой стороны, Богосян (1989, 515) понимает эту загадку Крипке как вопрос о том, что вообще определяет условия содержания и смысла --- ''[обладание] условием корректности'', тогда как загадка Крипке лишь косвенно, через вопрос о намерениях субъекта в прошлом, касается этого. Это, конечно, явно неправильное прочтение. Как верно отмечает Форбс (1983, 226), это Райта (1980) (а не Крипке) интересует определяющее условия содержания или значения. (Форбс считает утверждение Райта ''более прямым и сложным'', чем таковое Крипке, и, думаю, он так считает из-за непонимания стратегии Крипке.)}.

Для ответа скептику что испытуемый субъект и в этот раз зчитал и, значит, его ответ про 57 согласуется с предыдущими касательно меньших наборов предметов, нужны факты о субъекте, лежащие в основе ее предыдущего и нынешнего понимания счета. Здесь для описания фактов --- касающихся, например, диспозиций субъекта --- фактов, которые объясняют, к чему приходит его понимание, я ввожу термин ''обосновывающие факты''.

\textit{Некоторая терминология} \quad Я отличаю ''факты соответствия'' от ''обосновывающих фактов'': первые --- это предполагаемые факты внешнего мира, которым соответствуют истинные осмысленные предложения, а вторые --- психологические/диспозиционные факты о субъекте или окружающей его среде, обеспечивающие его способность понимать истинные осмысленные предложения, и, значит, определяющие эти предложения как понимаемые соответствующими фактам внешнего мира. Как вскоре будет ясно, вызов смыслового скептика напрямую оказывает давление на предполагаемые обосновывающие факты, и это, в свою очередь, подрывает факты соответствия.

Слово ''факт'' похоже на те коварные слова, с указания на которые я начал общее введение. Я не собираюсь использовать его в философском смысле, в частности, говорить о сущностях-\textit{фактах}, которым соответствуют предложения (полагаю, что в метафизическом смысле \textit{нет ничего} чему бы они соответствовали). Просто в некоторых из них есть термины, которые отсылают друг к другу, и то, что эти термины обозначают, называется ''фактами''. (См., однако, главу 6 с уточнениям этой идеи для работы с распространенными случаями предложений с несоотнесенными терминами).\footnote{См. также Аззуни (2012c).} Я также не имею в виду метафизически нагруженное понятие ''граундинга'' или что там еще обсуждают во всякой сложной литературе, которая расцвела в последнее время.\footnote{См., например, Коррейя и Шнайдер (2012).} ''Факт'', ''обоснование/граундинг'' и ''обосновывающий факт'' в моем случае обычные дотеоретические термины.

Еще одна терминология требует описания: Крипке различает ''скептические'' и ''прямые'' решения парадокса следования правилам. Прямые направлены на поиск фактов, обосновывающих то или иное понимание субъектом счета. Скептические же признают, что обосновывающие факты не существуют и объясняют понимание субъектом счета как-то иначе.

\textit{Назад к диалектике} \quad Как уже отмечалось, именно предполагаемые обосновывающие факты (или система таких фактов) объясняют истинность того, что в прошлом субъект понимал, как считать, а не как зчитать. Эта модель обосновывающих фактов \textit{связана} с субъектом --- либо с его психическими состояниями, либо с субличностными событями или структурами, лежащими в основе психических состояний, позволивших ему научиться считать. Обосновывающие факты о понимании субъекта должны давать ответ на вопрос: как психологические состояния субъекта --- что он думал, столкнувшись с задачей --- позволили ему понять, что он считает, или: как нейрофизиологические и прочие субличностные репрезентации активируются для счета.\footnote{Кэри (2009), например, предлагает «онтогенетическое» описание того, как ребенок — уже обладающий врожденными субличностными когнитивными системами (параллельное выхватывание небольших наборов и квантификаторы естественного языка), применимыми к конкретным числовым задачам — может в течение полутора лет с помощью индукции и аналогии (так называемая «Квайнианская самонастройка») понять некоторые важные свойства чисел, например бесконечность их ряда, и научиться применять их в подсчете предметов. ПОразительная особенность парадокса следования правилам, как мы увидим, в очевидной нерелевантности эмпирической гипотезы Кэри и ей подомных, предложенных учеными-когнитивистами, для его решения.}

Крипке, я считаю, налагает три ограничения на объяснение структуры обосновывающих фактов --- диспозиционных или других --- которые составляют понимание субъектом счета, и, значит, ответ смысловому скептику.\footnote{Гинсборг (2011, 228) приводит три возражения, которые она (и другие) выдвинула Крипке против диспозиционной теории, возражения эти напоминают (хотя и есть существенные отличия) требования, которые, согласно моему прочтению, он предъявляет к обосновывающим фактам.} \textit{Требование бесконечности.} Любой субъект насчитал лишь конечное число наборов предметов, так что предполагаемая его способность считать всякие другие наборы провоцирует следующие размышления. Во-первых --- он, возможно, считал только, например, яблоки и груши, во-вторых --- он, возможно, считал только наборы из не более чем 57 предметов. Однако ясно, что его навыки счета подразумевают способность считать наборы отличные от прошлых и количественно и качественно, а если испытуемый не осознал эту нейтральность счета и почему-то чувствует, что не может сосчитать, скажем, красные предметы или предметы в определенных коробках или больший чем всякий предыдущий набор предметов, то, очевидно, что-то не так. Возможно он систематически пропускает числа, начинает заново при достижении какого-то числа или даже фиксируется на определенном числе как на ответе и говорит что остальные объекты не в счет.\footnote{Конечно, некоторые из этих ''ошибок'' происходят во время обучения счету, но не большинство. Дети проходят определенные стадии по мере приобретения навыков счете, и ученые-когнитивисты говорят о них как ''знающих один'', ''знающих два'', ''знающих три'', ''знающих четыре'', ''знающих подмножество'' и, наконец, ''знающих количественный принцип''. ''Знающие одно'' умеют отличать один объект от многих, но не могут пока различать наборы предметов по количеству предметов в них, аналогично ''знающие два'' умеют еще различать наборы из не более чем двух предметов по эти предметов в них количествах, и т.д.. Также на определенном этапе освоения навыков счета дети пропускают числа, но, научившись считать, уже никогда этого не делают. Подробное описание этих процессов и ссылки на соответствующую литературу см. в Кэри (2009) и Баттерворт (1999).} Под ''пониманием'' мы ведь подразумеваем способность решить всякую задачу по счету, сколь бы она не отличалась от уже до этого решенных.

Второе ограничение --- это \textit{требование обоснования}: когда испытуемый посчитал предметы и так узнал их количество, его ответ оправдан. Ему \textit{следовало} продолжать так же, учитывая его \textit{осмысление}, не случайно его способ счета дает ожидаемый ответ.

Не менее важно и то, что взрослые и дети убежденны в своей оправданности в таких ответах --- они ведь \textit{считают}, что научились считать, что поняли, что \textit{значит} ''счет'', и если кого-то спросить, \textit{почему} он считает именно так, он ответит ''Потому что \textit{так} это и делается'' или ''Потому что \textit{это и есть} счет'' или ''Потому что \textit{так} принято''. Если ребенок или взрослый считает каким-то странным, необычным образом, он будет оправдываться указанием на одинаковость ответов полученных его способом и обычным.\footnote{''Я сначала группирую их по пять штук, потому что такие группы проще распознать, затем считаю их и умножаю на пять'' --- такое описание может дать ребенок, и дает он его ровно потому что уверен в одинаковости результатов этого его нового метода и обычного (строгого перечисления).}

Обратите внимание, что предложенные обоснования --- что это \textit{так} и делается и т.д. --- за исключением редких случаев, феноменологически достоверны, т.е. субъект, изучив процедуру счета, не будет уже колебаться относительно ее корректности и сразу даст объяснение: ''Это \textit{и есть} счет''. И мы также принимаем это его оправдание от третьего лица и говорим ''Он \textit{понимает}, что значит ''счет'''' или ''Он умеет считать''.

И здесь есть одна тонкость. Согласно моему описанию ''обоснования'', у нас обычно есть \textit{два} ключевых ожидания от субъекта в отношении счета или какой-нибудь другой задачи. Во-первых --- если он умеет считать, то все что он при этом делает, согласуется с его \textit{намерением} (с тем, что он \textit{имеет в виду}). Иногда ведь мы терпим неудачу в выполнении задачи именно из-за \textit{несоответствия} между \textit{делаемым} и \textit{намереваемым}: не делаем того, что ''должны'' были сделать. (''Посмотри что ты наделал! Разве ты \textit{этого} хотел?''.) Второе же ожидание в том, что то, что субъект делает --- действительно \textit{счет}. Могут ведь произойти и другие ошибки: субъект может неправильно решить поставленную задачу, решив, что ему ''нужно'' делать что-то отличное от того, что действительно нужно. ''Нужность'' здесь иная: дело не в том, что он должен был действовать так, как он поступил, учитывая, что и как он понимал, а в том, что он намеревался выполнить одну задачу, но следовало то ему намереваться выполнить другую. Феноменологическая уверенность, значит, ощущаемая при выполнении поставленной задачи, включает оба ожидания и оттого двумя способами и может быть подорвана.

Не так уж много литературы есть об этой ''нормативности'' касательно соблюдения правил и того, насколько нагруженной она должна быть.\footnote{Дебаты о ''нормативности содержания'' или ''нормативности значения'' --- т.е., как нормативность участвует в следовании правилам и определении значения,- впрочем, продолжаются. Некоторый взгляд на нормативность значение приписывается и Крипке, потому что, например, согласно Гинзборг (2011, 228), он утверждал, что ''человек, \textit{склонный} или бывший \textit{склонен} определенным образом реагировать в определенной ситуации, не обязательно \textit{должен} так реагировать''. См., например, Богосян (1989, 2003), Глюер (1999), Викфорсс (2001). Как я уже говорил (и еще буду повторять), ''нормативность'', ''рациональность'', ''нормы'', ''правильность'' или ''уместность'' касаются этой темы только в смысле предполагания ими успешного следования правилам, а не в смысле приполагания самим следованием правилам какого-то из них. Частично дело в том, что используемое здесь ''должен'' лишь гипотетично и может быть понято аналогично таковому в, например, ''Если хотите жить, вы должны не прыгать с моста''. (''Если хотите жить, я бы советовал ....'') Или ''если действительно хотите жить, было бы неуместно прыгать с моста, не так ли?'' (Представьте, что это говорит ангел-хранитель в фильме, потенциальному самоубийце.) Гиббард (2003, 85) пишет: ''... стоит ли нам гулять --- может зависеть от погоды, но это ведь не делает погоду нормативной в каком-то особом философском смысле''. Итак, здесь я на стороне отвергающих ''нормативность значения''. Подробнее об этом я говорю в Разделе 2.3.} Я буду понимать ее в ее тонком, облегченном варианте. ''Обоснование'' же буду понимать как \textit{согласованность} того, что \textit{имеется в виду} субъектом, и того, что им действительно делается. То есть требование обоснования значит достаточность ''встроенного'' в то, что субъект имел в виду, для подтверждения соответствия того, что он делает --- тому, что он имел в виду. Это и лежит в основе его феноменологической уверенности --- он может ''объяснить'', почему им делаемое и есть им намереваемое и другими от него требуемое. То есть здесь фигурирует как многообразие того, что имеется в виду --- оно определяет соответствующее поведение, --- так и \textit{наше} понимание этого значения и им подразумеваемого.

Требование обоснования тесно связано с \textit{требованием ошибки}: наша уверенность в способе и результатах счета справедлива с точностью до случайной ошибки. Любой ведь может ошибиться при подсчете предметов и так получить неверный ответ --- в этом нет ничего необычного. Мы, однако, четко отличаем такую ошибку от непонимания концепции счета, и --- что особенно важно --- даже систематические ошибки от случаев недостаточного усвоения счета и случаев когда \textit{с субъектом} просто что-то не так.

Отвлечься и проглядеть какой-нибудь предмет или дважды его посчитать --- значит ошибиться. Другое дело, когда испытуемый из раза в раз намеренно считает, например, все красные объекты дважды. Если он при этом действительно думает, что считает, то, похоже, он не понимает что значит считать, а иногда даже можно сказать, что и неспособен этому научиться. Важный признак ошибки --- возможность ее \textit{признания}. (''Вы пропустили вот этот'' --- ''Упс, сейчас исправлю''.)

Как бы мы ни описывали способность конкретного субъекта считать и как бы мы его диспозицию, преобретенную в ходе обучения счету, ни характиризовали, надо оставить место для способности давать \textit{неправильные} ответы и этих ответов неправильность \textit{признавать}. Странно, конечно, называть это способностью, но именно таково требование. Мы часто получаем неправильные ответы, даже прекрасно разбираясь в счете, и любое описание схемы обосновывающих фактов нами используемых для объяснения способности считать --- способности, раз уж на то пошло, выполнять любую данную задачу --- должно оставлять место для возможности ошибок и возможности этих ошибок распознания.

Следует еще раз подчеркнуть, насколько ошибки --- даже систематические к ним \textit{тенденции} --- совместимы с нашим пониманием таких понятий как счет, сложение, вычитание, умножение и деление. Некоторые люди просто потрясающе хорошо считают --- быстро и точно и даже большие числа --- но врядли можно сказать, что они как-то более глубоко и точно \textit{понимают} саму концепцию счета. Они, наверное, знают больше соответствующих трюков и обходных путей и \textit{запомнили} больше фактов о числах, да и просто быстрее оперируют ими. Даже кто-то вроде меня, кто почти гарантированно допустит какую-то элементарную ошибку, считая в уме или даже на бумаге, не считается плохо разбирающимся в \textit{понятиях} счета, сложения и т.д.. Что же \textit{действительно} требуется, так это способность \textit{распознать} допущенную ошибку --- без этого субъект не может считаться полностью усвоившим рассматриваемые концепции.\footnote{Как я укажу далее, это требование распознавание хотя и применимо к счету и прочим арифметическим операциям, все же не является обязательным для всякой концепции, которую мы пытаемся понять --- даже наоборот: часто смысл термина считается понятным даже тогда когда не понятно, когда некоторые (или даже все) его использования неверны. (Например, я не буду считать кого-то не понимающим концепцию Бога, даже если я не согласен с ним во всем или почти во всем, что он о Боге утверждает.)}

\section{Почему интроспективных ресурсов недостаточно}

Я начну с предложения, которое приписываю Витгенштейну Крипке. Дело в том, что ресурсы для понимания счета имеют лишь два возможных источника. Первый --- интроспективное понимание, то есть осознание чего-то, равносильное пониманию счета. Второй --- склонности к моделям поведения, сознательному или бессознательному. Лишь после признания интроспективного варианта безуспешным, сможем мы обратиться ко второму.

Вспомним смысл вызова скептика: почему субъект всегда считал а не зчитал, что на это указывает? Вполне естественно сначала ответить исходя из первого --- интроспективного --- варианта. Субъект, например, обычно считается \textit{распознающим} нужную для подсчета закономерность потому что видел некоторое количество примеров и теперь, значит, \textit{видит} как действовать аналогично. Но (Крипке (1982, 18)) ведь никакое конечное число примеров не определяет закономерность полностью. И точно так же любая попытка вменить субъекту осведомленность о правиле или алгоритме, совместимом с предыдущими случаями его применения своих способностей к счету и которое, значит, определяет ответы на последующие задачи --- любая такая попытка безуспешна от множественности интерпретаций конечного набора предыдущих задач и корректных на них ответов. Ведь любое такое правило можно понимать по-новому хотя и все еще совместимо с тем, что ''имеет в виду'' субъект, и со всеми прошлыми его подсчетами.\footnote{Крипке ставит проблему смыслового скептика от первого лица, то есть как проблему совместимости моей прошлой практики и нынешнего мышления с ''квожением'' вместо предполагавшегося сложения. Я же поставил проблему эту от третьего лица и касательно счета и ''зчета''. Учитывая, что мы обычно позволяем себе описывать феноменологию третьих лиц (во многом так же, как я это и сделал выше), это, думаю, не вызовет каких-то сложностей. Если же читателя использование такой обыденной практики все же не устраивает, он может легко переформулировать мои тезисы и рассуждения в терминах первого лица --- это не на что существенно не повлияет.}

Из этого теперь видно, как требование бесконечности Крипке исключает естественное описание необходимой модели обоснования фактов в терминах \textit{интроспекции}. Как и отмечали многие философы, проблема не в приобритении способности давать правильные ответы для \textit{бесконечного} числа новых случев --- нет --- нужна способность давать правильные ответы для \textit{новых} случаев вообще. Даже если испытуемый во второй раз сталкивается с тем, что кажется нам точно таким же заданием на счет, он все же может сделать что-то другое, но при этом описать это как ''то же самое''. (Ну ведь и правда --- второе задание, может, он решает уже во вторник, а не в среду, или в полнолуние, а не в новолуние, во всяком случае он \textit{позже} это делает.)\footnote{Крипке (1982, 52, сноска 34) неявно признает такую точку зрения, когда цитирует Витгенштейна: ''Если я знаю это заранее, какой от этого знания мне прок позже? То есть: как мне знать, что делать с этим более ранним знанием, когда шаг уже сделан?'' (Витгенштейн (1956, I, §3))}

Проблема в том, что возможные ментальные состояния, охватывающие понимание правил посредством воплощения их в лингвистических формах --- даже если с использованием кванторов --- или во что-то другое, например, в мысленных образах --- визуальных, кинестетических, или как воспоминаний о ранее выполненных задачах --- эти возможные ментальные состояния, тем не менее, надо \textit{применить} к текущей, новой, задаче. И то, как они применяются, указывает на то, \textit{как они интерпретируются} --- это не что-то заранее фиксированное тем что сейчас можно интроспектировать. Многие думают, что это урок Витгенштейна (1958) из параграфа 139 и близлежащих. Патнэм (1981, 20) излагает его так: ''Феноменологи не видят, что хотя описываемое ими является внутренним \textit{выражением} мысли, \textit{понимание} этого выражения --- понимание собственных мыслей --- это, тем не менее, не \textit{являние}, а \textit{способность}''.\footnote{Райт (1984, 771) же более осторожен, говоря: «Интуитивно, понимание выражения скорее способность, чем склонность».}

Здесь следует выделить несколько моментов. Как широко отмечалось (например, Райт (1989, 109)), для оценки достаточности интроспективных ресурсов для определения соответствия того, что имеет в виду субъект сейчас, тому, что он имел в виду раньше, Крипке идеализирует сознательный доступ к своим прошлым психическим состояниям, как бы давая субъекту знать \textit{все} о своей прежней психической жизни и поведении, так что \textit{воспоминания о своих намерениях} становятся частью интроспективных ресурсов, и тогда, конечно, можно не сомневаться, что субъект именно считать намеревался, а не зчитать. Он, возможно, явно думал о слове ''считать'', например, или само понятие ''счет'' как-то иначе задействовалось в ее мышлении --- через образы или понимание примитивной целесообразности делаемого.\footnote{См., например, Гинсборг (2011).} Почему же подобных воспоминаний может быть недостаточно для ответа скептику, почему нельзя сказать ''Это и есть то, что я \textit{имею в виду сейчас}, ведь это ровно то, что я textit{имел в виду раньше}''?

Крипке (1982, 51) действительно отвергает версию, согласно которой, состояние ''состояние сложения'' является примитивным или \textit{sui generis}, потому что природа состояния \textit{sui generis} остается ''совершенно загадочной''. Из-за этого его ответа (и его обоснования) некоторые философы обвинили Крипке в предвзятости относительно антиредукционистских ответов смысловому скептику.\footnote{Богосян (1989, 527), Фодор (1990, 135-136), Макдауэлл (1984) и другие; дополнительные примеры см. в Гинсборг (2011, 229), сноска 5. Следует добавить, что термин ''проблема следования правилам'' может вводить в заблуждение. Речь ведь о том, что \textit{всякое} в субъекте --- ресурс для продолжения того же, что он делал раньше, и ресурс этот не обязательно буквально ''правило'' или имеющая ''значение'' часть публичного языка. Это важно, потому что, похоже, некоторые комментаторы понимают Крипке как работающего в рамках какого-то из такого рода ограничений.} Гинсборг (2011, 229), например, пишет, что поддерживающие антиредукционизм касательно значения критики возражали, говоря, что ''факт намерения или следования правилу следует определять в чисто натуралистических терминах''.\footnote{Действительно, Гольдфарб (1985 например 476) понимает Крипке как бросающего вызов ''любой предполагаемой физикалистской редукции значения''.} Это, однако, недооценка мысштабы и методы смыслового скептика Крипке, ведь предположение просто таки интроспективного схватывания своих знаний субъектом и такого \textit{схватывания} достаточности для ответа скептику, действительно делает эти интроспективные ресурсы совершенно загадочными. Как, в конце концов, содержимому сознания субъекта (словам публичного языка, словам мысленного языка, ментальным состояниям, образам и чему угодно еще) удается существовать связанно и репрезентировать \textit{именно эту}, а не другую, вещь? Один из даваемых ответов --- некоторые ментальные состояния (или слова, или что-то еще) просто фактически \textit{способны} на это, в частности, есть \textit{внутренние} смысловые состояния, в которых мы иногда и находимся, \textit{отдавая себе в этом отчет}. Увы, но к чему уж интроспекция явно не имеет доступа, так это к механизму устранения \textit{неоднозначности}. Интуитивность описания Крипке смыслового скептицизма сама  по себе показывает, что \textit{все} воспоминания о выполнении определенных задач --- воспоминания, \textit{включающие} использования слов, концепций и образов, различные впечатления касательно \text{намерений} и правильности делаемого --- все они вполне совместимы как со счетом, так и со зчетом. Проблема для субъекта здесь в отсутствии у его интроспективного понимания счета ''содержания, достаточного для исключения всех нежелательных интерпретаций его прежнего понимания'' (Райт (1984, 765)).

А значит обвинение в предвзятости относительно антиредукционизма некорректно. Скорее уж вариант примитивного (нередуцируемого) смыслового состояния используется в качестве ответа смысловому скептику дважды, и оба раза неуспешно: первый раз он безуспешен в рамках исследования доступных субъекту интроспективных источников, второй --- при рассмотрении диспозиционных подходов (это будет обсуждаться в Разделе 2.3).

Прежде чем перейти к диспозициям, дам еще два замечания. Во-первых, некоторые философы обращаются к стандартным нашим способам обобщения конечного числа случаев чтобы применить их к нашему доступу к собственному ментальному содержанию. Например, на основании некоторых наблюдений, мы можем сделать вывод, что все вороны черные. Почему подобное же схватывание общности не может возникнуть от созерцания воспоминаний о предыдущих задачах и их выполнении?\footnote{См., например, Форбс (1983-1984, 234-235).} Крипке противопоставляет этому важный факт касательно нашего опыта намерения делать что-то конкретное вообще и опыта намерения считать в частности: \textit{это не гипотеза}. (Момент этот, кстати, возникнет еще в качестве возражения против диспозиционных подходов.) Мы просто \textit{знаем} свое намерение --- знаем, что имеем в виду --- а не выдвигаем индуктивно или как-то еще касательно этого гипотезы. Второе же наблюдения касается схожего возражения смысловому скептику, выдвигаемого Райтом (1984, 776-777):

\smallskip

\textit{Особенность нашего интуитивного понятия намерения такова, что скептический аргумент бессилен против него. Намерение --- наряду с мыслью, настроением, желанием и ощущением --- таково, что субъект имеет к его содержанию непосредственный, авторитетный доступ, и что содержание его может быть открытым и обобщенным так, что относиться ко всем ситуациям определенного рода.}

\smallskip

Это, несомненно, коректно характеризует ''феноменологию значения''\footnote{См. также Гольдфарб (1985) и Гинсборг (2011).}, но смысловой скептик \textit{успешно} оспаривает такого опыта достоверность лишь задавшись вопросом о в этом интроспективном опыте \textit{конкретном} источнике необходимого объема и общности того, что имеется в виду --- о том, \textit{что}, значит, определяло, что имеется в виду счет, а не зчет.\footnote{Гольдфарб (1985, 474) предполагает, что фрегевский ''непосредственный доступ к сфере чувств'' --- вполне убедительный ответ смысловому скептику. Тейт (1986), возможно, предлагает тот же ответ (здесь я не уверен). Во всяком случае, кажется уместным замечание Рассела о преимуществах воровства перед честным трудом. Предполагается, что нужно указать, что такого \textit{в субъекте}, что позволяет схватить ему именно счет а не зчет. Указание на определенные умственные его способности в качестве такого признака --- это, конечно, не ответ, как и утверждение, что определенные объекты (функции) --- просто вещи, которые субъекты \textit{могут} схватить.} Почему вообще наше якобы непосредственное \textit{ощущение} обобщенности и определенности того, что имеем в виду, пригодно в качестве ответа смысловому скептику? Почему тогда убедительность смыслового скептицизма не является само по себе доказательством возможности признания нами опыта намеревания лишь своеобразным интроспективным \textit{долговым обязательством}, исполнение которого зависит намереваемой и реализуемой функции?

\section{Как три ограничения Крипке блокируют диспозиционные подходы}

Именно здесь берутся за тот или иной диспозиционный подход. (Частично для того, чтобы найти ресурсы (\textit{осознаваемые или нет}) \textit{у} субъекта, которые указали бы на \textit{принужденность} субъекта к определенному намерению --- намерению считать, например.) Начну с простой формулировки этого подхода:

\smallskip

  \textbf{ДИС 1} \quad Понять ''счет'' --- значит быть готовым дать [правильный] ответ на вопрос о количестве предметов в наборе.

\smallskip

Это, впрочем, не работает, потому что совместимо с умением давать правильные ответы в действительности \textit{не} считая. Например, мы (взрослые люди, младенцы и многие животные) не считаем группы из одного, двух или трех предметов, потому что \textit{распознаем} их сразу таковыми.\footnote{Это так называемая ''субитизация'' и, похоже, она ограничена четырьмя предметамв случае взрослых и тремя --- в случае детей и некоторых животных. См. Мандлер и Шебо (1982).} А можно ведь представить и того, кто мог бы так \textit{распознавать} наборы любого размера (по, например, заполнению пространства объектами разных форм и размеров). Такой человек мог бы даже не осознавать --- и даже не быть способным осознать --- возможности линейно упорядочить наборы по количеству предметов в них. Так что следует еще зафиксировать конкретный способ подсчета:

\smallskip

\textbf{ДИС 2} \quad Если некто освоил ''счет'' и его спросили о количестве предметов в наборе, он склонен указать на каждый предмет один за другим, называя числа начиная с ''1'' при указании на первый, число следующее за ''2'' --- при указании на второй, и назвать ответом число, названное при указании на последний.\footnote{ДИС 2 неправомерно выделяет лишь \textit{один} метод. Как уже отмечалось в Разделе 2.1, субъект может овладеть другими подходящими (возможно, упрощенными) методами. Я оставил попытки модифицировать ДИС 2 так, чтобы избежать этой проблемы, потому что модификации эти не повлияли бы на общую диалектическую этой главы траекторию.}

\smallskip

Что ж, давайте посмотрим, насколько ДИС 2 соответствует требованиям Крипке. Начнем с требования бесконечности. И проблема в том, что мы обычно \textit{не} склонны действовать так, как в нем описано --- а он ведь требует такого поведения при любых или почти любых обстоятельствах. Если, например, набор достаточно большой и субъект устал считать, то он уже не сможет или откажется выполнять предписания ДИС 2. И это, похоже, верно в отношении всех людей. Вообще, эта их склонность чувствительна не только к размерам наборов, но и к свойствам составляющих их предметов и к общему состоянию субъекта. Если, например, предметы сильно блестят, или расположены слишком близко друг к другу, или очень маленькие, или ползают друг по другу, то субъект будет склонен вовсе отказаться от такого задания (и бежать в попытках спастись), чем прилежно заняться их подсчетом. То есть, если верить ДИС 2, \textit{никто вообще} не освоил счет.

Как отмечает сам Крипке (1982, 30–32), лучше охарактеризовать эту склонность как то, что человек \textit{сделал бы}, будь, например, его мозг больше, а глаза не столь чувствительны... Впрочем и так модифицированный ДИС 2 не работает: возможно все эти изменения \textit{в нашем строении} заставят нас как раз таки отказаться от правильного выполнения задачи. Мы же не можем гарантировать, что изменения, направленные на, казалось бы, устранение избегания или неправильного выполнения задачи счета, не повлияют на людей каким-то еще другим, мешающим выполнению задач, образом.\footnote{Факты нашей биологии (например, строение мозга), физические законы и т.д. --- все это имеет значение. Популярное рассмотрение этой проблемы с картинками и диаграммами см. в Фокс (2011).}

А что насчет третьего ограничения? Можно подумать, что ошибка ДИС 2 в неучтенности того простого эмпирического факта, что мы допускаем ошибки. Мы ведь обычно не требуем от тех, кто еще только учится считать (да и от самих себя), стабильно верного счета при любых обстоятельствах --- мы по разным причинам делаем исключения, например когда субъект устал, болен или пьян, или предметы такие, что их неудобно считать, и даже допускаем казалось бы \textit{необъяснимые} неудачи при даже привальном выполнении задания. (''Ой, точно, \textit{о чем} я вообще думал?'') То есть несмотря на такие вот отклонения от предписываемого ДИС 2, мы все же обычно готовы признать субъекта умеющим считать. Но почему бы тогда не исключить такие случаи, позволив в них все же считать субъекта умеющим считать?

Потому что это предвосхищает основание против альтернатив. Рассмотрим, например, последовательность* --- нечто, отличающееся от последовательности лишь в небольшом количестве случаев. Точно так же можно описать субъекта как понимающего счет*, исключив \textit{задачи, в которых он склонен отклоняться от счета*}. И что же тогда укажет нам на использование им счета а не счета*?

Это исправление могло быть предложено и ранее --- для удовлетворения требования бесконечности ДИС 2: исключения можно использовать для учета бесконечного количества случаев, в которых субъекты склонны не выполнять задачу.\footnote{Эти модификации (с использованием контрфактуалов и с использованием исключений) на самом деле представляют одну и ту же стратегию, грамматически по-разному реализованную: в первом случае нарушения ДИС 2 упаковываются в контрфактические события, а во втором исключения содержатся в антецедентах изъявительных кондиционалов.} Но это также предвосхищает основание против альтернативных подходов --- ДИС 2* например --- где ''следующее'' заменяется на ''следующее*'' и где различия проявляются только в тех ответах, что субъект не дает.

Проблема ДИС 2 в том, что он пытается определить склонность субъекта при столкновении с задачей счета через \textit{корректные} его процедуры, но --- как мы только что видели --- даже рассмотрение склонностей этих как имеющихся у реальных субъектов и использование исключений для случаев этих склонностей непроявления не помогает. Сторонники понимания испытуемым некой нестандартной формы счета могут использовать этот же ход.

Рассмотрим, наконец, требование обоснования. Здесь проблема в том, что диспозиционная точка зрения предлагает в действительности не используемые субъектами, для обоснования счета в прошлом, ресурсы. Крипке (1982, 23) пишет:

\smallskip

\textit{Должен ли я обосновать свое нынешнее убеждение, что я имел в виду счет, а не зчет с точки зрения гипотезы о прошлых моих склонностях?}

\smallskip

Я понимаю его так: \textit{только лишь} описание склонности как обоснования для \textit{счета} не сочетается с нашим феноменологическим впечатлением касательно понимания другими людьми их схватывания таких концепций, как счет. Только посмотрите, насколько странно это бы звучало:

\smallskip

\textit{Я умею считать, ведь склонен отвечать на вопрос ''Сколько здесь предметов?'' указывая на каждый из них один за другим и произнося ''1'' на первом из них, следующее за ''1'' число --- ''2'' --- на втором и т.д. до последнего, произнесенное на котором число и называю ответом.}

\smallskip

Скажи кто-нибудь так, он скорее подумает о себе как о выучившемся какой-то бессмысленной, нелепой процедуре \textit{вместо}, собственно, счета, или как об описывающем только \textit{побуждения}, но не образ действий, им предпринимаемый согласно выученной концепции.\footnote{Крипке (1982, 17) пишет: ''Мы, обыкновенно, рассматривая математичекое правило вроде сложения, понимаем себя руководствующимися им в каждом новом случае''. И (Крипке (1982, 10)) ''Я следую указаниям ...''.} Мы заподозрим это, потому что мелочное это его обоснование вообще-то ничего не говорит о том, почему же описание его склонностей \textit{имеет хоть какое-то отношение} к умению считать.

Ранее я уже упоминал, что мы часто доказываем свое умение демонстрацией соответствующего поведения, вроде ''Думаешь, я не умею считать? Ну, смотри! Раз, два, ...''. Но это же не значит, что счет в такой демонстрации и \textit{заключается}. Скорее, демонстрирующий предполагает знание собеседником счета и \textit{это свое предположение} и свое знание счета ему и показывает через демонстрацию своих навыков и этих навыков осознание.

Обратите внимание, что (согласно моей интерпретации) акцент Крипке на подрыве диспозиционных подходов требованием обоснования заключается не в этого обоснования ''нормативности'' в противовес диспозициям --- нет --- диспозиционные подходы попросту не соответствуют пониманию своей обоснованности субъектами, нарушают очевидные об \textit{опыте} такого себя осознания факты.

\section{Попытка защиты диспозиционных подходов}

Многие комментаторы Крипке защищают диспозиционный подход аналогией между использованием контрфактуалов для описания поведения стекла, соли или газов в воображаемых обстоятельствах и таковым их использованием для описания действий субъекта,\footnote{Блэкберн (1984, 289-290), Форбс (1983-1984, 229-230), Фодор (1990, 94-95). Среди одобряющих этот ответ, например, Богосян (1989), Гинсборг (2011). Райт (1984, 771), напротив, категорически отвергает его, указывая на интуитивно большую схожесть понимания со способностью, нежели со склонностью.} предполагая включенность в первые как исключений (''\textit{ceteris paribus}''), так и идеализаций, которые, согласно Крипке, бессильны подкрепить ответ смысловому скептику, хотя и, сталкиваясь с требованиями бесконечности, предположительно диспозиционно их преодолевают. В бесчисленном множестве разных обстоятельств структурные факты, определяющие диспозиции стекла и соли, определяют также и бесчисленное множество исключений: ''соль \textit{не} растворяется в холодной соленой воде'', ''соль \textit{не} растворяется в воде, если Бог вмешивается и регулирует \textit{соответвенно} движение молекул'' и т.д..

Но есть существенное различие между стеклом, солью и пр. и считающим субъектом.\footnote{В этом и следующих пяти абзацах приводится защита неприятия Крипке диспозиционных подходов, им самим, и, насколько могу судить, кем либо другим, однако, не данная.} Различие это в использовании лежащих в основе идеализированных описаний стекла и пр. структурных фактов для определения этих описаний ложными, а не истинными. Дело ведь не в том, что соль всегда расстворяется или что газы ведут себя подобно скоплениям маленьких твердых шариков --- дело в \textit{склонности} соли вести себя именно так, \textit{как она себя и ведет} исходя из своих структурных свойств. То есть \textit{строго говоря}, соль \textit{не} ''водорастворима'', а лишь растворяется в воде в широком \textit{таких-то} диапазоне обстоятельств (но не растворяется в \textit{других}), однако первые настолько особенные, что ярлык ''водорастворимая'' все же оказывается полезен. Ну и аналогично газы --- в широком диапазоне обстоятельств --- ведут себя \textit{примерно} как скопления маленьких твердых шариков.

Другое дело --- счет. Изучая нейрофизиологию способностей к счету, \textit{можно} обнаружить, что испытуемые склонны делать нечто отличное от просто счета --- хотя и \textit{нельзя} обнаружить, что они не понимают, что такое счет. Эмпирически, конечно, допускается обнаружение верности характиристики человеческих склонностей как функции зчета; но ведь это \textit{не} значило бы, что испытуемые зчитают, а не считают --- не было бы доказательством, что они не умеет считать, а умеют зчитать, скорее лишь указало бы на слонность совершать определенные систематические ошибки при счете.

Не буду, впрочем, давать прогнозов на основе мысленного эксперимента, а лишь опишу текущие исследования касательно математические способностей и что они предполагают об наших математических склонностях \textit{прояснить}. Первое, что следует отметить --- открытия касательно математических склонностей \textit{на самом деле} показывают их отклонение от представляемого нами правильными способами решения. Философы предположили, что нужные для ответа скептику Крипке диспозиции можно ''стратифицировать'', отделив первичные, нужные для правильного счета --- от вторичных, правильному счету мешающих.\footnote{См., например, Блэкберн (1984, 290) и Форбс (1983-1984). Гольдфарб (1985, 477) пишет: ''Редукционист может, например, заявить, что в будущем физиологическая психология раскроет два различных на науных основаниях механизма. Состояния первого уровня будут соответствовать языковой компетентности и, будучи несдерживаемы, приводят к лишь правильным ответам, состояния же второго уровня соответствуют признакам мешающим и так объясняющим ошибки''. Но, хотя генерирующие \textit{правильные} функции бесконечные диспозиции и \textit{возможны нейрофизиологически} --- касательно арифметических способностей это было исключено еще к началу 1990-х годов.} Пусть это различие и в чем-то соответствует нашему опыту решения математических задач --- мы действительно совершаем ошибки, которые, как \textit{иногда} затем признаем, вызванны были мешающими факторами --- это не свойство задействующихся в нашей арифметической компетенции предрасположенностей.\footnote{Вот некоторая литература по этой теме: Кэри (2009), Деэн (1997). Ее, конечно, гораздо больше --- это ведь очень активная сейчас область исследований, потому что ''диспозиции'' крайне сложны и запутаны и о них не получается говорить как о лишь механически следуемых нами (покуда ничто не мешает и не вызывает грубые ошибки) воплощенных арифметических правилах.} Интересно, что наш интуитивный числовой ряд подчиняется закону Вебера: числа расположены неравномерно, группируясь по мере возрастания, также у нас есть нечеткое, аналоговое ''чувство'' количества объектов в группе.

Определяй наши диспозиции наше понимание чисел, у нас, возможно, было бы и более одной используемой при счете системы чисел: одна --- нечеткая, но открытая, другая --- строго конечная, а третья --- подчиняющаяся закону Вебера. И сторонник диспозиционного подхода должен, конечно, объяснить, как этот ошеломляюще сложный клубок нейрофизиологичесвких способностей обеспечивает такое простое, стратифицированное понимание стандартных арифметический функций вкупе с условиями \textit{ceteris paribus}, исключающими поведенческие отклонения от этих функций.

Это ведет нас еще к одному разрушительному для эмирически обнаруживаемых диспозиций моменту. Даже если описанные сложности с ними не были обнаружены текущими исследованиями, наше признание эмпирической возможности отклонения субличностных арифметических способностей, от необходимых для корректного решения соответствующих задач, указывает на отличие нашего понимания усвоенности счета от \textit{только лишь} склонности к выполнению задач на него. Именно \textit{в этом смысле} следует проводить различие между нашими склонностями и тем, что мы \textit{должны} делать --- \textit{правильными} действиями.

Это, однако, поднимает следующую проблему: учитывая исчерпаемость ресурсов (к пониманию счета и прочих математических концепций) субъекта диспозициями, остается вовсе неясным определяющее субъекта как намеревающегося считать. Иначе говоря, что фиксирует смысл --- содержание --- его концепции \textit{счета}? Что, в конце концов, \textit{могло бы} помочь нам здесь, если не его диспозиции?\footnote{Фиксация содержания --- проблема, по мнению Богосяна (1989), Крипке и поднимает, и именно ее непосредственно поднимает Райт (1980) (см. примечание 4). Но соображения, им при этом используемые, отличаются от приведенных мной: он (2003, 496) пишет ''Раньше я недооценивал силу [требования бесконечности]''.}

Здесь уместны два замечания. Во-первых --- проблема фиксации содержания возникает здесь в диалектическом поиске в субъекте \textit{чего-то} определяющего усвоение им в прошлом счета, а не зчета, и, значит, счета им сейчас и намеревание. Мы могли бы, конечно, оспорить его намерение, исходя из несостоятельности диспозиционных подходов, но это потребует дополнительных предположений, которые, думаю, Крипке не делает. Аргумент против диспозиционных подходов смысловой скептик сопроваждает аргументами против интроспективных ресурсов.

Во-вторых --- ''нормативность'' арифметических законов или нашего таких законов понимания. Проблема диспозиционных подходов, в конечном итоге, в отличии склонности реагировать определенным образом в определенных ситуациях от должности так реагировать (Гинзборг (2011, 228)). Будет лучше --- во всяком случае, не столь запутывающе --- сказать, что это все значит, а именно: лишь негативная характеристика описания диспозиций как недостаточного для каких-либо утверждений касательно его умения считать. Слова ''правильно '' и ''должен'' \textit{лишь} вбивают клин между описанием возможного поведения субъекта и реальной функцией (например, функцией последовательности), рассматриваемой через ее этим субъектом понимание --- и ''редукция'' не удается, вот и \textit{все}. Аналогично, сказать ''правильность --- нормативный вопрос'' (Богосян (2003, 36)) --- значит не что иное, как отличие фактической функции от той, которую субъект \textit{склонен} выполнять.

Заманчиво, наверное, предположить неизменность или исчезновение различия Хомского между знанием и умением вместе с различием первичным и вторичным в диспозициях --- но это не так. Обе стороны --- знание и умение --- эмпирически чувствительны: ''эффективность'' --- не просто свалка всякого неграмматического поведения субъекта. Синтаксис языка субъекта можно пересматривать на основании эмпирических результатов, и, кроме того, неумение тоже эмпирически ограниченно и проверяемо. Нарушение памяти, например, нельзя сходу считать причиной ошибок в синтаксисе --- факт этого нарушения открыт эмпирической проверке. Вторая причина различия --- отсутствие в случае синтаксиса ключевого арифметического фактора --- правильности. Как я уже указывал, универсальные диспозиции к выполнению счета с помощью необычных (конечных, нечетких и т.д.) систем счисления --- показатель не не-счета, а лишь отличия нарушенного синтаксиса от предполагаемого лингвистом.

\section{Почему прямое социологическое решение не работает}

Как мы увидели, диапазон индивидуальных диспозиций очень мал и они не соответствуют ограничениям Крипке. Иначе говоря, они не дают психологические/индивидуальные обосновывающие факты для каждой возможной задачи счета (где обосновывающие факты --- достаточные для определения фактов соответствия вроде ''\textit{Каждая совокупность объектов имеет одно и только одно число, называемое количеством}''). Потому и возникает соблазн дополнить диспозиции контрфактуалами. Это, конечно, очень естественный ответ на проблему, ведь мы часто оправдываемся, указывая на усталость, которая и помешала выполнить задачу --- мы так абстрагируем характеристики нашего понимания счета и способности к нему от прочих факторов, ограничивающих наши к счету возможности (почти так же действия машины Тьюринга абстрагированы от физических ограничений). Именно поэтому кажется, что ограничение бесконечности удовлетворимо дополнением диспозиций контрфактуалами.

Однако, стратегия эта не удовлетворяет ограничению обоснования. Действительно: сам факт сомнения в способности таких дополненных диспозиций выполнить требуемую от них работу указывает, что диспозиции --- \textit{какие бы они не были и как бы не дополнялись} --- не являются \textit{стандартами} правильного счета. Будь они таковыми, не было бы и речи об их правильности --- \textit{какой бы ответ они не давали}. Но как бы слонности субъекта к счету не выглядели, кажется возможным сомневаться в их ответов правильности и более того, дают ли наши диспозиции такие ответы. Было бы странно услышать от субъекта после подсчета им что-то вроде ''Видите ли, это правильный ответ, потому что именно его я был расположен дать''.\footnote{Это чрезвычайно важный аспект нашего понимания следования правилам, и обусловлен он глубокими фактами нашей психологии, например тем, как мы учитываем (или не учитываем) субличностные способности касательно сознательно решаемых задач. Это интересное взаимодействие между личностным и субличностным, впрочем, не имеет значения для первоначальных моих моделях приватного следования правилам (см. Главу 3), однако включается в пятой модели в Главе 5.}

Это побуждает искать стандарты правильного счета где-то еще. Более того, современная научная практика предполагает незначительность склонности \textit{людей} к правильному счету --- на них в этом редко полагаются, в большинстве своем используя теории и инструменты (буду называть их ''калькуляторы''). Итак, похоже, учитывая широкую распространенность калькуляторов, единственно значимой для нас диспозицией становится диспозиция \textit{принимать} результаты, полученные с эти калькуляторов помощью. Исторически, использование счетов, счетных досок, веревок с узлами и прочего для счета --- указывает, что значимость лишь этой диспозиции касательно счета не нова. Наша способность считать точно и приблизительно не опирается на что-либо разумно характеризуемое как диспозиция к счету --- мы же не рассматриваем в качестве этой диспозиции таковую к принятию результатов, выдаваемых [изобретенными другими людьми] инструментами.

Обсуждая в Разделе 1 требование обоснования, я подчеркивал наше чувство оправданности в выборе методов счета, и это справедливо даже касательно [уверенного в себе] ребенка --- он будет озадачен заявлением, что все им делаемое как счет --- на самом деле зчет. Крипке весьма успешно использует этот феноменологический факт против диспозиционной теории: наша вера в обоснованность корректности нами делаемого не удовлетворяется одним лишь признанием диспозиций делать это определенным образом. Факт этот (в сочетании с некоторыми другими) может означать необходимость оценивать правильность поведения согласно социальным стандартам.

Учтите ведь, что испытуемый обыкновенно не слишком уверен в своих способностях к счету. Но даже уверенный в своих способностях может в них усомниться в определенных обстоятельствах.\footnote{Здесь часто приводится неуверенная реакция в некоторой ситуации --- например, в классе, где были даны [очень простые] инструкции ''поднимайте руку, когда учительница держит зеленую карточку, и опускайте --- когда она держит красную'', и где однако учительница и большинство учеников согласованно эти инструкции нарушают.} Это потому, что ключевым для понимания любой задачи является ''способность'' на ошибки и возможность инструментов или других субъектов на эти ошибки указать, и потому, что масштабы этих ошибок не имеют предела. Поэтому субъект может разувериться в \textit{чем угодно}, ранее, казалось бы, несомненном. А вот если бы стандартом поведения были \textit{лишь} его склонности, такая неуверенность была бессмысленна. И \textit{это}, очевидно, показывает, что рассматриваемые стандарты --- внешние к кому-либо конкретному и локализованы в тех, кто его окружает --- в сообществе.

Можно предположить, что, упоминаемые Крипке как часть феноменологии и как обычно плохо сказывающиеся на диспозиционном ответе, ''нормативные'' факты, а именно (как указано в сноске 26) --- ''Обычно, рассматривая математическое правило --- например, сложение --- мы думаем о себе как о руководимых им в каждом новом случае'' и ''Я следую указаниям'' --- недвусмысленно указывают на наше такое относительно следуемых правил впечатление как вызванное чувством их усвоенности \textit{посредством научения другими}. То есть очевидной частью феноменологии является уверенность в наученности другими как источник уверенности в теперь согласно научениям делании --- поэтому так легко поставить эксперименты, подрывающие эту уверенность: ''Все остальные со мной не согласны, может это потому что я все же не усвоил то, что, как казалось, давно знал? Или может я вовсе психически болен?''.

Нахождение стандартов в сообществе (в его, точнее, диспозициях) может также объяснить результат мысленного эксперимента, герой которого осознает субличностный источник своих способностей к счету. Он внезапно начинает видеть ярко-красные цифры на каждом предмете, цифры эти связаны одна с другой по порядку, и последнюю его что-то \textit{побуждает} произнести ответом.\footnote{Что это за побуждение? Ну, подобное побуждению почесаться или поесть, например --- этому можно сопротивляться, но ограниченно.} Естественная реакция здесь --- страх, что эти напрашивающиеся ответы на самом деле \textit{не} верны --- из-за очевидной оторванности побуждения от нашего понимания своей способности считать: мы научились этому у других, и с какой стати будем теперь полагаться на это внезапное побуждение как нашей наученности соответствующее?

Эти соображения подсказывают прямое \textit{социологическое} решение парадокса: если и существует некоторый паттерн обосновывающих фактов, который указывает на использование именно счета в противовес зчету, то паттерн этот обнаруживается в коллективных установках сообщества. Крипке (1982, 111) затрагивает это решение и отмечает, что теория такая ''будет открыта по крайней мере некоторым из тех же критических замечаний, что и исходная (индивидуально-диспозиционная)''.

Почему же прямое социологическое решение \textit{точно так же} не удовлетворяет трем ограничениям Крипке, как и индивидуальное? \textit{Ограничение бесконечности}: диспозиция сообщества --- как бы она не определялась в терминах совокупности индивидуальных диспозиций и общественно-доступных инструментов --- все еще ограничена в своем диапазоне и подвержена отклонениям от правильного счета (и это верно даже для современного общества с его мощными вычислительными инструментами). И контрфактуальное дополнение диспозиций сообщества сталкивается с теми же препятствиями, что и в индивидуальном подходе. \textit{Ограничение бесконечности}: описание диспозиций сообщества в тот или иной момент времени не указывает на этого общества оправданность используемого им способа счета, не доказывает, то есть, что в очередной задаче используемая в силу коллективных диспозиций концепция счета --- та же, что и в предыдущих. Мы ведь \textit{не} думаем --- даже концептуализируя задачи счета как выполняемые коллективно, --- что наш способ счета по определению должен быть таким, как если бы ощественные практики были понимались такими же, как раньше. Наконец, \textit{ограничение ошибки}: коллективное решение не дает логического пространства для возможности ошибок в расчетах, причем не только отдельных субъектов, но и всего сообщества. К тому же, решение через коллективные диспозиции предвосхищает основание против альтернатив так же, как и решение через индивидуальные.

\section{Условия не истинности, но утверждаемости}

Давайте тогда обратимся к скептическому решению Витгенштейна Крипке. Оно (Крипке (1982, 66)) признает, что ''отрицательные утверждения скептика неопровержимы'', то есть признает отсутствие, подходящей для ответа на вопрос о счете/зчете, модели обосновывающих фактов. Согласно Крипке, решение Витгенштейна состоит в отказе от ''условий истинности'' в пользу предложений о понимании субъектом операторов нумерации, а также ''условий истинности'' в целом. Т.к. модели обосновывающих фактов нет, то нет ничего дающего необходимые и достаточные условия истинности предложений о понимании субъектом этих [нумерационных] предложений. Но вместо них можно использовать "условия утверждаемости": условия, при которых субъект имеет \textit{право} --- в сообществе --- утверждать что он (или кто-то другой) понимает счет, в их понятие, значит, заложена принадлежность субъекта признающему наличие таких условий обществу.\footnote{В этом разделе я пытаюсь изложить позицию Крипке (1982, 74-93) касательно условий утверждаемости. Дуглас Паттерсон (24.09.2009 --- электронная почта) указал, что, похоже, общая замена условий истинности условиями утверждаемости приводит к наличию этих условий для в ином случае бессмысленных предложений (''в обществе я имею шуметь, например''). Он далее предполагает здесь непоследовательность, пока в разрешениях нет незаконной апелляции к условиям истинности.

Но я, хотя в конечном счете и отрицаю необходимость заменить условия истинности условиями утверждаемости, думаю, это неверно. Конечно, люди имеют право использовать или не использовать предложения без необходимых и достаточных условий применения. Здесь есть два варианта: либо (1) форма предоставления таких прав является ''смыслом'', либо (2) в силу определенных условий, согласно которым "значения" должны соответствовать (например, они должны быть необходимыми и достаточными условиями применения), предложения эти не имеют "значений". Но мне кажется, что в оба варианта не приводят к бессвязности построенной на ''правах'' практике утверждения или этой практике нужде в дополнительных условиях --- условиях истинности.

Замена условий истинности условиями утверждаемости (как Крипке (1982, 86) трактует Витгенштейна) не исключает обычного использования слов ''истина'' и ''ложь''. Именно потому все еще можно использовать вместо условий утверждаемости --- условия истинности, если понимать их ''дефляционно'', а не как требующие соответствия фактам, что я и указываю в Главе 6. Иначе говоря, я предполагаю, что Крипке (а, возможно, и Витгенштейн) встроил в понятие ''условий истинности'' соответствие фактам (Крипке (1982, 72): ''Декларативное предложение получает свое значение в силу своих \textit{условий истинности} --- в силу соответствия фактам, без которых оно быть таковым не может''). Но я отрицаю необходимость такого понимания ''условий истинности''.}

Крипке предполагает, что условия утверждаемости --- условия права А утверждать, что он использует сложение --- заключаются, грубо говоря, в его уверенности в своей способности отвечать правильно на последующие вопросы, тогда как его ответы могут подлежать исправлению другими. И что он также имеет право, (Крипке 1982, 90) ''опять же, с учетом исправлений со стороны других'', оценивать чей-то ответ как правильный, если бы его он дал и сам. Отрицание понимание другим счета оправдано только если ответы его отличаются от ответов отрицающего, а отрицание собственного понимание оправдано при неуверенности в ответах или способах их нахождения.

Как выше указано, ''сообщество'' связано с условиями утверждаемости двояко. Во-первых, отсылка к нему встречается в формулировки самих этих условий: ''Мы призваны признать временную природу наших диспозиций --- признать законное право \textit{других} нас поправлять''. Во-вторых, условия утверждаемости имеют смысл только в контексте общества, ведь они полагаются на понятие права в этом обществе что-то утверждать или отрицать. Опасения по поводу правомерности своих утверждений могут испытывать от общества изолированные --- называемые мной \textit{Робинзонами Крузо}.

Согласно интерпретации Витгенштейна Крипке (1982, 86-88), непоследовательность изолированного такого Робинзона, следующего правилам именно в его невозможности функционировать вне условий утверждаемости. То есть если он берется считать, то действует ''не колеблясь, но \textit{вслепую}'' (Крипке (1982, 87)). Скептический парадокс указывает на, при ограничении такими Робинзонами, отсутствие ''каких-либо условий истинности или фактов, могущих доказать согласованность настоящих намерений с прошлыми'' (Крипке (1982, 89)). То есть для Робинзона нет разницы между уверенностью в следовании правилу и действительном ему следовании, и поэтому рушится идея правильного или неправильного следования ему. Как пишет Витгенштейн (1958, параграф 202):

\smallskip

\textit{Думать, будто подчиняешься правилу --- не значит ему подчиняться, и значит нельзя подчиняться ''в частном порядке''; иначе бы для подчинения достаточно было одной только в нем уверенности.}

\smallskip

Крипке (1982, 110), впрочем, подчеркивает, что из этого \textit{не следует} невозможность описать Робинзона как правильно или неправильно соблюдающего правила: для этого достаточно чтобы описывающий относился к Робинзону этому как к члену своего общества.

Заметьте: неспособность обосновать второе и особенно третье требования Крипке с использованием индивидуальных диспозиций прямо противоречит возможности такого описания. И Крипке, кстати, сам указывает на включение Робинзона в свое сообщество как на необходимый, для оценки его как считающего, шаг --- только так мы можем применить к нему условия утверждаемости.\footnote{Проводоится различие между одиночными языками --- языками с единственным носителем у каждого, и языками приватными (''языками ощущений'') --- о которых пишет Витгенштейн (Гольдфарб (1985)). Блэкберн, Гольдфарб и Богосян считают скептическое решение Крипке не исключающим существование одиночных языков, и здесь я с ними не согласен.}

\textit{Заключительные замечания} \quad Мастерское изложение Крипке создает у большинства уверенность в скептическом его подходе \textit{оправданности} логическими контурами разбираемой проблемы. Три ограничения возникают из простых наблюдений за нашей практикой следования правилам и ограничения эти совершенно явно исключают [прямые] диспозиционные решения (что и было показано в этой главе). Факты соответствия, в свою очередь, подрываются не метафизически --- напрямую подрывая способность наших утверждений что-либо правильно или неправильно описать --- но из-за недостатка у нас ресурсов на обеспечение такого этих утверждений соответствия. Ведь, говоря о счете, речь не об отсутствии функции следования одного за другим --- речь об отсутствии необходимых обосновывающих фактов, что подрывает наши способы мышления и разговора \textit{об} этой функции. И, получается, фактов соответствия тоже нет: ментальные элементы субъекта или его языка не могут соответствовать тому, чему мы дотеоретически считаем их соответствующими, потому что субъект, значит, неспособен подразумевать соответствие своих утверждений или мыслей этому. (Это очень важная тема, она еще будет возникать в следующих главах, особенно в Главе 4, где основное внимание будет направлено на льюизианское отрицание основанности фактов соответствия только лишь на ресурсах субъекта.)

Отсутствие соответствия (говоря в рамках естественного понимания свойств идиомы истинности) тогда подрывает анализ всяких связанных с утверждениями практик --- его придется заменить анализом условий утверждаемости. Наконец, естественные психологические факты о нашей способности устанавливать для себя стандарты (от нас самих требующие определять наше им соответствие) заставляют использовать стандарты общественные для убедительности следования правилам.

Диалектика здесь тоже очень широка по своему охвату. Она выражена в терминах арифметических функций, где наиболее естественно говорить о ''следовании правилу''. Но, как отмечает Крипке, она в равной степени убедительна как аргумент в пользу совпадения настоящего намерения и прошлого.

В последующих главах этой книги я покажу, как такие выводы отклонить.

\chapter{Две версии Робинзона Крузо}

\qquad

\textbf{Аннотация} \quad В этой главе я начинаю анализ проблемы следования правилам, используя диспозиционные языки --- в которых термины применяются именно так, как субъекты применять их склонны. Я покажу, что эмпирические обстоятельства (и склонности испытуемых), будучи достаточно благоприятны, позволяют изолированным испытуемым следовать правилам. То есть испытуемые эти могут успешно оценивать и использовать для успешной навигации в мире такие языки. Диспозиции их, впрочем, в некотором роде искусственны. Но примеры, описанные в последующих главах, будут гораздо более естественными --- будут гораздо больше походить на нас.

\qquad

\section{Сообщество идиолектов}

Один из выводов, делаемых в этой книге --- возможность приватных моделей следования правилам. Я вывожу ее поэтапно, используя несколько разных Робинзонов, каждый из которых может участвовать в разных видах такого правилам следования. Прежде чем начать эту серию мысленных экспериментов, я в этом разделе кратко представлю воображаемое сообщество, которое --- в отличие от \textit{нашего} --- понимает значения своих слов в терминах своих к их использованию склонностей. Я рассмотрю некоторые детали, преводящие к подобной языковой практике провалу и покажу способы ей все же --- как не удивительно --- существовать.

\textit{Диспозиционно-смысловые языки} \quad Представьте сообщество, в котором каждый действительно намеревается использовать всякое слово --- ''таблица'', ''счет'', ''сложить'', ''сова'' и т.д. --- как значащее ровно то, к чему он склонен в данный момент его применить. Например, под словом ''собака'' он имеет в виду то, что у него возникает желание назвать собакой, под ''столом'' --- то, что склонен называть столом и т.д.. Таким образом, один и тот же предмет в одних обстоятельствах может быть собакой, а в других --- кошкой, а большинстве случаев он вообще ничем не является. (Экстенсии терминов в этих языках --- не наборы объектов, а наборы пар объект-конекст, где контексты индивидуализированы по времени, положению в пространстве и другим стандартным факторам.)

Может показаться, что такой язык вовлекается в фатальную цикличность, ведь, например, значение слова ''собака'' задается формулировкой содержащей само это слово ''собака''. Да, замкнутость возникла \textit{бы}, понимай люди в таком обществе формулировки эти как \textit{инструкции} к использованию слов, но ведь они их так \textit{не понимают}. И диспозиции --- даже в ходе обучения приобретенные --- автоматически \textit{диктуют} им ответы, и потому формулировка значения слова ''собака'' независимо характеризует его значение как ''применимое к этим предметам тогда, когда есть побуждение его так применить''. Назовем такие языки ''диспозиционно-смысловыми'', а слова в них --- имеющими ''диспозиционный смысл''.

Требования Крипке для этих языков либо теряют силу, либо тривиально выполняются. Требование обоснования тривиально выполняется: используемые в данный момент термины языка чувствительны только к побуждениям в данный момент говорящих. (Что говорящие имеют в виду тогда-то и там-то, --- это что они чувствуют необходимость иметь в виду тогда-то и там-то.). У них нет намерений (хотя они и могут делать прогнозы) касательно того, что они будут иметь в виду в будущем, и они не понимают себя как обусловленных тем, что имели в виду в прошлом. Итак, на вопрос скептика ''Откуда вы знаете, что то, что имели в виду раньше, соответствует тому, что имеете в виду сейчас?'' они отвечают

\smallskip

\textit{Я помню, что под словом ''кошка'' подразумевалось именно то, что мне тогда хотелось подразумевать, и я помню, к чему это слово применял, так что если вы спросили, имею ли я в виду в данных момент, что чувствую необходимость иметь в виду в данный момент так же, как я делал это и раньше, то мой ответ, конечно, да, ведь я прекрасно помню, что тогда происходило. Если же вопрос ваш подразумевал не это, то я не знаю, что еще он мог подразумевать.}

\smallskip

То есть скептический вызов не может даже быть сформулирован, потому что обращение к чьим-то прошлым предрасположенностям \textit{действительно оправдывает}, что он имел в виду тогда, и что сейчас он действует согласно сейчас им подразумеваемому --- странная речь из Части 2.3 вовсе не странная в \textit{таком} обществе. Другой способ указать на провал смыслового скептика перед носителями этих языков: хотя и есть обобщение того, что они имеют в виду --- \textit{они имеют в виду то, что иметь в виду побуждаемы} --- обобщение это не часть намерения говорящего, когда он что-то имеет в виду, то есть они не \textit{намерены} всегда иметь в виду то, что им хочется иметь в виду, они просто всегда так делают.

Требование ошибки также теряет силу, ведь что применения субъектом слова всегда привальное, независимо от его характера или поведения --- нельзя поступать ''плохо''. Наконец, никто в таком обществе ни в коем случае не намерен ''складывать'' или ''умножать'', или еще какие-нибудь слова применять подобно \textit{нам}: каждый намеревается ''сложить'' или ''умножить'' только в смысле соответствия своих диспозиций этим действиям в этот момент, так что и понятие ''счет'' применимо только если диспозиции его применить позволяют.

Возможным недостатком такого диспозиционно-смыслового ''языка'' является отсутствие общих слов у не имеющих \textit{совершенно} одинаковые в совершенно одинаковых контекстах, диспозиции: если у них есть хоть какие-нибудь отклонения в диспозициях относительно многих слов, то результатом будет скорее семья идиолектов, чем что-то вроде ''общественного языка''. Более того, они признают это и признают невозможность каких-либо переводов с одного идиолекта на другой (по крайней мере подавляющего большинства слов). Такая какофония идиолектов может показаться безнадежной для общения!

На самом деле такая модель не будет бесполезной для большей части общения, по крайней мере если идиолекты отличаются незначительно --- сбои тогда будут возникать лишь изредка. Вообще успешность общения между такими субъектами зависит от величины отклонений среди их идиолектов. И даже если почти все их слова содержат существенные межличностные отклонения, успешное общение все еще возможно --- если есть переводы.

Вот пример: пусть В --- член такого общества --- имеет в своем словарном запасе слово ''диспозиция'', и пусть слово это \textit{точно} применимо к диспозициям С и к его собственным. Тогда В может осознать совершенную определенность своего понимания слов --- например ''стол'', ''апельсин'', ''бежать'', ''умножать'' --- его диспозициями к таких слов применению. Аналогичное он может осознать касательно С. Затем В осознает, что не может напрямую перевести идиолект С на свой: ''стол'', например, как его С использует, не имеет тех же условий применения, что ''стол'' В --- их диспозиции ведь различаются. И все же В может перевести ''стол'' С так: ''стол'', как его использует С, означает на идиолекте В ''''стол'', как его склонен использовать С''. То есть если слово ''диспозиция'' применимо к диспозициям других людей --- если, значит, они разделяют это слово и условия применения его соответствуют диспозициям каждого из них --- то взаимный перевод слов в таком сообществе возможен.

Конечно, можно представить подобное сообщество с отклонениями касательно \textit{каждого} из слов, то есть и слова ''диспозиция'' тоже. И пусть даже способ использования ''диспозиция'' В таков, что ''''стол'' как С склонен использовать это слово'' \textit{не} имеет тех же условий использования, что ''стол'' С (потому что ''диспозиция'' В не соответствует ''диспозиция'' С). Осознав это, В увидит, что он и С говорят на разных идиолектах, которые к тому же не поддаются взаимному переводу. Смогут ли они, несмотря на невозможность перевода, общаться --- зависит от степени различия условий применения используемых ими слов.

Итак, невозможность перевода идиолектов и, значит, несостоятельность таких языков --- во всяком случае, для общения --- не следует так сразу из определямости значений слов через диспозиции их использующих.\footnote{Я благодарю Дугласа Паттерсона (электронная почта 24.09.2009) за предупреждения меня об этих проблемах при обсуждении с ним изложенной модели.}

Тем не менее, есть поводы для беспокойства касательно ценности таких языков для выживания даже в эмпирически удачных обстоятельствах. Пусть, например, диспозиции одинаковы у всех людей в нем, могут ли тогда они \textit{последовательно} применять к объектам эти свои слова? И почему их диспозиции не могут заставлять их называть одно и то же то столом, то --- в следующий же момент --- как-то иначе? Ведь \textit{тогда} даже будучи разделяемы всеми в обществе, диспозиции эти могут привести к катастрофическим результатам. Способность таких идиолектов обеспечивать общение, \textit{помогающее пережить} их носителям взаимодействие с вещами в мире --- способность эта зависит исключительно от эмпирических фактов о диспозициях каждого из них и в том числе от этих диспозиций (а значит, и применений слов) стабильности.

Пусть единообразные такие склонности в обществе крайне ''осторожны'': распространена, например, склонность не применять слово ''стол'' до тех пор, пока целевой объект не будет внимательно изучен. Такие диспозиции могут до некоторой степени стабилизировать применение слов в разных контекстах, в том числе --- слов, касающихся вычислений --- ''сложение'', ''вычисление'' и т.д.. Пусть члены общества склонны применять слово-число как результат вычислений, но только когда вычисления эти были дважды проверены определенными способами (значит, например, торопливо вычислять они не склонны --- они даже не назвали бы такое вычислением). Если при этом каждый имеет диспозицию давать одинаковые ответы после таких вычислений завершения --- пусть даже \textit{одинаково} ошибочные --- то идиолекты остаются согласованы, применение слов происходит стабильно.

Что ж, предварительный вывод таков: \textit{практическая реализуемость} диспозиционно-смысловых языков зависит исключительно от эмпирических условий. Например, от того, какими именно диспозициями обладают субъекты в таком обществе, и --- что особенно важно --- как диспозиции эти приводят к единообразиям среди них. Конечно, я еще не рассматривал, требуется ли такой языковой практике определенное единообразие вещей в самом мире, которое могло бы как-то соответствовать единообразию в носителях языка диспозициях. И еще, я не ответил на ожидаемый от некоторых философов вызов касательно в действительности не соответствия \textit{концепций} носителей таких языков и предметов реального мира. Ведь у них, казалось бы, есть лишь склонность более или менее скоординированно испускать случайные \textit{шумы}. Первый из этих вопросов будет рассмотрен при рассмотрении других видов Робинзонов, второй же будет рассмотрен уже в следующем разделе.

\section{Робинзоны в эмпирически благоприятных обстоятельствах}

Представьте, что наши Робинзоны питаются в основном кокосами. Добыча кокосов, впрочем, трудна: приходится избегать различных крупных хищников. Чтобы выжить, Робинзоны научились считать кокосы: сколько их есть сейчас, сколько будет съедено за день и, значит, сколько дней до следующего рискованного поиска.\footnote{Я предполагаю, что Крузо-1 может изобретать слова, причем даже для сложных понятий --- например, для количеств предметов --- и делает это на основе опыта и врожденных диспозиций. Сомнительно, конечно, что, даже будучи щедро одарены природой полезными диспозициями, так изолированные люди на это способны. Однако сейчас не оспариваются эмпирические предположения по этому поводу. Если способ Крипке выводить возражения против приватных языков на основе парадокса верен --- особенно важны здесь второе и третье требования --- то не важно, насколько изощренными диспозициями наши Робинзоны одарены: парадокс всяко остается.} Он также изучил территорию, где добывает пищу, определил, где больше всего встречается кокосов, и, значит, куда ему за ними лучше лезть. В общем, навыки счета оказались жизненно ему необходимы.

В описание этого сценария включен один неоспоримый факт: наборы предметов --- кокосов в данном случае --- имеют определенный числовой размер. Так что, вообще то, не удивительно, что при их подсчете --- и при подсчете, значит, дней до следующей охоты --- важно \textit{правильное} понимание Робинзоном-1 чисел, ведь ему нужно сосчитать сколько ''действительно'' кокосов на дереве.\footnote{То есть нужно соответствие один-к-одному между применяемыми числами и кокосами, к которыми он их применяет, и соответствие это должно сохранять отношение порядка, чтобы количества можно было сравнивать.} И применение Крузо-1 прочих его слов: ''кокос'', ''тигр'' и т.д. должно оцениваться на адекватность так же. Кокосы --- еда для него, он --- еда для тигров. Его слово ''кокос'', значит, должно выделять определенный набор предметов, являющихся для него пищей, ''тигр'' --- набор предметов, которых лучше избегать.

Такое понимание ''правильности'' отлично от ранее описанного в книге, особенно в Главе 2 --- там оно применялось к имеемому в виду отдельными субъектами и этого соответствию имеемому в виду ранее. Важно, что для Крузо-1 ''правильность'' значит соответствие его ответа тому, например, что находится на дереве, чтобы действия из результатов его счета следующие позволили ему собрать нужное количество кокосов.

Думаю, следует признать, что Крузо-1 действительно \textit{понимает} придуманные им слова: ''кокос'', ''тигр'', ''1'', ''2'' и т.д.. Это потому что он \textit{придал} им диспозиционные значения: они значат то, к чему его склоняет (в любое время и в любом месте) применять их. Но, учитывая указанное только что понимание ''правильности'', можно все же вопрошать: почему придание словам диспозиционных значений не приводит к полной катастрофе? Пусть наш Крузо-1 находится в ситуации, где (1) его диспозиции к слов применениям стабильны и (2) вещи кажутся ему такими, какие они в самом деле есть. Например, если он думает, что видит кокос, то он \textit{действительно} видит кокос. Некоторые могут счесть такое изложение тенденциозным. В конце концов, что такое \textit{для Крузо-1} кокос и что тогда значит стабильность его диспозиций?\footnote{Я описывал Крузо-1 и его мир с точки зрения нашего языка, так что фразы вроде ''оценить условия Крузо-1 на адекватность'' подразумевают оценки в наших терминах. Требуемое для этого будет рассмотрено дальше, особенно в Разделе 3.3 и в Главах 4 и 7, а пока давайте смотреть на это как на предварительных способ говорить об изолированном таком Крузо и о проблемах, с ним связанных, хоть и способ этот в дальнейшем будет пересмотрен.}

Но позвольте поставить вопрос по-другому. \textit{Мы} живем в мире, где тени иногда выглядят как кокосы, а кокосы иногда похожи на свернувшихся котов. Иногда даже требуется тщательная проверка, действительно ли выглядящие одинаково вещи одинаковы. То есть почти все имеет свойства, которые в сочетании с ограничениями наших чувств приводят к множеству ошибок касательно действительной вещей природы. Из-за чувств Крузо-1, из-за небольшого количества существующих в его мире видов вещей и ограничений этих вещей возможных различий --- из-за этого все в его мире так аккуратно подпадает под тот или иной \textit{узнаваемый им вид}. Он может даже распознавать и различать объекты с первого же взгляда, он \textit{не может} ощутить, что одно и то же в разное время кажется ему разным. Он не сможет, например, понять, что то, что выглядело как тигры, кокосы или белки, теперь выглядит совсем иначе. И он, наконец, не способен допускать вычислительных ошибок: если в результате счета он \textit{уверен}, что на дереве 6 кокосов, то это потому что на нем \textit{действительно} 6 кокосов.

В таком замечательном мире с такими завидными эпистемическими способностями Крузо-1 может наделять свои слова смыслом диспозиций и быть вполне успешным. Он может, например, понять, что ''7'' значит количество предметов, о котором он склонен думать, что их 7. Он может иметь в виду ''кокос'' применительно именно к тому, к чему он склонен применять это слово. Поскольку его идиолект --- диспозиционно-смысловой, три требования Крипке либо теряют силу, либо тривиально выполнимы. И, кстати, раз требование бесконечности теряет силу, то некоторые не будут думать о Крузо-1 как о \textit{считающем} в том же смысле, что и мы. Хотя я и допускаю, что он никогда не совершает ошибок, я не думаю, что он может считать сколь угодно большие наборы предметов. Т.к. есть верхний предел размера доступного для подсчета набора, есть только конечное число числовых слов.\footnote{Поскольку его числа конечны, я не буду говорить о нем как о владеющем нашей их концепцией. Впрочем, не так уж очевидно, что это стоит отрицать --- числовые понятия то у него имеются. Я сейчас обсужу другой аспект этой проблемы.}


Крузо-1 все же имеет вразумительную языковую практику --- вразумительный идиолект --- в которой слова значат ровно то, что он склонен ими называть, и благодаря очень хорошим обстоятельствам этот идиолект полезен для него --- даже незаменим. И это так даже несмотря на невозможность неправильного применения им слов.

Кто-то, возможно, укажет на \textit{невозможность} когерентной языковой практики без возможности так ошибиться. (И то же скажет, значит, касательно общества идиолектических носителей, рассмотренном в Разделе 3.1.) Он будет также отрицать соответствие слов Крузо-1 понятиям, обладающим которыми можно было бы его считать: понятием количества, кокоса и т.д.. Крузо-1, каким я его здесь представил --- утверждал бы этот оппонент --- не имеет [реальных] \textit{концепций}, ведь его их применение не предполагает ими его \textit{руководства}. Он, вместо этого, признает, что имеет лишь потребность вербально \textit{реагировать} на вещи в мире. (Грубо говоря, у него есть только полезный набор речевых \textit{тиков}, но не язык.)

Однако три соображения подтверждают лишь то, что его концепции не совсем те же, что у нас, а не что у него вовсе их нет. Во-первых, есть сопровождающее использование им своих слов ментальное содержание. Хотя его понимание своих же слов подразумевает диспозиционное их значение, это не мешает им соответствовать наборам вещей (в конкретных обстоятельствах), наборам похожим и наборам имеющим определенные количественные свойства. Действительно, такое психичесвкое содержание иногда составляет его диспозиции (например, он склонен относиться к таким-то предметам как к одинаковым вне зависимости от своего их \textit{восприятия}). Во-вторых, и \textit{наши} некоторые концепции, подобно таковым его, имеют схоже диспозиционное значение. Рассмотрим, например, понятие \textit{боли}. Это, как известно, концепция, касательно применения которой ошибаться невозможно (по крайней мере, от первого лица).\footnote{Широко распространено мнение, что Витгенштейн всячески бросал вызов таким концепциям. Но указываемые им проблемы, похоже, исходят непосредственно из соображений соблюдения правил, и потому сомнительно настаивать на них здесь. (Здесь я, кажется, согласен с Крипке (1982, 3), когда тот пишет: ''''аргумент приватного языка'' применительно к ощущениям --- лишь частный случай гораздо более общих соображений о языке, ранее обсуждавшихся ...''. См. также его сноску 47, 60-61 и текст, которому она соответствует.) Я предлагаю в дальнейшем использовать другие способы --- уже предлагавшиеся другими философами --- для подрыва этой очевидности касательно использования понятия боли от первого лица, их я обсуждаю в Разделе 5, описывая также, как они действуют в диспозиционно-смысловых языках.} Последнее, и, возможно, самое важное: ничто не мешает Крузо-1 использовать свои концепции так же, \textit{как мы}: описывать вещи в мире, рассуждать о них и принимать на основе этих рассуждений решения о действиях. Как и мы, он может решить, например, залезть на одно дерево, а не на другое (потому что, скажем, на другом меньше кокосов), или он может бросить вызов группе опасных хищников поскольку их в ней только два. Поразительные эти факты о его способностях соответствуют аналогичным и о наших: те наши концепции, к которым ''правильность'' или ''неправильность'' не применима --- ''боль'', например --- кажутся нам все же концепциями, которые можно использовать с различаемыми по правильности для описания мира, рассуждения и принятия решений.

И последнее наблюдение: было бы ошибкой думать, будто Крузо-1 в таких обстоятельствах считает себя всеведущим. У него, конечно, нет понятия ошибки: он не понимает, как можно думать о не-кокосе --- что это кокос. И все же способен понять невежественность: это когда он еще не сосчитал кокосы на дереве --- и не знает, сколько их, или когда не знает, есть ли тигр за следующим поворотом.

Здесь есть тонкость: следует различать два случая. Первый: Крузо-1 сталкивается с набором предметов --- с кучей песка, например --- которые у нет желания считать. Он тогда не склонен применить какое-либо из счетных слов, но и не расположен отрицать их применения возможность: для его идиолекта нет ответа на вопрос, сколько здесь песчинок. Он буквально не знает ответа --- ответа нет. Но есть и другой случай: когда он чувствует, что захоти он --- мог применить числовое слово --- и склонен отрицать, например, что на дереве кокосов 1, 2 или 3. Он знает, что их больше трех, не без подсчета --- без задействования, значит, своей диспозиции для указания их количества --- он не знает, сколько их.\footnote{Сложности возникают, если попытаться это различие заострить: потребуется различение диспозиций Крузо-1 применять слово от других диспозиций, с которыми у первых возможен конфликт. Впрочем, для моей аргументации здесь это и не важно --- не важно, обоснованно различие такое или нет --- так что воздержусь от дальнейшего обсуждения.}

В удачных обстоятельствах наш первый Робинзон удачлив именно благодаря простоте мира, а также идеальному соответствию имеющемуся в нем своих чувств и установок: он ведь не нуждается даже в понятии ''правильного применения'' слов. В этом, конечно, отличие его идиолекта он нашего с вами языка, но это, тем не менее, язык --- и язык несомненно полезный. Таким образом, описанный мысленный эксперимент уже сузил масштаб витгентшейновского аргумента Крипке против приватного следования правилам. Оставляя в стороне, значит, небольшое злоупотребление термином приватности касательно этого случая, получаем неожиданный результат: при определенных эмпирических обстоятельствах все же существуют полезные приватные языки: диспозиции и мир следователя приватным правилам таковы, что нет разницы между его таким следованием и лишь убежденностью о нем.

\section{Робинзон с двумя наборами сознательных диспозиций}

Давайте теперь представим второго, немного более реалистичного Робинзона --- ''Крузо-2'', в том же самом мире --- мире совершенно различимых объектов, делящихся на узнаваемые виды. Здесь, значит, как и прежде, есть продукты питания (кокосы, креветки и т.д.) и хищники (тигры, кабаны и др.), а органы чувств Крузо-2 --- при оптимальной работе --- прекрасно отличают что угодно от чего угодно.

''Оптимальная работа'' в абзаце выше призвана указать на определенную нестабильность: иногда наш Робинзон видит вещи так, как они есть, а иногда --- иначе. Например, когда свет падает иначе, или он устал, или поспешил с принятием решеня, или слишком много выпил --- тогда ему может хотеться назвать \textit{юкосом} то, что обычно хочется --- кокосом, или тенью то, что обычно хочется назвать тигром.

Первый, грубый, способ показать разницу между первым и вторым Крузо --- указать на полезность второму учиться распознавать, когда он может доверять своим диспозициям. Крипке (от имени Витгенштейна), впрочем, отрицает такую для него возможность. Он (1982, 112, сноска 88) пишет:

\smallskip

\textit{в отсутствие скептического парадокса могло бы показаться, что человек помнит свои ''намерения'' и может одно такое воспоминание о них использовать для исправления воспоминания другого --- парадокс указывает наивность таких идей, на их бессмысленность. В конечном счете, ведь, у субъекта могут быть противоречивые грубые наклонности, и исход будет зависеть только от его воли.}

\smallskip

И далее (1982, 112, сноска 88):

\smallskip

\textit{Ситуация тут не такая же, как если бы отдельные субъекты имели разные и независимые воли и, при принятии в сообщество их нового субъекта, считали, что могут положиться на его реакции --- ничего такого нет между субъект и им же самим.}

\smallskip

Здесь я не согласен. Рассмотрим два возможных случая. Первый --- не столь реалистичный, в нем Крузо-2 может различать два внутренних своих диспозиционных состояния: первое вызывает один набор диспозиций, другое --- другой (назовем для простоты первое ''отдохнувшим'', а второе --- ''усталым''). Во втором случае, уже более реалистичном, побуждения Крузо-2 меняются в зависимости от времени и места, но он все же часто может классифицировать их и распознать, в каком диспозиционном состоянии находится сейчас, произошел ли свдвиг в его диспозициях. Второй случай я рассмотрю в Разделе 5.2.

В первом же случае у Крузо-2 есть выбор, какой из \textit{двух} возможных диспозиционно-смысловых языков\footnote{Альтернативный подход --- индивидуализация диспозиционных языков так, чтобы Крузо-2 всегда лишь на одном таком говорил. От этого, впрочем, ничего здесь не зависит: достаточно бы было лишь по-другому расставить пункты в дальнейшем рассуждении.} воспринимать всерьез. То есть, хотя он \textit{говорит} всегда на том языке, на котором говорить склонен --- несмотря на это, он все же способен решить, что следует определенному из двух таких языков доверять. Он признает, что различные диспозиционные состояния заставляют его придавать словам разные значения, и что один набор диспозиций лучше, чем другой --- один ''достоин доверия'', другой же --- нет. Пусть, например, когда он отдохнул, его диспозиции соответствуют миру так же, как диспозиции Крузо-1. Но поведение его меняется от усталости. Крузо-2, однако, замечает такое довольно легко: когда он устал, его слово ''кокос'' имеет специфичное для усталого состояния значение. Такие ''уставшие'' кокосы иногда уже не подходят ему и могут даже навредить, и в отдохнувшем состоянии он видит, что некоторые из них --- \textit{юкосы}.

Способ Крузо-2 осознавать полезность доверия ''отдохнувшему'' своему состоянию и недоверия, значит, ''уставшему'' --- его способность обнаруживать явные и неявные \textit{грубые закономерности}.\footnote{Грубые закономерности я рассматривал еще в Аззуни (2000) и в Аззуни (2010b).} Например, ''кокосы сытные, а от юкосов только тошнит'' --- две очевидные грубые закономерности. Если ему в отдохнувшем состоянии доступны слова ''есть'', ''кокосы'', ''юкосы'', ''питает'' и ''болеть'', то он может сформулировать на нем такие грубые закономерности. Есть эти слова у него и в уставшем состоянии --- некоторые юкосы он и в нем способен от кокосов отличить. Но в уставшем состоянии есть другая грубая закономерность: ''некоторые кокосы сытные, а от некоторых только тошнит''. (Она сохраняется, если Крузо-2 также сохраняется, если он видит --- наоборот --- некоторые кокосы как юкосы.) С другой стороны, усталость и отдых не влияют на осознание им своего состояния --- болен он или здоров --- а от юкосов он как раз заболевает.

Даже если у него слов ''сытый'' и ''больной'', он все равно способен распознать им соответствующее, просто нужные грубые закономерности теперь будут \textit{неявными}. У него, значит, если \textit{концепции} сытости и болезни даже вне нужных для них слов, и используя эти понятия, он также уловит соответствующие грубые закономерности --- которые пусть и не сможет выразить себе же самому.

Заметьте, что он так признает, что соответствующий отдохнувшему состоянию диспозиционный язык более надежен, чем соотвутствующий уставшему: ведь схыватываемые им грубые закономерности точнее --- успешней предсказывают результаты его действий. Такие закономерности --- лучшие в отдохнувшем состоянии, и худшие в уставшем --- можно сформулировать с использованием его концепции чисел. Крузо-2 может осознать, что перед собиранием кокосов --- когда у него они закончились --- ему надо выспаться, чтобы на следующий день точнее находить именно их. (Короче говоря, это тоже знание --- неявное --- признаваемое даже будучи явно в словах выразимым.) Он не хочет ведь насобирать юкосов и потом выбрасывать их или --- еще хуже --- съесть и заболеть.

\section{Что для Крузо-2 "лучшее соответствие миру"?}

В прошлом разделе я предлагал представить, что диспозиции Крузо-2, когда он отдохнул, соответствуют миру так же, как таковые Крузо-1.

Разговоры о миросоответствии --- метафоричны, хотя, конечно, некоторые философы предпочитают относиться к ним серьезно. Можно ли сказать, что к признанию ''отдохнувшего'' языка более полезным Крузо-2 приходит через лучшее его терминов соответствие миру?

Это именно практический вопрос о способе Крузо-2 распознавать, какой из языков лучше. Примерно, конечно, понятно, как ему это удается: один набор диспозиций доставляет ему неприятности, другой же --- нет. Может даже возникнуть соблазн сформулировать так: он \textit{думал} о чем-то как о кокосе, но когда его съел и заболел, то понял --- это был юкос! Это, конечно, неправильно. Давайте использовать обычные кавычки для указания на принадлежность слова к нашему языку, r-кавычки --- для указания на принадлежность к ''отдохнувшему'' языку и t-кавычки --- на принадлежность к ''уставшему'' языку. Нельзя сказать, будто он думал он предмете как о ''кокосе'', но затем обнаружил, что был неправ. Он, однако \textit{не} обнаружил, что предмет является \textsuperscript{t}кокосом\textsuperscript{t}, потому что это был \textsuperscript{t}кокос\textsuperscript{t}. Мы, однако, \textit{можем} сказать, что он обнаружил, что это был не \textsuperscript{r}кокос\textsuperscript{r}, но не в том смысле, что он \textit{думал}, будто это \textsuperscript{r}кокос\textsuperscript{r}. Он ведь не пользовался ''отдохнувшим'' языком, произнося соответствующие всем этим ''кокосовым'' словам звуки, он произносил \textsuperscript{t}кокос\textsuperscript{t} --- корректно его, значит, применяя. (И это верно даже когда он заболел --- главное, чтобы не был уставшим.)

Однако, проблема его может выражена на любом доступном ему языке. Судя по тому, как он говорит на своем ''усталом'' языке, проблема эта в том, что от некоторых \textsuperscript{t}кокосов\textsuperscript{t} он заболевает, а от некоторых --- нет. Но это ведь не значит, что они не \textsuperscript{t}кокосы\textsuperscript{t}. И, с другой стороны, он заболевает от любого \textsuperscript{r}юкоса\textsuperscript{r}.

Здесь \textit{нам}, возможно, хочется сказать: ''отдохнувший'' язык \textit{на самом деле} описывает кокосы и юкосы такими, \textit{какие они есть}. Ведь слово \textsuperscript{r}кокос\textsuperscript{r} для Крузо-2 соответствует своей экстенсией \textit{нашему} ''кокос'', а наше слово всегда указывает верно. Но что не так с экстенсией \textsuperscript{t}кокос\textsuperscript{t}? \textit{Мы} думаем, что некоторые \textsuperscript{t}кокосы\textsuperscript{t} --- кокосы, но некоторые \textsuperscript{t}кокосы\textsuperscript{t} --- юкосы.\footnote{Возможно, пример следует немного разъяснить. Некоторые кокосовые пальмы заражены паразитической лозой, растущей внутри и на них и имеющей свои орехи. Крузо-2 в отдохнувшем состоянии может отличать эти орехи от принадлежащих пальме. Мы, конечно, понимаем, что юкосовое дерево --- это и не дерево даже, но Крузо-2 \textit{этого} не знает.} Но что с того? Слова в каждом диспозиционно-смысловом языке Крузо-2 имеют разные области применения --- они варьируются в относительно разных наборов, и в \textit{этом} нет ничего неправильного.

\textit{Нам} может хотеться сказать в ответ: множество наборов предметов, категоризированных на ''отдохнувшем'' языке Крузо-2, лучше согласуется с реальными между ними сходствами, а множество наборов, классифицированных на ''уставшем'' --- хуже. И добавить: ''отдохнувший'' язык \textit{сопоставляет} свои слова с действительно существующими естественными видами, а ''уставший'' --- нет. То есть что множества кокосов и юкосов никак не вложены друг в друга.

Здесь, однако, две проблемы. Первая: что дает нам право такое заявлять? Ее мы рассмотрим далее --- особенно в Главе 4. Вторая: Крузо-2 в любом случае не имеет доступа к таким соображениям. У него есть \textit{две} фразы: \textsuperscript{t}настоящее\textsuperscript{t} сходство и \textsuperscript{r}настоящее\textsuperscript{r} сходство. То есть каждый из его языков по своему реальные сходства характеризует. У Крузо-2 есть и две фразы, соответствующие нашей ''естественные виды''. С каких же слов помощью должен Крузо-2 объяснять идею ''более точного'' отражения мира словами на одном языке, чем словами на другом.

Обратите внимание: это не значит (по крайней мере, \textit{пока что}) нашу невозможность понять идею превосходства одного языка над другим в смысле соответствия миру. Мы ведь это и делаем, сравнивая экстенсии его слов с \textit{нашими}. Вопрос вот в чем: как \textit{Крузо-2} должен наделить для себя это различие содержанием, выходящим за рамки превосходства одного языка над другим в смысле ориентации в мире через информацию от грубых закономерностей?

Давайте пока отложим вопрос о нашем праве говорить о соответствии слов Крузо-2 миру через воображение значений за пределами обоих диспозиционно-смысловых языков --- значений, свойственных языку Бога --- языку, каждое слово которого, благодаря Его могуществу, относится к набору, содержащему только совершенно естественных видов представителей, и фраза ''естественный вид'' в котором относится только к множеству таких наборов. (Не спрашивайте, как Бог заставляет этот язык так делать --- он же Бог, и среди его сверхъестественных способностей есть и такая.) Используя свой божественный язык, он может описать ситуацию Крузо-2 так: ''Есть действительно похожие вещи, которые, значит, действительно принадлежат к одним и тем же естественным видам''. И действительные сходства вещей \textit{влияют} на Крузо-2. Один из языков Крузо-2 --- ''отдохнувший'' --- схватывает эти действительные сходства, а другой --- нет, поэтому, выбирая ''отдохнувший'' язык, Крузо-2 может ориентироваться в мире успешней. Ведь в первом случае он классифицирует вещи согласно действительному их сходству, а во втором ему это не удается.

Такой ''божественный'' способ смотреть на вещи, как уже отмечалось, недоступен \textit{Крузо-2}. И все же, его выбор ''отдохнувшего'' языка вместо ''уставшего'' рационален, потому что подразумевает распознавание лучшего в отношении способности поддерживать свое существование в мире. Крузо-2 видит, что добьется большего успеха с одним языком, и меньшего --- с другим. Он, впрочем, не может явно уловить эту разницу, как это может Бог, ведь что бы он имел в виду, говоря о ''соответствии''?

Обратите внимание: проблема Крузо-2 в том, что наличие способности понимать превосходство одного языка над другим касательно достигаемого успеха в мире --- не необходимо и достаточно для лучшего ''соответствия'' языка миру. У него есть еще, конечно, способность воображать один из языков всячески превосходней всех прочих в отношении успешности его носителя в мире, но и это не необходимо и достаточно для ''соответствующего'' миру языка. Думаю, Крузо-2 неспособен понять, что значит ''соответствие'' языка миру. (Я также предполагаю, что эта проблема --- проблема понимания Крузо-2 соответствия совершенного языка Бога миру --- глубже, чем кажется. Возможно, и \textit{мы} это не понимаем --- вопрос этот я рассмотрю далее в Разделах 7.2 и 7.3.)

Почему же Крузо-2 не может понять ''соответствие'' миру? Его языки --- диспозиционно-смысловые, и понятия, значит --- тоже. Поэтому понимание им всякой концепции следует описывать в смысле его склонности к концепции этой применению --- это и есть то, что упускается в предположительном его взгляде на превосходящесть одним языком другого касательно их соответствия миру. Он ведь вовсе не так их сравнивает, он сравнивает с точки зрения успеха в мире --- это не то же самое, что ''соответствие'' ему.

Он может лишь официально принять один из языков и отрицать, что диспозиции ''уставшего'' языка не определяют значения его слов. Мы можем подумать, что он будет делать это на ''отдохнувшем'' языке: ''Я устал, поэтому \textit{думал}, что несу кокос, но нет, я \textit{ошибался} --- это был юкос''.

Но это не верно, потому что оба его языка --- диспозиционно-смысловые. Он не использует те термины, которые не склонен использовать, поэтому, когда устал --- и говорит на ''уставшем'' языке --- его отрицание определения ''уставшими'' диспозициями, слов на ''уставшем'' языке --- \textit{бессмысленно}. Поэтому он не скажет ''Я устал, и потому подумал, что несу кокос'', имея в виду ''отдохнувший'' кокос. Ведь когда он устал, он вообще не использует этот термин, так что как же он может ''думать'', что несет \textsuperscript{r}кокос\textsuperscript{r}, когда устал?

И это так, даже если признать его знание, к чему относятся термины на другом языке. Он может \textit{сказать} ''Если бы я отдохнул, я бы подумал, что это \textsuperscript{r}юкос\textsuperscript{r}'' и быть правым, но правота эта не делает правильной формулировку ''Я думал, что у меня есть \textsuperscript{r}кокос\textsuperscript{r}''.

Поскольку Крузо-2 знает, что для него существует два диспозиционно-смысловых языка, с одним из которых он достигает большего успеха, чем с другим, он может описать свою ситуацию металингвистически, сказав: ''Я использовал менее полезное \textit{слово} \textsuperscript{t}кокос\textsuperscript{t}'', потому что устал. Он при этом все еще считает себя совершившим ошибку, но ошибка эта --- в использованном слове, и не касается \textit{мира}. Или --- что то же самое --- он может сказать ''У меня были неправильные диспозиции, когда я произносил это слово, потому что я устал'': здесь он думает о себе --- что он устал и потому употребил неправильные диспозиции. В первом же случае он указывал на два разных слова --- соответствующие разным диспозициям --- неправильное из которых он употребил. Но для него разница между этими случаями чисто терминологическая.

Еще раз: Крузо-2 может предпочесть один язык другому только потому, что тот позволяет ему лучше ориентироваться в мире, а не потому, что он --- как Бог --- видит, что один из них ''соответствует'' миру лучше. Относительная успешность одного языка перед другим, однако, объективный факт --- но не тот, что \textit{он} мог бы выразить, говоря о ''соответствии'' очерчивания языками этими мира так же как мир очерчен сам собой.

Мы вернулись к понятию \textit{соответствия}. Некоторые философы здесь могут думать, что Крузо-2 способен понять \textit{соответствие} как \textit{объяснение} превосходства одного языка над другим: более совершенный язык лучше соответствует миру, и \textit{потому} лучше подходит для в нем навигации. Крузо-2 --- сказали бы они --- занимается выводом наилучшего обяснения своего успеха: успех его должен обусловлен быть тем \textit{фактом}, что термины в нем лучше отражают то, каков мир. Его успех, должно быть, обусловлен совпадением способов его терминов делить мир с тем, как мир действительно разделен своими стыками. Он, конечно, прямо не видит, что это так --- делает вывод только на основании своего успеха.

Однако, все еще есть старая проблема: что \textit{имеет в виду} Крузо-2, говоря нечто вроде ''подходит лучше'', ''делит мир по стыкам''? (Как он может \textit{понять} эти метафоры?) Как им под ними подразумеваемое выходит за рамки лучшего в мире ориентирования? Метафоры ведь эти \textit{не могут} значить ''при ''правильном'' описании мира наилучший язык описывает мир так же'', потому что понятие ''правильного описания'' мира Крузо-2 может быть только описанием, ведущим к успеху в этом мире. Так что же они значат?

Вот, например, карты --- некоторые из них, мы можем думать, ''совершенно'' описывают местность, которую должны описать. Но как это может значит что-то кроме ''использование этих карт исключает ошибки''? Впрочем, мы, похоже, думаем, что как-то да выходит --- мы же можем представить лист бумаги с линиями на нем, а так же местность, этим линиям совершенно соответствующую. (Линии на карте изоморфны контурам местности \textit{с учетом} масштаба.) Однако \textit{образ} такой сталкивается с той же проблемой --- он по прежнему лишь описание местности ''без языка и понятий'', абстрагированное от ''визуальных'' наших диспозиций, отчего и думаем, будто карта соответствует этой якобы бессловесной (неконцептуальной) характеристике местности. Так что, похоже, проблема Крузо-2 --- и наша проблема тоже.\footnote{Множество философов на протяжении долгих лет утверждали --- вопреки, например, корреспондентной теории истины, --- что слова нельзя сравнить с реальностью. Не всегда, впрочем, очевидно, что они, говоря такое, имеют в виду. Но некоторые, похоже, имеют в виду здесь разъясненное.}

Пусть Крузо-2 на самом деле зчитает (а не считает), тогда \textit{Бог бы сказал} ''Крузо-2 поступает плохо --- относится к некоторым разным по количеству наборам предметов как к по количеству одинаковым''.\footnote{Похоже, Бог скажет так лишь если не ''переведет'' термины Крузо-2 на свои собственные --- по той же причине, по которой отдохнувший Крузо-2 не может сказать, что, когда устал, то \textit{думал}, что несет \textsuperscript{r}кокос\textsuperscript{r}, и почему мы не можем сказать, что он \textit{думал}, будто у него есть кокос, когда у него был юкос. Бог же \textit{может} сказать, что ''уставший'' язык Крузо-2 заставляет его относиться к кокосам --- как к юкосам, а к различным количествам --- как к одинаковым, не описывая при этом его \textit{мысли}. Бог также может сказать, что слова Крузо-2 не выявляют реальный различий и сходств --- тех, что влияют на его (Крузо-2) благополучие --- и именно потому он плохо поступает. Тема эта также поднимается в секциях 3.5 и 6.4} С другой стороны, пусть на острове есть еще кто-то (\textit{Пятница}, например), и пусть они одинаково отдохнувшие/уставшие. Тогда, если Пятница (мудро) решит вести переговоры с Крузо-2 только в отдохнувшем состоянии, но Крузо-2 (по глупости) ведет переговоры с Пятницей, будучи уставшим, то Пятница однозначно будет успешней Крузо-2. (Потому что он, например, поймет, что может использовать Крузо-2 как \textit{денежный насос}.) Пусть Пятница описывает это, будучи отдохнувшим, тогда он скажет, что раз Крузо-2 \textit{думает}, что у него 5 кокосов, то он готов обменять их на 5 сардин в пятницу, когда на самом деле у Крузо-2 не 5, а 57 кокосов.\footnote{Здесь снова возникает проблема из сноски 12: Пятница ведь может сказать это, ''переведя'' термины ''уставшего'' языка Крузо-2 на свои, но это будет неверное описание мышления Крузо-2.}

Достоверность таких заявлений Пятницы лишь в том, что он успешно пользуется ими в отношении Крузо-2 --- у него по прежнему нет для них Божественных оснований (хотя, скажем, его язык и язык Бога допускают дословный перевод в смысле соответствия применения слов Пятницей --- таковому Бога). Бог знает, что знает, потому что способен заставить свои слова соответствовать миру метафизически --- возможно, кстати, потому, что сотворил мир таким вот соответствующим своим словам. Пятница же не обладает такими сверхспособностями, поэтому может знать лишь то, что у него дела идут намного лучше, чем у Крузо-2.

Можно, конечно, подумать, что вступительное описание Крузо-1 и его мира в Секции 3.2 --- его описание \textit{с точки зрения Бога}. Но наверное \textit{нам} лучше не пытаться приписывать себе такое в сколь-нибудь большей степени, чем Пятнице --- лучше сказать, что мы Крузо-1 с ''пятничной'' точки зрения описываем. И тогда идеальное соответствие диспозиций Крузо-1 его окружению значит ''Будь мы на одном острове с Крузо-1, мы не смогли бы превратить его в ''денежный насос'' --- не смогли бы его эксплуатировать''.

Но вернемся к изолированному Крузо-2, оценивающему два своих языка. Повторим: тот очевидный факт, что он успешней с одним из них, чем с другим, и позволяет ему их оценивать. Сделав выбор, он может выразить свое признание превосходства выбранного языка --- как мы видели ранее --- либо критикуя свои диспозиции, либо критикую язык. Но это не предполагает (и предполагать вовсе не может), что он знает сам факт метафизического соответствия --- что термины в ''лучшем'' языке очерчивают реальность точнее, чем в другом.

И тем не менее, при подходящих обстоятельствах Крузо-2 может и осознать уместность своего доверия \textit{некоторым} диспозициям и недоверия --- другим. Для этого ему нужно, конечно, распознать (некоторые из) своих состояний. \textit{Значит} это его предпочтение \textit{не} произвольно, ведь он опирается на объективные факты об успешности различных используемых им языков. ''Успех'' здесь, значит, включает способность Крузо-2 оценить свое благополучие, к изучению чего и перейду я в следующем разделе.

\section{Интроспекция}

Я предположил, что Крузо-2 различает свои диспозиции на основе успеха при их использовании. Я также предполагаю, что он может оценивать свое благополучие после событий, которые на него влияют, и даже осознать, как эти события влияют на него. Для этого не обязательно предполагать его в этом вопросе объективность или что он всегда прав --- достаточно предположить, что он достаточно хорошо распознает, когда, например, от съеденного ему стало плохо, или когда животное стало опасным, или когда причинило ему вред, и т.д.. Допустив эти возможности, мы можем говорить о нем, как о способном учиться на своем опыте.

Сразу возникает проблема: разные Крузо могут иметь разные ''системы ценностей''. Описанные в этой книге Крузо --- существа с простыми, понятными ценностями: они хотят ''человеческого комфорта'', выжить, избежать боли и прочих неудобств, им также нужна еда, и желательно та, от которой не тошнит. Но возможно ведь другие Крузо: презирающие земные удобства, желающие быть съеденными, принимающие боль и дискомфорт и смерть. А понятие успеха ведь с конкретными ценностями и соотносится.\footnote{Я благодарю Митча Грина (устное общение --- 08.08.2009) за то, что он поднял этот вопрос и побудил меня к написанию следующих абзацев.} Рассмотрим еще раз случай, когда Крузо говорит на двух языках: ''уставшем'' и ''отдохнувшем''. Несомненно, следующий за этим его успех касательно ценных для него ''самоубийственных'' ценностей будет достигнут принятием решений в уставшем состоянии.


Очевидно невозможно отрицать связь успеха Крузо с тем, чего он хочет. То есть ''успех'' должен быть, хотя бы частично, соотнесен с целями и желаниями конкретного рассматриваемого Крузо.\footnote{''Частично'' --- потому что есть некоторые сложности: иногда мы учитываем при оценке успеха других оценку и их ценности, считая, например, неудачником достигшего ценностей, которые считаем на самом деле не ценными вовсе. Сложности эти, впрочем, не относятся прямо к понятию ''успех'' как оно здесь используется.} По этому поводу, однако, следует сделать два замечания. Во-первых, одна из целей книги --- представить несколько видов Крузо с широким диапазоном диспозиций в различных обстоятельствах, в которых они могут следовать приватным правилам. Я намерен в конечном счете, в Главе 5 укоренить полученную убежденность в траектории изменения приводящих к объективному успеху диспозиций Крузо. Конечно, мы \textit{не} можем приписать такому Крузо \textit{какой угодно} набор диспозиций и ожидать, что он будет успешно следовать правилам, поэтому я в этой книге придерживаюсь простых случаев: тех, где Крузо хочет выжить, нуждается в комфорте и т.д., а ''нездоровых'' Крузо отложу на другую работу.\footnote{Под словом ''нездоровый'' я не имею в виду, что у таких Крузо есть ценности, которые следовало бы игнорировать или исключать, и я не утверждаю, будто Крузо эти не могут следовать частным правилам несмотря на свои такие ценности --- многие, очевидно, могут. Речь лишь о рассмотрении случаев, не слишком усложняющих обсуждение --- чтобы легче продемонстрировать возможность следования приватным правилам в самых разных обстоятельствах.}

Во-вторых, хотя ''успех'' должен быть соотнесен с ценностями Крузо, он должен быть и объективным --- объективным фактом того, что Крузо становится лучше (согласно его ценностям), когда, например, он принимает такое-то количество кокосов, а не другое и т.д., объективный факт лучшей среди прочих группировки предметов в своем мире.

Этот ответ, однако, поднимает еще две проблемы. Во-первых, такие объективные факты, кажется, должны быть выражены на Божественном языке, и в то же время --- укоренены в пропозициональных установках Крузо (чего \textit{он} хочет и чего не хочет).\footnote{Своим вниманием к этой проблеме я обязан Стивену Шифферу (устное общение --- 08.08.2009).} Но любое подобное описание пропозициональных установок Крузо должно быть --- как мне кажется --- выражено на том или ином языке. Пскольку он ясно формулирует, что он хочет съесть кокос и избежать тигра, то отношение такое должно быть выразимо на всяком языке, которы он использует, и, конечно, в языке этом должны быть \textit{его} слова ''кокос'' и ''тигр'', например. Но его язык, похоже, не позволяет описать соответствующие неудачи. Вспомните, как уставший Крузо-2 хотел \textsuperscript{t}кокос\textsuperscript{t}. Он ведь и получил \textsuperscript{t}кокос\textsuperscript{t} --- где же здесь ошибка? Описывая такие установки на предположительно Божественном языке, мы уже не выражаем пропозициональные установки Крузо-2 также, как он сам их выражал и признавал.

Значит первое, что нужно сделать, --- это отказаться от идеи, что успех Крузо-2 обясняется терминами ''кокосы'' и ''тигры''. Нужно формулировать с точки зрения удовлетворение его более примитивных желаний --- голода, страха, безопасности и т.д. --- только так, посредством неявных (или явных) содержащих их грубых закономерностей (а не только с помощью \textsuperscript{t}кокоса\textsuperscript{t} и \textsuperscript{r}кокоса\textsuperscript{r}), следует обозначать его успехи.

Но, похоже, это лишь отодвигает проблему на шаг назад. Вариант первый: у Крузо-2 есть слова и для обозначения этих психических состояний: \textsuperscript{t}голод\textsuperscript{t} и \textsuperscript{r}голод\textsuperscript{r}, \textsuperscript{t}страх\textsuperscript{t} и \textsuperscript{r}страх\textsuperscript{r}, \textsuperscript{t}комфорт\textsuperscript{t} и \textsuperscript{r}комфорт\textsuperscript{r}, и т.д., и для них возникает та же проблема, что и для слов о кокосе и тигре. Вариант второй: у Крузо-2 нет слов для описания этих психологичегских состояний. Но как это поможет уйти от вопроса касательно языка его пропозициональных установок? Он, конечно, в принципе способен \textit{осознать}, что голоден или напуган, но будучи уставшим, может думать, что голоден, когда на самом деле сыт, и напуган, когда на самом деле спокоен. То есть даже без явных слов Крузо-2 проблема сохраняется, потому что он по-прежнему обременен двумя наборами концепций.

У этой проблемы есть изящное решение: припишем Крузо-2 определенный инстроспективный доступ к своем же состояниям, чтобы он обычно не думал, например, что голоден, когда на самом деле сыт, или испытывает боль, когда на самом деле ее не испытывает. То есть свяжем такие его самовосприятия с соответствующими его психическими состояниями. Крузо-2 так будет прав относительно \textit{некоторых} своих состояний, а значит, выраженные в понятиях боли, дискомфорта, голода и т.д. его успехи решат проблему. Ведь пропозициональные установки Крузо-2 --- его желания и надежды --- объективны, когда выражены в терминах не кокосов или тигров, а боли, дискомфорта и голода, и эти же понятия тогда использовал бы Бог для описания установок Крузо-2. Более того, понятия эти негибки: они не меняются от усталости.

Это решение исключает опасность создания бессвязности в рассматриваемом мысленном эксперименте и его модификациях, а так же избегает следствия, согласно которому Крузо-2 на самом деле не способен сколь-нибудь убедительно следовать приватным правилам. Однако все же есть другие две опасности. Во-первых, постулируя обладание Крузо таким доступом к собственным внутренним состояниям, мы отклоняемся от предположения использования ими диспозиционно-смысловых языков. Например, вместо диспозиции применять слово ''голод'', мы говорим уже о применимости этого слова только в отношении действительного голода. Было бы поспешным предполагать, будто у голодного всегда возникает желание применить слово ''голодный'' в отношении себя. Вообще, находиться \textit{в} каком-то состоянии и иметь \textit{слова} для его обозначения --- разные вещи, ведь состояние само по себе может и вызывать диспозиции, но не обязательно диспозиции касательно соответствующих слов применения. Вторая же опасность в превращении Крузо-2 в настолько уж искусственное существо, что соответствующие ему случаи использования приватных языков будет трудно воспринимать всерьез. К этим проблемам я возвращаюсь в оставшейся части этого раздела.

Давайте описывать Крузо-2 как находящегося в состояниях: усталого, отдохнувшего, голодного, испытывающего боль, а так же как имеющего понятия: ''голоден'', ''испытывает боль'' и т.д.. Верно ли, что он имеет диспозицию применять эти понятия \textit{тогда и только тогда}, когда находится в соответствующих состояниях? Конечно нет, это слишком сильное и, вообще то, не нужное, утверждение. Мы, однако, предположили нечто подобное в связи с его отдохнувшим состоянием и такими вещами как кокосы, например. Но давайте быть реалистичными касательно его интроспективных концепций. Крузо-2, даже когда он отдохнул, не обязательно иметь термины \textsuperscript{r}боль\textsuperscript{r}, \textsuperscript{r}страх\textsuperscript{r} и т.д., применяемые к его же состояниям тогда и только тогда, когда и бы Бог применил \textit{Свои} термины к этим состояниям. То есть использование им этих терминов, возможно, иногда отклоняются от такового Богом. (Например, можно страдать от заболеваний, вызывающих диспозицию применить касательно себя понятие жажды, тогда как Бог бы так не сделал.\footnote{Декарт (1993, 58) говорит в этом отношении о водянке.}) И термины Крузо-2 \textsuperscript{t}боль\textsuperscript{t}, \textsuperscript{t}страх\textsuperscript{t} и т.д., возможно, еще больше отклоняются в использовании им от такового Богом. Тогда следует допустить, что у него есть пары интроспективных понятий: \textsuperscript{t}страх\textsuperscript{t} и \textsuperscript{r}страх\textsuperscript{r}, например.

Если эти парные концепции слишком сильно отклоняются друг от друга --- и, значит, от предположительно описываемых ими состояний --- как их бы видел Бог), то Крузо-2 не сможет сколь-нибудь долго выживать. Но если нет --- если \textit{по большей части} они согласуются --- то они предоставляют Крузо-2 способность достаточно точно распознавать успех и неудачу своих действий, а вместе с ними --- и лучшие в смысле достижения успеха языки. Более того, посторонние, описывая его желания, смогут довольно точно использовать свои слова ''страх'', ''голод'' и т.д. для описания его же состояний.

Стоит отметить присущесть более или менее успешных корреляций самоописаний с действительным уровнем своего комфорта --- нашей самооценочной психологии. Несмотря на насыщенность современной популярной психологической литературы результатами исследований, показывающих, сколь сильно мы способны заблуждаться касательно собственных же мотиваций и сколь плохо порой оцениваем свои способности и действия --- не смотря на это --- мы все еще достаточно хорошо осознаем боль и дискомфорт. Действительно, когда дело касается базовых удобств, мы довольно точно понимаем, когда нам, например, стало лучше, чем раньше. Еще раз: наши навыки в этой области не обязательно должны быть совершенны, чтобы была возможность следовать приватным правилам. Но конечно, эти наши навыки прямо отражаются на возможности им следовать: навыки эти более чем достаточны.

Я начал говорить о нашем понимании своих психических состояний и попытался упростить способ говорить о них, обращаясь к точке зрения Бога. Но то же, что касается терминов Крузо-2 для описания психических состояний (и терминов используемых в его отношении другими), касается и его терминов для предметов в мире. Например, мы можем заменить наш ''объективный'' способ сравнения терминов психических состояний Крузо-2 с нашими, на таковой Пятницы. Вместо того, чтобы говорить о лучших/худших корреляциях между его самоатрибуциями и самими этими состояниями, мы можем говорить об использованиями этих терминов с точки зрения приносимого \textit{успеха}. Так, испытывать боль, голод или страх --- значит находиться (по отношению к набору ценностей) в худшем положении, чем если бы их не испытывал. А наличие слов для этих негативных состояний помогает их устранять и лучше ориентироваться в мире.

Исложенных соображений, пожалуй, чтобы убедить в не столь уж исследовательско-стерильном характере описываемых случаев следования приватным правилам. Кстати, соображения эти применимы и к воспоминаниям Крузо-2, и это важно, ведь память имеет решающее значение для способности сравнивать свои языки --- ведь при этом нужно \textit{вспоминать} свои прошлые психические состояния. И для описания их нужен набор понятий. Я, конечно, принял в этом отношении неявную идеализацию, предложенную Крипке (и упомянутую в Главе 2) --- о том, что субъект прекрасно помнит все свои прошлые такие состояния. Но предположение это нельзя так уж легко принять касательно носителей диспозиционно-смысловых языков, ведь даже терминам для собственных психических состояний тогда надо придавать смысл диспозиций. Я уже предположил, что это вполне возможно --- возможно допустить достаточное соответствие значений этих терминов реальным состояниям субъекта (как описал бы их Бог). Но память может показаться нечтом в этом отношении особым, особенно учитывая недавнюю психологическую литературу о ее неточности.\footnote{Смотрите, например, учебник Баддели и др. (2015).} Опять же, совершенно точная корреляция здесь \textit{не} требуется --- Крузо-2 не обязательно помнить всякое нечто тогда и только тогда, когда оно действительно произошло. Нужна лишь достаточная корреляция между двумя языками Крузо-2 --- в отношении своих состояний и языка Бога --- чтобы он мог языки эти оценить. Бог бы мог сказать, что Крузо-2, склонный систематически заблуждаться касательно произошедшего, например, с \textsuperscript{t}кокосом\textsuperscript{t} после отдыха, долго не протянет. Насколько же именно точная корреляция нужна между его воспоминаниями и тем, что происходит --- вопрос эмпирический. В рамках рассматриваемой серии мысленных экспериментов будем считать, что она достаточна.

Крузо-2 кому-то может показаться совсем уж неестественным, учитывая его интроспективную терминологию --- он ведь знает, что говорит на диспозиционных языках. Понятно, что он думает, что кокос --- это то, что он кокосом склонен называть, но имеет ли смысл утверждение, что, например, \textsuperscript{r}воспоминание о вчерашнем поедании кокоса\textsuperscript{r} --- это то, что он склонен (прямо сейчас) им называть? Впрочем, такой образ мышления о своих состояниях и нам не чужд. В частности (и несмотря на утверждения Мура касательно памяти\footnote{Мур (1962, 210-211).}), мы часто говорим нечто вроде ''помню, как положил бумажник в пальто, но очевидно, что на самом деле этого не делал''. Мы говорим и что иногда что-то помним, или думаем, что голодны или что хорошо отдохнули, когда на самом деле совершенно ясно, что это не так, и применять эти фразы заставляют нас лишь соответствующие \textit{диспозиции}. Это указывает на диспозиционный смысл фраз вроде ''думаю, что хочу пить''. И ведь действительно --- говоря подобным образом, крайне трудно осознать свою неправоту.

\section{Последние замечания}

Убедительность некоторых понятий (например, об ''оправданном'' или ''неоправданном'' применении слова, или о ''правильном'' его применении) утрачивает силу, когда речь заходит о носителях диспозиционно-смысловых языков, что и предсказывает анализ Крипке парадокса следования правилам. И все же, диспозиционно-смысловые языки, даже в представленных простых формах --- вполне успешны как приватные языки и даже полезны для своих носителей. Более того, как я показал, такие языки могут \textit{объективно} сравниваться их носителями \textit{самостоятельно}. Носители их будут, значит, объективно определять, какой язык \textit{им} лучше подходит для ориентации в мире. И это так, несмотря даже на их непонимание ''соответствия миру''. Это, я считаю, удивительные результаты.

Несмотря на оговорки относительно памяти и доступа к своим психическим состояниям (в прошлом разделе), Крузо-1 и Крузо-2 остаются, конечно, нереалистичными в психологическом плане. Особенно нереалистично, что им известна связь их языков с диспозициями (а у нас, как правило, не так). В Главе 5 я вернусь к разработке мысленных экспериментов о Крузо, наделив их теперь более реалистичной психологией. В Главе 4 же я постараюсь подточить некоторые естественные против моих утверждений возражения, особенно из Раздела 3.4, что якобы понятия ''деление по стыкам'' и ''соответствие миру'' утратили свою актуальность в контексте Крузо-1 и Крузо-2 --- такие указания дает в своей работе Дэвид Льюис.

\chapter{Референциальный магнетизм}

\qquad

\textbf{Аннотация} \quad В этой главе я рассмотрю влиятельный ответ Дэвида Льюиса на проблему следования правилам (посмертно названный ''референциальным магнетизмом''). В основополагающих его работах можно проследить три различных подхода. Первый говорит о структуре мира как метафизически обеспечивающей ресурсы, дополняющие наши для определения референции. Дело в том, что слова (понятия) имеют определенное значение, выходящее за пределы психологических и нейрофизиологических ресурсов любого человека и общества. Второй подход рассматривает интерпретаторов естественных языков согласно требованиям семантической теории (наряду с базовой научной практикой) для наложения определенной референции на термины этих языков и естественные виды как релаты видовых этих языков терминов. Третий рассматривает \textit{априорное} конститутивное навязывание естественных видов, требуемое ''единственной игрой в городе'' --- муровскими фактами определенной референции и предполагающими определенную референцию семантическими теориями. В этой голове я постараюсь показать, что ни один из этих подходов не работает.

\qquad

\section{Прямые решения через референциальный магнетизм}

Прямые решения парадокса следования правилам пытаются сохранить классическую метафизику соответствия между нашими утверждениями и миром. Они пытаются сделать это показав, например, что скептик не заметил в действительности имеющихся у нас ресурсов для определенности ссылок на вещи в мире --- ресурсов для, например, фиксации нечта как именно счета а не зчета. Или они могут пытаться дополнить недостаточные способности субъекта тем, что за его пределами. Смысловой скептик Крипке утверждает, что необходимые ресурсы --- обосновывающие факты --- не найти в нашей психологии или нейрофизиологии. Семейство прямых решений, критикуемое мной в Разделе 2.5, пытается ответить на этот вызов дополнительными ресурсами в сообществе говорящего или его \textit{переводчика}.

Семейство решений, тесно связанное с работами Дэвида Льюиса, обнаруживает нужные ресурсы либо в самом мире, либо --- в зависимости от интерпретации подхода Льюиса --- в наилучших наших об этом мире и себе теоритезациях. Утверждается нечто в мире, следующее из корреспондентной метафизики или научной практики, способное предоставить искомые обосновывающие факты. Это второе семейство прямых решений --- подходы референциального магнетизма --- столь же популярно, как и версии социологического ответа, приписываемые Крипке Витгенштейну.

Я покажу, почему ни одни из этих подходов не работает, а в последующих главах покажу, почему они и не нужны вовсе. Литература по метафизике, серьезно относящаяся к подходу Льюиса, лишь прикоснулась к текстовой сложности его своей позиции представления. Дело в том, что стиль Льюиса --- как и прочих хороших стилистов-философов, Юма, например --- маскирует сложности и двусмысленности содержания. В соответствующих обсуждениях Льюиса (Льюис 1983, 1984) я выделяю три позиции, которые разительно отличаются своими предположениями. Его риторика и аргументы показывают некоторую двусмысленность среди все трех позиций, хотя я и отмечаю, что он придерживается одной конкретной. Ни одна из этих позиций, впрочем, не отвечает смысловому скептику Крипке, и по той же причине не рассматриваем ''парадокс Патнэма'' --- хотя на последнем я и не буду останавливаться. Все три ответа рушатся от давления определенных внутренних несостыковок.

\section{Может ли Крузо использовать реальность как стандарт для своих слов?}

Некоторые философы оспорят мое утверждение в Разделе 3.4, что Крузо-2 не может сравнить сходства своих ''уставших'' и ''отдохнувших'' слов с миром. Они скажут, что Крузо-2, конечно, сравнивает эти два языка вопрошая не какой из них лучше для ориентации в мире, а какой набор слов лучше всего схватывает реальность. Я уже указывал на \textit{невозможность} этого, обратное утверждение настолько привлекательно, что заслуживает отдельной главы.

Вполне естественно, конечно, думать, что Крузо-2 осознает свое желание соответствовать словами миру, а не просто наборам вещей, к которым он \textit{склонен} относиться как к одинаковым. Но как он может \textit{понять}, что достиг этой цели? И как к ней собирается двигаться? В Секции 3.4 я поддерживаю идею существования у Бога языка, в котором каждое слово отсылает к естественному виду, и Его фразу ''естественный вид'', а также фразы рода ''такой же вид, как ...'' соответствуют только реально существующим естественным видам. Нет нужды объяснять, почему язык Бога таков --- ведь Он всемогущ --- что очень удобно. Но нужно объяснить, как может Крузо быть способен на такое.

Итак, рассмотрим нового Крузо --- Крузо-3. Представьте, что его островной мир схож с таковым Крузо-1, но его чувства не такие же, как у Крузо-1 и Крузо-2 (даже отдохнувшего), а схожи с нашими --- достаточно хороши, хотя порой и могут обмануть. Возможность такого обмана не значит, что он будет смешивать в один класс то, что Бог бы распределил по разным. И пусть он так же слегка неуклюж в счете. (Значит Бог будет думать, что рассматривает количества предметов в определенных наборах как одинаковые, когда Бог бы рассмотрел как разные и наоборот.) Чувства и навыки счета Крузо-3 не столь плохи, чтобы он погиб, но все же \textit{достаточно плохи}, чтобы он часто ощущал негативные от этого последствия. Наконец, он \textit{намеревается} использовать свои слова не только лишь как то заставят диспозиции, но и согласно реальности.

Понадобиться объяснить, как Крузо-3 должен понимать \textit{свои} слова ''кокос'', ''тигр'', ''естественный вид'' и т.д., чтобы \textit{они} такому его намерению соответствовали. Во-первых, Крузо-3 не может понять стандарты своих слов как определяемые лишь только диспозициями --- вместо этого он должен понимать, что стандарты эти следуют из реального устройства мира. Но как \textit{он} может добиться этого? Ведь все, что у него есть --- лишь \textit{реальные} диспозиции применять определенные слова --- больших способностей у него нет. Как же перейти от его действительных диспозиций (\textit{фактических} стандартов) к вещам в мире так, чтобы последние стали стандартами? Иначе говоря, как реальный мир манифестирует себя в намерениях нашего Крузо-3? Он же не может, например, \textit{подумать} или сказать: ''Стандарт применения моего слова ''кокосы'' --- \textit{реальные} кокосы'', ведь ''реальный'' --- такое же \textit{его} слово, как и ''кокос''. Кажется, что попытки Крузо-3 использовать свои слова для обозначения чего-то вне его диспозиций --- попытки, подобно Мюнхгаузену, вытянуть себя из болота за собственные волосы.

Этот третий Крузо, как и предыдущие --- должен думать при каждом пременении слова ''кокос'', например, что \textit{полагается} на свою диспозицию группировать вещи так, как они на соотносятся самом деле. Как иначе? Значит, он применяет слово (или понятие) ''кокос'' ко всему, что кажется ему на это похожим. И все же, предполагается его \textit{понимание} значения слова ''кокос'' как относящегося не ко всему, что ему лишь \textit{кажется} похожим на кокосы, а ко все, что действительно схоже с ними. (Но, учитывая, что это \textit{его} понимание --- ''похоже'' должно быть также \textit{его} словом.)

Здесь возникают \textit{две} проблемы. Во-первых, Крузо-3 должен как-то \textit{заставить} свои слова соответствовать миру. А во-вторых --- как он должен \textit{думать} или \textit{понимать} свои слова согласно этому предположению?

Давайте пока оставим эти загадки в стороне. Одним из необходимых следствий такого образа мышления, как у Крузо-3, будет его отношение к своим характеристикам предметов --- как к лишь оправданно применимым (они ведь могут применяться только согласно его диспозициям их применять). Ему, значит, придется считать себя способным на \textit{ошибки}, но понятие ошибки здесь существенно отлично от такового у Крузо-2 в отношении своих ''усталых'' диспозиций. Крузо-2 их как ошибочные относительно ''отдохнувших'' диспозиций как стандарта, и признает ошибки как использование ''неправильных'' слов (или ''неправильных'' диспозиций с правильными словами), то есть: использование слов или диспозиций, которые \textit{менее ценны}, че те, что он использовал бы вотдохнувшем состоянии. Но ведь предполагается, что Крузо-3 имел в виду нечто иное: стандартом должен быть \textit{мир}, а не какие-то диспозиции. А значит, его ошибка не в выборе языка, а в \textit{самих предметах} --- он что-то не так понял касательно ни самих и с тем, как их следует группировать.

Вернемся к первоначальным вопросам. Во-первых, как намерения Крузо-3 позволяют его словам ''похожий'', ''кокос'', ''естественный вид'' и т.д. соответствовать ''мировым стандартам'', т.е. соответствовать реальным естественным видам в мире? \textit{Что-то} должно этому способствовать, так что естественна такая этого вопроса постановка: каковы \textit{факты} о мире и Крузо-3, определяющие его слово (концепт) ''кокос'' как соответствующее правильному реальному набору предметов, а не какому-то другому? Что определяет слово ''сходство'' группирующим вещи в наборы, соответствующие реальным видам? Вопрос этот особенно неприятен из-за того, что, следуя примеру Крузо-3, \textit{не удается} получить соответствующим слову ''кокос'' все и только то что ему кокосом действительно является.

Во-вторых, как может Крузо-3 вообще мыслить о ''вещах как они есть'', независимо от его собственной предрасположенности к о них мышлению, так, что даже \textit{допустить} стандартом эту ''реальность'' вещей?

Его предполагаемое понимание значений своих слов создает пространство между тем, к чему его слова ''кокос'', ''естественный вид'', ''настоящий'' и т.д. относятся ''на самом деле'' и его ресурсами для их с этим ''на самом деле'' соотнесения. Вопросы сосредоточены на фактах, которые бы определяли --- независимо от Крузо-3 --- что значат все эти слова. Дело в том, что диспозиции Крузо-3 исчерпывают его ресурсы для такого определения, они исчерпывают все, что он о словах своих способен думать. Что же здесь может помочь? (Что может это изменить?)

Здесь множество влиятельных философов-метафизиков испытывают искушение ссказать, что эту всю работу за Крузо-3 делает сам \textit{мир} --- объекты в нем \textit{сами} сортируются по естественным видам, и, например, термин ''кокос'' относится к естественному виду.\footnote{Льюис (1983, 1984) --- многие философы последовали его примеру. См., например, Сайдер (2009), особенно 400-401, где суггестивный язык --- подобно ''референциальным магнитам'' и ''идеальным переводчикам'' --- используется для риторического содействия продвижению внушения. См. также мою работу (2000), особенно часть III, параграф 5, где представлена ранняя версия аргумента против такого подхода к референции. Подход Льюиса содержит уровни приемлимости, а не простую дихотомию между чистыми естественными видами --- совершенно подходящими для обозначения наших терминов --- и неприемлемо неестественными видами --- совершенно неподходящими. Об этом всем я еще расскажу в этой главе.} Вот почему такой подход им может показаться рабочим: пока Крузо-3 может быть уверен в схожести \textit{нескольких} предметов --- в их, значит, принадлежности одному естественному виду (\textit{это} не столь уж большое требование\footnote{Ладно, возможно все же большое, но на это возражение я не буду отвечать здесь.}) --- и пока предметы в мире ведут себя не слишком проблематично, слово ''кокос'', например, будет относиться ко всем кокосам и только к ним. Льюис (1983, 47) пишет:

\smallskip

\textit{Чтобы создать ссылку, нужно два нечта --- две стороны --- и мы не найдем искомое ограничение, если будем искать его не той стороне. Референция частично состоит из наших слов и мыслей, но частично и в приемлимости референта --- и эта приемлимость зависит от естественных свойств.}

\smallskip

Первый --- грубый --- способ интерпретировать это предположение таков: естественные виды в мире действуют как ''референциальные магниты'', семантически \textit{притягивая} референции из слов Крузо-3 к себе, так что его слово ''кокос'', например, отсылает ко всем тем же предметам --- кокосам --- что и тот, который он так назвал, и только к ним.

К такой грубой интерпретации есть контраргумент: независимо от аккуратности организации предметов в мире (миры уже не могут быть существенно аккуратней тех, в которых живут первые два Крузо), нам все же нужно объяснить способность \textit{Крузо-3} указывать своим словом ''кокос'' на все кокосы и только на них. В его слова --- \textit{ни в какие} его слова --- это никак не заложено,  не заложено их соответствие нерегулярным частям реальности. (Это же \textit{его} слова, так что как вообще такое условие могло быть в них заложено, и кем?) Иначе говоря: дополнительные метафизические факты о естественны видах не помогут вне указания \textit{чувствительности} Крузо-3 к ним в ходе использования таких слов как ''кокос''. Чувствительность здесь, конечно --- не примитивная ''семантическая'': семантическая чувствительность должна основываться на предшествующих наших способностях так или иначе взаимодействовать с миром и определять его аспекты.\footnote{Здесь уместна аналогия с электромагнитным излучением: его гораздо больше, чем мы можем видеть своими глазами. Пчелы, например, видят ультрафиолетовый свет, а могли ли указывать на него словами древние греки? Нет, и в основном потому, что не видели его. А мы можем, потому что у нас есть способные на это инструменты. См. Аззуни (2004a).} Назовем это требованием чувствительности. Кстати, в случаях Крузо-1 и Крузо-2 оно удовлетворяется через идеализацию их чувств. Но в случае Крузо-3 оно не выполнено.

В работах Льюиса есть указания на его отвержение этого требования. Например, он пишет (1983, 54-55):

\smallskip

\textit{Мы понятия не имеем, как решить проблему интерпретации, рассматривая все свойства как равные касательно включения в содержание, ведь это означало бы решить ее без достаточных ограничений. Только имея независимое, объективное различие между свойствами, и налагая как конститутивное ограничение --- \textit{априорную} презумпцию в пользу приемлемого содержания, можем мы осмысленно говорить о хоть каких-то решениях этой проблемы. То есть любое правильное решение должно учитывать эту презумпцию. Ведь нет никакого контингентного психологического факта, которому можно было бы довериться.}

\smallskip

И несколькими абзацами позже добавляет (1983, 55):

\smallskip

\textit{Естественные свойства присутствуют в содержании наших установок потому что естественность является необходимой частью в них присутствующего --- а не потому что мы созданы для проявления какого-то особого интереса к естественным свойствам или что естественность эта возникает в них от нашего к ним интереса.}

\smallskip

Похоже, что Льюис отвергает требование чувствительности: он предлагает \textit{априорное} конститутивное положение о референции и содержании вместо раскрытия реальных ресурсов, которые дополняли бы способности индивида в создании определенности референции. Ключевые фразы здесь --- ''нет никакого контингентного психологического факта, которому можно было бы довериться'' и ''а не потому, что мы созданы для проявления какого-то особого интереса или что естественность эта возникает в них от нашего к ним интереса''. Итак, Льюис считает достаточным констатировать, что мир обставлен аккуратно --- предметы рассортированы по видам, и этого достаточно как ограничения на допустимые референты для наших слов, и никакие дополнительные психологические факты о предполагаемой чувствительности не нужны --- они не имеют отношения к такой сортировке.

Но именно такой отказ от требования чувствительности превращает заявление Льюиса в подобие ad hoc или постулирование несводимой ''семантической силы'' (Сайдер 2011, 27), и поэтому Патнэм так упорно к нему цепляется. Льюис (1984, 67) защищается, говоря:

\smallskip

\textit{Патнэм относит мое решение к основанным на сверхъестественном постижении, уподобляет ''мир... сортирует вещи по видам'' нелепости вроде ''мир дает вещам их имена'' (RT&H, стр. 53)! Недавно он даже назвал мое рассуждение об элитных классах ''жутким'' и ''будто бы средневековым''. Но что ж плохого в этой ''средневековости''? Если речь о том, что в средневековье люди признавали объективные связи в мире --- а они это, как я понимаю, делали, как реалисты, так и номиналисты --- что ж, флаг им в руки. Но, думаю, эгалитаризм классификаций --- не сугубо средневековое понятие, скорее уж он особенность нашего века.}

\smallskip

Этот ответ, однако, никак не касается возражения касательно отсутствия \textit{чувствительности} к естественному делению мира. Замечания Льюиса в отношении требования чувствительности --- о другом, на что я и укажу, когда буду рассматривать две другие интерпретации его взглядов.

\textit{Промежуточный вывод} \quad Конечно, возможность Крузо-3 \textit{заставить} свое слово --- сколь бы фундаментальным оно не было --- соответствовать миру по его намерению и загадочной дополнительной помощи в этом со стороны самого мира --- бессмысленна. Непонятно даже как он мог бы \textit{помыслить} такое. Ясно также, что это не зависит от изоляции Крузо-3 --- едва ли \textit{сообщество} имеет больше шансов поднять себя за собственные же волосы, чем отдельный субъект.\footnote{Некоторых людей, конечно, можно поднять --- но только по отношению к другим, оставшимся на земле.} Не поможет и введение подолнительных метафизических \textit{ad hoc} девайсов вроде ''свойств'', ''концепций'' Фреге и прочих \textit{абстракций}, которые Крузо-3 должен якобы ''схватить'' и намеренно связать со своими словами. Ведь с какой стати он ''схватывает'' именно то, что ему нужно, а не другое, немогущее ''соответствовать'' миру накладываемыми на слова Крузо-3 контурами?

Заметьте: это возражение отдельно от тех, что я процитировал у Снайдера (использование ad hoc и постулирование нередуцируемой семантической силы). Оно, вообще, следует из более фундаментального возражения: предложенное Льюисом --- непоследовательно, по крайней мере в том виде, в котором он его изложил. Вы можете говорить о мире как о структурированном так, как только захотите --- это чистая метафизика и не стоит обсуждения. Ведь именно презумпция самого исследования приводит к парадоксу следования правилам: мы ищем обоснование использования субъектом счета, а не зчета, или отсылания им на кокосы, а не на какой-то другой отвратительно хаотичный набор объектов. Поэтому требование чувствительности необходимо: постулируемая структура должна быть нами уловима (у нас, так сказать, должны быть антенны для нее). Если Льюис против этого требования, то пусть же мотивирует свой от него отказ. В следующем разделе я рассмотрю интерпретацию Льюиса, которая изложение такой мотивации и предполагает.

\section{Нисходящий подход к референциальному магнетизму}

Сайдер (2011) слишком поздно стал пренебрегать описанной в предыдущем разделе интерпретацией референциального магнетизма (и той, что он предложил сам (2009)). И это несмотря на то, что он все еще говорит о заявлении ''предикаты должны обозначать естественные свойства и отношения в правильной интерпретации'' (стр. 17) как экстерналистское потому что ''референция не определяется одним лишь использованием --- помимо этого, естественные свойства и отношения являются ''внутренне приемлимыми значениями'' или ''опорными магнитами''''. То есть слово ''кокос'' Крузо-3 отсылает ко всем кокосам и только к ним, потому что набор из всех кокосов --- подходящее значение, а все прочие наборы --- нет. Сайдер, как я уже упомянал в прошлом разделе, указывает, что некоторые видят в предложении Льюиса ad hoc, а некоторые --- ''постулирование нередуцируемой ''семантической силы''''. Но, как отмечает Снайдер, ''ни одно из обвинений не обосновано'' (2011, 27), потому что референциальным магнетизм можно понимать подобно Уильямсу (2007, раздел 2) --- выводя его из ''хорошо мотивированной и более общей доктрины о теоретической добродетели''.\footnote{В обращении Снайдера к Уильямсу (2007) есть некоторая ирония. Уильямс в этой своей статье ясно дает понять, что подход, избранный Снайдером, терпит неудачу, сталкиваясь с фатальным для него аргументом неопределенности. Я не буду останавливаться на аргументе Уильямса против его интерпретации Льюизианского референциального магнетизма --- с референциальным магнетизмом Льюиса есть и более фундаментальные проблемы. Хочу лишь официально заявить, что не вижу, как ссылка Сайдера на богатую метафизическую структуру может помочь в ответе на возражение Уильямса против этой версии референциального магнетизма. (Примечательно, что Сайдер (2011) даже и не указывает на убежденность Уильямса в нерабочести поддерживаемого Сайдером же подхода. И потому он не отвечает на возражение Уильямса.)}

Это хорошая стратегия, но только если она действительно сработает. Общие ее контуры таковы: хорошие научные теории обладают оптимальным сочетанием нескольких достоинств, среди которых простота, (дедуктивная) сила и соответствие данным. Простота требует, чтобы примитивные термины \textit{научной теории} содержали сравнительно естественные свойства и отношения. ''Сравнительно естественные'' потому что, хотя --- согласно второй точки зрения, приписываемой Льюису --- фундаментальные физические теории должны использовать термины, выделяющие фундаментальные --- совершенно естественные, значит --- свойства и отношения, в специальных науках все же будут термины дл лишь сравнительно естественных свойств и отношений. Поскольку семантика --- наука, она наследует это требование простоты, и потому ее термины --- ''отсылает'', например --- должны касаться сравнительно естественных предикатов и отношений. Потому то и \textit{наша} семантическая теория Крузо-3 требует отношение его слова, например, ''кокос'' ко все кокосам и только к ним, а ''подсчет'' --- к сравнительно естественному счету в противовес неестественному зчету.

Заметьте: эта точка зрения не требует от Крузо-3 (или нас) какой-то сверхъестественной чувствительности. Дело ведь не в том, что естественные виды и отношения сами по себе \textit{играют} роль в фиксации референций. (Таким обраом, название ''референциальный магнетизм'' --- как и многое цитированное у Льюиса выше --- вводит в заблуждение, если говорить о нем в этой второй интерпретации.) Просто есть ограничение на использование термина ''отсылает'', ведь это термин специальной науки --- семантики --- и, значит, должен соответствовать сравнительно естественному виду (\textit{сравнительно с другими научными добродетелями} --- например, с соответствием данных). Привлекательной в этой интерпретации Льюиса является неуместность его зловещий выражений вроде ''априори'', ''предполагает'' и ''составляет''. Ведь рассматриваемое ограничение --- эмпирическое --- и непосредственно следует из обычной научной практики.

Есть у этой стратегии, конечно, и недостаток: она требует \textit{правильной} реализации \textit{актуальной} научной практики, и, таким образом, становится ''заложником удачи'', ведь если реальная научная практика не такова, как ее изображает Уильямс (и Льюис, с этой точки зрения), то уже поэтому стратегия потерпит неудачу.

Как ясно видно из моего предварительного обсуждения, Уильямс в этой своей статье преследует две цели: мотивировать свой интерпретационализм и на основе текста доказать, что эта точка зрения принадлежит Льюису. Второй вопрос я оставлю в стороне, сосредоточившись только на корректности интерпретационализма Уильямса\footnote{В этой сноске отмечу свои сомнения в правоте Уильямса касательно Льюиса. Уильямс структурирует точку зрения Льюиса так, что утверждаемое им о науке применимо непосредственно к семантике (поскольку она --- особая наука), и что так Льюис выводит свой интерпретационализм из своих более широких взглядов на науку. Однако  нет никаких текстуальных свидетельств именно такого структурирования Льюисом своих взглядов. (И никаких свидетельств не предоставляет Уильямс. За противоположным взглядом обращайтесь, например, к Уильямс (2007, 371), сноска 21. Кстати, и никаких новых текстовых подтверждений Уильямс (2015) не приводит.) Я же рассматриваю использование Льюисом синтаксической простоты, например, как инструмента общего назначения, используемого всякий раз, когда он нужен. И ничего существенно не изменилось бы, выясни мы действительные взгляды Льюиса по этим вопросам. К предложению Уильямса я отношусь так же, как к интерпретации Витгенштейна Крипке --- как к интересному и важному предположению, которое следует оценивать отдельно от его исходному философу позиции соответствия.} --- думаю, он не работает.

Именно анализ простоты как теоретической добродетели подводит философа науки, согласно Уильямсу, к необходимости постепенного разделения между естественными и другими видами. Рассмотрим две теории микрофизики: T и T', где T' --- ''то же, что T, но с дополнительным, ''избыточным'' законом, не порождающим новых предсказаний'' (Уильямс (2007, 371)). Пусть, например, это дополнение в T' касается ''поведения частиц в номологически невозможных обстоятельствах: при движении быстрее скорости света, например'' (Уильямс (2007, 371)).

Есть разные ответы на вопрос о том, что делает законы истинными. Согласно Юмианству, например, закон истинен тогда и только тогда, когда соответствует ''конкретным фактам'' (Уильямс (2007, 372)).\footnote{Юмианство --- исходная позиция, предполагаемая здесь: ее придерживается Льюис, в ее рамках работает Уильямс, и я, кстати, тоже Юмианец --- но с той лишь оговоркой, что истинность предложений возможна и без их с миром соотнесенности (это имеет значение в Главе 6, но сейчас пока эти вопросы можно оставить в стороне).}

Но если выбор теории основывается только на фактах, которым она должна соответствовать, то T и T' неотличимы. Как пишет Уильямс (2007, 273): ''Назовем это аргументом от дополнительного мусора, аргумент этот угрожает Юмианству неопределенностью справедливости избыточного закона в T'''. Обратите внимание на логическую простоту этого замечания. Пустые обобщения вида \begin{math}(x)(Gx \rightarrow Hx)\end{math}, где G не определено --- верны, но являются ли они от этого \textit{законами}? Если юмовская теория не налагает никаких дополнительных ограничений на законоподобие, то ответ должен быть положительным.

Итак, Уильямс заключает (от лица Юмианцев и Льюиса), что нужно что-то еще. Есть, конечно и другие достоинства теорий: например, простота (насколько экономна и изящна теория) и дедуктивная сила. Лучшая научная теория, согласно Уильямсу (2007, 372) --- ''имеющая оптимальное сочетание простоты, [дедуктивной] силы и соответствия фактам''.

Льюис понимает простоту синтаксически. Уильямс пишет: ''одна аксиоматизированная теория проще другой, если имеет меньше аксиом и синтаксически менее сложна''.\footnote{Уильямс (2007, 375), сноска 27 ''оставляет в стороне'' вопросы измерения такой сложности. Известно, однако, что попытки в этом направлении не увенчались успехом, в том числе из-за серьезных технических препятствий, на которые намекает и Уильямс. Речь в том числе об отсутствии у формальной простоты чего-либо общего с взглядами ученых касательно простоты теорий. Я здесь оставляю без внимания всю эту запутанную философию науки, потому что признание позиции Уильямса (и Льюиса в этой интерпретации) не поможет их делу. Отмечу лишь, что Уильямс (2015, 373) решительней в своей оппозиции: ''Я совершенно не имею понятия, каким должно быть ''определение'' обычного предмета мысли и разговора --- обуви, веревки и сургуча --- если их следует черпать из микрофизики, и вы, думаю, тоже. Нет вообще никаких оснований полагать таких определений возможность''. Проблема эта, конечно, касается также и отношений между специальными науками и микрофизикой --- а также микрофизикой и другими разделами физики! (Этот вопрос я обсуждаю в Разделе 4.4). Уильямс (2015) рассматривает и другие варианты, но и они терпят неудачу.} Но есть приемы, позволяющие превратить сколь угодно синтаксически сложную теорию --- в сколь угодно синтаксически простую (и притом эквивалентную).\footnote{Обсуждение Уильямсом этого вопроса просто-таки вымучено. Он указывает, что трюк этот не окончателен, ведь даваемая им простая теория не имеет дедуктивной --- а значит, и теоретической --- силы. Тем не менее, исключение синтаксически простых аксиоматизаций --- согласно его интерпретации Льюиса --- основное оправдание дополнения понятия простоты различием между ''совершенно естественными'' или ''фундаментальными'' свойствами и остальными. Но если это единственное оправдание, то Уильямс фактически признал (явно в сноске 23: ''случай не имеет решающего значения'') неудачность своего разделения на естественное и не-естественное.} Для обхода проблемы нужно ввести градацию природных свойств, и второй решающий ход именно здесь и происходит. Уильямс (2007, 373) пишет:

\smallskip

\textit{''Чтобы [введение естественных и не-естественных] свойств способствовало объективному анализу синтаксической простоты, нужно постулировать \textit{объективное} различие между ними --- между элитными свойствами и кучей мусора''}

\smallskip

До сего момента речь была лишь о \textit{чисто теоретических} достоинствах (и так и должно быть). Юмианец требует \textit{объективности} различия между законами и, однако, этих различий полностью из свойств теории следования. Нам теперь --- внезапно --- предстоит ввести \textit{метафизическое различие} в мире, мотивированное желанием избежать ''субъективности''. Мы вынуждены пойти на этот шаг, потому что синтаксические конструкции простоты безуспешны, а соглашаться с ее субъективностью мы не готовы, не готовы принимать ее лишь проекцией кажимостей нам чего-то простым (Уильямс (2007, 372)). Но этот шаг бросает тень на Юмианство, которое призван поддержать. Мы постулируем подлинную метафизическую структуру, которая играет решающую роль в отличии T от T'. Юмианский подход вместо этого трактует их как точно описывающие соответствующие закономерности и исключает вторую потому что она лишь отвлекает внимание нереальными случаями (я думаю, это ''субъективный'', но ключевой для ''дружелюбности к пользователю'' вопрос). Что же не так с этим подходом? Юмианцы так же могут признать пустые обобщения --- законами, но совершенно бесполезными. Почему постулирование богатой метафизики для \textit{Юмианца лучше}, чем признание существования бесполезных и пустых законов --- почему признание бесполезности многочесленных законов \textit{хуже}, чем постулирование богатой метафизики и укоренение в \textit{ней} различия между законами и простыми обобщениями?\footnote{Вот причина считать постулирование такой метафизики лучшим вариантом: законы должны действовать контрфактуально, а законы с пустыми антецедентами не имеют пустых антецедентов в некоторых возможных мирах. Но лучшее \textit{Юмовское} решение \textit{этой} проблемы, похоже, в указании условий отношения возможных миров к контрфактуалам --- а не в использовании метафизики для исключения законов с избыточными, но пустыми условиями.} Оставляя в стороне возражения последних двух абзацев: как метафизическое различие естественного и не-естественного влияет на семантику как специальную науку и как оно выводит, в частности, референциальный магнетизм? Идея такова: как только мы сможем измерять простоту аксиоматизации (с точки зрения синтаксической сложности и приемлимости --- естественности --- меры того, к чему относятся примитивные термины), мы получаем добродетель простоты, занимающую место \textit{среди прочих}, и это касается в том числе и семантики.

Итак, ''считаем мы или зчитаем?'' --- вопрос о том, проще ли семантическая теория, использующая наш термин ''счет'' для обозначения счета, той, что использует ''счет'' обозначая зчет. Предполагается, что мы должны прийти к постулированию референциальной связи между ''счетом'' и счетом проще, чем постулировать связь такую между ''счетом'' и зчетом --- частично потому, что счет более естественен, чем зчет.

Однако даже допустив большую простоту аксиоматизации семантики ''счета'' как счета --- мы все же не получим то, чего хотим. Потому что Льюис, когда говорит о сравнительно естественных свойствах как об \textit{априорном} конститутивном ограничении содержания, будто вовсе \textit{не учитывает} прочие научные добродетели (''соответствие фактам'', например). Даже после принятия всех ходов Уильямса (многие из которых я оспаривал), результат все еще довольно слаб: естественные свойства функционируют \textit{наряду с прочими научными добродетелями}, которые можно использовать для оптимальной оценки. Это [диалектически] позиция куда слабее той, к которой [риторически] стремился Льюис.

В частности, \textit{вернулся} угрожающий парадокс следования правилам --- теперь под видом ограничения на соответствие фактам, и значит все психологические факты, явно исключаемые Льюисом --- также возвращаются как снова релевантные. Ведь те из них о нас, что показывают отсутствие у нас ресурсов для отличения сложения от квожения и счета от зчета --- релевантны фактическим данным, которые семантическая теория должна учитывать. Мы отвергли общую теорию относительности \textit{не потому}, что она сложнее Ньютоновской. История физики, вообще --- история замены менее сложных теорий все более сложными из-за несоответствия первых фактам. Неужели семантикам вместо того же разрешено быть \textit{настолько} ленивыми? Это такая прерогатива практиков специальных наук? (\textit{Надеюсь}, нет.)

\textit{Промежуточное заключение} \quad Реконструкция Уильямсом подхода Льюиса не удалась по довольно интересной причине: если естественность понимать как компонент простоты, то ее роль в семантике уже учтена смысловым скептиком. Ведь это важное предположение о (натурализированной!) семантике --- если мы на что-то ссылаемся, то потому, что для этого у нас присутствуют ресурсы. То есть добродетель ''соответствия фактам'' здесь превосходит простоту, как это часто в науке и бывает.\footnote{Это, кстати, дает еще один повод против оригинальности реконструкции Уильямсом подхода Льюиса. Льюис, как я уже указывал в Разделе 4.2, кажется, отвергает требование чувствительности, но реконструкция Уильямса требует его принятия.}

Важный методологический момент здесь --- использование учеными заведомо ложных теорий из-за их простоты.\footnote{См. Аззуни (2000), особенно Часть I, параграф 2, касательно этого вопроса и литературы по философии науки, в которой высказывается то же. Но, конечно, продолжающееся широкое использование различных аспектов ньютоновской физики --- пример повсеместно известный.} Но это подпадает под ''идеализацию''. Это не то, что предлагают Уильямс и другие референциального магнетизма сторонники --- они хотят разрешить парадокс следования правилам, а не признавать его силу, рассматривая введение градуированные естественные свойства как ложную, но простую семантику.

\section{Муровские факты и аргумент единственной игры в городе}

Позиция Льюиса открыта и для другой интерпретации --- гораздо более непримиримой, чем таковая Уильямса. Речь о сочетании признания фактов Мура и версии аргумента о единственной игре в городе.\footnote{Фодор (1975) широко известен (среди философов, психологов и лингвистов) использованием фразы ''единственная игра в городе'' и приводит версии этого аргумента от имени различных тезисов в философии сознания (особенно репрезентативной теории). Один из самых милых намеков на этот аргумент --- цитата Линдона Джонсона (1975, 27): ''Я единственный имеющийся у вас президент''. Аргумент единственной игры в городе --- это, конечно, версия заключения к лучшему объяснению, но в случае, когда объяснение есть только одно.} Я начну с очень краткого указания на то, как именно рассуждения Льюиса эту интерпретацию мотивируют.

Во-первых, Льюис (1983, 46) рассматривает аргумент радикальной неопределенности Патнэма как опровергающий ''отсутствие дополнительных ограничений на референцию'' кроме касающихся мыслей и слов субъекта. Возможно, тот же урок можно извлечь и из нашего парадокса. Льюис добавляет (1983, 47): ''Наш язык действительно имеет довольно определенную интерпретацию (Муровский факт!), поэтому должно быть какое-то ограничение, не созданное нашим соглашением \textit{ex nihilo}''.

На основании этих цитат Льюису можно приписать следующую точку зрения.\footnote{Думаю, что эта позиция на самом деле Льюису и принадлежит. Интерпретация же Льюиса Уильямсом исключена как корректная по мной уже изложенным причинам. Описанный в Разделе 4.2 взгляд исключен из-за непроработанности.} Во-первых, стандартная (успешная) семантика требует определенных интерпретаций наших слов вместо референциальной неопределенности. Во-вторых, значит, семантика требует градуированного различия между естественными и не-естественными свойствами. В-третьих же, это единственная игра в городе. В этом разделе я оспорю первое и третье утверждения.

Начнем с естественного вопроса: в чем именно муровский факт касательно определенной интерпретации нашего языка заключается? Ну, как не иронично, это непростой вопрос, и нам придется рассмотреть различные возможности. Первая --- что эта банальные факты о референциальных практиках населения. Факты эти на самом деле --- скорее данные \textit{для} специальной науки семантики, чем для принадлежащей \textit{ей} серьезной теоретической доктрины. Например, когда Крузо-3 называет что-то кокосом, это, зачастую, действительно кокос. Не всегда, конечно, ведь он подбирает и вещи, которые называет ''кокосами'', позже отрицая, что они кокосы (и может исправиться даже еще раз). Это и есть муровские факты.\footnote{Что ж, нужно оговориться, ведь факты эти об использовании и они гораздо сложнее, чем предполагает Льюис. Она также, как правило, невидимы для пользующихся естественным языком (что делает их не столь удачными на роль муровских фактов кандидатами), и Хомский регулярно указывал на это. Наконец, они сопротивляются пониманию с помощью стандартных семантических теорий, так приглянувшихся Льюису. Подробности этого см. в моем обсуждении идей Хомского и Петровского далее в этом разделе. См. также Аззуни (2013).} Есть также множество практик, связанных со счетом: что мы говорим, когда считаем, на что при этом указываем как на нам \textit{известное}, и как себя и других попровляем.

Это, скажем так, гигантский массив вербальных и невербальных муровских фактов. Но --- к несчастью для Льюизианцев --- непонятно, как Льюис мог говорить о чем-либо подобном, ведь именно это все имеет в виду скептик --- именно это как основание для своего парадокса использует. То есть муровские факты здесь --- общая для обеих сторон монета, и Льюизианцу они не особо помогают.

Давайте попробуем еще раз. Может ли быть так, что \textit{обобщения}, обнаружимые в стандартной семантике --- муровские факты, на которые Льюис и ссылается, описывая язык наш как имеющий ''достаточно определенную интерпретацию''? (Похоже именно такие вот формулировки злят своей расплывчатостью). Как эти обобщения могли бы выглядеть?

Что ж, нам приходится гадать, посколько Льюис нам ничего здесь не сказал. Один из простых наборов примеров-кандидатов --- бикондиционалы Тарского --- предложения формы

\smallskip

\textit{''Снег бел'' истинно тогда и только тогда, когда снег бел.}

\smallskip

То есть все утверждения получаемые заменой ''снег бел'' и его закавыченного варианта на любое другое утверждение.

Но это не помогает Льюису, потому что предлагаемые даже бесконечным числом бикондиционалов Тарского ограничения --- слишком грубы для фиксации референции использованных в них терминов. Эти бикондиционалы лишь координируют значения истинности двух наборов предложений --- это недостаточно. Проблема эта, конечно, обобщается на теории, на которые повлияли работы Тарского: истиностно-кондициональные семантические. Это потому, что основные ограничения на референции терминов таких теорий непосредственно из бикондиционалов Тарского и следуют. Прочие ограничения, налагаемые такими теориями --- композиционны, но не налагают дополнительных ограничений на референции (во всяком случае, недостаточны для их определенности).\footnote{Будь они достаточны, теоремы обосновывающие результаты Патнэма о неопределенности --- результаты перестановок и теоремы Левайтхайма-Скулема --- не были корректны. Технические подробности см. в Аззуни (2003).}

Дела с этой разновидностью Льюизианства даже хуже, чем я осмеливался описать: можно показать, что даже истиностно-кондициональные семантические теории, успешно фиксирующие использование контекстно-зависимых выражений (вроде ''Эта лампа не горит'') --- не определяют референцию согласно требованиям сторонников референциального магнетизма.\footnote{Подробно об этом см. в Аззуни 2010b, Глава 5. Я также возвращаюсь к этой теме в Главе 6 этой книги.} И это, конечно, усугбляет ситуацию, ведь контекстно-зависимые муровские факты о референциях --- не просто ''еще больше языка'' для новой интерпретации открытого, не нечто гораздо сильнее: выглядят как контакт с реальным миром. И если даже они не помогают, то что поможет?

Другие виды семантических теорий --- порывающие с истиностно-кондициональностью --- тоже не способны предоставить необходимые Льюису муровские факты. Рассмотрим, например, ''структурно-семантические теории'' Каца (1972), Каца и Фодора (1963) и Джекендоффа (1972). Такие теории предназначены сопоставлять каждое выражение естественного языка с таковым репрезентативного --- языка мысли (как в Фодор 1975) или абстрактного языка (как в Кац 1981). Оставим в стороне, что Льюис определенно \textit{не} о таких теориях писал, ведь он (1970) утверждал, что отображения такие не обеспечивают интерпретации естественного языка --- проиходя из одной системы обозначений в другую, что указывает на неизвестность нам условий каждой из них по отдельности.\footnote{Подтверждает ли это подобные семантические теории? Этот вопрос можно оставить в стороне. Касательно аргумента в пользу этих семантических теорий см. Аззуни (2010b), Раздел 5.3. Возражения, на которые я там отвечаю, см. в Льюис (1970), Хиггинботам (1990) и Лепор (1983).} Важно, что (и Льюис, кстати, тоже так считает) теории такие не способны предоставить определенные референции на термины естественных языков в мире --- это просто не то, чем они занимаются.

Здесь можно возразить, что я сосредоточен не на том, что имел в виду Льюис, когда он писал: ''Наш язык действительно имеет довольно определенную интерпретацию (муровский факт!)''. Но что же я упустил? Я рассказал банальные факты нашей референциальной практики, рассмотрел семантические этих практик теории, и даже расширил обсуждение до семантических теорий языков, где они бихевиористски связаны с жестами, совершаемыми нами к самому миру --- например, когда указываем на лампу, говоря, что она не горит. Так что трудно удивляться сомнениям оппонентов Льюиса касательно его смутных намеков на ''определенные ссылки'' и ''муровские факты''.

Я, значит, опровергаю, что муровские факты могут использоваться Льюисом против скептиков, и объединение этих муровских фактов в гигантскую теорию не поможет. Это, так сказать, \textit{принцип множества петель}: если один, используя только себя не может подняться в воздух --- то и несколько тысяч таких же не смогут.

И еще два, последних, замечания. Во-первых, я обрисовал свой аргумент против использования Льюисом муровских фактом на определенном уровне абстракции, и дополнительные обоснования против этого маневра я представлю в следующих двух главах: используя следующие, психологически более реалистичные версии Крузо, покажу, что успешная языковая практика вместе с приписыванием истинности и ложности этого языка предложениям и со стандартной для него семантикой истинности --- полностью совместима с ''неопределенной референцией''. А именно: уступка смысловому скептику никак не влияет на приписывание истины или стандартной семантики этому языку. Это, значит, будут \textit{примеры} бесполезности муровских фактов для позиции Льюиса.

Второй момент касается статуса семантики как науки. В нескольких абзацах я сослался на не-истинностно-кондициональные семантики, в которых референция не является центральным понятием. И это лишь верхушка айсберга гораздо более серьезного вызова подходу Льюиса. Вопрос в том, будет ли определенная референция играть роль в семантике --- если и когда та станет наукой --- ведь только определенную референцию Льюис и мог бы противопоставить скептику.\footnote{Это касается не только ''Мурианского'' Льюиса, но и более умеренной его интерпретации из Раздела 4.3.} Заметьте, что в семантике тогда определенная референция должна играть центральную ''объясняющую'' роль --- действовать как стержневое ''теоретическое'' понятие.\footnote{Дэвидсон предполагает это по поводу ''референции''. См., например, Дэвидсон (1973, 74).} Сайдер, например, пишет: ''некоторые философы отвергаю объяснение, основанное на референциях'' и добавляет: ''некоторые утверждают, что конвенциональное значение кодирует гораздо меньше референтных свойств, чем принято считать, но даже так --- референции (по крайней мере, референции говорящего) отводится \textit{некоторая} объяснительная роль'', и для иллюстрации цитирует Хомского (2000) и Петровского (2003).

Это крайне некорректное прочтение Хомского и Петровского, выдающее непонимание фундаментальности их противостояния чему-либо на поле основанного на референциях объяснения. Во-первых, такое ''референциальное объяснение'' --- намек на условно-истинную референтную семантику вроде семантики Тарского --- на что и указывает ''чем принято считать''. Но Хомский и Петровский, а также множество других лингвистов, категорически отвергают такой подход. Вопрос для них заключается в том, поддаются ли термины естественного языка эмпирически чему-либо вроде референции как она в ''мейнстримной'' позиции понимается. Хомский категорически отвергает идею книги, на которую ссылается Сайдер, и показывает, что имена и родственные термины естественного языка не имеют ничего общего с терминами, поддающимися референтному отношению в стиле Тарского. И Петровски (2003) подчеркивает то же, приводя еще и соответствующие примеры.\footnote{См. Хомский (2000, 37, 180-1, 35-36) и Петровский (2003, 226-233).}

Петровский (2003, 222) делает часто повторяемое наблюдение что ''отсылают не слова, но люди''. Этот лозунг, будучи невнимательно прочтен, создает впечатление возможности работы с устойчивыми к стандартной семантической обработке терминами естественного посредством замены термина-референции-как-использованного-говорящим или термина-референции-в-контексте-как-использованного-говорящим --- на термин-референцию. Действительно, предположение Сайдера о серьезном восприятии ''референции-говорящим'' Хомским и Петровским упускает силу их возражений --- а они ведь ясно показывают, что ни один простой прием такого рода не справляется с этими примерами.\footnote{Вот что пишет Хомский (2000, 37): ''Лондон настолько несчастен, уродлив и загрязнен, что его следует разрушить и перестроить за 100 миль от него''. ''Лондон'' здесь сопротивляется стандартной трактовке Тарского. И это явление широко распространено в естественном языке: одно использование имени --- ''Лондон'' --- в одном и том же предложении включает в себя несопоставимые своими референциальными целями предикаты. См. также Петровский (2003) о попытках изменить истинностно-кондициональные семантические подходы для таких примеров систематической обработки.} Подчеркиваемая Хомским и Петровским идея --- что любая наука о семантике вряд ли будет включать что-то вроде определенной референции требуемого ответом Льюиса типа. Ведь любое определенное референциальное отношение --- такое как истинность и ложность произнесенных нами предложений --- эффект ''неразрешимого взаимодействия'', как проницательно выразился Петровский (2003, 218). То есть оно не играет объяснительной роли в семантике --- вместо этого будет схоже с концепцией ''трения'' в физике --- \textit{объяснимой} как результат множества эффектов, многие из которые не-семантические --- сама по себе не будет полезной теоретической идеей.\footnote{Хомский часто прямо критиковал стандартное понимание референции, называя его бесполезным для лингвистов, например в Хомский (2000, 40). Обсуждение этого аспекта его мышления см. в Аззуни (2013). Заметьте контраст с первым абзацем Уильямса (2007, 361): ''Подвопрос, на котором здесь концентрируюсь, касается семантических свойств языка: в силу чего такое имя как ''Лондон'' относится к чему-то как предикат, также как ''большой'', например?''. С понятием референции есть две проблемы. Во-первых --- как ее понимать в семантике, во-вторых --- ее в ней теоретическая/объясняющая роль. Сайдер, говоря об ''объяснительной роли'', сосредоточивается на втором, Уильямсон же в первом абзаце (2007) --- на первом. В подходах Тарского они совпадают. Хомский и Петровский настаивают на подходе к семантике, в котором оба вопроса этих вынесены на второй план из-за теоретической неразрешимости.}

Выдвигаемые Хомским и Петровским аргументы против референциального объяснения касаются способов, с помощью которых имена и родственные выражения можно эмпирически продемонстрировать в естественных языках, и способы эти противоречат требуемому референциальной семантикой. Помимо этих проблем, есть и второй способ для родственных терминов нарушать ограничения естественности --- в специальных науках и в обыденной жизни они, кажется, выделяют свойства довольно неестественные (по крайней мере, относительно терминологии физики). Льюис ясно говорит об этом и вводит для решения такой проблемы ступенчатую естественность, градации в которой соответствуют результату синтаксической сложности определений менее естественных свойств с точки зрения более естественных. Льюис (1984, 66), говоря об отношении элитных свойств (физических) к не столь элитным (например ''палки, камни, кошки, книги, звезды''), пишет:

\smallskip

\textit{Менее элитные таковы оттого что связаны с более элитными цепями определимости. Так, довольно длинные цепи ведут к умеренно элитных классов --- кошек, карандашей и луж --- но цепи к совершенно безнадежным много длиннее.}

\smallskip

В связи с попытками сведения специальных наук к фундаментальным есть много независимой литературы о возможности таких цепочек определений.\footnote{См. Ким (1993), Патнэм (1967), Фодор (1974). Обсуждение и цитирование дополнительной литературы см. в Аззуни (2010b), Раздел 4.2.} Так называемая проблема множественной реализации: специальные научные термины кажутся неестественными дизъюнкциями базовой терминологии. Льюис, в своем очевидно на это ответе, предлагает дефиниционно-синтаксическое решение.

И это довольно странно, ведь литература по философии науки указывает на невозможность здесь определений. (Осознано и обнародовано это было еще в 1960-х годах.) Эти знания также стимулировали исследования супервентности как инструмента для \textit{семантического} закрепления связи специальной научной терминологии с базовой без реальных определений. Но возможность изменения подхода Льюиса к градуированной естественности через замену синтаксического ее критерия на семантический --- не то, чем я буду сейчас заниматься. Ведь есть и более серьезная проблема.

Заключается она в том, что --- в том числе благодаря контрпримерам --- становится ясна неподдаваемость соображениям естественности терминологии повседневной жизни и науки --- вместо этого употребление терминов в этих областях очевидно чувствительно к таким поверхностным явлениям как случайные исторические факты о порядке открытий. Сайдер (2011, 32) признает такие простые примеры, как ''амулет'' лингвистического сообщества для обозначения всех предметов из золота и одного конкретного --- из серебра. И пример этот кажется вполне подходящим для подхода градуированного натурализма. Но есть множество примеров --- из самых разных наук и как реальных, так и мысленно-экспериментальных --- показывающих незначительность проблем ''разрезания по стыкам'' для терминов для естественных видов.\footnote{См. примеры ЛаПорта (1996), Дюпре (1981), Аззуни (2000) и Уилсона (1982). Аналогичную точку зрения высказывает Санделл (2012).} Проблема --- как всегда --- в совершенной случайности актуальной научной терминологии. Если бы, как убедительно указывает ЛаПорт (1996, 117-20), мы открыли планету, подобную Земле, но с тяжелой водой --- произойди это до открытия изотопов, свойства тяжелой воды (что, например, она убивает рыбу, крабов, водные растения, замерзает при иной чем обычная вода температуре и от обычной воды отделима) исключали бы ее отождествление с водой \textit{даже после их открытия}. Семантика родственных терминов --- научных и других --- будет, похоже, очень сложной, и вряд ли какую-то естественность вообще будет включать.\footnote{Дальнейшее этого обсуждение см. в Аззуни (2010c), Раздел V.}

\section{Заключительные замечания}

В этой главе я проанализировал три возможных подхода к ''референциальному магнетизму'' (все они заимствованны из работы Дэвида Льюиса, но все --- в различных комбинированных формах --- очень популярны среди и сторонников, и противников). Одна версия использует естественные виды для оказания некоего семантического влияния на референтные свойства наших терминов. Другая --- рассматривает ступенчатую естественность как часть набора достоинств, должных присутствовать у всякой научной теории (включая семантику). Третья же пытается показать, что других вариантов для семантики нет. Контраргументы этим трем позициям теперь, думаю, ясны. Первая интерпретация действительно заслуживает презрения, обрушенного на нее Патнэмом, ведь (1) \textit{не} соответствует научной практике и (2) в любом случае \textit{не} дает необходимого результата: перевешивания соображениями простоты, проявляемых скептиком натуралистических опасений по поводу ссылочных ресурсов. Третья же интерпретация безуспешна просто потому, что в городе есть несколько конкурирующих игр --- и все чтят муровские факты рецеренциального использования и даже муровские семантические законы (если таковые есть) --- но которые отвергают центральную роль определенной референции, являющуюся ядром ''мейнстримной'' семантики.

Должен также отметить, что Дорр и Хоторн (2013, 3) описывают продолжающуюся дискуссию между ''энтузиастами-натуралистами'' и ''скептиками-натуралистами'', где различие в ''отношениях'' между отвергающими различие между естественными и неестественными свойствами и энтузиастами которые его ''принимают'' и ''одобряют''. Они еще высказывают сожаление о построенности этого разногласия вокруг ''автобиографических утверждений''. Но --- независимо от этого точности как в современной метафизике дебатов описания --- возражения в этой главе не имеют практически ничего общего с вопросами богатства метафизики мира. Речь скорее о (1) ресурсах, доступным агентам для фиксации своих референций средствами постулируемой метафизики и (2) должна ли эта метафизика быть встроена в теории о них.

В следующей главе я возобновляю свой анализ парадокса с помощью все более сложных Робинзонов.

\chapter{Кривые успеха}

\qquad

\textit{Аннотация} \quad В этой главе обсуждаются еще два Робинзона Крузо. Крузо 4, имея интроспективный доступ к своим диспозициям, осознает, что говорит на диспозиционно-смысловом языке, но из-за постоянной этих диспозиций смены не может думать о себе как о говорящем на разных языках --- вместо этого думает о своем языке как о включающем иногда плохие, а иногда хорошие, диспозиции. Крузо 5 же не имеет (или имеет очень ограниченный) доступ к своим диспозициям, и, как я показываю, должен быть подобен нам: думать о терминах своего языка как о подчинанных стандартам, обусловленным образом существования субъектов в мире, а не их диспозициями.

\qquad

\section{Крузо с непостоянными диспозициями}

Живет он в том же мире, что и предыдущие из Глав 2 и 3, и, подобно им, осознает себя как говорящего на диспозиционно-смысловом языке, где актуальные его побуждения определяют референции его слов. У него, как и у предыдущих Крузо, отличная память: он может вспомнить и предметы, к которым применял слова, и состояния, в которых при этом находился. В отличие от Крузо 2, однако, диспозиции Крузо 4 постоянно меняются, и более того --- демонстрируют положительную ''кривую обучения''. То есть, определяющие применение им слов, смысловые побуждения постоянно совершенствуются в точности идентификации, различении и вынесении суждений о сходстве объектов в мире.

С точки зрения Бога, вместо положительной кривой обучения Крузо 4 замечает новые внутренние состояния смысловых побуждений, в которых его побуждения применять слова позволяют ему с тем же или большим успехом взаимодействовать с миром, чем предыдущие. У Крузо 2, напомню, было два внутренних состояния: усталое и отдохнувшее. Крузо 4 же, в отличие от Крузо 2 и от нас --- признает наличие у себя большого множества внутренних состояний, влияющих на его смысловые побуждения, он так же различает их и признает, что часто находится в новых, ранее не испытываемых. Поскольку новые состояния приносят ему больший успех, он решает всячески избегать действий вне новых таких состояний.

Конечно, у него не всегда есть \textit{выбор} --- ему, например, приходится убегать от преследующих его тигров вне зависимости от внутреннего состояния. Но он обнаружил возможность провоцировать крайне полезные состояния определенным поведением: например, избегать употребления алкоголя, держаться хорошего освещения и тишины. Он не может \textit{непосредственно} себя заставить находиться в определенном состоянии (так же как мы не способны заставить себя быть спокойными, не чувствовать голод и не спать), но он все же способен управлять ими посредством провоцирующего или угнетающего их поведения.

Благодаря некоторой удаче и частой смене состояний, этот Крузо взаимодействует со средой все лучше и лучше, что и отражает положительная кривая \textit{успеха}. И она же верна для применения им счетных слов: раньше еда заканчивалась раньше ожидаемого, а хищников оказывалось больше чем как ему казалось их должно быть, но с освоением им новых методов счета подобные неприятности почти сошли на нет, и иногда он замечает прогресс даже в течение дня.

Обратите, однако, внимание на представленные здесь идеализации. Во-первых, Крузо-4 переживает изменения своих состояний, в том числе изменения в алгоритмах счета. Это, впрочем, не столь уже неправдоподобно.\footnote{Обсуждение аргументов за существование феноменологии когнитивных состояний см., например, в Стросон (1994), Питт (2004), а также в статьях Бэйн и Монтегю (2011). На данный момент я не спорю за или против этой позиции. В Разделе 5 я, впрочем, допускаю, что Крузо-4 осознает либо состояния осознанного своего рассуждения того или иного типа, либо состояния с этим осознанием коррелирующие. Стоит добавить, что я (как и другие) интерпретирую Витгенштейна как отрицающего существование такой феноменологии, но ее существование или несуществование, я считаю, --- не априорный факт осознания своих внутренних состояний.} Также он признает воздействие своего поведения на свое состояние --- например, посчитать что-нибудь или различить кокос проще в состоянии, соответствующем хорошему освещению. Более важная --- и не столь естественная --- идеализация в отсутствии неверных шагов изобретении Крузо-4 новых способов ведения своих дел --- его методы все лучше и лучше. Это вторую идеализацию можно немного ослабить: пусть Крузо-4 \textit{иногда} обнаруживает все же неуспешность некоторых новых своих состояний --- но это так, нечасто.

С точки зрения Бога Крузо-4 продолжает изобретать все лучшие и лучшие алгоритмы счета --- и получать правильные (или очень близкие к правильным) ответы все чаще. Раньше, например, он считал кокосы как получится, а теперь группирует и считает сначала те, которые отложи он на потом --- скорее всего бы упустил.

Богу, конечно, очевидно отношение слов Крузо-4 к общим его диспозициям и только к ним. Так как смысловые диспозиции Крузо развиваются так же, Бог видит его развивающим постоянно меняющуюся \textit{серию} языков, и слова более поздних лучше соответствуют реальности. (Бог может говорить об этом потому что он --- и только он --- \textit{видит} соответствие слов реальности.)

Как и предыдущие, этот Крузо разбил предметы на группы схожих, чтобы одинаково их называть. Но он не может непосредственно сравнить диспозиционно-смысловые \textit{языки}, как Крузо-2, потому что его диспозиции и состояния всегда \textit{существенно} меняются. Крузо-2 может индивидуализировать свои слова в соответствии с двумя своими состояниями --- уставшим и отдохнувшим --- а Крузо-4 вынужден понимать себя как всегда говорящего на \textit{одном}, хотя и меняющемся, языке. Референции слов в этом языке постоянно меняются --- в соответствии с диспозициями. Поэтому для него ответом на вопрос ''Использую ли я эти слова так же, как и раньше?'' будет ''Да, ведь я все использую их согласно смысловым побуждениям''. Т.е. Крузо-4 все еще понимает свой язык как диспозиционно-смысловой.

Потому Крузо-4 не может индивидуализировать свои слова и сказать --- как мог бы Крузо-2 --- ''Я использовал неправильное слово ''кокос''''. Вместо этого он \textit{способен} сказать только ''Я использовал слово ''кокос'' с неправильными диспозициями'', ведь он не допускает указание прежними своими \textit{диспозициями} на то же, на что указывают его диспозиции сейчас. Говоря так, он признает отношение нынешнего его использования слова ''кокос'' к нынешним диспозициям, и указывает причиной прежнего использования диспозиции прежние.

Он, однако, понимает, что требования текущих диспозиций могут отвергнуты быть уже завтра --- завтра же его диспозиции могут не согласовываться с теми, что у него сегодня. Но как ему это сказать? Возможно, так: ''Сегодня ''кокос'' указывает на этот предмет сегодня, но уже завтра может на него не указать, если мое состояние изменится [в лучшую сторону]''. Не буду останавливаться на подробном анализе такого использования --- для него в любом случае вполне разумно хотеть сказать нечто такое, ведь он осознает изменчивость референций своих слов при неизменности их определенности диспозициями.

Решающее значение для языковой практики такого Крузо имеет положительная кривая успеха, вызванная определенным его поведением. Заметьте: он не думает, будто диапазон его слов ''совпадает'' с реальным устройством естественных видов --- он не пытается объяснить свой успех в метафизических терминах. Принятие им новой политики маневрирования в мире и использования слов знаменует новое внутреннее состояние --- для него это лучший способ создать положительную кривую \textit{успеха}, и в этом смысле значения слов Крузо-4 можно понимать как искреннее руководство собственными намерениями. Он делает рациональный выбор благодаря памяти о предыдущих своих состояниях и соответствующего им успеха, и он может разными способами новые такие состояния вызывать. Общая его политика: стараться принимать решения и действовать только в новых или достаточно хороших внутренних состояниях.

Теперь два важных замечания. Во-первых, успехи взаимодействия Крузо помещены мной в аккуратный и простой метафизический контекст, и я даже привел пример, где кривая успеха соответствует кривой \textit{обучения} соответствию словами реальной сгрупированности вещей в естественные виды.

Но важно понимать, что такие как бы фоновые метафизические положения не имеют решающего значения для \textit{согласованности} практик Крузо приватного языка. Последовательность такой практики для разных Крузо значит, во-первых, положительное такой практики влияние на благополучие Крузо, а во-вторых --- его способность оценить, где нужно, эти практики по отношению друг к другу.\footnote{Способы такой оценки, конечно, не нужны Крузо-1 из Главы 2, ведь он никак и не может повысить свой успех.} Решающее значение имеет способность оценивать успех альтернативных подходов взаимодействия с миром --- вот что действительно нужно для создания положительной кривой успеха.

Фоновая метафизика не обязательна здесь потому что Крузо может создать такую положительную кривую по \textit{самым разным} метафизическим причинам. Например, не обязательно (с точки зрения Бога) приближение серии языков Крузо к естественным сходствам в мире. Нужно, конечно, \textit{какое-то} позитивное диалектическое взаимодействия между устройством мира и способами изменения Крузо своих диспозиций, но вряд ли это необходимо включает предположенные мной метафизические взаимодействия. Может, метафизических видов вовсе нет, а Крузо защищен от этого вмешательства в создание положительной кривой строго локальным и ограниченным характером своих взаимодействий с миром. Трудно утверждать, что история науки вплоть до сегодняшнего дня не демонстрирует подобную же закономерность вместо постепенного схождения наук к некоему окончательному набору метафизических видов.

Есть стиль философа, чувствующий здесь необходимость некоторого \textit{умозаключения}, \textit{объяснения}. Успех, например, следует объяснять все более тесным соответствием между терминами и миром.\footnote{Бойд (1991), Патнэм (1978, 21).} Пусть, например, у Крузо-4 есть теория --- элементарная, но подлинная --- о кокосах, белках и т.д.. Его успех тогда можно \textit{объяснить} все большим соответствием его теорий миру, но такому соблазнительному ''выводу'' следует противостоять.

\textit{Всегда} возможны эпистемические \textit{тупики}. Успех сам по себе не исключает свою метафизическую \textit{лишь} локальность --- ничто вообще не исключает возможности эпистемического тупика. Факт этот часто выражают в терминах ограниченной \textit{индукции}.\footnote{Описанное здесь понимание принадлежит Юму (1961).} Индукция всегда способна на неудачу --- и не важно, насколько успешной она до этого была.

Это все особенно важно еще и по другой причине: Крузо, как я утверждал в Главе 3, не может судить об успехе посредством метафизических сравнений. Поэтому его способность создавать положительную кривую успеха необходима и достаточна для последовательной им практики приватного языка. В дальнейшем, когда я предполагаю наличие у Крузо такой практики, я предполагаю наличие у него способов реализовать такую кривую в ряди приватных языков им изобретаемых.

Второй момент здесь --- оговорка к первому: Крузо-4 был описан как имеющий такую кривую успеха, но в целом она совместима с достаточно длительным негативным для него периодом, когда его способность к суждению нарушена, и, осознавая это, он прячется, пока не ''придет в себя''. Так может быть когда он, например, пьян. Но пока у него есть \textit{какое-то понимание} своего состояния, его практика приватного языка будет последовательной, ведь даже будучи неспособным \textit{улучшить} свои диспозиции, изменив состояние, он может ''переждать'' до той поры, пока его состояние и диспозиции не придут в норму. (Он может попытаться вспомнить, где лучшие диспозиции совпадают с нынешними, хотя, конечно, негативные состояния способны влиять на память и суждение даже такого рода.) Решающее для связности практики приватного языка --- способность классифицировать смысловые побуждения с точки зрения их влияния на кривую успеху, и распознать их он умеет интроспективно.

Какие виды развивающихся состояний совместимы с последовательностью практики приватного языка --- эмпирический вопрос. Если, например, способности Крузо стремительно падают из-за болезни Альцгеймера, то такая практика будет невозможна. Или может просто его диспозиции действуют произвольным образом, и он вообще не может их контролировать. Я не готов --- да и не должен --- описывать конкретные условия процессов развития диспозиций изолированных индивидов хотя бы потому, что вопрос этот касается не только человека, но и его окружения. Я лучше опишу любой набор [меняющихся] диспозиций Крузо как набор диспозиций, вызывающих такую согласованность существованием положительной кривой успеха, которую он распознает и поддержание которой провоцирует (диспозиции ИКПЯП --- индуцирующие когерентность приватной языковой практики). Наличие или отсутствие такого набора диспозиций, как было отмечено, зависит не только от самого человека, и мир, в котором он находится, не может быть оценен им самостоятельно через сравнение с его диспозициями --- он может только признать способствование вызванных им изменений диспозиций поддержанию положительной кривой успеха.

\section{Робинзон с диспозициями ИКПЯП и умозрительным доступом к ним}

Крузо-2 и Крузо-4 нереалистичны из-за почти идеального интроспективного доступа к своим диспозициям к использованию слов.\footnote{Точнее, им дана почти идеальная корреляция между внутренними состояниями и этих состояний концепциями --- такое я обычно называю интроспективным доступом или осведомленностью.} Трудно определить, какова реалистичная степень нашего доступа к своим таким диспозициям, но мне хотелось бы сейчас рассмотреть Крузо-5, у которого --- в основном --- \textit{нет} такого интроспективного доступа: его диспозиции полностью субличностны. Это, конечно, не значит отсутствие у него доступа ко многим его внутренним состояниям, просто по большей части он не особо то и осознает как состояния эти влияют на его побуждения.

В этом смысле опыт успеха Крузо-5 в мире очень похож на опыт тренировки стрельбы по цели --- по мере все большего успеха в этом, в нем происходят многочисленные нейрофизиологические сдвиги, хотя он этих изменений и не осознает: только что становится все лучше и лучше в стрельбе. Он, конечно, может знать и использовать какие-то приемы для этого, но может и без таких знаний обойтись.

Это, впрочем, не значит отсутствие интроспективного доступа Крузо-5 к своим состояниям --- такое \textit{было бы} нереалистично. У него определенно есть доступ к тем состояниям, которые ему нужны для отслеживания своего успеха. У него есть слова и понятия, соответствующие доступным диспозициям, такие как ''боль'' и т.д.. Пусть также его диспозиции --- диспозиции ИКПЯП. Раз у него есть лишь косвенный доступ к своим диспозициям, то я опишу совсем другой способ ему думать о значениях слов своего языка --- очень похожий на наш.

Начнем с того, что Крузо-5, не осознавая своих склонностей касательно большинства слов --- осознает только видимые изменения в объектах мира. Объекты эти для него то такие, то другие --- иногда \textit{представляются} ему так, а иногда иначе --- иногда представляются обладающими одними способностями, а иногда другими. Иногда, полагаясь на свои такие представления он достигает успеха, а иногда нет.

Так как у него нет интроспективного доступа к своим диспозициям касательно слов, то он должен косвенно распознавать изменения в них по изменениям взаимодействия с соответствующими объектами. Как уже подчеркивалось, Крузо-5 \textit{действительно} имеет интроспективный доступ к определенным своим \textit{состояниям}. Помимо признания боли, страха и т.д., он может также признавать, что устал, пьян или испытывает головокружение --- и делать соответствующий вывод об изменении диспозиций. Но --- как предполагает мое использование слова ''вывод'' --- он признает это лишь \textit{косвенно}, ведь из предыдущего опыта уже усвоил отличность явления ему вещей в подобных состояниях от такового в состоянии обычном. Вещи могут быть, например, нечеткими, неудобными и т.д..

Аккуратный и эффективный метод для него классифицировать способы представления ему объектов --- это обращать внимания на \textit{внешний вид} объекта и с чем он схож. Заметим, что \textit{ни один} Крузо не может судить об адекватности своих слов реальному миру, а значит проводимое различие не может быть различием между реальностью и тем что ''кажется''.

Как же он тогда его проводит? Пусть он сначала узнает, что объекты перемещаются самыми разными способами, и расплываются, когда он нажимает на боковые части своих глаз, а также меняют цвета при смене освещения. Как же он может рассортировать объекты на ''лишь таковыми кажущиеся'' и ''действительно являющиеся такими''. Полная об этом история, конечно, очень сложна\footnote{Она конечно же различна для разных чувств и включает разные виды автоматических субличностных способностей.}, но я ведь и не хочу ее здесь всю давать --- я хочу дать лишь историю \textit{иллюстративную}, показывающую существование способа Крузо-5 проводить словесные различия между действительным и кажимостями, и понимать и исправлять свои насчет этого ошибки.

Ответ мой, думаю, не удивит дочитавших до этого места: Крузо-5, как уже отмечалось, обладает диспозициями ИКПЯП. Он, как правило, описывает объект как он есть а не каким он ему лишь является, и удается ему это все лучше. Лучшие на данный момент смысловые побуждения --- не обязательно те, которые у него \textit{сейчас}, ведь он можеть быть \textit{сейчас} уставшим, и, осознавая это, не доверять текущим кажимостям. Речь скорее о побуждениях в кажущихся ему наиболее оптимальными состояниях.

Важно, что проводимое Крузо-5 различие между кажущимся и действительным --- не систематично, он не способен научиться ему целиком и сразу, и признает возможность ошибаться в нем. То есть как бы он сейчас не представлял объекты, он знает, что может затем признать за собой ошибку, и как действительное может оказаться кажимостью, так и кажимость --- действительным.

Как я уже указал, это различие видимости и реальности лежит в основе возможности Крузо-5 говорить о себе как о совершившем ошибку. Его самоисправления сопровождаются описанием своих прежних заявлений как ошибочных по сравнению с текущими, и, конечно же, последовательность так различающего ошибки и их исправления Крузо-5 приводит к диспозициям ИКПЯП.

Здесь, впрочем, не место подробно останавливаться на этом и уточнять чтобы приблизить способы Крузо-5 говорить и думать об ошибках к таковым у нас. Но учитывая мое описание, легко охарактеризовать различение кажимости и реальности касательно значений своих слов на фоне ошибок и исправлений. Он уверен, что применяет слова к совокупностям вещей и что называет разные вещи. Раз он уже признает возможность чего-то показаться чем-то иным чем оно на самом деле (и он может потом в этом отношении исправиться), то, значит, он понимает свою возможность верить что нечто --- кокос (потому что таким ему \textit{представляется}), когда на самом деле это не кокос. Исправляясь же, он считает, что вместо кокоса это оказался юкос, и теперь он \textit{знает} что это на самом деле.

Решающее значение для способности говорить о кажимости и действительности --- косвенное осознание своих диспозиций, ведь иначе --- осозновай он все время прямо (интроспективно) свои диспозиции --- у него бы и мысли не возникло делить представления на кажимости и действительность и вместо этого он бы сосредоточен был (как Крузо-2 и Крузо-4) на наиболее успешных изменениях в своих диспозициях. Но для Крузо-5 это не вариант, ведь он не осознает актуальных таких изменений, а может лишь впоследствии о них подумать --- подумать, почему в ходе или после определенных своих действий называет нечто кокосом. Он может даже признать способы классифицировать вещи обусловленными только его диспозициями, и никак не тем, какой ''на самом деле'' мир.

Давайте, однако, сосредоточимся на менее философски настроенном Крузо-5. Думая о своих практиках, он видит, что использует слова для группировки предметов по их сходству, и эти сходства иногда понимает правильно, а иногда --- нет. И вместо того, чтобы думать, будто успех заставляет его группировать предметы иначе, чем раньше, он ''ставит телегу впереди лошади'': думает, будто успех зависит от его правильной --- \textit{как в реальном мире} --- этих предметов группировки. Он, значит, склонен говорить ''Мне лучше понять, что этот предмет --- кокос, прежде чем заберусь на дерево''.

Говоря так о своих ошибках, ему естественно использовать придуманные им слова ''кокос'', ''юкос'' и т.д. для обозначения различных предметов, и слова ''1'', ''2'' и т.д. для обозначения различных их количеств. Не допустив ошибок, он их правильно обозначает --- правильно обозначает, к чему относятся эти слова. Он, конечно, думает о своих диспозициях как о \textit{ненадежных} --- в любой момент он может ошибиться. И он думает о себе как о понимающем, когда хуже обычного распознает кокосы или их количество --- замечает, что с тренировкой справляется с этим все лучше.

С точки зрения Бога, ''практика'' Крузо-5 заключается в следующем: он меняет способы решения своих задач --- иногда в мелочах, а иногда значительно --- и распознает (сознательно или несознательно), когда изменения эти приводят к положительному изменению его кривой успеха, и так со временем он может развить набор диспозиций до диспозиций ИКПЯП. То есть большую часть времени, экспериментируя с модификациями способов решения задач, он имеет более одного варианта развития своих диспозиций, и если сумеет правильно их развить, то это положительно скажется на кривой его успеха --- что он также может распознать. Говоря о нескольких вариантах, я имею в виду, что, например, он может считать точно или быстро, может пересчитывать предметы небольшими группами и суммировать их количества или же считать их любым другим способом. Может выполнять задание тщательно, но медленно, или торопиться, и т.д.. То есть в его действиях важно, что изменения в его поведении влияют на его ответы. Менять же свои действия его заставляет результирующая определенный способ решения задачи кривизна изгиба кривой успеха.

Это с точки зрения Бога, точка же зрения самого Крузо другая: он все лучше распознает виды предметов и лучше справляется с задачами, по мере практики делает меньше ошибок.

Заметьте, что использование им нормативных выражений (''исправлять свои ошибки'', ''быть правым'' и ''правильно выполнять задачу'') --- последовательно и согласуется с его приватной языковой практикой благодаря диспозициям ИКПЯП. То есть он исправляет себя и без его ведома ему это полезно из-за характера изменений в его диспозициях --- изменения эти обеспечивают положительную кривую успеха.

Крузо-5, как я уже указывал, серьезно заблуждается относительно своего применения слов к предметам в мире. С точки зрения Бога, его слова и то на что они ссылаются --- колеблется, хотя он этого и не признает, думая, что они фиксированы в своих референциях, но что диспозиции его могут приводить к ошибкам. Поэтому его представления о собственном языке во многом совпадают с нашими. Он, конечно, уверен, что \textit{понимает}, что означают его слова, что может распознавать совокупности одинаковых вещей несмотря даже на свою склонность к ошибкам, что знает, какие функции выделяют его арифметические изобретения. Найдя правила для вычислений, он думает, что теперь всегда им следует, что эти функции определены \textit{не} в терминах его диспозиций, и подкрепляет это возможностью самоисправления. И действительно, он, конечно, может исправиться позже, и даже прямо во время вызывающего ошибки состояния --- его осознавая.

Крузо-5, как я ранее указывал, останется скорее всего удовлетворенным этими своими убеждениями касательно практики приватного языка, если его мышление не примет необычный ''философский'' оборот --- только тогда он сможет озадачиться вопросами вроде ''Откуда мне знать разницу между действительным и кажимым?'' и ''Что кроме моих диспозиций могло бы определять значение слова ''кокос''?''.

Еще раз подчеркну важное изменения точки зрения: Крузо-5 считает причиной своего успеха лучшее распознание схожих предметов, но на самом деле все \textit{наоборот}: это его склонность группировать вещи так субличностно меняется. Направляет процесс --- делая его непроизвольным --- вызванность изменений в его диспозициях большим успехом к которому они \textit{приводят}.

Часто утверждают, что бессвязной практику приватного языка делает невозможность ошибки --- Крузо изолирован, а для понятия правильности нужен стандарт, с которым он свою практику мог бы сравнить, и стандарт этот, казалось бы, должен быть внешним.

Но, вообще-то, это все не нужно --- достаточно понимания Крузо опровержимости своих слов (возможность, значит, в них ошибки) и последовательного с течением времени самоисправления. То есть нужна непроизвольность самоисправления и получение пользы от него.

Думаю, в этом и предыдущем разделах я показал, что, во-первых, обладающий определенными типами диспозиций и способностями их изменять и затем оценивать влияние этих изменений на успешность навигации по миру, Крузо способен действовать в направлении получения все большей пользы, и, во-вторых, если его диспозиции [интроспективно] ему доступны, то он будет говорить не об изменениях --- но об исправлениях своих способов классификации предметов и решения задач, будет описывать себя как обучающегося лучше распознавать вещи.

\section{Не принуждаясь правилами --- но руководствуясь}

В некотором смысле Крузо-5 настолько хорош, насколько Крузо вообще могут быть, ведь он даже неотличим от нас в некоторых аспектах. Чтобы обосновать это, я в следующей главе обращусь к другим аспектам его языка, показав, что язык его нуждается в анализе с точки зрения условий истинности, а не утверждаемости, и что приписываемая им своему языку семантика --- стандартная и обусловлена истинностью.

Но в оставшейся части этого раздела хочу еще несколько громче заявить о сравнимости опыта Крузо-5 при следовании правилам подсчета с таковым у нас. Это важно потому что основной факт, на который указывает Крипке и который и я обсуждал в первой главе --- это наше ощущение правилами именно руководства, а не принуждения.\footnote{Крипке (1982, 17) пишет: ''Обычно, рассматривая математическое правило вроде сложения, мы думаем, что им руководствуемся в каждом новом нужном случае''.} Этот факт, я утверждаю, прямо связан с субличностными диспозициями, которые и порождают смысловые диспозиции в основе нашего счетного поведения. И поскольку доступ Крузо-5 к своим диспозициям аналогичен нашему --- он также чувствует себя руководствующимся, а не принуждаемым. Когда чьи-то диспозиции субличностны, он может сосредоточиться на предметах в мире --- а не на чем-то внутри себя вроде тех же диспозиций --- может осознать лишь свою прежних впечатлений об этих предметах коррекцию, и отсюда феноменологическое впечатление руководства и возникает. У любого была бы точка зрения совсем другая --- знай он о своих диспозициях и том, как пытается их менять для большего успеха. Только в этом впечатления Крузо-4 о соотношении между диспозициями, словами и миром отличаются от таковых у Крузо-5.

Вспомним пример из Раздела 2.5, где человек при счете испытывает яркую визуализацию цифры и явное желание ее назвать --- это и есть неведомое нам субличностное побуждение, укоренившееся в нас вместе со способностью считать. Побуждение это, кстати, не обязательно крайне убедительно --- возможно, ему легко противостоять. (Поэтому ''принуждение'' --- слишком мелодраматическое описание.) И все же, никто не назвал бы такое ''руководством'' --- скорее выбором поддаться побуждению или сопротивляться, как в случае голода или зуда. Сознательной потребности в еде можно сопротивляться, и слова вроде ''принуждение'' здесь явно лучше подходят --- никто не говорит о своем руководствовании чувством голода касательно еды.

\section{Ошибочно ли называть стандарты слов Крузо-5 ''внешними''?}

Действительно ли Крузо-5 ошибается, думая, что использует внешние стандарты для своих слов? Не обязательно. С одной стороны, он, конечно, ошибается: думает, что его слова соответствуют миру, и что от этого стандарты и берет, что стандарт для правильного использования слова ''кокос'' --- действительный кокос. Теоретик ошибок укажет на неправильность таких суждений. Стандарты ведь исходят из собственных диспозиций Крузо-5 и того факта, что они меняются так как если бы задавались реальным внешним миром. Однако с другой стороны --- он прав. Он не понимает механизм работы референций, но это, учитывая обстоятельства, лишь мелочь. Именно его успех постепенно формирует стандарты --- и так мир (глобально) вводит набор стандартов для его использования слов. Это, конечно, никакой не референциальный магнетизм --- ни одна из обсуждаемых в Главе 3 его версий. Во-первых, его успех может быть метафизически локальным --- особый случай, который будет отброшен лишь тогда, когда его влияние на мир соответствующим образом расширится.\footnote{Это, кажется, схоже с выявлением категорий [аристотелевской] физики признанием особым случаем среды, включающей трение.} Но, все же, мир --- или, точнее, успех Крузо во взаимодействии с миром --- играет здесь важную роль в формировании диспозиций ИКПЯП, и это очевидным образом экстернализирует стандраты в мир. (Полагаю, некоторые прагматики были бы очень довольны этой второй версией истории о Крузо-5.)

Завершу это главу сравнением использования слова ''действительно'' Крузо-5 и более ранними. Ранние считали себя говорящими на диспозиционно-смысловых языках: применяли слова к тому, к чему их побуждало их применить --- вопрос ''Это действительно кокос?'' допускал в качестве ответа только тривиальные ''Да'' или ''Нет'' в зависимости от требований текущих побуждений. Крузо-5 же может использовать слово ''действительно'' на основе своего различения кажимости и реальности, может более осмысленно спросить ''Это \textit{действительно} кокос?'', ведь нынешние смысловые побуждения --- как ему кажется, по крайней мере --- не дисктуют ему конкретный ответ.

Для Бога, конечно, применение слова ''действительно'' Крузо-5 колеблется, как, впрочем, и применение им всех прочих слов. Так что Крузо-5 ошибается, говоря ''Возможно, это не кокос'' или когда пьян: ''Это не кокос, потому что раньше я видел, что это юкос''. То есть ''действительно'' продиктовано текущими диспозициями Крузо-5 --- и не более. Однако, он прав, ведь может позволить этому своему слову учитывать не только нынешние его диспозиции --- но и лучшие. Разницу между этими взглядами я рассмотрю в Главе 7, особенно в Разделах 7.2 и 7.3.

\chapter{Атрибуции истинности и ложности в приватных языках}

\qquad

\textbf{Аннотация} \quad В этой главе я продолжаю исследовать язык Крузо-5 --- изолированного следователя правилам --- и показываю, что его язык (как и наш) поддерживает атрибуции истинности и ложности. То есть ему, как и нам, полезно иметь возможность назвать некоторые группы утверждений истинными, а некоторые --- ложными. И более того, он, если захочет, сможет разработать истинностно-кондициональную семантику для своего языка --- и это несмотря на подкрепленность фактами соответствия не всех предложений его языка. Все это, конечно, применимо и к нашим естественным языкам.

\qquad

\section{Как Крузо-5 говорит об истинности и ложности}

Крузо-5, как мы видели, имеет последовательную практику разговоров о тиграх, кокосах, белках и прочих предметах мира. Может сказать, например: ''Там три тигра'' или ''На этом дереве шесть кокосов''. Но может сказать и больше: ''То, что там три тигра --- истина'' или ''Неправда, что на этом дереве семь кокосов''. Может также сказать, что ''думал, что там было три тигра, но это ложь --- на самом деле их там \textit{четыре}''. Крузо-5 может почти так же как мы говорить об истинности и ложности, потому что обладает поддерживаемой диспозициями ИКПЯП последовательной практикой применения своих слов, основанной на ошибках и исправлениях.

Впрочем до сих пор позволяемое мне им обращение к истинности и ложности избыточно --- его можно устранить, если изложить содержащее его высказывание иначе. Он может сказать, например: ''Там не три тигра'' вместо ''Неправда, что там три тигра'', и истина здесь не является существенной для целевого выражения. \textit{Мы} можем говорить об истинности и ложности там, где это действительно нужно: ''Все, что Эйнштейн сказал вчера об общей теории относительности --- правда'' --- это не избыточно, потому что иначе нужно будет заменить предложение перечислением все того, что вчера сказал Эйнштейн. Конечно, если он сказал только \begin{math}E = p^2c^2 + m^2c^4\end{math}, то можем сказать ''Эйнштейн сказал ''\begin{math}E = p^2c^2 + m^2c^4\end{math}''''.

Крузо-5, оказывается, нуждается в упоминаниях истинности и ложности так же, как и мы. Он может с гордостью сказать себе, например: ''Ряд вещей, которые я вчера сказал --- правда''. Если он не помнит точно, что именно он сказал, то указание на истинность здесь не лишне, ведь он не может перечислить все эти свои вчерашние высказывания. Это важные и вовсе не избыточные использования слов ''истина'' и ''ложь'', а Крузо-5 нуждается в оценке своей своей надежности в определенных состояниях. ''Когда пьян, часто думаю о вещах, не соответствующих действительности'' --- может заключить он, например, и эта мысль, выраженная неизбыточным использованием истинности, может оказаться чрезвычайно важной для него. Это особенно актуально потому, что он (как и мы) часто помнит общие контуры события (например, что много ошибался), не помня при этом деталей (в чем \textit{именно} он был не прав --- \textit{какие} это были утверждения).

Всего этого вполне достаточно, чтобы он имел полное и устойчивое представление об истине --- по крайней мере, по мнению дефляционистов.\footnote{Есть много разновидностей дефляционистов касательно истины, необходимые различия между ними см. в Аззуни (готовится к печати). Аргументы в пользу моей любимой позиции см. в Аззуни (2006, 2010b).} То есть его самоатрибуции истинности и ложности связаны, ведь (1) основаны на предшествующей связной практике исправления ошибок и (2) его фразы об истинности и ложности связаны с другими его предложениями посредством подчинения бикондиционалам Тарского --- все предложения формы

\smallskip

''На этом дереве три кокоса'' истинны тогда и только тогда, когда на этом дереве три кокоса.

\smallskip

Крузо-5 говорит/думает об истинности и ложности так же, как и мы. Более того: он (так же, как и мы) говорит о своих словах, \textit{отсылающим} к предметам или их классам в мире. И подобно его самоприписываниям истины и лжи, он понимает, что может ошибаться в отношении имеемого в виду им под этим словом. Может ошибаться, например, что это кокос, или что все кокосы съедобны. Раз он думает, что может ошибаться, то стандартом своей правоты считает \textit{действительность}.

Он использует слово ''кокос'' для обозначения всех кокосов и только их, слово ''пять'' --- для обозначения количественного объема всех наборов из пяти объектов и только их. Он говорит о правильном понимании того, что нечто является кокосом, а иногда и о неправильной в этом убежденности. И говорит, что совершает ошибки при подсчете кокосов и позже исправляет их. Именно положительная кривая успеха, индуцируемая изменениям в его диспозициях, делает эту практику последовательной --- и вообще возможной --- наряду с его разговорами об истинности, ложности и референциях.

Представление Бога о практике приватного языка Крузо-5 сильно отличается от такового Крузо-5 (что неудивительно, конечно). С точки зрения Бога --- как я описывал это в Главе 5 --- у Крузо-5 нет ни одного стабильного языка, а есть только постоянно меняющаяся их серия, или --- что то же самое --- его слова постоянно меняются в своих референциях, референции эти колеблются. С точки зрения Бога, слова Крузо-5 относятся только к тому, что сейчас диктуют ему лучшие его диспозиции. Но экстенсии его понятий продолжают меняться из-за положительной кривой успеха/обучения. Учитывая одну из возможных фоновых метафизик, Бог понимает, почему успех Крузо-5 продолжает расти: экстенсии его понятий все больше соответствуют действительно существующим естественным видам (которые и создал Бог). Учитывая же совершенно иную метафизику --- у Бога же, в конце концов, если много вариантов на этот счет --- Бог видит успех Крузо-5 гораздо более иронично: как обусловленный эпистемическим тупиком. В этом случае Крузо-5 пользуется тем, что Бог считает сугубо локальным --- и, возможно, временным --- успехом.

Представьте случай, когда Крузо-5 обладает набором диспозиций, которые в какой-то момент не дают ответа на вопрос ''Сколько кокосов на этом дереве?'': кокосов на нем слишком много, чтобы он мог последовательно все их сосчитать. И после нескольких безуспешных попыток он вынужден признать, что полученные им ответы слишком различаются, и, значит, он не доверяет своим способностям в этой задаче. Пусть позже его диспозиции меняются, и он уже дает устойчивый --- один и тот же из раза в раз --- ответ. Думает, например: ''Только теперь я знаю, что на том дереве кокосов 57. Ну конечно, их и было всегда 57, просто до сих пор я этого не знал. Это всегда было правдой, но только сейчас я смог обнаружить ее. Ну, возможно, есть деревья с такими множеством кокосов, что я никогда не узнаю, сколько там их, но их там точно какое-то определенное число''.

Действительно ли на этом дереве было 57 кокосов до того, как он, казалось бы, развил свою способность счета достаточно для выдачи такого ответа? Это я задаю вопрос \textit{на языке Крузо-5}, не на своем. Буду писать его слова в \textsuperscript{c}таких\textsuperscript{c} кавычках, итак, вопрос. Соответствует ли факту \textsuperscript{c}на этом дереве 57 кокосов\textsuperscript{c}?\footnote{Вопрос этот грамматически верен и для моего языка --- точно так же, как ''Соответствует ли фактам фраза ''Chaque champignon est vénéneux''?''. Выражаю благодарность Дугласу Паттерсону (электронная почта 24 сентября 2009 г.) за его жалобы на более раннюю этого формулировку.}

Я рассматривал один заманчивый способ решения этом проблемы в Разделе 2.2: описать факт в терминах развития диспозиций Крузо-5 при заданных определенных обстоятельствах. Раз [когда-нибудь] он будет обладать диспозицией, диктующей \textsuperscript{c}57\textsuperscript{c} в качестве ответа на вопрос, то это касается некоторого факта. Но Крузо-5, как видели, считает, что на любой пальме строго определенное количество кокосов, и думает так даже о тех, количество кокосов на которых никогда не сможет сосчитать. Итак, поддадимся мы искушению охарактеризовать факты о количестве с точки зрения диспозиций или нет --- Крузо-5 явно выводит свои слова --- об истинности и ложности в том числе --- за рамки диспозиций. Для нас --- и для Бога --- нет никаких обосновывающих фактов, эти его утверждения подтверждающих, и нет, значит, фактов соответствия, которые могли бы их истинность определять.

Мы --- и Бог --- можем захотеть перевести термины Крузо-5 на наши языки. В частности, перевести его числовые термины, чтобы обосновать истинностное значение для его утверждения \textsuperscript{c}на этом дереве 57 кокосов\textsuperscript{c} и прочих подобных независимо от предоставления его диспозициями обосновывающих фактов. Перевод такой, однако, сталкивается с исходным парадоксом следования правилам: нет никаких оснований для легитимизации перевода его слова как нашего ''число'' вместо ''число*''. И непонятно, какие [априорные] ограничения можно [законно] наложить на практику перевода, которая оправдала бы нас (или Бога) в определенном переводе слов Крузо-5. Если исходный парадокс, описанный в Главе 1, вообще валиден, то он валиден и здесь.\footnote{Заметьте, что сила этого утверждения частично в отказе от референциального магнетизма, который я рассматривал в Главе 4. Это, впрочем, не означает, что перевод слов Крузо-5 на наш язык невозможен. Он, конечно, возможен, только не следует обманываться, думая, будто тем самым будет разрешен наш парадокс. Я еще разберу это в Разделе 6.4, но сейчас скажу вот что: парадокс применим как к нам, так и к Крузо-5. Поэтому --- при условии наличия других ограничений --- мы можем оправдать перевод, хотя и не такой, который бы дал на обосновывающие факты.}

Признающий согласованность практик приватного языка различных Крузо мог бы возразить, что показанное заключается лишь в том, что условия утверждаемости не требуют сообщества, несмотря на утверждение Витгенштейна Крипке об обратном. Раз изолированный Крузо-5 имеет набор диспозиций ИКПЯП (независимо от осознания этого), то он способен к практике приватного языка, регулируемой условиями утверждаемости. Пусть, например, такой Крузо-5 приобрел диспозиции для сложения (разумеется, только до определенного числа). Может тогда использовать это как условие утверждаемости: Крузо-5 имеет право утверждать, что он имеет в виду сложение под знаком ''плюс'' потому что уверен, что может дать правильные ответы на новые аналогичные вопросы при условии, конечно, исправлений ''его лучшим я'' --- характеризуемым относительно его диспозиций ИКПЯП.\footnote{Не думаю, что это может сработать, но и не буду останавливаться для подробностей.}

Это то, что я \textit{не} утверждаю. Вместо этого я говорю, что язык Крузо-5 можно рассматривать как истинностно-кондициональный --- очень похожий на наш. Об этом расскажу в следующий трех разделах.\footnote{В целях аргументации в рамках этой книги я придерживаюсь позиции о полезности и применимости к естественным языками истинностно-кондициональной семантики. Многие лингвисты (и философы, с лингвистикой знакомые), впрочем, не согласятся со вторым утверждением. См., например, обуждение Хомского и Петровского в Разделе 4.4. Что же касается первого утверждения, см. Аззуни (2008), где я высказываю опасения касательно возможной тривиальности [большинства] истинностно-кондиционального семантического анализа языков.}

\section{Условия истинности, двувалентность и тезис о широком неведении}

Одно и того, что нужно для поддержки утверждения об истинностно-кондициональности языка Крузо-5 --- отрицание абсолютно критического предположения Витгенштейна Крипке. Предположение это, конечно, высказывается не только им --- оно, вообще, широко распространено: условия истинности \textit{требуют} фактов соответствия. Но как утверждают дефляционисты\footnote{См. цитаты в сноске 1, посвященные как моей конкретной истинностно-дефляционистской позиции, так и обсуждению соответствующей литературы. Ниже же я несколько вкратце рассмотрю свою эту позицию.} касательно истины, предложения языка могут иметь ''условия истинности'' даже если истина не подразумевает соответствие.\footnote{Некоторые философы, отрицая связь истинности с соответствием, отрицают также и соответствие всякой истины фактам. Но нам нет нужды занимать столь радикальную позицию касательно нашего Крузо-5 --- нужно только соответствие фактам не всех его истин. И это хорошо согласуется с моей дефляционистской позицией, как вскоре я и укажу.}

Я не буду защищать свой дефляционизм от конкурирующих взглядов --- это уже сделал в других работах\footnote{См. Аззуни (2006, 2010b), например.} --- так что лишь узложу свое понимание и некоторые в его пользу аргументы. Во-первых, как отмечено в сноске 8, я широко защищал свою точку зрения в других местах, так что если понадобится дополнительная информация --- она легко доступна. Во-вторых, я понимаю интерпретацию Крипке стратегии Витгенштейна как ход ''единственной игры в городе'': мы \textit{принуждены} к скептическому социологическому решению парадокса следования правилам потому что другого выхода и нет. Как видно из обсуждения референциального магнетизма в Главе 4, реакции Льюиса как на витгенштейновский парадокс и на ''парадокс'' Патнэма схожи по духу, и потому то Льюис так часто говорит о своем ответе как ответе на \textit{reductio} --- риторический маневр, которому падражают многие его последователи. Моя цель, значит, не столько в [аргументированном] противопоставлении этим двум семействам альтернатив своего скептического решения --- в первую очередь я стремлюсь представить еще одну убедительную позицию для подрыва стратегий ''единственной игры в городе''.

Начну с использования нашего языка, вот ряд связанных утверждений о нашем --- а может, и любом --- естественном языке:

\smallskip

(1) Естественный язык содержит предложения с не-относящимися терминами.\footnote{Я использую здесь слово ''относиться'' в метафизическом смысле --- как характеризующее отношения между словами и предметами, то есть ''не-относящиеся'' не значит ''излишние''. Так, слова ''Микки Маус'' не относятся к Микки Маусу, ведь его не существует. Слово ''относится'' в этом смысле не то же что в естественном языке --- там то ''Микки Маус'' относится к Микки Маусу --- это отдельный термин. Подробный разбор такого употребления этого слова см. в Аззуни (2012a).}

(2) Но несмотря на (1), многие эти предложения истины или ложны.

(3) С (1) и (2) совместимо, что часть естественного языка можно рассматривать как согласующуюся с бивалентной классической логикой.

(4) Значит (согласно (3)) часть естественного языка можно рассматривать как обладающую стандартной истинностно-кондициональной семантикой.

(5) Язык Крузо-5 в этом отношении такой же, как и наш, поэтому (1)-(4) касается и его тоже.

\smallskip

Существенные оговорки: в (3) говорится о лишь совместимости потому что я не хочу брать на себя обязательство утверждать возможность установления даже на небольших участках естественного языка этой подчиненности классической логике --- в этом я сомневаюсь, хотя и думаю, что в начале 20-ого века произошло событие, результатом которого и стала привычка нормативно навязывать такую логику естественным языкам (в той мере, в какой они используются в науках, конечно). Это большая тема, к которой я сейчас не буду обращаться --- сейчас о ней я больше и не могу сказать.\footnote{См. Аззуни (2006, 2009, 2013), особенно последнюю главу.} Для наших целей правильной формулировкой будет ''Соображения Витгенштейна Крипке не побуждают заменять условия истинности условиями доказуемости''.

Вернемся к представленному мной в пунктах (1)--(5) наброску обоснования. В языковой практике --- частной или публичной --- часто бывает нужно говорить о предложениях как об истинных или ложных, даже если в них есть не-относящиеся термины.

Вот примеры:

\smallskip

(1) Минни Маус накогда не изображалась в кино как водопроводчик.

(2) Есть схожие с реальными людьми мультипликационные персонажи.

(3) Есть стратегии к преодолению гнева, многие из них можно найти в книгах по саморазвитию.

(4) Значимых греческих богов столько же, сколько и богинь.

(5) Джеймс Бонд изображен в романах Яна Флеминга учтивым и утонченным, хотя в последних фильмах не изображается таким.

\smallskip

Все эти предложения имеют истинностные значения, но не потому, что упоминаются ''Микки Маус'', ''Минни Маус'', ''Джеймс Бонд'': дело не в том, что ''мультипликационные персонажи'' или ''стратегии'' или ''боги'' имеют непустые экстенсии.\footnote{Крипке (2013), как известно, предерживается влиятельной конкурирующей точки зрения: вымышленные имена действуют притворно (в художественных контекстах --- романах, пьесах и т.д.). Они используются и в негативных эксистенциалах. Точка зрения Крипке, в общем, сложна и приписывает такого рода именам двусмысленность.} Принятая здесь точка зрения --- что предложения эти имеют истинностные значения несмотря на содержание не-относящихся терминов и отсутствие правильно или неправильно характеризующих их метафизических фактов (например то, чем действительно являются описываемые ими вещи).

Как я только что предположил, естественным подходом здесь являются ''создатели истины''. Предложение касается разных объектов, и именно их расположение определяет истинность или ложность о них предложений. Если предложение говорит о них так, как они есть на самом деле --- оно истинно, иначе --- ложно. Но если в нем есть не-отсылающие термины, то так о нем уже не скажешь. Вместо этого есть некая наша практика, которая им истинностные значения и присваивает --- причем она чувствительна к соответствующим аспектам мира, и потому мы вынуждены к ней прислушиваться.

Например, есть практика вымысла. Художественные произведения --- романы, например --- состоят из предложений, не имеющих истинностных значений --- авторы и читатели лишь притворяются, будто они есть. Но говоря, например, о Джеймсе Бонде в (5), мы видим, что наши предложения вполне истинностны благодаря отношению к соответствующим сюжетам. В художественных случаях это работает довольно просто: персонаж изображен определенным образом --- как и реальные люди могут быть изображены --- и истинность предложения о нем определяется соответствием этому изображению.

Здесь, впрочем, возникает одна проблема: очевидно же, что хотя многие предложения о том же Джеймсе Бонде истинностны благодаря соответствию или несоответствию выдуманной истории, несчетное количество предложений о выдуманном истинностного значения вовсе не имеют, несмотря на частое их использование. Некоторые философы быстро пришли к выводу о невозможности описания художественной литературы двувалентной классической логикой --- и что нужна некоторая ей альтернатива. Другие считают, что предложения с не-относящимися терминами следует трактовать небуквально --- как семантическое притворство, например.

Но столь радикальные стратегии нам ни к чему: мы же можем рассматривать предложения о предметной области как принадлежащие к двум категориям: с известными истинностными значениями, и с неизвестными. То есть, даже зная, что предложение такого значения не имеет, мы можем все равно говорить так, будто оно его имеет, отрицая впрочем знание об этом --- и это звучит вполне естественно. Например, читая о Джеймсе Бонде, мы можем иногда задаваться вопросом, не существовал ли он в реальности. Правильный ответ здесь в том, что ответа нет --- вообще. Художественная практика определяет факты о вымышленном персонаже их описанием автором.\footnote{За некоторыми исключениями: например, факты об известности персонажа и т.д. --- такие вызываны реакцией реальных людей на истории о них или их изображения.} Так, Джеймс Бонд в романах Яна Флеминга не изображен как разумная рептилия из космоса, и мы знаем это не потому, что в любой из этих книг есть предложение это подразумевающее, а потому, что Ян Флеминг просто не пишет о таком. Но вопрос о, например, его социопатии уже более тонок, и однозначного ответа на него не видно.

Мы выражаем незнание относительно тех истинностных значений предложений о вымышленном, что не фиксированны соответствующими нашими практиками. И как уже упомянал: мы можем говорить ''не знаю, действительно ли Джеймса Бонда изображали социопатом''. Это выражение подобно таковому при в действительности фиксированном художественной практикой ответом на этот вопрос при нашем, однако, о нем незнании, как ''не знаю, сколько дочерей Лир изображено в пьесе Шекспира, но, если надо, могу поискать про это в Википедии''. На самом деле, наши речевые практики во многих сферах жизни в принципе не способны диктовать истинностные значения тех или иных предложений. Это верно и в математике, где доказательства --- за исключением некоторых особых случаев --- превосходят истину, и в обыденной жизни тоже ничто не определяет, например, лысые некоторые люди или нет. В таких случаях мы говорим о незнании фактов --- независимо от того, есть ли факты метафизические.

Некоторые философы считают, что такой способ говорить об истинностности --- некорректен.\footnote{Прист (2011, 361), говоря о первичности истины перед доказательством в математике, пишет: ''часто бывает, что мы не можем установить ни A, ни не-A, и может показаться, что у нас есть какие-то пробелы в истинностности. Как же тогда нам оправдать использование классической логики? Она ведь подразумевает возможность либо A либо не-A, и классическое объяснение дизъюнкции, значит, не работает''.} Они утверждают, что наши логические принципы должны учитывать метафизические факты --- факты о ''действительном'' наличии определенного истинностного значения. И в этом случае пробелы будут явно учтены путем изменения логики --- вместо замалчивания их выражениями невежества. Но почему? Отказ от классической бивалентной логики довольно дорог --- не в последнюю очередь из-за ее лучшей согласованности с нашими практиками неявного вывода.\footnote{Как я указываю в работе, цитируемой в сноске 10, дело не в иллюстрированности той или иной реализации классической логики нашими практиками вывода --- просто наши методы неявного вывода лучше всего ей регламентируются, так что к ним стоит относиться как к соблюдающим эти принципы как некую норму.}

Принятие бивалентной классической логики сопровождается терминологией знания/незнания, которая обрабатывает оба вида неопределенности истинностного значения. ''Не знаем'' --- говорим мы, как если бы не знали достаточно, чтобы точно сказать, лыс ли конкретный человек, или не знали, применим ли к нему этот термин. Более того, как ясно показывают многочисленные случаи неопределенности в обычном языке, часто вовсе нет четкой линии, отделяющей лишенной метафизической истинностности выражения от тех, о которых нам просто недостаточно известно. Одна из причин --- возможность изменения класса ''определимых'' утверждений изменениями в наших методах определения истинностных значений утверждений или в применении наших терминов.\footnote{Такое понимание фразы ''не знаю'' я называю ''тезисом широкого неведения''. См. Аззуни (2010b 91-93). Этот способ заполнения пробелов в истинностных значениях может быть назван ''совместимость классической логики с фактическими пробелами в истинностных значениях посредством выражения невежества'', и впервые был применен мной в Аззуни (2000), Часть IV, парагра 6.}

Разумеется, некоторые философы утверждают, что наши способы говорить у себе как о незнающих требуют наличия соответствующего факта, о котором мы, однако, тоже не знаем. А если нет факта о применимости слова к конкретному предмету, то мы и не знаем, применимо ли оно. Но именно в таких вот случаях все говорят о невежестве (''не знаю, лысый ли он --- никто не знает''). То есть наше понимание предиката не указывает на его применимость в конкретной ситуации --- и в такие моменты корретно сказать о невежестве. То есть речь не о действительном существовании определяющего применимость факта. Это все точка зрения Уильямсона (1994).

Вполне естественно считать, что если не всякое предложение определено в своей истинности, то ''никто не имеет права утверждать истинность принципа бивалетности'', как пишет Уокер (1989, 33). Здесь следует быть осторожным. Сказать --- прямо --- что есть предложения ни истинные и ни ложные --- значит отрицать бивалентность. Потому я и решил использовать не столь жесткое ''определенно''. Но говорить о существовании предложений, истинность или ложность которых мы не знаем --- не значит сделать хоть что-нибудь из указанного выше. Сказать же, что есть предложения, истинность или ложность которых мы [скорее всего] никогда не узнаем --- также не значит отрицание бивалентности, потому что мы не делаем однозначного предсказания и вообще не связываем непознаваемость с несуществованием. То есть сказать, что нет факта истинности или ложности --- не значит сказать что его действительно нет. (Обратное бы значило отказ от бивалентности.)

Мы говорим, например: ''Петр или лысый, или нет'', и используя свойство бикондиционала Тарского (и несколько других простых принципов), можем сказать также ''''Петр лысый'' или истинно, или ложно''. Утверждается ли здесь, что истинностное значение ''Петр лысый'' или истина, или ложь? Первое утверждение --- всего лишь дизъюнкция двух утверждений. Не то что бы из такого разделения нельзя сделать ряд выводов, но все они приводят дополнительные факты касательно конкретного предложения ''Петр лысый''. А ведь пока мы не знаем, лысый он или нет, мы не можем вывести из этой дизъюнкции что-то конкретное. (И именно это многие имеют в виду, говоря о ней как о тавтологии.) Второе утверждение аналогично: нельзя сделать какой-либо конкретный вывод о Петре или его лысости.

Можно возразить, что во втором утверждении говорится об истинности или ложности --- приписывается разделительное свойство. Ответ: это все равно что сказать ''Микки Маус'' известней Хиллари Клинтон. Микки Мауса не существует, и все же это предложение о нем истинно благодаря нашим связанным с мультипликацией практикам. ''''Петр лысый'' или истинно, или ложно'' --- так же истинно именно благодаря нашим [логическим] практикам, а не неким метафизическим фактам. В чем же тогда проблема? На фразу про Петра можно жаловаться с тем же успехом, что на аналогичную про Микки Мауса --- в обоих случаях соответствие практике употребления слов об истинности и ложности превосходит отношения корреспонденции, которые пытаются навязать теоретики истины. Так тем хуже для этих теоретиков и их навязывания.

Мы, как уже отмечал, можем спорить об истинностности так же, как о приписывании свойств несуществующему. Но если мы можем принять один способ выражения, то можем и другой. А значит, можем говорить об утверждении обладающем свойством быть истинным или ложным (даже если оно на самом деле ничему из этого не соответствует --- даже если утверждений нет, как, вероятно, утверждают номиналисты) точно так же, как об обладающим таким свойством несуществующем объекте. Альтернативные способы говорить --- это, конечно же, способы отказа от стандартного бивалетного дискурса (путем введения дополнительных или ''определенных'' значений истинности). Я же предлагаю включить эту сложность в разговоры о знании и незнании и так оставить нетронутыми и наши разговоры об истинности и ложности, и связанные с ними разговоры об истинностности.\footnote{В введении (сноска 4) я высказывал озабоченность касательно фразы Крипке ''осмысленные повествовательные предложения должны претендовать на соответствие фактам'' и заметьте, что в некотором смысле это можно и принять.}

\section{Применение бикондиционалов Тарского и истинностно-кондициональной семантики к нашему языку}

Стандартная истинностно-кондициональная семантика при применении к языку без контекстно-зависимых терминов (таких как ''это'', ''он'', ''я'') поддерживается набором бикондиционалов Тарского или --- в альтернативном случае --- так называемыми кондиционализациями. Принято считать, что истинностно-кондициональные теории такого рода обеспечивают связь между языком и миром, и ключевой момент здесь тот же, что и в прошлом разделе. Как мы уже видели, говорить об истинности и ложности утверждений можно и без отношения терминов в них к чему-то в мире --- отношения эти не обязательно релевантны. Так же и истинностно-кондициональная семантика применима как к языкам с отсылающими к миру терминам, так и к языкам без таковых --- отношения языка и мира и здесь не обязаны быть релевантны. Заметьте нейтральность: ''не обязаны быть релевантны'' не значит ''не релевантны''.

Путь в некотором очень простом языке есть лишь один предикат --- ''является кокосом'' и два имени --- ''Джули'' и ''Майк'', и пусть ''кокос'' относится ко всем кокосам и только к ним, а ''Джули'' и ''Майк'' --- отдельные вещи, этими именами обозначаемые. Пусть так же в этом языке есть связка ''&'' и квантор ''есть''. Вот тогда игрушечная истинностно-кондициональная семантическая теория для этого языка:

\smallskip

''Джули --- кокос'' истинно тогда и только тогда, когда Джули --- кокос.

''Майк --- кокос'' истинно тогда и только тогда, когда Майк --- кокос.

''Что-то --- кокос'' истинно тогда и только тогда, когда что-то --- кокос.

Если S1 и S2 --- утверждения, то ''S1 и S2'' истинно тогда и только тогда, когда S1 истинно и S2 истинно.\footnote{Знакомый с такого рода теориями читатель заметит здесь много вольностей: например,, я исключил обсуждение различий между метаязыком и языком объектным, использую переменные в кавычках и скрываю подробности о кванторах, включая вопросы удовлетворения. Все это, конечно, можно исправить --- и во многих местах это сделано, в частности, в моих работах, например, в Аззуни (2008) или Аззуни (2010b).}

\smallskip

Заметьте, что последнее утверждение с миром определенно не соприкасается, а вместо этого сопоставляет значения истинности одних утверждений с таковыми значениями утверждений других. Однако первые три пункта в этом отношении смотрятся иначе. Они, конечно же, сопостовляют значения истинности: первое утверждение, например, сопоставляет истинность истинности утверждений ''''Джули --- кокос'' истинно'' с ''Джули --- кокос''. Но утверждения в правых частях этих бикондиционалов все же, кажется, описывают нечто в мире, потому что ''используются'' а не ''упомянаются''. Так, фраза ''''Джули --- кокос'' истинно'' говорит об утверждении, фраза ''Джули --- кокос'' говорит уже нечто о мире --- о том, что Джули --- оказывается --- кокос. Вывод, казалось бы, таков: подобные семантические теории обязательно говорят что-то о мире, и делают это благодаря бикондиционалам Тарского.\footnote{Это очень распространенная точка зрения, и многие в ней видят преимущество таких вот семантических теорий перед теми, что не решаются на раскавычивание. Я уже обсуждал это в Разделе 4.4. Еще два примера см. в Лепор (1983) или Ладлой (1999).}

Рассмотрим, однако, язык, который допускает утверждения подобные приведенным мной в Разделе 6.2. Пусть в нем есть предикат ''изображен водопроводчиком'' и два имени --- ''Микки Маус'' и ''Минни Маус'' (ни одно из которых к ним не относится), и пусть, конечно же, ''изображен водопроводчиком'' относится ко всем вещам, изображенным водопроводчиками, и только к ним. Пусть также в нем есть связка ''&'' и квантор ''есть''. Вот тогда вторая семантическая теория:

\smallskip

''Минни Маус изображена водопроводчиком'' истинно тогда и только тогда, когда Минни Маус изображена водопроводчиком.

''Что-то изображено водопроводчиком'' истинно тогда и только тогда, когда что-то изображено водопроводчиком.

Если S1 и S2 --- утверждения, то ''S1 и S2'' истинно тогда и только тогда, когда S1 истинно и S2 истинно.

\smallskip

Здесь, как и в прошлом случае, имеем сопоставления значений истинности разлчиных предложений, но о мире речи уже вовсе нет, ведь нет ни Микки, ни Минни Маус. А семантическая теория при этом имеет ту же форму.

Некоторые могут указать, что я не вижу в своей комнате слона: ведь данные условия истинности утверждений, не содержащих относящих к миру терминов --- не обеспечивают необходимых и достаточных условий истинности, потому что левые части утверждений связывают утверждения с несуществующим, а правые части описывают условие истинности в терминах этого несуществующего. Но ведь несуществующее не имеет свойств, как же тогда эти условия могут быть условиями истинности --- как могут раскрыть обстоятельства, в которых рассматриваемые утверждения истины или ложны? Разве не это должны делать истинностно-кондициональные теории?

Рассмотрим условие:

\smallskip

(6) ''Микки Маус изображен трубопроводчиком'' истинно тогда и только тогда, когда Микки Маус изображен трубопроводчиком.

\smallskip

Как оно должно быть связано с действительностью касательно ''Микки Маус изображен трубопроводчиком'', как оно может определять истинность заковыченного в левой части утверждения?

Оно, конечно же, не может --- во всяком случае, само по себе. И здесь мне стоит деликатно взять быка за рога (или, точнее, слона за его хобот): ''Условия истинности'' --- вводящая в заблуждение терминология, изначально использовавшаяся для описания определенного вида семантический теорий из-за ложного убеждения, что они должны определять условия истинности утверждений. С точки зрения семантики, здесь обсуждаемой, условия эти могут быть заданы набору выражений точно так же, как в прочих семантических традициях, и, более того, условия эти будут совместимы со значениями истинности, которые утверждения имеют на самом деле. Например, истинность ''''Микки Маус изображен водопроводчиком'' истинно'' коррелирует с истинностью ''Микки Маус изображен водопроводчиком'' несмотря на то, что ''Микки Маус'' ни на что в действительности не ссылается. Практика касательно вымысла, применяемая в отношении таких терминов, как ''Микки Маус'', порождает значения истинности для обоих этих предложений (или ни для одного из них --- что то же самое). То есть традиционные истинностно-кондициональные теории не дают тогда условий истинности в любом случае. Или делают это примерно так же, как

\smallskip

(7) ''Наполеону поклонялись древние греки'' истинно тогда и только тогда, когда Наполеону поклонялись древние греки.

\smallskip

Здесь утверждение в левой части истинно тогда и только тогда, когда истинно то, что в правой, и раз то, что в правой --- ложно, то и то что в левой --- тоже. И то же можно сказать о (6).

Рассмотрим теперь утверждения совсем другого рода. Слова вроде ''то'' как ''та ваза'', или ''она'' применимо к человеку в этой комнате, или даже ''стол'' понимаемое как относящее к определенному столу. Высказывания с использованием такого рода терминов описываются расширениями истинностно-кондиционального подхода. Может показаться, что рассматриваемые теории ограничиваются контекстами --- это, однако, не так. Рассмотрим сначала довольно нетрадиционный подход к семантике употребления такого рода терминов\footnote{Это подход я заимствовал из Ладлоу (1999), часть 3.}, подразумевающий использование контекстно-зависимых выражений в обеих половинах условий, например:

\smallskip

(i) ''Эта ваза уродлива'' истинно тогда и только тогда, когда эта ваза уродлива.

(ii) ''Она --- галлюцинация'' истинно тогда и только тогда, когда она --- галлюцинация.

(iii) ''Хоббит здесь --- не настоящий'' истинно тогда и только тогда, когда хоббит здесь --- не настоящий.

(iv) ''Я голоден'' истинно тогда и только тогда, когда я голоден.

\smallskip

Заметьте, что контекст произнесения обеих половин этих бикондиционалов должен быть один и тот же. Набор условий истинности, учитывающий отличие контекста интерпретатора от контекста автора смещает контекстно-зависимые выражения так:

\smallskip

(iii*) ''Хоббит здесь --- не настоящий'' истинно тогда и только тогда, когда хоббит там --- не настоящий.

(iv*) ''Я голоден'' истинно тогда и только тогда, когда он голоден.

\smallskip

Истинностно-кондициональная интерпретация чьих-то высказываний в момент их произнесения может включать контекстно-зависимость правой половины бикондиционалов, соотнося ее с контекстом интерпретатора. И такая семантика работает не только когда все термины в утверждении относятся к чему-то в мире --- напротив --- утверждения с условиями истинности любого типа допускают ни к чему не отсылающие термины по обе стороны бикондиционала. Если кто-то может находиться в контексте, где можно указать на галлюцинацию с помощью якобы не относящих ни к чему терминов, то и условие истинности эти термины использующее также корректно. Короче говоря, термины в правой половине могут обращаться к тому же, что и термины в левой, даже если термины эти не относят к чему-либо в мире.\footnote{Это, впрочем, не лучшая постановка вопроса. См. Аззуни (2012a), где избран лучший способ об этом говорить.} А контексты --- фон, на основе которого даются интерпретации использования контекстно-зависимых терминов --- не должны ограничиваться использующими лишь обозначающие нечто реальное термины.

Впрочем, как я уже упоминал, приведенный выше набросок --- не стандартный подход к контекстно-зависимым утверждениям. Содержание таких увтерждений чаще выражают с помощью ''кондиционализаторов'' --- описаний их предполагаемых референтов. И вместо (i), (ii), (iii), (iii*), (iv) и (iv*) мы тогда имеем:

\smallskip

(i**) Высказывание u в момент времени t субъектом s ''Эта ваза уродлива'' истинно тогда и только тогда, когда объект o (ваза) обозначенный субъектом s посредством ''Эта ваза'' в момент времени t существует и уродлив.

(ii**) Высказывание u в момент времени t субъектом s ''Она --- наллюцинация'' истинно тогда и только тогда, когда объект o обозначенный субъектом s посредством ''Она'' в момент времени t существует и является галлюцинацией.\footnote{Возможно, содержание слова ''она'' --- ''женщина'' --- можно понизить до уровня ''представленное женщиной''. Это зависит от свойств, которые галлицинируемым объектам следует предписывать, и это не то, что считаю важным здесь обсуждать. См. лучше главу 2 Аззуни (2010b).}

(iii**) Высказывание u в момент времени t субъектом s ''Хоббит здесь не настоящий'' истинно тогда и только тогда, когда объект o обозначенный субъектом s посредством ''Хоббит здесь'' в момент времени t существует и не является настоящим.

(iv**) Высказывание u в момент времени t субъектом s ''Я голоден'' истинно тогда и только тогда, когда s в момент времени t голоден.\footnote{(i**)-(iv**) призваны служить простой иллюстрацией широко охарактеризованного семейства подходов, так что при их формулировке я не затронул вопросы того, как именно и сколь подробно следует характеризовать кондиционализации и обошел стороной то, как именно содержание высказывания способствует условиям его истинности. Я также не разбирал вопросы несоответствия между демонстративными не отсылающими к чему-либо в мире, утверждений, когда демонстрируемый объект на момент высказывания оказался в другом месте, отчего правая часть бикондиционала стала неверна. См. Лепор и Людвиг (2000, 230-238) касательно критики различных подходов к кондиционализации таких сложных демонстративов. Впрочем, уже, пожалуй, ясно, что замечания мои по поводу не отсылающих к миру утверждений не будут заменой (i**)-(iv**) соответствующими сложными альтернативами.}

\smallskip

\textit{Краткое содержание этого раздела} \quad Из сказанного об истинностно-кондициональных семантических теориях следует, что --- хотя ''условия истинности'' действительно представляют необходимые и достаточные условия для истинности утверждений --- фактические значения истинности часто из них не следуют. Рассмотрим утверждение:

\smallskip

(8) Кларк Кент изображен идентичным Супермену почти во всех комиксах о нем.

\smallskip

Оно истинно тогда и только тогда, когда Кларк Кент изображен идентичным Супермену почти во всех комисках о нем. Супермена не существует, но есть факты о наших практиках вымысла о нем.\footnote{Сравните это с аналогичными фактами о Дональде Трампе и о том, как его изображают (в различных СМИ). Это факты о том, что действительно существует. В случае с реальными существами, причины изображать их именно такими могут иметь непосредственное отношение к ним. Но в случае Супермена и прочих вымышленных персонажей это не так --- Супермен не существует и, значит, не обладает какими-либо свойствами.} То есть наука семантики сама по себе не дает все об отношениях языка и мира --- и оно должно быть дополнено изучением самих этих отношений. А в случае предложений с ни на что не отсылающими в мире терминами, отношения эти сложны и запутанны, и потому их и нельзя характеризовать простым изложением условий истинности.

Здесь некоторые могут подумать, что семантика как наука должно охватывать условия истинности в существенно определенном смысле --- что они должны характеризовать возможные обстоятельства истинности и ложности подобно тому, как описание треугольника замкнутой трехсторонней фигурой из отрезков обуспечивает необходимые и достаточные условия бытия треугольником для всякой фигуры. А не говорить нечто вроде ''чтобы быть треугольником, необходимо и достаточно быть треугольником''. Сомневаюсь, что это верное понимание истинностно-кондициональной семантики.\footnote{Пояснения см. в Аззуни (2008)}

Могу, впрочем, привести некоторые иллюстрации своего понимания эмпирических границ семантики, которые бы не допускали бы присвоения фактических значений истинности утверждениям в контексте. Рассмотрим высказывание ''Это Санта Клаус'' ребенка, указывающего на изображение Дэниела Деннета, и сравним его с тем же, когда он указывает на изображение Санта-Клауса. То, что изображение Дэниела Деннете (а значит, и он сам) находится в области действия указательного жеста ребенка, делает его критерием истинности для его утверждения ''Это Санта Клаус'', и раз Дэниел Деннет --- не Санта Клаус, то утверждение это ложно. Но для второго случая подобного критерия истинности нет, потому что Санта Клауса не существует, и тогда именно изобразительная практика определяет истинность утверждения.

В общем, есть много асистематической сложности в отношениях между определяющим значения истинности утверждений и самими утверждениями. Описание критерия истинности ''Это Санта Клаус'', например, требует лишь упоминания определенной изобразительной практики и ее к указываемому изображению отношения. Но если ребенок, указав на изображение Санта Клауса, говорит ''Это не Микки Маус'', то опираться потребуется уже указывать на разницу в изображении несуществующих вещей. Нечто подобное просходит с ''2 + 2 = 4'' с точки зрения номиналистов --- приходится приводить соображения незаменимости в отношении целых областей прикладной математики, что может потребовать обращения и к физическим объектам, к которым она применяется.

Итак, даже если философ или лингвист считает уместным обеспечение семантической теории условий истинности, из этого еще не следует, что она должна также объяснять, как получаемые значения истинности обусловленны реальностью. Ведь отношение наших высказываний к миру имеет смысл только когда все, о чем в них сказано --- существует.\footnote{На мой взгляд, впрочем, это не работает с естественными языками. См. обсуждение Петровского и Хомского в Разделе 4.4. Есть эмпирическая возможность отношениям язык-мир быть теоретически неразрешимыми независимо от отношения используемых терминов к миру. Я позаимствовал большую часть обсуждения в этом разделе из Аззуни (2012b, 264), лишь смягчив слишком категоричные утверждения. (Последнее предложение предназначено для читателей, интересующихся такими вопросами.)}

\section{Бикондиционалы Тарского и применительно к языку Крузо-5}

Как я утверждал в Разделе 6.2, язык Крузо-5 такой же, как и наш --- так же соответствует истинностно-кондициональной семантике. С его языком дела обстоят даже лучше --- в мысленном эксперименте я оговорил, что когда его термины действительно ссылаются на мир, нет никакой сложности слов вроде ''Лондон'', о которой говорят Петровский и Хомский: термины ''кокос'' и ''юкос'', слова для счета и прочие --- имеют простые референциальные отношения к вещам и их наборам, столь любимые истинностно-кондициональными семантиками. Просто обосновывающие факты касательно Крузо-5 всегда вызывают пробелы и другие недостатки, которые хотя и меняются, но никогда не исчезают.

Значит, если он захочет, то сможет воспользоваться истинностно-кондициональной семантикой так же, как и мы, ведь его выражения истинности и ложности подобны нашим. Действительно: предполагая, что назвал два объекта в своем мире ''Джули'' и ''Майк'' и что у есть экзистенциальный квантор ''нечто'' и ''&'', он может использовать первую игрушечную теорию, которую я дал в Разделе 6.3:

\smallskip

''Джули --- кокос'' истинно тогда и только тогда, когда Джули --- кокос.

''Майк --- кокос'' истинно тогда и только тогда, когда Майк --- кокос.

''Что-то --- кокос'' истинно тогда и только тогда, когда что-то --- кокос.

Если S1 и S2 --- утверждения, то ''S1 и S2'' истинно тогда и только тогда, когда S1 истинно и S2 истинно.

\smallskip

А как насчет указания им на знание и незнание? Похоже --- как разбиралось в Разделе 6.3 --- он, как и мы, может их использовать, благодаря чему может использовать язык согласно бивалентной логике. (''На этом дереве находится строго определенное количество кокосов, хотя я и не знаю сколько --- и даже не узнаю никогда.'')

Но вот что может беспокоить: манера речи Крузо-5, кажется, применяет истинность и ложность к утверждениям даже более неопределенным, чем таковые с не относящимися к чему-либо в мире терминами. Ведь в мире есть вещи, имеющие отношение к оценке истинности таких утверждений, как

\smallskip

(1) Минни Маус никогда не изображалась в фильмах водопроводчиком.

\smallskip

Может даже показаться, что можно операционализировать процедуру поиска для определения истинности или ложности подобных утверждений. Но сравните его с утверждением о невозможности сосчитать предметы в некотором наборе (потому что их количество превосходит диспозиционные ресурсы Крузо-5) --- в этом случае ведь тоже их должно быть какое-то конкретное количество?

Да, потому что это аналогично истинности или ложности предложений с неотносящими к реальности терминами. И в этих случаях Крузо-5 --- как и мы --- говорит, что не знает.\footnote{В целях дальнейшего обсуждения в этом разделе я буду теперь отличать слова Крузо-5 с c-кавычками от гомофонного их использования Богом или нами.} Он исходит из того, что на каждом дереве с кокосами их определенное число --- это следует из его понимания отношения слова ''кокос'' ко всем кокосам и только к ним вне зависимости от его диспозиций это слово применять, и понимания применения счетных слов. Он не осознает колебание референций своих слов от колебания соответствующих им диспозиций, и потому у него есть один способ сказать о незнании, не различающий между тремя случаями: (i) когда не \textsuperscript{c}знает\textsuperscript{c} то, на что диспозиции могут дать ответ, (ii) когда не \textsuperscript{c}знает\textsuperscript{c} то, на что диспозиции смогут дать ответ, но не сейчас и (iii) когда не \textsuperscript{c}знает\textsuperscript{c} то, на что диспозиции никогда не дадут ответ.

Следует отметить, что слова \textsuperscript{c}истина\textsuperscript{c} и \textsuperscript{c}ложь\textsuperscript{c} --- это слова Крузо-5, и потому (согласно Богу), их референции колеблются так же, как референции прочих его слов. Практике с \textsuperscript{c}истина\textsuperscript{c} и \textsuperscript{c}ложь\textsuperscript{c} естественным образом соответствует указание на незнание употреблением термина \textsuperscript{c}незнание\textsuperscript{c}, не различающего между случаями (i)-(iii).

Перейдем теперь к вопросам о том, как Бог или мы должны говорить о языке Крузо-5. Должен ли Бог принять убедительность использования Крузо-5 слов \textsuperscript{c}истина\textsuperscript{c} и \textsuperscript{c}ложь\textsuperscript{c}? Он, конечно, может это сделать, используя язык самого Крузо-5. Его этого языка понимание может принять такую форму: \textsuperscript{c}Звезд либо столько-то, либо нет\textsuperscript{c}. Но Бог знает истинность/ложность каждого предложения на своем языке, тогда как в язык Крузо-5 такого не предполагает. Он знает, например, как заполнить пробел в утверждении \textsuperscript{c}Количество звезд --- \textsuperscript{c}, вообще, он всеведущ в своем языке, но всеведение это должно быть ограничено, если он собирается использовать язык Крузо-5, потому что истинность/ложность превосходит всякие факты касательно соответствия миру.

Эти утверждения, строго говоря, не имеют ничего общего с пониманием Богом языка Крузо-5, в отличие от мыслей Крузо-5, ведь то же верно даже если у Крузо-5 нет языка --- а только мысли и концепции. Эти мысли означают тогда диспозиции, и потому Бог признает их полную определенность диспозициями. Отсутствие обосновывающих фактов касательно мыслей Крузо-5 приводит к тому же и для Бога: когда Он решает понять мысли Крузо-5, его всеведение ограничивается, и он уже не знает \textsuperscript{c}количество предметов во многих наборах\textsuperscript{c}.\footnote{Здесь c-кавычки значат не цитаты Крузо-5, а характеристики его мыслей.}

У Бога, значит, есть только три варианта интерпретации языка Крузо-5. Он, конечно, может принять язык --- и, соответственно, истинностно-кондициональную семантику --- Крузо-5, и утратить всеведение (ведь на языке Крузо-5 есть много утверждений, истинность которых не установлена).

Второй же вариант --- предусмотреть произвольный перевод и присвоить значения истинности утверждениям Крузо-5 на основе этого перевода. На любом дереве --- согласно языку Крузо-5 --- есть определенное количество кокосов. Конечно, варианты перевода этого Богом совершенно произвольно из-за отсутствия обосновывающих фактов для Крузо-5. Вариантов много, и выбрать из них какой-то один не представляется возможным --- вот снова парадокс следования правилам. Из-за отсутствия обосновывающих фактов Бог не может [обоснованно] требовать непроизвольных ответов вместо этих фактов --- у Него нет оснований для сколь-нибудь определенного выбора.

Третий вариант --- принять подход к семантике, который я отстаивал в другой работе --- отказаться от необходимых и достаточных условий истинноти и вместо этого дать только необходимые и только достаточные (на основе диспозиций Крузо-5). Этот подход, в отличие от стандартных истинностно-кондициональных подходов Тарского, не приводит к переводу языка Крузо-5 на язык Бога, а позволяет Богу --- до некоторой степени --- описать семантические свойства языка Крузо-5. И этот третий подход мне больше всего нравится, хотя сейчас я не могу описать его подробней или аргументировать.\footnote{См. Аззуни (2007, 2008).}

Наш подход к языку Крузо-5 такой же, как у Бога --- у нас есть те же три варианта, что и у него. На практике, однако, третий вариант может оказаться непреодолимым, ведь мы не всеведущи и потому у нас не будет знания о характере Крузо-5 пресущего Богу. Впрочем, если наши диспозиции более или менее совпадают с таковыми Крузо-5, то мы можем выбрать второй вариант --- заниматься прямым переводом его терминов на наш язык --- и не столкнуться с произвольностью перевода, как Бог, за исключением, может, произвольности нашими собственными терминами индуцированной.

Конечно, Бог утратит всеведение и если наш язык попытается понять.

\section{Некоторые заключительные замечания}

В этой и предыдущих главах я показал, как приватные речевые практики (особенно таковые Крузо-5, что особенно ценно от его с нами сходства) совместимы с возможностью говорить об истинности и ложности благодаря превосходству истинности/ложности над фактами. Способ говорить о них доступен как в пределах собственного языка такого Крузо, так и за его пределами --- при переводе его языка на наш, например. Более того, этот его язык совместим с истинностно-кондициональной семантикой --- применительно к нему и им самим, и нами.\footnote{Следует отметить, что это предвосхищено в некотором смысле еще Витгенштейном Крипке. Крипке (1982, 86) приписывает Витгенштейну точку зрения, согласно которой возможность говорить об истинности и ложности совместима с его скептическим решением парадокса следования правилам. Значит возможность эта совместима с понятием истины вне соответствия миру. Дополнительная же идея, приписываемая мной сторонникам дефляционизма касательно истины в том, что понятие истины вне соответствия фактам --- и есть вся истина, которая кому-либо нужна. (Подробности см. в Аззуни (готовится к печати).) Такое полнофункциональное понятие истины лишь ошибочно рассматривается как требующее соответствия миру.}

Важно понимать, что любое решение парадокса следования правилам через замену условий утверждаюмости условиями истинности --- не просто устраняет убедительность приватного следования правилам, а устраняет также и убедительность важных аспектов самоприписываемой нами способности приватно мыслить. И именно поэтому решение Витгенштейном Крипке этого парадокса столь шокирует --- получается, что наши мысли могут быть убедительны только при соответствии публичным средствам коммуникации. Подразумевается, значит, своего рода публичный сентенциализм: мысль убедительно только если нормативно ограничена стандартами общества. Но из моего решения это не следует.

Как я упомянал в первой главе, многие философы пытались сделать социологический вывод из парадокса следования правилам. Эта ошибка --- на мой взгляд --- симптом слишком узкого понимания требуемого от последовательной нормативной практики исправления себя и других --- понимания, будто она трубет фиксированного стандарта, внешнего по отношению к исправляемому. Такое узкое понимание и заставляет прийти к выводу, что внешний стандарт должен быть локализован в сообществе, к которому человек принадлежит --- и это соответствует рассмотренному мной в Разделе 2.5 прямому социологическому решению. Другой же подход --- Витгенштейна Крипке --- радикальней и заменяет условия истинности условиями утверждаемости. Но из-за природы условий утверждаемости практика исправления может быть убедительно применена лишь в публичной обстановке --- а значит, тоже трубет стандартов общества.

Но думая о своих обычных практиках, мы понимаем противоречие им любого социологического решения. Ведь считается, что один может быть прав, а все остальные --- ошибаться. То есть допускается способность одного человека классифицировать предметы зависит от наличия у него определенных навыков и способности видеть те различия, которые другие видеть не способны. Мы также допускаем, что правоту такого человека могут никогда и не признать. И более того --- все общество может вечно ошибаться.\footnote{Крипке так описывает свои опасения касательно скептического решения Витгенштейна: ''Разве не может человек допускать, что общество способно тотально и всегда ошибаться, никогда эту свою ошибку не исправив?. Трудно сформулировать такое сомнение в рамках Витгенштейна, поскольку оно похоже на вопрос о возможности вечной ошибки --- а согласно ему, такого факта нет'' (Крипке (1982, 146)). Крипке затем отмечает, что избегал обширной касательно этого дискуссии, потому что ему тогда ''возможно, пришлось бы отказаться от роли защитника и толкователя в пользу роли критика''.} Можно представить такое, например, представив Пятницу, чья практика подсчета успешней, чем у любого другого человека, и потому он может обманывать их для еще пущего успеха.

Мой подход допускает возможность тотальной ошибки и ошибки всех кроме одного, потому что целесообразность исправления человеком собственных практик в нем обосновывается с точки зрения последующего большего успеха во взаимодействии с миром. То есть убедительность чьей-то правоты или неправоты основана на взаимодействии его диспозиций с окружающей средой \textit{независимо от его этого осознавания}.

У нас есть сильная интуиция, что Крузо способен разработать связный язык --- и сопровождающие его концепции --- который сможет успешно применять к объектам своего мира в соответствии со своими же намерениями, и что язык не использует стандарты общества. Именно эта интуиция делает ''аргумент о приватном языке'' столь шокирующим --- ведь, казалось бы, наше (собственное) умение считать позволяет нам считать хоть до бесконечности.\footnote{Крипке (1982, 21-22) пишет: ''Иногда, размышляя над этим, я ощущал какой-то ужас, и даже сейчас чувствую уверенность, что в моей голове есть что-то --- значение, которое я придаю знаку ''плюс'' --- которое указывает, что мне слудет делать во всех будущих случаях его использования. И я не предсказываю, что буду делать ... но указываю себе, что делать следует в соответствии с его смыслом. (Если бы я сейчас сделал прогноз своего поведения, он бы был существенен только оттого, что уже имеет смысл, с точки зрения даваемых мной самому себе инструкций, задаваться вопросом соответствия моих намерений.) Но думая о том, что у меня в голове --- какие инструкции могу там найти?''}

Подход мой, будучи скептическим, подтверждает эти интуиции лишь только в том смысле, что и мы, и Крузо-5 имеем опыт ''понимания'' как действовать дальше. В ''понимании'' этом нет фактов, могущих обосновать эту истину, и это показывает, что говоря о ''понимании'' --- как и о чем угодно другом --- мы не учитываем рамки ''фактов'' --- психологических или любых других, которые о нас можно бы было установить. ''Понимание'' --- совершенно обычное слово, используемое нами для обосзначения уверенной компетентсности в определенных задачах, так что неудивительно, что, как и в случае со словом ''кокос'', числовыми словами --- да со всеми словами вообще --- стандарты нестабильны и колеблются.\footnote{В следующей главе я вернусь к теме ''колебания'' касательно терминов ''более высокого уровня''.}

В случае ''понимания'' колебание референций включает более поздние и ''успешные'' диспозиции, обусловленные обучением сначала счету на пальцах, и затем уже с использованием различных инструментов --- устройств вроде счетов, обозначений вроде арабской нотации и т.д.. На каждом этапе освоения новых подходов к счету мы постепенно устраняем ошибки: ранние методы счета приводили к не столь оптимальной кривой успеха. Мы, однако, думаем, будто продолжаем считать так же, как всегда намеревались --- и так оно и есть. В следующей главе я рассмотрю отличия между этими двумя способами мышления о нашем языке --- и, значит, мышлении Крузо-5 о своем.

\chapter{Корреспонденсткая метафизика и перспектива Бога}

\qquad

\textbf{Аннотация} \quad В этой главе рассматриваются два оставшихся вопроса. Во-первых, опасения, что, подрывая референциальный магнетизм, я исключил возможность совместимости эмпирического для нас устройства мира с некоторой версией корреспондентской метафизики --- эмпирически обоснованного соответствия между языком и миром. Предыдущие главы, возможно, породили беспокойство еще и по поповду предположения моим походом точки зрения Бога --- рассматриваемой им же как не доступную для понимания ни нами, ни следователями приватным правилам. Проблемы эти связаны, и потому в этой главе рассмотрю их обе, а также два подхода к языку: контрастивистский и некотрастивистский, но не буду указывать на правильность какого-то из них.

\qquad

\section{Введение}

Думаю, мое решение парадокса и его следствия уже ясны, ведь ключевые их элементы изложены в предыдущих главах. Была показана необходимость в диспозициях ИКПЯП для последовательной приватной практики исправления ошибок --- такая практика позволяет приписывать своим утверждениям истинность и ложность, а также использовать истинностно-кондициональную семантику. Также в Главе 6 я указал, в какой степени посторонние способны использовать тот же аппарат в применении к языку следователя приватным правилам.

Эта глава, значит, посвящена устранению некоторых проблем, которые (я думаю) могут беспокоить некоторых философов, а также вопросам, которые уже несколько раз до этого здесь поднимались.

Одна из проблем --- якобы исключение моим подрывом референциального магнетизма возможности совместимости эмпирического для нас устройства мира с некоторой версией корреспондентской метафизики --- эмпирически обоснованного соответствия между языком и миром. Надеюсь, предыдущие главы --- особенно Глава 4 --- не создали такого впечатления. И все же, в заключение книги хочу такие мнимые проблемы рассмотреть.

Мне также указывали на основания полагать, будто подход мой методологически основан на точке зрения Бога, которую он сам и отрицает как возможную для нас и для приватным правилам следователей. Этот вопрос я рассмотрю в последующих двух разделах, затем уже вернувшись к первому.

\section{Контрастивизм и нон-котрастивизм}

Вот что может беспокоить некоторых читателей\footnote{Во многом тем, как я эту мнимую проблему излагаю, я обязан Дугласу Паттерсону (элетронное письмо 24.09.2009).}: я, очевидно, ввел стандарт объективности, взглянув глазами Бога на то, как следует применять наши слова, и стандарт этот отличается от того, как мы обыкновенно их применяем. В предыдущих главах я также предположил, что Крузо-2, Крузо-3 и Пятнице недоступна перспектива Бога. И, что еще драматичней, они не могут понять даже некоторые необходимые для характеристики перспективы Бога термины --- не могут понять, как слова могут лучше или хуже соответствовать миру. Они могут, впрочем, противопоставлять один язык другому в терминах индуцируемых ими кривых успеха, и могут даже опровергать требования одного языка с точки зрения другого благодаря переводу терминов с одного из них на другой.

Возможна такая формулировка: никто, включая нас, не может избежать своих собственных языковых рамок, концептуальных схем и пр.. Но она подразумевает, что Крузо-5 --- и мы --- не можем понять лучшее или худшее соответствие языка миру, и, значит перспективу Бога. И оттого перспектива эта становится бессмысленной. Любой контраст между ней и нашей сводится на нет, ведь мы не можем понять этот контраст в перспективе Бога. И, значит, лучшее понимание правильности слов может быть получено только с точки зрения их использования и отклонения от использования другими. Но, думаю, все немного сложнее, чем кажется на первый взгляд. Я уже высказывал свою позицию по этому вопросу --- в Разделе 3.4 --- но он заслуживает отдельного разбора.

Начну с во многом одинаковости нашей картины работы слов и таковой Крузо-5. Пусть, например, наш термин ''кокос'' соответствует кокосам --- мы уверены в этом даже несмотря на потенциальную ненадежность наших диспозиций касательно его использования, и, вообще, так понимаем взаимосвязь между (почти) всеми своими словами и диспозициями.\footnote{То, как мы от первого лица применяем определенные слова --- ''боль'', например --- кажется нам подразумевающим свободу от возможности какой-либо ошибки. Натуралистическое объяснение этому я дам позже в этой главе.}

Одна из точек зрения здесь --- контрастивизм --- предполагает, что эта книга показала неверность такой картины, и что на самом деле мы (и Крузо-5) используем постоянно меняющуюся серию идеолектов, определяемых диспозициями. Мы и Крузо-5, значит, страдаем от неизлечимой иллюзии (поскольку наши диспозиции к применению слов почти полностью субличностны) устойчивости референций наших слов.\footnote{Мы, конечно, осознаем возможность этих слов изменений в том, к чему они относятся, но рассматриваем такой процесс как спорадический и медленный --- как процесс, который странно и печально осуждал Уильям Сэфайр.} Мы --- но не Крузо-5 --- думаем о своем языке как о публичном\footnote{То есть иллюзия в понимании своего языка как общественного объекта --- в отсутствии ощущения, что каждый из нас говорит на своем собственном идиолекте, который, в лучшем случае, достаточно и нужным образом похож на используемые собеседниками. См. Аззуни (2013).}, хотя согласны с устойчивостью референций своих слов. Но из-за больших различий в знаниях (как среди носителей одного языка, так и --- с течением времени --- для любого конкретного носителя) мы сильно различаемся в способностях правильно применять эти слова. Мы думаем, например, что ''кокос'' не меняет своего значения, и что со временем мы можем разве что узнать, что некоторые вещи, прежди называемые кокосами, на самом деле были не кокосами --- и наоборот. Согласно контрастивизму, мы ошибочно понимаем факт колебания значений своих слов как эпистемический, думая, что это знания об объектах соответствующих нашим словам колеблются, а не слов этих референции.

Этим заблуждениям соответствует семантическая теория, ошибочно нами к нашему языку обыкновенно применяемая. Теория эта предполагает, например, бивалентность повествовательных предложений определенного класса и возможность их описания принципами Тарского.

Согласно контрастивизму, на самом деле все иначе: диспозиции ИКПЯП порождают смысловые диспозиции для каждого предложения S: определенные обстоятельства, в которых S будет утверждаться, отрицаться, или ответом будет, например,  ''Я не знаю''. Эти смысловые диспозиции со временем меняются из-за индуцируемости этими изменениями большего успеха взаимодействия с миром и друг другом. Природа этих изменений такова, что (i) заставляет их наслаждаться иллюзиями общего стабильного языка и (ii) позволяет их носителям говорить об истинности и ложности утверждений на этом мнимом языке и об этого условиях. В действительности, однако, утверждения таких условий истинности не имеют --- по крайней мере тех, что представлены бикондиционалами Тарского, а вместо этого возникают для каждого идиолекта из диспозиций ИКПЯП говорящего на нем.\footnote{Заметьте, что ''условия истинности'' в понимании контрастивиста не соответствуют использованию слов ''истина'' и ''ложь''. Говорящие не считают свои утверждения истинными от склонности к их произнесению --- потому что, согласно контрастивисту, они предполагают ложное представление о своем языке.}

Здесь можно подумать, что я слишком уж узко описал ресурсы диспозиций ИКПЯП для обоснования семантических теорий. Мы можем взглянуть на них в хронологическом порядке и сформировать условия истинности согласно последним из них --- ведь только в отношении наиболее позндих диспозиций и только тогда, когда они в итоге стабилизируются касательно определенных утверждений, мы сможем говорить о действительных значениях истинности для этих утверждений. Но при этом многие утверждения по-прежнему остаются без значений истинности и более того --- мы обычно не считаем устойчивый вердикт необходимо правильным.\footnote{Единственное, что мы можем сделать --- это охарактеризовать условия истинности предложений с точки зрения наших диспозиций к их утверждению или отрицанию без совершения ''ошибки'', но это значило бы вновь столкнуться с парадоксом на уровне семантической теории: понятие ''ошибки'' может иметь содержание только в отношении диспозиций говорящих в какой-то уже другой момент.}

Наконец, контрастивист диагностирует определенную ошибку касательно стандартов наши слов: уверенность, что мир предоставляет нормативные стандарты для их применения (слово ''кокос'', например, верно применяется только тогда, когда применяется к кокосам). Согласно контрастивисту, наши стандарты возникают из диспозиций и диспозиций этих развития в ответ на некое влияние мира.

Контрастивизм, значит, постулирует резкий контраст между реальностью наших языков и их определенностью диспозициями и иллюзией у языков этих носителей касательно них. Контрастивизм я, кстати, неоднократно приписывал Богу, что, впрочем, не делает его правильным (???). Нон-контрастивист же отрицает, что взгляд Бога на наши языки подразумевает ложность всякого другого на них взгляда. Разница здесь скорее как между лицами как их обычно видим и как выглядит их анатомия под кожей. Пекспектива с точки зрения диспозиций ИКПЯП (с точки зрения, значит, ''анатомии референции'') описывает то, что можно назвать ''инженерией'' референции. Но, несмотря на эту перспективу, на слова все же относятся к тому, к чему они относятся --- в частности из-за позитивного развития коллективных диспозиций ИКПЯП. Нон-контрастивизм предполагает, что слово ''кокос'', например, относится к кокосам --- это референтное отношение между родовым термином и объектами в мире, которое и называется референцией. ''Отношение соответствия'' --- это отношение между слово ''кокос'' и определенными объектами мира --- кокосами.

Вспомним различие между обосновывающими фактами и фактами соответствия, данное в Главе 1. Нон-контрастивист утверждает, что отношение соответствия лишь частично поддерживается обосновывающими фактами и, значит, лишь частично реализуется фактами соответствия. То есть отсутствие (определенных) фактов соответствия --- из-за отсутствия обосновывающих фактов у отдельных лиц (взятых индивидуально или вместе). Во-первых, как мы видели, диспозиции --- индивидуальные или коллективные --- к применению слов могут вообще ничего не определять касательно большого количества вещей. Во-вторых, диспозиции могут ошибочно определить (с более позней точки зрения), чем являются предметы.

Описывая так отношение референции, нон-контрастивист просто распространяет и на само слово ''референция'' наш способ говорить о количестве кокосов, например. Поэтому, говорит он, не следует отрицать ни отношение слова ''кокос'' к кокосам, ни характеристику словом ''относится'' референтного отношения --- ведь только некоторые случаи такого отношения выходят за рамки фактов.

Чтобы прояснить точку зрения нон-контрастивизма, следует отличить применение слова от того, к чему оно относится. То, к чему оно (сейчас) применено --- это то, к чему [коллективные] диспозиции его пользователей заставляют применить его. Согласно нон-контрастивизму, колеблится не референция слов --- то есть речь не о постоянной замене слов новыми в новых языках --- а этих слов применение. Референции же не меняются (если только слово не отброшено или не изменено в своей референции явно).

Нон-контрастивист не отрицает, что происходит изменение значения, но считает, что есть резкая разница между изменениями в применении слова и изменениями в его значении. Стандарты применения слов --- а значит и то, как распознаются ошибки --- следует характеризовать в зависимости от колебания применения этих слов, а не их референции. И потому --- как и из перспективы Бога --- стандарты применения слов и колеблются. Бог прав касательно нашего применения слов и признания нами ошибок, но --- как нон-контрастивисты утверждают --- мы не должны концептуально связывать нашу практику реализаций стандартов этих с тем, на что указывают слова.

Как я указывал в Разделе 5.4, это не значит отсутствие внешних стандартов употребления слов. Теории референциального магнетизма потому и терпят неудачу, что пытаются обеспечить локальные характеристики внешних стандартов слов, передаваемых от слова к миру: именно кокосы --- то, что обеспечивает стандарты для использования слова ''кокос'' --- это, конечно, неверно. Если наше слово ''кокос'' после некоторого колебания стабилизируется на определенном наборе объектов --- это удача, позволяющая хорошо согласовать наши диспозиции его применять и мир. Но удачу эту нельзя рассматривать как играющую ключевую роль --- какую-либо роль вообще --- в установлении стандартов для референций наших слов. Это удачное соответствие, значит, не может объяснить --- или обосновать --- эти стандарты. Стандарты эти были бы в силе, даже если такое вот удачное положение не возникло, ведь мир в любом случае вмешивается и устанавливает стандарты только в смысле вознаграждения или наказания нас за результаты наших диспозиций и этих диспозиций с течением времени изменение. И, значит, у нас со временем все получается только если мы можем вызывать в себе диспозиции ИКПЯП, которые и индуцируют положительность кривой успеха.

Согласно нон-контрастивисту, мы понимаем применение слова как его референцию, когда больше не допускаются ошибки. Но ''ошибка'' здесь аналогична ''незнанию'' как о нем говорилось ранее --- слова выходят за рамки любого описания --- сейчас или когда-либо еще --- того, как мы могли бы изменить свои диспозиции, чтобы получить другой --- ''правильный'' --- ответ. (Все наши слова, согласно нон-контрастивисту, в лучшем случае частично подтверждаются фактами соответствия тому, к чему отсылают.) Таким образом, эта негативная характеристика референции слова позволяет говорить нам, например, о существовании определенного количества предметов (независимо от существования соответствующих обосновывающих фактов, указывающих, какое именно это количество). Аналогично мы можем говорить и о словах касательно наборов объектов, независимо от существования обосновывающих фактов, которые это референтное отношение бы подтвердили.

Есть, в общем, разные возможности. [Коллективные] диспозиции могут колебаться, прежде чем стабилизируются, и наша склонность применять это слово уже больше не изменится (хотя, конечно, это не относится к нашим диспозициям применять прочие слова). Более того, вполне возможно, что эти диспозиции фактически характеризуют то, к чему слово и отсылает. Такие слова как ''боль'' достаточно стабильны сразу --- рассматриваемые состояния и диспозиции использовать определенные слова (или иметь определенные понятия) уже достаточно скоординированны в ходе эволюции. То есть может существовать тесная причинная связь между определенными состояниями субъекта и его склонностью применять слово ''боль'', например. Такого рода ограничение локальности обычно применимо к интроспективным терминам, и может объяснить их большую устойчивость в сравнении с применяемыми касательно ''внешного'' субъекту. Это, впрочем, не значит, что диспозиции к применению эти интроспективно-устойчивых слов не могут быть дестабилизированны --- неврологические расстройства и искусственно вызванные переживания указывают на обратное.

Другой случай --- стабилизация слова в его применении касательно некоторых предметов при все еще колебании в применении касательно других. Например, наша способность применять все большие количественные слова к все большим наборам предметов --- такой случай. Этот растущий набор диспозиций, вполне вероятно, характеризует все большие и большие коллекции предметов, к которым можно применить количественные слова. Могу, наконец, быть и слова, применение которых никогда не становится стабильным. Думаю, некоторые термины неудавшихся научных теорий могут именно так и выглядеть, и то же верно, пожалуй, для таких терминов как ''призрак'' или ''ведьма''.

Если слово применимо с пользой, то потому, что наши диспозиции его применять устойчивы касательно хотя бы некоторых предметов и со временем могут стать устойчивыми касательно каких-нибудь еще. Но всегда есть подходящие области применения слов, которые ускользают от стабильности --- и даже от применения --- и они всегда будут таковыми. И все же, как мы говорим о возможности чего-то быть кокосом или нет вне зависимости от определнности ответа диспозициями --- так же говорим и об отношении слов к другим предметам вне зависимости от наличия обосновывающих фактов, которые бы подтвердили факты соответствия, определяющие это референтное отношение.

В Главе 6 я рассматривал механизм критериев истины (truth-makers) --- определенность истинности утверждения существованием описываемых им предметов. Есть много разных версий этого представления, но все они предполагают невозможность истинности или ложности утверждения о том, чего не существует. Если утверждение истинно/ложно, то есть нечто, в отношении чего оно истинно/ложно. Думаю, следует сопротивляться такому предположению, и нон-контрастивизм дает возможность это делать --- способ, отличающийся от аргументов, приведенных мной в других работах. Нет фактов соответствия слова ''кокос'', например, потому что обосновывающие факты --- о нашем этого слова использовании --- такие факты не определяют. И все же, утверждения о соответствии этого слова предметам истинны или ложны.

А все потому, что слово ''ссылаться'' --- да и все прочие семантические понятия, если уж на то пошло --- часть нашего языка (как соответствующие понятия Крузо-5 --- часть его языка). Поэтому бивалентность применима и к этим терминам, и к идиомам ''знания''/''незнания'', которые я обсуждал в Главе 6.

Здесь следует отметить, что нон-контрастивизм не ограничивается утверждениями и языком --- он распространяется на мысли и концепции. Мы понимаем, что наша концепция кокоса заключается в соответствии все кокосам и только им --- даже если нет фактов, определяющих, что такое все кокосы и только они). То есть в наши концепции --- в наше понимание концепций --- заложена их применимость к тому, к чему они применимы, хотя обосновывающие факты о нас не определяют соответствующие условия соответствия (те самые необходимые и достаточные условия) для классификации всего в мире как корректного или некорректного объекта применения этой концепции. Проще говоря, мы считаем себя способными усвоить свои концепции даже если не слишком хороши в их успешном применении.

Последнее, что здесь отмечу --- отличную сочетаемость подхода Тарского к понятиям ''истина'' и ''отсылает'' и истинностно-кондициональной семантики с нон-контрастивизмом. Сочетаемость эта обязана успешному функционированию дефляционистских понятий ''истина'' и ''отсылает'' вне зависимости от терминов, к которым они применяются. Поскольку ''истина'' и ''отсылает'', взятые в рамках истинностно-кондициональных подходов, нечувствительны к присутствию или отсутствию или природе отношений язык-мир --- они нечувствительно и к возможной неспобосности обосновывающих фактов подтвердить правильность использования слов.

\end{document}

