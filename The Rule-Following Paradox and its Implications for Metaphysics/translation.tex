\documentclass[12pt]{book}

\usepackage{polyglossia}
\setdefaultlanguage[indentfirst=true,forceheadingpunctuation=false]{russian}
\setotherlanguages{english}

\RequirePackage[a4paper, headsep = 0.5 \headsep, left=2.5cm, right=2.1cm, top=2cm, bottom=2.1cm]{geometry}

\setmainfont{Times New Roman}
\newfontfamily\cyrillicfont{Times New Roman}[Script=Cyrillic]

\title{Парадокс следования правилам и его последствия для метафизики}
\author{Джоди Аззуни}
\date{2017}
\begin{document}

\maketitle

\tableofcontents

\chapter{Введение}

\qquad

\textbf{Аннотация} \quad Здесь я дам краткое содержание всей книги и проясню несколько методологических моментов, а также укажу причины избегания таких терминов как ''понимание'', ''значение'' и ''факт''. Я опишу некоторые существенные различия моего понимания соблюдения правил от такового Крипке, например - и в этом разница между моим подходом и подходом большинства занимающихся этой темой философов - я уделяю большое внимание ''следованию правилам'' в его применении к задачам в ходе взаимодействия с миром, и этого его отличия от, например, разного рода арифметических упражнений вроде сложения чисел.

\qquad

Центральным теме этой книги - ''парадоксу следования правилам'' - является термин ''понимание'': как в выражениях ''она понимает, как складывать'', ''он понимает, как продолжать считать числа больше десяти'' и ''она понимает эти концепции''. Уместно предупреждение касательно такой методологии: ''понимание'' здесь обозначает поразительно сложную человеческую способность, включающуюю субличностные, сознательные/феноменологические, социологические и нормативные элементы. Понятие это слишком сложное для \textit{использования} в философском анализе - должно скорее быть его \textit{объектом}. Это, впрочем, едва ли единственное такое опасное в своей сложности слово: еще одной ловушкой могут стать ''объяснение'' и ''смысл''. Причем даже повторяющиеся и очевидные неудачи, похоже, не мешают философам продолжать попытки строить сколь-нибудь точный анализ на таком песчаном фундаменте.

Куайн редко комментирует свой подход к занятию философией - даже его признания натурализма и экстенсионализма довольно специфичны по содержанию и применению - и тем не менее, я нашел у него редкую методологическую медитацию, стоющую \textit{запоминания}. Куайн (1981, 184) пишет:

\smallskip

\textit{Согласно описанию моих взглядов Шуденфреем: ''предложения заменили мысли, а склонность к согласию заменила веру''. Имеет ли он ввиду отсутствие для меня большего чем вера и мысли? Читая дальше, понимаю, что так и есть. Значит он неверно меня понял.}

\smallskip

Куайн продолжает:

\smallskip

\textit{Позиция моя в том, что понятия мысли и веры - достойные объекты философского и научного анализа - и столь же плохи как этого анализа инструменты. Если кто-то решается их таковыми использовать, то загадкой для меня остается что же, как он счел, более нуждается в анализе чем они сами.}

\smallskip

Но давайте все же придерживаться термина ''понимание'', раз уж именно ему эта книга непосредственно посвящена. Общее для нескольких философских традиций предположение основанности всякого понимания набора концепций на понимании правил этими концепциями управляющих. Это, похоже, отражает понимание значений слов и, например, концепций из таких простых математических практик, как счет. Хорошим пример здесь - шахматы: как только человек усвоит правила игры в шахматы, он поймет, как в них играть вне зависимости от обстоятельств.

Некоторые сторонники такого подхода говорят о нас как об обладающих ''предрассположенностью'' к пониманию этих правил и даже этих правил (будучих воплощенных в предрассположенностях) обладателях и что предрасположенности эти позволяют нам вести себя соответствующим образом в ситуациях при проявлении такого рода ''понимания'' - например, при подсчете апельсинов в корзине.

Огромное количество литературы посвящено так называемому ''парадоксу следования правилам'' - противоречию аспектов нашей практики естественной картине нашего понимания правил. Крипке (1982), например, развил масштабную философскую индустрию вокруг проблем, которые, по его мнению, Витгенштейн поднял для всякого обосновывающего понимание простых математических правил применением набора диспозиций подхода. (Крипке, впрочем, не претендует на однозначную присущесть этой точки зрения Витгенштейну; и он, и другие, понимают что она ему, возможно, не пренадлежит, откуда и пошел термин ''Крипкенштейн''. Я избегаю этого термина, предпочитая использовать такие фразы как ''Витгенштейн Крипке'' или контекст.)

Проблема следования правилам, вкратце, такова: мы понимаем себя как следующих правилам и, пройдя соответствующую подготовку, действительно на это способных. Но средства, с помощью которых мы понимаем себя способными, согласно Крипке, не подходят для этой задачи. Кандидаты на развитие такой вот способности следовать правилам это, например, интроспективное их понимание, разные привычки и склонности их выполнять в рамках соответствующих задач и т.д. - кандидаты эти показаны Крипке как неспособные гарантировать все необходимые элементы нашей компетентности в этим правилам следовании. Он отвечает на парадокс ''скептическим решением'' - допускающим неудачу ''смыслового скептика'' и меняющим понимание нашей практики следования правилам несколькими важными способами. Среди этих способов отказ от понимания значения в терминах ''условий истинности'' и замена их ''условиями утверждаемости'' - условий с точки зрения соответствия индивидуальных практик следования правилам таковым сообщества.

Версия парадокса Витгенштейна о следовании правилам, предложенная Крипке, оказала большое влияние. Один из вопросов которые я буду рассматривать - это как она была понята как подрывающая индивидуалистскую картину математической практики - точку зрения, согласно которой отдельные люди (буду называть их ''\textit{Робинзонами Крузо}'') независимо от сообщества могут осмысленно заниматься математикой и, значит, иметь ''приватные языки''. Парадокс отрицает применимость фраз вроде ''правильный подсчет'' к таким людям, потому что эти нормативные понятия осмысленно применимы только относительно стандартов сообщества.

Одним из исходных элементов моего альтернативного ''скептического решения'' будет отрицание этого шокирующего следствия: оно не следует даже если витгенштейнианские возражения Крипке против диспозиционных подходов к следованию правилам по большей части верны. Это потому, что мое решение не будет ставить стандарты сообщества выше индивидуальных. Более того, оно не не будет заменять условия истинности условиями утверждаемости, что важно для решения Крипке, отдающего предпочтение сообществу. Вместо этого мой подход фокусируется на том, как склонность к следованию правилам позволяет нам более или менее успешно взаимодействовать с миром. Предположим, что их практики демонстрируют склонности, которые меняются, постепенно оптимизируя успешные события (например, расчет неободимого на несколько дней количества еды). Тогда соответствие между индивидуальными следованиями правилам возникнет даже без явной связи этой практики с сообществом\footnote{Чтобы не перегружать объяснение и сделать его более ясным, я здесь пока опускаю детальное описание того, как это работает. В последующих главах эти детали будут прояснены.}.

Льюис (1983) в статье почти столь же влиятельной, как таковая Крипке (1982), связывает рассматриваемую версию парадокса с тогдашним антиреализмом Патнэма (1981) и предлагает общее решение. Он настаивает на подходе почти полностью встраивающем в ссылки на термины естественного вида (а также на математические термины вроде ''сумма'') \textit{предпосылку} метафизических ограничений на возможные их расширения и утверждает, что она \textit{необходима} для ответа и на рассматриваемый парадокс, \textit{и} на антиреализм Патнэма. Последующее поколение аналитических метафизиков назвало такой ответ ''референциальным магнетизмом''. В главе 4 я покажу, что семейство таких Льюизианских подходов к проблеме (далее подходы ''относительного магнетизма'') - не работает и даже не требуется.

Льюис не использовал термин ''референциальный магнетизм'', фраза эта, видимо, пошла от Ходса (1984, 135), вопрошающего, как так получается, что

\smallskip

\textit{мы все в конечном итоге говорим на языках, в которых из всех возможных нумераторов (то есть функций \begin{math}F\end{math} 2ого типа, переносящих концепты 1ого типа на объекты, где для любых из этих концептов \begin{math}X\end{math} и \begin{math}Y\end{math} верно \begin{math}F(X)=F(Y)\equiv (Q_{E}x)(Xx, Yx)\end{math}) словосочетание ''количество ... '' обозначает стандартный нумератор? Почему стандартный нумератор является ''референциальным магнитом'', ''подтягивающим'' ссылку с помощью этой фразы (чего не могут сделать его нестандартные конкуренты)?}

\smallskip

''Референциальный магнетизм'' - вещь тонкая, так что неудивительно, что последующее поколение философов превратно его истолковало.

Вот схема глав книги.

В главе 2 я возвращаюсь к обсуждению Крипке парадокса следования правилам для диспозиционных подходов к числовой компетентности, а также к альтернативному решению парадокса приписываемого им Витгенштейну. Соображения этой главы - в значительной степени исходные соображения Крипке. Меня при этом не беспокоят вопросы витгенштейновской экзегезы - насколько корректно Крипке присваивает парадокс Витгенштейну\footnote{По этому вопросу, впрочем, есть литература, например: Блэкбор (1984), Гольдфарб (1985, 1992), Макдауэлл (1984), Тейт (1986)} - его формулировка проблем, стоящих перед смыслодиспозиционалистом, важна вне зависимости от присущести Витгенштейну. Есть, впрочем, два изменения. Во-первых, в рассуждениях Крипке центральным примером выступает субъект, практикующий арифметическое сложение, и скептический вызов о согласованности всего известного и испытанного им ранее с возможностью им практики на самом деле чего-то от сложения отличного, я же заменяю сложение более элементарной задачей - задачей счета - и соответствующей возможностью практиковать на самом деле что-то от него отличное. Это, конечно, незначительное изменение, и Крипке (1982, 17) сам мимоходом поднимал вопрос о счете.

Второе же изменение уже существенное и ключевое для моего решения парадокса. В моем случае центральный пример о субъекте, который считает \textit{вещи}, то есть в его задаче используются \textit{не только} числа. Дети так и учатся считать: они \textit{применяют} числа к задачам распознавания количеств предметов, и на приобретение таких знаний уходит несколько лет\footnote{Тщательное изучение этого вопроса с цитированием соответствующей литературы см. в Кэри (2009) и Баттерворт (1999), глава 3, разделы 1-4. Классическим исследованием является Гельман и Галлистел (1986)}. Это важно от рассмотрения многими философами ''социальных решений'' парадокса, приоритизирующих сообщество над индивидуумом, как единственных успешных. Одна из причин такого рассмотрения - пренебрежение соображениями о применении математических концепций \textit{в мире}. Точно так же применение математики к миру находится по большей части за кулисами дискуссий Крипке и - как побочный эффект - за кулисами дискуссий многих этой литературы комментаторов.

В творчестве же самого Витгенштейна это не столь закулисно - он часто приводит примеры про людей, считающих предметы. Да и Крипке тоже обращает внимание на ''числовые размеры'' наборов подсчитываемых предметов, когда мимоходом обсуждает ''зчет'' в рамках текста Витгенштейна. Но если подсчет \textit{предметов} занимает центральное место в примерах которым скептики бросают вызов, то становятся заметны вполне последовательные формы приватно-языковых практик. В частности, \textit{понятие диспозиций приводящих к когерентным таким практикам}, которое я ввожу в главе 5, не опирается ни на что за пределами контекста применения концепций к вещам в мире. Поэтому, конечно, мое скептическое решение парадокса не годится для бестелесных картезианских сущностей, от скуки вечно считающих числа в уме.

Оставляя в стороне эти разногласия с Витгенштейном Крипке, добавлю, что я частично согласен с одной важной леммой, которую Крипке выводит и парадокса и которую он (1982, 78-79) изредка подтверждает:

\smallskip

\textit{Надо отказаться от ''естественной предпосылки'' что осмысленные повествовательные предложения должны соответствовать фактам.}

\smallskip

Добавляя:

\smallskip

\textit{Прежде чем мы сможем приступить к скептической проблеме, надо прояснить картину этого соответствия фактам.}

\smallskip

Я \textit{частично} с ним согласен - соображения следования правилам указывают на необходимость признания несоответствия \textit{некоторых} значимых повествовательных предложений фактам. Впрочем, не думаю, что это установлено для \textit{всех} таких предложений. В главе 6 я покажу, почему соображения следования правилам требуют лишь \textit{частичного} отказа от корреспондентской метафизики. Обсуждение это в то же время подкрепит мою особую форму дефляционизма относительно истины\footnote{См. Аззуни (2006) и гл. 6 этой книги. Интерпретация приведенной выше формулировки Крипке затруднена его фразов ''должен подразумеваться''. Я считаю, что осмысленные повествовательные предложение могут и даже ''должны'' претендовать на соответствие фактам - и это совместимо даже с некоторыми из тех, что не столь уж соответствуют. (Убедительность этого толкования, конечно, зависит от значения термина ''смысл'' - оно, все же, не совсем ясно, не ясно также, исключает ли ''уборка'' Крипке ''корреспондентность'' полностью или лишь частично.) См. обсуждение использования идиом знания/неведения в разд. 2, а именно сноску 15 гл. о моих способах маневрирования среди всего этого.}.

Здесь стоит отметить еще вот что. Диспозиционалистская надежда на посвященность значительной части анализа Крипке (1982) дроблению, находит \textit{основания} отношения соответствия между значимыми истинными предложениями и фактами в ''разуме'' (в широком смысле этого слова) человека, следующего правилам. Есть факты \textit{о человеке} и его \textit{возможностях}, которые и определяют отношения соответствия значимых истинных предложений фактам в мире. Но, думаю, соображения следования правилам указывают на неудачу этого диспозиционного проекта: диспозиции \textit{бессильны} в том, чего большинство диспозиционалистов от них требует, и получается так во многом по причинам, приведенным Крипке от имени Витгенштейна.

Я отвергаю, однако, главный тезис, который многие отсюда выводят. Крипке упорно говорил о себе как о всего лишь толкователе, но множество философов утверждают, будто соображения вроде представленных в главе 2 необходимо приводят к соотнесенности или воплощенности стандартов следования правилам вообще и математической практики в частности в \textit{обществе}, в котором человек эти правила изучает. Как говорит сам Крипке (1982, 109):

\smallskip

\textit{Что действительно отрицается, так это то, что можно было бы назвать ''частной моделью'' следования правилам - моделью, в которой понятие следующего данному правилу человека надо анализировать просто с точки зрения фактов об этому правилу следователе и только о нем, то есть не отсылая к сообществу.}

\smallskip

Одна из целей этой книги - показать, что несмотря на правоту Крипке относительно природы парадокса и последовавших на него ответов, модель приватного следования правилам остается этим парадоксом не затронутой. Контуры логического пространства в этой проблемной области более запутанны и тонки чем мыслитли конца прошлого века осмеливались полагать.

В общих чертах обрисовав изложение Крипке его способа решения парадокса - его, значит, интерпретацию Витгенштейна о замене условий истинности условиями утверждаемости, - я в последующих главах приведу доводы в пользу уже другого - не приоритезирующего общественные стандарты над индивидуальными - решения. Также в этой главе я покажу неуспешность диспозиционного подхода, расширяющего круг необходимых диспозиций до обладаемых всеми в обществе.

В главе 3 я рассмотрю диспозиционно-смысловые языки - действительно приватные языки. Их носители (называю их ''Крузовцами'') - в отличие от нас - относят каждое слово к тому, к чему их предрасположенности склоняют их его применять. Я далее начинаю исследование сферы применения таких языков и показываю как понятия вроде ''ошибка'', ''обоснование'' и ''концепция'' в них если и остаются, то лишь в некой скромной форме, впрочем эти результаты не столь существенны для приватного следования правилам. Во всяком случае, в этой и в 5ой главе покажу как они или в них необходимость нивелируются в таких приватных языках. Еще в этой главе я рассмотрю содержание фраз вроде ''согласованность с миром'' и ''разрезание мира по стыкам'' для носителей этих языков. Неожиданным выводом будет возможность использования таких языков для взаимодействия с миром при широком диапазоне благоприятных эмпирических обстоятельств, и более того - носители этих языков могут объективно сравнивать разные диспозиционно-смысловые языки чтобы определять, какой из них лучше взаимодействует с миром. Они, впрочем, не будут использовать понятия вроде ''лучшего соответствия'' языка миру или ''правильности'' применения слова. Здесь следует указать, что ''благоприятные эмирические обстоятельства'' значит нечто большее чем просто характеристики внешнего мира потому что она также охватывает эмпирические факты о, например, механизмах \textit{развития}\footnote{Есть забавное следствие моего решения парадокса, на которое укажу лишь в рамках этой сноски: возникает проблема с пониманием искусственных языков как неких полезных инструментов, которые, значит, можно принимать и отвергать в зависимости от целей (такое понимание описано у Карнапа (1956)). Ведь как можно сравнивать достоинства и недостатки таких языков не прибегая к некоторому метаязыку? Именно здесь эта загадка и решается.} диспозиций таких говорящих.

В главе 4 я сделаю паузу в анализе диспозиционно-смысловых языков, чтобы контраргументировать решения поставленных мной перед такими языками проблем касаемо отсылкам к наборам вещей в мире, основанные на референциальном магнетизме. Здесь же я рассмотрю три его версии. Первая понимает структуру мира метафизически обеспечивающей ресурсы для дополнения того, что люди в сообществе привносят для определения отсылок. Наши слова (концепты) имеют определенное значение, выходящее за пределы психологических и нейрофизиологических ресурсов любого человека и даже за пределы того, с чем любое таких людей сообщество могло бы справиться. Вторая понимает интерпретаторов естественных языков так, как того требует - наряду с базовой научной практикой - семантическая теория для наложения определенной референции на этих языков термины и на естественные виды как на корреляты видовых терминов этих языков. Третья понимает \textit{априорное} конститутивное навязывание естественных видов как требуемое в качестве ''единственной игры в городе'' муровскими фактами определенной референции и предполагающими определенную референцию семантическими теориям. Разумеется, ничто из этого не защищает от референциального магнетизма, что я и показываю.

В главе 5 я представляю окончательную версию моего Крузо (Крузо 5) - он психологически близок нам тем, что не понимает свой язык \textit{как} диспозиционно-смысловой, ведь его диспозиции к использованию терминов ему (как и нам) неведомы. Я показываю, однако, как его осознание возможности использования терминов для улучшения общего его благосостояния позволяет ему навязывать последовательные стандарты этой своей приватной языковой практике.

Более подробно язык Крузо-5 я изучу в главе 6, показав способ применить к нему стандартную семантику функциональной истинности и то, как он, подобно нам, естественным образом может использовать идиому истинности. Это должно продемонстрировать, что ''скептическое решение'' Крипке через замену условий истинности условиями утверждаемости - не единственное такое решение парадокса. Замена эта, впрочем, нужна для вывода о невозможности приватных языков.

В главе 7 я завершу то, что не завершил до этого: разберу методологическую роль точки зрения Бога в моем подходе к оценке последователей приватных правил. Для этого я рассмотрю два возможных взгляда на языки Крузо (и наши языки): первый понимает их как постоянно меняющиеся в отношении того, к чему относятся их термины, а второй считает референцию фиксированной. Противопоставив их, я покажу, в чем смысл точки зрения Бога\footnote{Впервые я предложил эти различные способы рассмотрения работы языка в Аззуни (2000), Часть IV.}.

Далее я рассмотрю \textit{следствия} моего решения парадокса для корреспондентской метафизики. Если кратко: эмпирически обоснованная метафизика \textit{возможна}. Что же невозможно, так это пресуппозициональная роль такой метафизики в философских объяснениях эмпирического успеха или в семантике.

Ну и наконец, я концептуально свяжу парадокс с проблемой индукции Юма.

\chapter{Версия Крипке парадокса Витгенштейна и его решение}

\qquad

\textbf{Аннотация} \quad В этой главе рассматривается описание Крипке парадокса Витгенштейна и его решение. Интерпретация Крипке вызвала довольно много критических комментариев, но меня парадокс представленный Крипке интересует в отрыве от интерпретационной корректности. Здесь я рассмотрю две дистинкции: прямые и скептические решения и обосновывающие факты и факты соответствия, а так же три ключевых момента. Во-первых - три требования Крипке для прямого решения парадокса: бесконечности, обоснования и ошибки. Во-вторых - почему прямое социологическое решение не работает. И в третьих - некоторые критические комментарии касательно следования правилам, возникшие после публикации соответствующей книги Крипке.

\qquad

\section{Постановка проблемы: три ограничения любого решения}

О том, кто успешно посчитал разные наборы объектов, мы можем сказать, что он научился считать. Это значит предположить, что он при следующих случаях будет продолжать считать ''так же'', несмотря на возможно новые виды придметов и большее их количество.\footnote{Некоторое неявное знание, нужное для успешного этой задачи выполнения, рассмотрено Гельманом и Галлистелом (1986, 77-82) как принципы подсчета: ''количественное слово'', ''нерелевантность порядка'' и ''абстрактность'': последнее слово в подсчете и есть количество подсчитанных предметов, потому порядок и вид считаемых объектов не важен. Описание процесса подсчета и трудностей преобретения соответствующего неявного знания см. в Кейси (2009, стр.241-244).} Пусть его научили системе счисления - правилу образования новых нумералов из предыдущих. Системы нумералов разительно отличаются этим от наборов словесных названий чисел в большинстве языков\footnote{Рассмотрение типов числовой лексики в разных языках см. в Баттерворт (1999, особенно стр. 52-62).}, потому что вторые обычно относятся к финитной записи чисел, требующей явного создания нового словаря для все больших чисел. Однако в системы счисления создание обозначений для все большего количества цифр уже встроено. Итак, как только субъект обрел способность считать и освоил определенную систему счисления, мы говорим что он \textit{понимает}, как считать\footnote{Дети, кстати, способны схватить неопределенную природу чисел, не выучив настоящую систему счисления. Например, Кэри (2009, 252) цитирует слова одной пятилетней девочки, представляющие спонтанное изобретение аргумента в пользу бесконечности числового ряда: ''Предположим, вы считаете тысячу самым большим числом, но вы же можете сказать тысяча и один, тысяча и два и так далее''. Я в своем изложении предполагаю, что испытуемый все же освоил систему счисления, потому что это помогает избежать сложностей с числовыми языками, превосходящими знания испытуемого о самом числе. Да и в нашей культуре дети в любом случае со временем осваивают системы счисления. Это достижение трудно еще и из-за возможной путаницы с сопоставлением терминов системы счисления со словами-числами естественного языка. (См., например, многочисленные по этому поводу результаты Деэна и его коллег. Некоторые из полученных ими данных указывают на потерю определенных способностей с в то же время сохранением других после инсультов и других специфически локализованных травм головного мозга.)}.

Пусть например теперь он пытается сосчитать предметы набора большего чем всякий до этого встречавшейся - набора из 57 предметов - и получает правильный ответ. Крипкианский скептик оспаривает, что этот ответ соответствует предыдущему пониманию задачи счета нашим испытуемым: в прошлом, он говорит, испытуемый никогда не считал - он зчитал\footnote{См. также Гельман и Галлистел (1986, 51), где рассматривается схожий интересный мысленный эксперимент.

Стратегия Крипке, заключающаяся в подрыве намерений субъекта через подрыв его намерений в предыдущих случай, подвергалась широкой критике и, как мне кажется, часто неверно понималась. Форбса (1983-1984, 225-226), например, в своей интерпретации пишет: ''утверждение об отсутствии факта о намереваемом субъектом в прошлом двусмысленно: либо у него не было определенного намерения в прошлом, либо намерение было, но неясно, такое же ли, что и сейчас''. Но контекст то явно указывает на верность первого варианта прочтения. Вызов субъекту со стороны скептика ведь в том, что, возможно, он зчитал, а не считал - вызов этот, значит, подрывает темпоральную идентичность намерений через подрыв всяких фактов о прошлом как определяющих соответствующее в настоящем. И факты о настоящем в противовес прошлому аналогичны таковым в прошлом - они так же неспособны отличить счет от зчета. С другой стороны, Богосян (1989, 515) понимает эту загадку Крипке как вопрос о том, что вообще определяет условия содержания и смысла - ''[обладание] условием корректности'', тогда как загадка Крипке лишь косвенно, через вопрос о намерениях субъекта в прошлом, касается этого. Это, конечно, явно неправильное прочтение. Как верно отмечает Форбс (1983, 226), это Райта (1980) (а не Крипке) интересует определяющее условия содержания или значения. (Форбс считает утверждение Райта ''более прямым и сложным'', чем таковое Крипке, и, думаю, он так считает из-за непонимания стратегии Крипке.)}.

Для ответа скептику что испытуемый субъект и в этот раз зчитал и, значит, его ответ про 57 согласуется с предыдущими касательно меньших наборов предметов, нужны факты о субъекте, лежащие в основе ее предыдущего и нынешнего понимания счета. Здесь для описания фактов - касающихся, например, диспозиций субъекта - фактов, которые объясняют, к чему приходит его понимание, я ввожу термин ''обосновывающие факты''.

\textit{Некоторая терминология} \quad Я отличаю ''факты соответствия'' от ''обосновывающих фактов'': первые - это предполагаемые факты внешнего мира, которым соответствуют истинные осмысленные предложения, а вторые - психологические/диспозиционные факты о субъекте или окружающей его среде, обеспечивающие его способность понимать истинные осмысленные предложения, и, значит, определяющие эти предложения как понимаемые соответствующими фактам внешнего мира. Как вскоре будет ясно, вызов смыслового скептика напрямую оказывает давление на предполагаемые обосновывающие факты, и это, в свою очередь, подрывает факты соответствия.

Слово ''факт'' похоже на те коварные слова, с указания на которые я начал общее введение. Я не собираюсь использовать его в философском смысле, в частности, говорить о сущностях-\textit{фактах}, которым соответствуют предложения (полагаю, что в метафизическом смысле \textit{нет ничего} чему бы они соответствовали). Просто в некоторых из них есть термины, которые отсылают друг к другу, и то, что эти термины обозначают, называется ''фактами''. (См., однако, главу 6 с уточнениям этой идеи для работы с распространенными случаями предложений с несоотнесенными терминами).\footnote{См. также Аззуни (2012c).} Я также не имею в виду метафизически нагруженное понятие ''граундинга'' или что там еще обсуждают во всякой сложной литературе, которая расцвела в последнее время.\footnote{См., например, Коррейя и Шнайдер (2012).} ''Факт'', ''обоснование/граундинг'' и ''обосновывающий факт'' в моем случае обычные дотеоретические термины.

Еще одна терминология требует описания: Крипке различает ''скептические'' и ''прямые'' решения парадокса следования правилам. Прямые направлены на поиск фактов, обосновывающих то или иное понимание субъектом счета. Скептические же признают, что обосновывающие факты не существуют и объясняют понимание субъектом счета как-то иначе.

\textit{Назад к диалектике} \quad Как уже отмечалось, именно предполагаемые обосновывающие факты (или система таких фактов) объясняют истинность того, что в прошлом субъект понимал, как считать, а не как зчитать. Эта модель обосновывающих фактов \textit{связана} с субъектом - либо с его психическими состояниями, либо с субличностными событями или структурами, лежащими в основе психических состояний, позволивших ему научиться считать. Обосновывающие факты о понимании субъекта должны давать ответ на вопрос: как психологические состояния субъекта - что он думал, столкнувшись с задачей - позволили ему понять, что он считает, или: как нейрофизиологические и прочие субличностные репрезентации активируются для счета.\footnote{Кэри (2009), например, предлагает «онтогенетическое» описание того, как ребенок — уже обладающий врожденными субличностными когнитивными системами (параллельное выхватывание небольших наборов и квантификаторы естественного языка), применимыми к конкретным числовым задачам — может в течение полутора лет с помощью индукции и аналогии (так называемая «Квайнианская самонастройка») понять некоторые важные свойства чисел, например бесконечность их ряда, и научиться применять их в подсчете предметов. ПОразительная особенность парадокса следования правилам, как мы увидим, в очевидной нерелевантности эмпирической гипотезы Кэри и ей подомных, предложенных учеными-когнитивистами, для его решения.}

Крипке, я считаю, налагает три ограничения на объяснение структуры обосновывающих фактов - диспозиционных или других - которые составляют понимание субъектом счета, и, значит, ответ смысловому скептику.\footnote{Гинсборг (2011, 228) приводит три возражения, которые она (и другие) выдвинула Крипке против диспозиционной теории, возражения эти напоминают (хотя и есть существенные отличия) требования, которые, согласно моему прочтению, он предъявляет к обосновывающим фактам.} \textit{Требование бесконечности.} Любой субъект насчитал лишь конечное число наборов предметов, так что предполагаемая его способность считать всякие другие наборы провоцирует следующие размышления. Во-первых - он, возможно, считал только, например, яблоки и груши, во-вторых - он, возможно, считал только наборы из не более чем 57 предметов. Однако ясно, что его навыки счета подразумевают способность считать наборы отличные от прошлых и количественно и качественно, а если испытуемый не осознал эту нейтральность счета и почему-то чувствует, что не может сосчитать, скажем, красные предметы или предметы в определенных коробках или больший чем всякий предыдущий набор предметов, то, очевидно, что-то не так. Возможно он систематически пропускает числа, начинает заново при достижении какого-то числа или даже фиксируется на определенном числе как на ответе и говорит что остальные объекты не в счет.\footnote{Конечно, некоторые из этих ''ошибок'' происходят во время обучения счету, но не большинство. Дети проходят определенные стадии по мере приобретения навыков счете, и ученые-когнитивисты говорят о них как ''знающих один'', ''знающих два'', ''знающих три'', ''знающих четыре'', ''знающих подмножество'' и, наконец, ''знающих количественный принцип''. ''Знающие одно'' умеют отличать один объект от многих, но не могут пока различать наборы предметов по количеству предметов в них, аналогично ''знающие два'' умеют еще различать наборы из не более чем двух предметов по эти предметов в них количествах, и т.д.. Также на определенном этапе освоения навыков счета дети пропускают числа, но, научившись считать, уже никогда этого не делают. Подробное описание этих процессов и ссылки на соответствующую литературу см. в Кэри (2009) и Баттерворт (1999).} Под ''пониманием'' мы ведь подразумеваем способность решить всякую задачу по счету, сколь бы она не отличалась от уже до этого решенных.

Второе ограничение - это \textit{требование обоснования}: когда испытуемый посчитал предметы и так узнал их количество, его ответ оправдан. Ему \textit{следовало} продолжать так же, учитывая его \textit{осмысление}, не случайно его способ счета дает ожидаемый ответ.

Не менее важно и то, что взрослые и дети убежденны в своей оправданности в таких ответах - они ведь \textit{считают}, что научились считать, что поняли, что \textit{значит} ''счет'', и если кого-то спросить, \textit{почему} он считает именно так, он ответит ''Потому что \textit{так} это и делается'' или ''Потому что \textit{это и есть} счет'' или ''Потому что \textit{так} принято''. Если ребенок или взрослый считает каким-то странным, необычным образом, он будет оправдываться указанием на одинаковость ответов полученных его способом и обычным.\footnote{''Я сначала группирую их по пять штук, потому что такие группы проще распознать, затем считаю их и умножаю на пять'' - такое описание может дать ребенок, и дает он его ровно потому что уверен в одинаковости результатов этого его нового метода и обычного (строгого перечисления).}

Обратите внимание, что предложенные обоснования - что это \textit{так} и делается и т.д. - за исключением редких случаев, феноменологически достоверны, т.е. субъект, изучив процедуру счета, не будет уже колебаться относительно ее корректности и сразу даст объяснение: ''Это \textit{и есть} счет''. И мы также принимаем это его оправдание от третьего лица и говорим ''Он \textit{понимает}, что значит ''счет'''' или ''Он умеет считать''.

И здесь есть одна тонкость. Согласно моему описанию ''обоснования'', у нас обычно есть \textit{два} ключевых ожидания от субъекта в отношении счета или какой-нибудь другой задачи. Во-первых - если он умеет считать, то все что он при этом делает, согласуется с его \textit{намерением} (с тем, что он \textit{имеет в виду}). Иногда ведь мы терпим неудачу в выполнении задачи именно из-за \textit{несоответствия} между \textit{делаемым} и \textit{намереваемым}: не делаем того, что ''должны'' были сделать. (''Посмотри что ты наделал! Разве ты \textit{этого} хотел?''.) Второе же ожидание в том, что то, что субъект делает - действительно \textit{счет}. Могут ведь произойти и другие ошибки: субъект может неправильно решить поставленную задачу, решив, что ему ''нужно'' делать что-то отличное от того, что действительно нужно. ''Нужность'' здесь иная: дело не в том, что он должен был действовать так, как он поступил, учитывая, что и как он понимал, а в том, что он намеревался выполнить одну задачу, но следовало то ему намереваться выполнить другую. Феноменологическая уверенность, значит, ощущаемая при выполнении поставленной задачи, включает оба ожидания и оттого двумя способами и может быть подорвана.

Не так уж много литературы есть об этой ''нормативности'' касательно соблюдения правил и того, насколько нагруженной должна быть эта ''должность''.\footnote{Дебаты о ''нормативности содержания'' или ''нормативности значения'' - т.е., как нормативность участвует в следовании правилам и определении значения,- впрочем, продолжаются. Некоторый взгляд на нормативность значение приписывается и Крипке, потому что, например, согласно Гинзборг (2011, 228), он утверждал, что ''человек, \textit{склонный} или бывший \textit{склонен} определенным образом реагировать в определенной ситуации, не обязательно \textit{должен} так реагировать''. См., например, Богосян (1989, 2003), Глюер (1999), Викфорсс (2001). Как я уже говорил (и еще буду повторять), ''нормативность'', ''рациональность'', ''нормы'', ''правильность'' или ''уместность'' касаются этой темы только в смысле предполагания ими успешного следования правилам, а не в смысле приполагания самим следованием правилам какого-то из них. Частично дело в том, что используемое здесь ''должен'' лишь гипотетично и может быть понято аналогично таковому в, например, ''Если хотите жить, вы должны не прыгать с моста''. (''Если хотите жить, я бы советовал ....'') Или ''если действительно хотите жить, было бы неуместно прыгать с моста, не так ли?'' (Представьте, что это говорит ангел-хранитель в фильме, потенциальному самоубийце.) Гиббард (2003, 85) пишет: ''... стоит ли нам гулять - может зависеть от погоды, но это ведь не делает погоду нормативной в каком-то особом философском смысле''. Итак, здесь я на стороне отвергающих ''нормативность значения''. Подробнее об этом я говорю в Разделе 2.3.} Я буду понимать ее в ее тонком, облегченном варианте. ''Обоснование'' же буду понимать как \textit{согласованность} того, что \textit{имеется в виду} субъектом, и того, что им действительно делается. То есть требование обоснования значит достаточность ''встроенного'' в то, что субъект имел в виду, для подтверждения соответствия того, что он делает - тому, что он имел в виду. Это и лежит в основе его феноменологической уверенности - он может ''объяснить'', почему им делаемое и есть им намереваемое и другими от него требуемое. То есть здесь фигурирует как многообразие того, что имеется в виду - оно определяет соответствующее поведение, - так и \textit{наше} понимание этого значения и им подразумеваемого.

Требование обоснования тесно связано с \textit{требованием ошибки}: наша уверенность в способе и результатах счета справедлива с точностью до случайной ошибки. Любой ведь может ошибиться при подсчете предметов и так получить неверный ответ - в этом нет ничего необычного. Мы, однако, четко отличаем такую ошибку от непонимания концепции счета, и - что особенно важно - даже систематические ошибки от случаев недостаточного усвоения счета и случаев когда \textit{с субъектом} просто что-то не так.

Отвлечься и проглядеть какой-нибудь предмет или дважды его посчитать - значит ошибиться. Другое дело, когда испытуемый из раза в раз намеренно считает, например, все красные объекты дважды. Если он при этом действительно думает, что считает, то, похоже, он не понимает что значит считать, а иногда даже можно сказать, что и неспособен этому научиться. Важный признак ошибки - возможность ее \textit{признания}. (''Вы пропустили вот этот'' - ''Упс, сейчас исправлю''.)

Как бы мы ни описывали способность конкретного субъекта считать и как бы мы его диспозицию, преобретенную в ходе обучения счету, ни характиризовали, надо оставить место для способности давать \textit{неправильные} ответы и этих ответов неправильность \textit{признавать}. Странно, конечно, называть это способностью, но именно таково требование. Мы часто получаем неправильные ответы, даже прекрасно разбираясь в счете, и любое описание схемы обосновывающих фактов нами используемых для объяснения способности считать - способности, раз уж на то пошло, выполнять любую данную задачу - должно оставлять место для возможности ошибок и возможности этих ошибок распознания.

Следует еще раз подчеркнуть, насколько ошибки - даже систематические к ним \textit{тенденции} - совместимы с нашим пониманием таких понятий как счет, сложение, вычитание, умножение и деление. Некоторые люди просто потрясающе хорошо считают - быстро и точно и даже большие числа - но врядли можно сказать, что они как-то более глубоко и точно \textit{понимают} саму концепцию счета. Они, наверное, знают больше соответствующих трюков и обходных путей и \textit{запомнили} больше фактов о числах, да и просто быстрее оперируют ими. Даже кто-то вроде меня, кто почти гарантированно допустит какую-то элементарную ошибку, считая в уме или даже на бумаге, не считается плохо разбирающимся в \textit{понятиях} счета, сложения и т.д.. Что же \textit{действительно} требуется, так это способность \textit{распознать} допущенную ошибку - без этого субъект не может считаться полностью усвоившим рассматриваемые концепции.\footnote{Как я укажу далее, это требование распознавание хотя и применимо к счету и прочим арифметическим операциям, все же не является обязательным для всякой концепции, которую мы пытаемся понять - даже наоборот: часто смысл термина считается понятным даже тогда когда не понятно, когда некоторые (или даже все) его использования неверны. (Например, я не буду считать кого-то не понимающим концепцию Бога, даже если я не согласен с ним во всем или почти во всем, что он о Боге утверждает.)}

\section{Почему интроспективных ресурсов недостаточно}

Я начну с предложения, которое приписываю Витгенштейну Крипке. Дело в том, что ресурсы для понимания счета имеют лишь два возможных источника. Первый - интроспективное понимание, то есть осознание чего-то, равносильное пониманию счета. Второй - склонности к моделям поведения, сознательному или бессознательному. Лишь после признания интроспективного варианта безуспешным, сможем мы обратиться ко второму.

Вспомним смысл вызова скептика: почему субъект всегда считал а не зчитал, что на это указывает? Вполне естественно сначала ответить исходя из первого - интроспективного - варианта. Субъект, например, обычно считается \textit{распознающим} нужную для подсчета закономерность потому что видел некоторое количество примеров и теперь, значит, \textit{видит} как действовать аналогично. Но (Крипке (1982, 18)) ведь никакое конечное число примеров не определяет закономерность полностью. И точно так же любая попытка вменить субъекту осведомленность о правиле или алгоритме, совместимом с предыдущими случаями его применения своих способностей к счету и которое, значит, определяет ответы на последующие задачи - любая такая попытка безуспешна от множественности интерпретаций конечного набора предыдущих задач и корректных на них ответов. Ведь любое такое правило можно понимать по-новому хотя и все еще совместимо с тем, что ''имеет в виду'' субъект, и со всеми прошлыми его подсчетами.\footnote{Крипке ставит проблему смыслового скептика от первого лица, то есть как проблему совместимости моей прошлой практики и нынешнего мышления с ''квожением'' вместо предполагавшегося сложения. Я же поставил проблему эту от третьего лица и касательно счета и ''зчета''. Учитывая, что мы обычно позволяем себе описывать феноменологию третьих лиц (во многом так же, как я это и сделал выше), это, думаю, не вызовет каких-то сложностей. Если же читателя использование такой обыденной практики все же не устраивает, он может легко переформулировать мои тезисы и рассуждения в терминах первого лица - это не на что существенно не повлияет.}

Из этого теперь видно, как требование бесконечности Крипке исключает естественное описание необходимой модели обоснования фактов в терминах \textit{интроспекции}. Как и отмечали многие философы, проблема не в приобритении способности давать правильные ответы для \textit{бесконечного} числа новых случев - нет - нужна способность давать правильные ответы для \textit{новых} случаев вообще. Даже если испытуемый во второй раз сталкивается с тем, что кажется нам точно таким же заданием на счет, он все же может сделать что-то другое, но при этом описать это как ''то же самое''. (Ну ведь и правда - второе задание, может, он решает уже во вторник, а не в среду, или в полнолуние, а не в новолуние, во всяком случае он \textit{позже} это делает.)\footnote{Крипке (1982, 52, сноска 34) неявно признает такую точку зрения, когда цитирует Витгенштейна: ''Если я знаю это заранее, какой от этого знания мне прок позже? То есть: как мне знать, что делать с этим более ранним знанием, когда шаг уже сделан?'' (Витгенштейн (1956, I, §3))}

Проблема в том, что возможные ментальные состояния, охватывающие понимание правил посредством воплощения их в лингвистических формах - даже если с использованием кванторов - или во что-то другое, например, в мысленных образах - визуальных, кинестетических, или как воспоминаний о ранее выполненных задачах - эти возможные ментальные состояния, тем не менее, надо \textit{применить} к текущей, новой, задаче. И то, как они применяются, указывает на то, \textit{как они интерпретируются} - это не что-то заранее фиксированное тем что сейчас можно интроспектировать. Многие думают, что это урок Витгенштейна (1958) из параграфа 139 и близлежащих. Патнэм (1981, 20) излагает его так: ''Феноменологи не видят, что хотя описываемое ими является внутренним \textit{выражением} мысли, \textit{понимание} этого выражения - понимание собственных мыслей - это, тем не менее, не \textit{являние}, а \textit{способность}''.\footnote{Райт (1984, 771) же более осторожен, говоря: «Интуитивно, понимание выражения скорее способность, чем склонность».}

Здесь следует выделить несколько моментов. Как широко отмечалось (например, Райт (1989, 109)), для оценки достаточности интроспективных ресурсов для определения соответствия того, что имеет в виду субъект сейчас, тому, что он имел в виду раньше, Крипке идеализирует сознательный доступ к своим прошлым психическим состояниям, как бы давая субъекту знать \textit{все} о своей прежней психической жизни и поведении.

\end{document}

