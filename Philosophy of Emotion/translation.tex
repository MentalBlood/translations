\documentclass[11pt]{book}

\usepackage{polyglossia}
\setdefaultlanguage[indentfirst=false,forceheadingpunctuation=false]{russian}
\setotherlanguages{english}

\RequirePackage[a4paper, headsep = 0.5 \headsep, left=2.5cm, right=2.1cm, top=2cm, bottom=2.1cm]{geometry}

\setmainfont{Times New Roman}
\newfontfamily\cyrillicfont{Times New Roman}[Script=Cyrillic, Ligatures=TeX]

\title{Философия эмоций}
\author{Кристина Тапполет}
\date{2023}
\begin{document}

\maketitle

\tableofcontents

\part{Философия и наука об эмоциях}

\chapter{Философия эмоций}

\section{Введение}

Представьте, что пытаетесь набирать текст будучи запертым в иглу, и ''комок слякоти из перегретого купола уже в третий раз падает в пишущую машинку, заканчивая вашу работу на сегодня'' (Бриггс 1970, 259). Фрустрация и гнев в такой ситуации вполне понятны. Но антрополог Джин Л. Бриггс обнаружила, что понятны они не для всех --- в своей книге ''Без гнева'' (1970) она описывает эмоциональную жизнь уткухихалингмиутов (утку, для краткости) --- общины инуитов в арктической тундре к северо-западу от Гудзонова залива, где она провела полтора года. На собственном горьком опыте она узнала, что утку выражают неодобрение остракизмом --- им запрещается злится. Конечно, к вспышкам гнева у детей они терпимы, но взрослые должны контролировать свои эмоции. Потому злость, а вместе с ней грусть и любовь они выражают куда реже, чем это делается, скажем, в средней североамериканской семье. Казалось бы --- печальный факт, но в их суровой природной среде жесткий контроль сильных эмоций и сопровождающего их насилия вполне оправдан.

Подобные нормы есть и в других культурах --- идеал жизни, лишенной эмоциональных потрясений многим показался привлекательным. Например, согласно Стоицизму, страх, гнев и зависть --- иррациональные импульсы, связанные с ложными суждениями, и от эмоций, значит, нужно освободиться, чтобы достичь спокойствия ума. Буддизм также призывает отказаться от ненависти, и даже от желания и любви --- состояний, влекущих привязанности, удерживающие от освобождения. Да и в современных обществах к эмоциям относятся как минимум с осторожностью --- страх, гнев, зависть, отвращения и даже сострадание или любовь, считается, мешают ясно мыслить.

Но указывается также, что эмоции могут быть и полезны --- страх помогает избежать опасности, отвращение --- употребления вредных веществ. Иногда мы даже стремимся испытать негативные эмоции без, казалось бы, какой-то пользы: прыгаем с парашютом, слушаем грустную музыку или смотрим фильмы ужасов. А гнев помогает справиться с противником, и иногда кажется оправданным --- ''праведным гневом''. Уж тем более трудно отрицать пользу положительных эмоций --- сострадания, любви, радости, восхищения и надежды --- к ним и удовольствию от них мы обычно стремимся, и они, похоже, ключевой компонент счастливой жизни. Принято считать, что жизнь, лишенная эмоций --- и положительных, и отрицательных --- была бы бедна и едва ли человечна. Кто бы выбрал таковую жизнь андроида Дейты из ''Звездного пути'' --- владеющего мощнейшим позитронным мозгом, но вовсе не испытывающего эмоции?

Итак, наше отношение к эмоциям неоднозначно --- кажется, что следует быть не подверженным им, и все же хочется вести полноценную, наполненную ими, жизнь. Эта амбивалентность проявляется не только в противоположных оценках эмоций, но и, собственно, эмоционально --- мы боимся гнева, отвращения, любви, и самого страха порой тоже боимся. И все же эти переживания нас привлекают --- и позитивные, такие как любовь и радость, и негативные, такие как гнев и страх. Но так думают большинство людей независимо от жизненного уклада и профессии, а что насчет философов? Конечно, философы тоже люди, и чувства их не могут принципиально отличаться. Так что и мысли их по поводу эмоций не отличаются фундаментально.

Чтобы дать представление о деятельности философов в этом направлении, я сделаю краткий исторический обзор философии эмоций. Обзор этот, конечно, избирателен и фокусируется лишь на нескольких основополагающих фигурах, тем самым игнорируя, быть может, не менее значительных --- Баруха Спинозу (1632-1677) и Адама Смита (1723-1780), например, а также не-европейских мыслителей --- таких как конфуцианский философ Мэн-Цзы (ок. 1040-221 гг. до н.э.) (Мэн-Цзы 2008, см. Вираг 2017; Ван Норден 2019; Вонг 2019) или буддийский философ Буддхагхоша (V-ый век) (Буддхагхоша 1984, см. Хейм 2013, Таске 2016), не говоря уже о женщинах-мыслителях, которым не предоставилась возможность оставить след в истории идей (см. Суперсон 2020).

\section{Эмоции: от Платона до Фомы Аквинского}

Современная западная философия берет свое начало в Древней Греции, и философия эмоций не исключение. И хотя ни Платон (ок. 429-347 гг. до н.э.), ни Аристотель (384-322 гг. до н.э.) не разработали законченную теорию эмоций (или страстей --- \textit{pathê} --- того, что мы претерпеваем, будучи по отношению к нему пассивны), они придерживались особых на них взглядов, оказавших огромное влияние --- и продолжающих влиять --- на мыслителей и по сей день (см. Прайс 2009). Оба они --- и Платон, и Аристотель --- считали человеческий разум (или ''душу'' --- \textit{psyche}) разделенным на части. Согласно ''Государству'' Платона, душа состоит из рациональной части --- разума, и низшей, нерациональной, делящейся на яростную (тумос) и страстная (эпитумия). Яростная часть ответственна за гнев, стыд, негодование, страх, гордость и восхищения, а страстная связана с телесными желаниями, такими как жажда и голод. Можно подумать, что эмоций касается только яростная часть --- но нет --- и Платон прямо утверждал, что эмоции сопровождают и некоторые телесные желания. Более того, он относил любовь к истине и мудрости к разумной части. Примечательно, что, согласно Платону, эмоции были не чисто психическими --- и они, и телесные позывы, он считал, соответствуют телесным нарушениям. Так, гнев, например, сопровождается кипением крови от мысли о несправедливости.

Центральное место в концепции души Платона занимает возможность конфликта между разными ее частями. Например, особый недостаток рациональности, называемый акрасией (слабость воли), Платон определял как совершаемое несмотря на веру в его неправильность, и объяснял в терминах конфликта между страстной и рациональной установкой. И чтобы таких конфликтов не было, рациональная часть, конечно, должна преобладать --- в ''Федре'' он сравнивает разум с возничим, ведущим двух тянущих в разные стороны крылатых коней. Платон, впрочем, считал яростную часть ближе и податливей разуму (в сравнении со страстной) и допускал возможность научиться, например, любить и ненавидеть так, чтобы конфликта с разумом не возникало.

У Аристотеля была схожая иерархия в его концепции, только он делил душу на две части: рациональную и не-рациональную --- ответственную за эмоции. В ''Никомаховой этике'' он включает в список эмоций телесные желания (эпитумию), а также гнев, страх, уверенность, зависть, радость, любовь, ненависть, тоску, соперничество и жалость. Как и Платон, Аристотель говорит об эмоциях как о затрагивающих не только душу, но и тело, описывая разгневанного при мысли об оскорблении как вскипающего. Он дает определения ряда эмоций, среди которых гнев, негодование, стыд, жалость, зависть и страх. Гнев --- это желание мести, сопровождающееся болью из-за действительного или воображаемого пренебрежения к себе или близкому. Примечательно, что для гнева характерна не только боль, но и удовольствие [предвкушения мести]. Эмоции, считал Аристотель, включают боль или удовольствие, оценку (например, веру в незначительность чего-то или его опасность), мотивацию, а также телесные изменения.

Как и Платон, Аристотель указывал на эмоции как на частый источник иррациональности --- конфликт между ними и разумом может привести к действию вопреки здравому смыслу --- акрасии. Но он также признавал за эмоциями возможность играть положительную роль --- считал их ключевыми для моральных/этических добродетелей и счастья. Моральные добродетели --- привычки, тесно связанные с ощущением нужных эмоций в нужное время по отношению к нужным предметам и людям (см. главу 10), так что и гнев может быть правильной реакцией. Добродетели, считал Аристотель, требуют воспитания, и эмоциональные реакции зависят от паттернов, усвоенных в общественных взаимодействиях. Как становятся хорошими игроками на лире, тренируясь на ней играть, так и становятся храбрыми, встречаясь лицом с опасностью и приучаясь испытывать при этом правильные эмоции. Идея воспитания эмоций оказалась, вообще очень влиятельной (см. главу 12).

Аристотель считал эмоции необходимыми для добродетели и счастья --- но в рамках Стоицизма, важной школе мысли, основанной в Афинах в III веке до нашей эра и просуществовавшей вплоть до римского периода, к эмоциям относились куда более критично. В работах стоиков можно найти подробное рассмотрение таких негативных эмоций как гнев, страх, горе, печаль, стыд, зависть, ревность и жалость. В отличие от Платона и Аристотеля, они не верили в разделенную душу, представляя ее как физическую субстанцию (пневму), смешанную с телом. Душа, однако, обладает управляющей способностью, через которую реализуется, среди прочего, способность к рассуждению. Эмоции же --- согласно стоикам --- иррациональны, их надо не просто контролировать, но --- в идеале --- полностью искоренить. Согласно Сенеке (ок. 1 до н.э. - 65 н.э.), ''Гнев следует не контролировать, но совершенно уничтожить, ибо какой контроль возможен на по сети своей злобным?'' (Сенека, ''О гневе'', iii. 42). То есть разделяя душу на части, стоики не согласовывали, но противопоставляли их. Наличие эмоций --- источник акрасии, влекущей скорые и разрушительные изменения в душе.

Что же не так с эмоциями, согласно стоикам? Эмоции подразумевают суждения о добре и зле, которые глубоко ошибочны. Горе, например, следствие убеждения в действительном наличии нечта плохого --- согласие с наличием такого и заставляет считать реакцию правильной. Однако такого рода убеждения основаны на ошибочном представлении о человеке и его месте в мире. Хороша же только добродетель, и только соответствующие ей действия обеспечивают истинное счастье. Считать же важным что-нибудь еще --- свою удовольствие или боль, например --- не просто плохо, но иррационально, ведь ни к чему кроме несчастья это не приведет. Стоики предлагали разные методы избавления от эмоций --- от когнитивной терапии, включающей философские рассуждения, до развития соответствующих привычек и даже музыкальной терапии.

\end{document}
