\documentclass[11pt]{book}

\usepackage{polyglossia}
\setdefaultlanguage[indentfirst=false,forceheadingpunctuation=false]{russian}
\setotherlanguages{english}

\RequirePackage[a4paper, headsep = 0.5 \headsep, left=2.5cm, right=2.1cm, top=2cm, bottom=2.1cm]{geometry}

\setmainfont{Times New Roman}
\newfontfamily\cyrillicfont{Times New Roman}[Script=Cyrillic, Ligatures=TeX]

\title{Философия эмоций}
\author{Кристина Тапполет}
\date{2023}
\begin{document}

\maketitle

\tableofcontents

\part{Философия и наука об эмоциях}

\chapter{Философия эмоций}

\section{Введение}

Представьте, что пытаетесь набирать текст будучи запертым в иглу, и ''комок слякоти из перегретого купола уже в третий раз падает в пишущую машинку, заканчивая вашу работу на сегодня'' (Бриггс 1970, 259). Фрустрация и гнев в такой ситуации вполне понятны. Но антрополог Джин Л. Бриггс обнаружила, что понятны они не для всех --- в своей книге ''Без гнева'' (1970) она описывает эмоциональную жизнь уткухихалингмиутов (утку, для краткости) --- общины инуитов в арктической тундре к северо-западу от Гудзонова залива, где она провела полтора года. На собственном горьком опыте она узнала, что утку выражают неодобрение остракизмом --- им запрещается злится. Конечно, к вспышкам гнева у детей они терпимы, но взрослые должны контролировать свои эмоции. Потому злость, а вместе с ней грусть и любовь они выражают куда реже, чем это делается, скажем, в средней североамериканской семье. Казалось бы --- печальный факт, но в их суровой природной среде жесткий контроль сильных эмоций и сопровождающего их насилия вполне оправдан.

Подобные нормы есть и в других культурах --- идеал жизни, лишенной эмоциональных потрясений многим показался привлекательным. Например, согласно Стоицизму, страх, гнев и зависть --- иррациональные импульсы, связанные с ложными суждениями, и от эмоций, значит, нужно освободиться, чтобы достичь спокойствия ума. Буддизм также призывает отказаться от ненависти, и даже от желания и любви --- состояний, влекущих привязанности, удерживающие от освобождения. Да и в современных обществах к эмоциям относятся как минимум с осторожностью --- страх, гнев, зависть, отвращения и даже сострадание или любовь, считается, мешают ясно мыслить.

Но указывается также, что эмоции могут быть и полезны --- страх помогает избежать опасности, отвращение --- употребления вредных веществ. Иногда мы даже стремимся испытать негативные эмоции без, казалось бы, какой-то пользы: прыгаем с парашютом, слушаем грустную музыку или смотрим фильмы ужасов. А гнев помогает справиться с противником, и иногда кажется оправданным --- ''праведным гневом''. Уж тем более трудно отрицать пользу положительных эмоций --- сострадания, любви, радости, восхищения и надежды --- к ним и удовольствию от них мы обычно стремимся, и они, похоже, ключевой компонент счастливой жизни. Принято считать, что жизнь, лишенная эмоций --- и положительных, и отрицательных --- была бы бедна и едва ли человечна. Кто бы выбрал таковую жизнь андроида Дейты из ''Звездного пути'' --- владеющего мощнейшим позитронным мозгом, но вовсе не испытывающего эмоции?

Итак, наше отношение к эмоциям неоднозначно --- кажется, что следует быть не слишком им подверженным, и все же хочется вести полноценную, наполненную ими, жизнь. Эта амбивалентность проявляется не только в противоположных оценках эмоций, но и, собственно, эмоционально --- мы боимся гнева, отвращения, любви, и самого страха порой тоже боимся. И все же эти переживания нас привлекают --- и позитивные, такие как любовь и радость, и негативные, такие как гнев и страх. Но так думают большинство людей независимо от жизненного уклада и профессии, а что насчет философов? Конечно, философы тоже люди, и чувства их не могут принципиально отличаться. Так что и мысли их по поводу эмоций не отличаются фундаментально.

Чтобы дать представление о деятельности философов в этом направлении, я сделаю краткий исторический обзор философии эмоций. Обзор этот, конечно, избирателен и фокусируется лишь на нескольких основополагающих фигурах, тем самым игнорируя, быть может, не менее значительных --- Баруха Спинозу (1632-1677) и Адама Смита (1723-1780), например, а также не-европейских мыслителей --- таких как конфуцианский философ Мэн-Цзы (ок. 1040-221 гг. до н.э.) (Мэн-Цзы 2008, см. Вираг 2017; Ван Норден 2019; Вонг 2019) или буддийский философ Буддхагхоша (V-ый век) (Буддхагхоша 1984, см. Хейм 2013, Таске 2016), не говоря уже о женщинах-мыслителях, которым не предоставилась возможность оставить след в истории идей (см. Суперсон 2020).

\section{Эмоции: от Платона до Фомы Аквинского}

Современная западная философия берет свое начало в Древней Греции, и философия эмоций не исключение. И хотя ни Платон (ок. 429-347 гг. до н.э.), ни Аристотель (384-322 гг. до н.э.) не разработали законченную теорию эмоций (или страстей --- \textit{pathê} --- того, что мы претерпеваем, будучи по отношению к нему пассивны), они придерживались особых на них взглядов, оказавших огромное влияние --- и продолжающих влиять --- на мыслителей и по сей день (см. Прайс 2009). Оба они --- и Платон, и Аристотель --- считали человеческий разум (или ''душу'' --- \textit{psyche}) разделенным на части. Согласно ''Государству'' Платона, душа состоит из рациональной части --- разума, и низшей, нерациональной, делящейся на яростную (тумос) и страстная (эпитумия). Яростная часть ответственна за гнев, стыд, негодование, страх, гордость и восхищения, а страстная связана с телесными желаниями, такими как жажда и голод. Можно подумать, что эмоций касается только яростная часть --- но нет --- и Платон прямо утверждал, что эмоции сопровождают и некоторые телесные желания. Более того, он относил любовь к истине и мудрости к разумной части. Примечательно, что, согласно Платону, эмоции были не чисто психическими --- и они, и телесные позывы, он считал, соответствуют телесным нарушениям. Так, гнев, например, сопровождается кипением крови от мысли о несправедливости.

Центральное место в концепции души Платона занимает возможность конфликта между разными ее частями. Например, особый недостаток рациональности, называемый акрасией (слабость воли), Платон определял как совершаемое несмотря на веру в его неправильность, и объяснял в терминах конфликта между страстной и рациональной установкой. И чтобы таких конфликтов не было, рациональная часть, конечно, должна преобладать --- в ''Федре'' он сравнивает разум с возничим, ведущим двух тянущих в разные стороны крылатых коней. Платон, впрочем, считал яростную часть ближе и податливей разуму (в сравнении со страстной) и допускал возможность научиться, например, любить и ненавидеть так, чтобы конфликта с разумом не возникало.

У Аристотеля была схожая иерархия в его концепции, только он делил душу на две части: рациональную и не-рациональную --- ответственную за эмоции. В ''Никомаховой этике'' он включает в список эмоций телесные желания (эпитумию), а также гнев, страх, уверенность, зависть, радость, любовь, ненависть, тоску, соперничество и жалость. Как и Платон, Аристотель говорит об эмоциях как о затрагивающих не только душу, но и тело, описывая разгневанного при мысли об оскорблении как вскипающего. Он дает определения ряда эмоций, среди которых гнев, негодование, стыд, жалость, зависть и страх. Гнев --- это желание мести, сопровождающееся болью из-за действительного или воображаемого пренебрежения к себе или близкому. Примечательно, что для гнева характерна не только боль, но и удовольствие [предвкушения мести]. Эмоции, считал Аристотель, включают боль или удовольствие, оценку (например, веру в незначительность чего-то или его опасность), мотивацию, а также телесные изменения.

Как и Платон, Аристотель указывал на эмоции как на частый источник иррациональности --- конфликт между ними и разумом может привести к действию вопреки здравому смыслу --- акрасии. Но он также признавал за эмоциями возможность играть положительную роль --- считал их ключевыми для моральных/этических добродетелей и счастья. Моральные добродетели --- привычки, тесно связанные с ощущением нужных эмоций в нужное время по отношению к нужным предметам и людям (см. главу 10), так что и гнев может быть правильной реакцией. Добродетели, считал Аристотель, требуют воспитания, и эмоциональные реакции зависят от паттернов, усвоенных в общественных взаимодействиях. Как становятся хорошими игроками на лире, тренируясь на ней играть, так и становятся храбрыми, встречаясь лицом с опасностью и приучаясь испытывать при этом правильные эмоции. Идея воспитания эмоций оказалась, вообще очень влиятельной (см. главу 12).

Аристотель считал эмоции необходимыми для добродетели и счастья --- но в рамках Стоицизма, важной школе мысли, основанной в Афинах в III веке до нашей эра и просуществовавшей вплоть до римского периода, к эмоциям относились куда более критично. В работах стоиков можно найти подробное рассмотрение таких негативных эмоций как гнев, страх, горе, печаль, стыд, зависть, ревность и жалость. В отличие от Платона и Аристотеля, они не верили в разделенную душу, представляя ее как физическую субстанцию (пневму), смешанную с телом. Душа, однако, обладает управляющей способностью, через которую реализуется, среди прочего, способность к рассуждению. Эмоции же --- согласно стоикам --- иррациональны, их надо не просто контролировать, но --- в идеале --- полностью искоренить. Согласно Сенеке (ок. 1 до н.э. - 65 н.э.), ''Гнев следует не контролировать, но совершенно уничтожить, ибо какой контроль возможен на по сети своей злобным?'' (Сенека, ''О гневе'', iii. 42). То есть разделяя душу на части, стоики не согласовывали, но противопоставляли их. Наличие эмоций --- источник акрасии, влекущей скорые и разрушительные изменения в душе.

Что же не так с эмоциями, согласно стоикам? Эмоции подразумевают суждения о добре и зле, которые глубоко ошибочны. Горе, например, следствие убеждения в действительном наличии нечта плохого --- согласие с наличием такого и заставляет считать реакцию правильной. Однако такого рода убеждения основаны на ошибочном представлении о человеке и его месте в мире. Хороша же только добродетель, и только соответствующие ей действия обеспечивают истинное счастье. Считать же важным что-нибудь еще --- свою удовольствие или боль, например --- не просто плохо, но иррационально, ведь ни к чему кроме несчастья это не приведет. Стоики предлагали разные методы избавления от эмоций --- от когнитивной терапии, включающей философские рассуждения, до развития соответствующих привычек и даже музыкальной терапии.

Стоиков часто высмеивали за их идеал --- существо, начисто лишенное эмоций, кажется недосягаемым для нормальных людей, но, справедливости ради, стоики и не рекомендовали жизнь, лишенную даже намека на эмоции. Допустимо чувствовать своего рода протоэмоциональную реакцию --- не соглашаясь с ней, не позволяя ей стать полноценной эмоцией, движущей к определенным действиям. Некоторые психические состояния, сейчас считаемые эмоциями, стоиками даже поощрялись --- например, восторг (если он вполне обоснован). Впрочем, даже с учетом этих уступок, идеал их может показаться чрезмерно требовательным и в целом нежелательным.

Средневековых философов тоже интересовали эмоции (их называли \textit{passio} или \textit{affectus}). Фома Аквинский (1225-1274), на подход которого значительно повлиял Аристотель, предложил наиболее обширную трактовку эмоций в своей ''Сумме теологии'' (Часть II -- 1.22--48 и 2.17--36) (см. Кнууттила 2004, Перлер 2018). Подобно стоикам, современники Фомы и он сам были сосредоточены на нормативном вопросе --- хороши или плохи эмоции, являются ли любовь, ненависть, надежда, отчаяние, страх, радость или гнев, добродетелью или грехом. Фома Аквинский, считавший эмоции движениями души, подчеркивал их побудительную силу. Эмоции --- это мотивация к действию: гнев, например --- желание мести. Мотивация эта связана с телесными изменениями и возникает из оценки чего-либо как приятного или болезненного или, вообще, значимого. Фома Аквинский называет эти категории ''формальными объектами'', и понятие это (как будет видно в Главе 2) до сих пор играет в философии эмоций важную роль. Поскольку оценки, на которых основаны эмоции, могут быть ошибочными, их, считал Фома, следует проверять разумом. Но искоренять эмоции не стоит --- достаточно держать их под контролем. Итак, Фома Аквинский следовал Аристотелю, утверждая необходимость для добродетели соответствующих разуму эмоций, и --- раз эмоции могут командам разума сопротивляться, контроль разума должен быть подобен политической власти над свободными сопротивляться. Рефлексия, считал он, способна не только ослабить, например, гнев, но и полностью его прекратить, и для этого достаточно помыслить вызывающее его в ином свете. Как будет видно из Главы 11, стратегия такой переоценки занимает центральное место в современных подходах к регуляции эмоций.

\section{Современная философия и эмоции}

Рене Декарт (1596-1650) (см. Шапиро 2020) обычно считается отцом современной философии. Декарт открыто презирал средневековые теории эмоций в своей книге ''Страсти души'' (1649/1989). Его занимала проблема отношения разума и тела, которые он считал отдельными субстанциями. Эмоции, согласно Декарту --- это восприятия, ощущения или волнения души, вызванные движениями животных духов --- мельчайших частиц, сообщающих душе готовность тела к определенным действиям. То есть эмоции затрагивают не только ум, но и тело. Примечательно, что животные (кроме человека), считал Декарт, не имеют разума и потому неспособны испытывать эмоции и даже просто чувствовать боль или удовольствие --- это резко контрастирует со взглядами его предшественников, допускающими способность животных испытывать эмоции даже не имея разума.

Но как взаимодействуют разум и тело? Это сложный вопрос, отвечая на который, Декарт разработал теорию, согласно которой, тело влияет на разум передачей своих движений посредством животных духов через шишковидную железу в основании мозга. Но эта теория, конечно, не может быть корректной --- если разум состоит из нефизической субстанции, то такой ''факс в душу'' не объясняет воздействие физического на разум.

Хотя Декарт и уделяет много внимания такому механистическому объяснению эмоций, его теория весьма схожа как с аристотелевской, так и со стоической. Так, говоря про эмоции как про волнения души, вызванные движением животных духов, он все же указывал, что некоторые из них включают оценку вещей с точки зрения вреда и пользы. Любовь и радость, например --- являясь одной из основных эмоций наряду с удивлением, ненавистью, желанием и печалью --- вызваны представлением своего объекта как хорошего и полезного[, при этом любовь, в отличие от радости подразумевает неизменность этих свойств со временем]. Эмоции обладают мотивационной силой и отличаются от оценочных суждений тем, что суждения требуют также активного согласия с оценкой. То есть можно выносить суждение о вещи как о хорошей, и все же невольно реагировать на нее как на плохую, не соглашаясь, впрочем, с этим представлением. Как будет видно в Главе 6, такие ''непокорные эмоции'' (Д'Армс и Джейкобсон, 2003) занимают видное место в современных дебатах о природе эмоций.

Декарт отличал эмоции от суждений еще тем, что эмоции могут вводить в заблуждение. Они часто искажают наше понимание реальности, заставляя верить, что все лучше или хуже, чем на самом деле, так что эмоции следует контролировать. Напрямую управлять эмоцией нельзя, но, считал Декарт, с первоначальной эмоциональной реакцией можно не соглашаться --- согласие же даст ей импульс к дальнейшему развитию. Косвенный контроль над эмоциями возможен посредством другой эмоции --- чувства собственного достоинства, способности свободно осуществлять свою волю. Декарт также выделял несколько способов изменить свои [первичные] эмоциональные реакции. Можно, во-первых, пытаться представить объект своих эмоций в ином свете. Или представить последствия эмоциональных реакций --- если они негативные, это будет постепенно притуплять порождающие их первичные реакции. В конце концов, частое столкновение с тем, что вызывает эмоциональные реакции, также постепенно притупляет их.

Другой известный философ этого периода --- Дэвид Юм (1711-1776) --- взял на себя задачу пересмотреть традиционную рационалистическую концепцию эмоций (см. Кохон 2018). Вместо поиска способов их контроля он прямо заявил, что эмоции играют --- и должны играть --- ведущую роль. В своем ''Трактате о человеческой природе'' он указал, что ''разум должен быть --- и был всегда --- только служанкой чувств'' (Юм 1739-1740/2000, Книга 2, Часть 3, Раздел 3). Роль разума --- полезная, но вспомогательная: он предоставляет информацию о мире и средствах к достижению наших целей, но цели эти предоставляются эмоциями. Разум сам по себе не обладает мотивационной силой и, значит, не в состоянии действительно противостоять эмоциям --- любая попытка такого мнимого противостояния предполагала бы своей причиной другие эмоции. В своем ''Трактате'' Юм утверждал, что эмоции не состоят в представлении вещей, а значит и не могут, подобно суждениям, оцениваться разумом как рациональные или иррациональные. Впрочем, в более поздней работе ''О стандарте вкуса'' (1757/1985a) Юм от этого тезиса отказался. Но в ''Трактате'' он также утверждал, что ''нет никакого противоречия в том, чтобы предпочесть смерть всего мира, царапине'' (Юм 1739-1740/2000, Книга 2, Часть 3, Раздел 3). Однако, Юм, конечно, решительно осудил бы такое предпочтение --- размышление о смерти мира вызывает ужас --- но рациональность или иррациональность здесь не при чем.

Юм выделяет два вида эмоций. Первый --- ''прямые страсти'' --- включает желание, отвращение, надежду, горе, радость, страх, отчаяние и противоположность страха --- чувство безопасности. Они вызываются соответствующим опытом или мыслью о добре и зле, что для Юма то же, что переживание удовольствия или боли или мысль о них. К так называемым же ''косвенным страстям'' относятся гордость, смирение, стыд, честолюбие, тщеславие, любовь, ненависть, зависть, жалость, злоба и щедрость --- помимо переживания боли или удовольствия, они требуют также дополнительных мыслей. Гордость за красоту своего дома, например --- удовольствие, направленное на самого себя и вызванное мыслью о принадлежности вам этого дома. Вообще, гордость соответствует положительному представлению о себе, а стыд --- отрицательному. То есть, согласно Юму, и прямые, и косвенные эмоции подразумевают оценку своих объектов, так что различие его теории с таковыми предшественников не так уж велико --- Юма отличает не столько описание природы эмоций, сколько понимание их взаимосвязи с разумом.

Такая инверсия отношения разума к эмоциям является ключевой для морального сентиментализма --- для Юма, моральные оценки берут свое начало из эмоций, а не из разума, и именно на основе удовольствия или неудовольствия в от восприятия людей и их действий, мы выносим моральные суждения. То есть добродетельным считается человек, характер которого вызывает одобрение, а одобрение --- это определенный род удовольствия. Удовольствие это опосредованно ''симпатией'' --- спонтанном ощущении удовольствия/неудовольствия при виде его у других. Так, доброжелательность одобряется потому что приносит удовольствие другим. Юм, как не странно, указывал на способность эмоциональных реакций вводить в заблуждение. Например, при рассмотрении личности врага мы склонны быть предвзятыми, из-за чего не можем дать ей справедливую оценку. Нам, значит, стоит принять нейтральную точку зрения, чтобы не допускать подобных отклонений.

Другой знаменитый философ --- Иммануил Кант (1724-1804) --- выступает против концепции морали Юма (см. Уилсон и Дэнис 2018). Юм указывал, что основание морали --- эмоции по отношению к другим, Кант же полагал ее основой практический разум --- способность определять волю и приводить к действию. Мораль, согласно Канту, возможна благодаря автономной воле --- той, что устанавливает сама себе закон. Такая воля должна игнорировать всякие привлекательные черты своих объектов, поскольку руководство ими значило бы руководство ''склонностями'', которые нам в действительности чужды. Следовать склонностям значило бы рабство --- такая воля не автономна. Поэтому эмоции не следует учитывать в принятии решения --- только так мы можем быть автономными агентами и следовать своим моральные обязательства.

Кант критиковал эмоции как основу для морали, указывая, что они слишком шатки для столь фундаментальной роли. В своей работе ''Основы метафизики морали'' (1785/1996) он утверждал, что такие ''эмпирические принципы'' не могут поддержать моральные законы, справедливые для всех [разумных] существ, ведь они привязаны к конкретной --- человеческой --- природе. Кант также отмечал ''бесконечные различия в степенях и видах чувств'', из-за чего они ''не могут предоставить единый стандарт хорошего и плохого'' (Кант 1785/1996, 4:442). Можно, впрочем --- независимо от вопроса об универсальности моральных принципов --- усомниться в справедливости такого осуждения эмоций, ведь наши рациональные способности так же зависят от нашей природы, а наши выводы могут различаться в разных обстоятельствах и даже сбивать нас с верного пути.

Кант, однако, не столь плохо относится к эмоциям, как можно было подумать. Так, одной конкретной эмоции --- уважению морального закона --- он отводит центральную роль. Эмоция эта возникает при рассмотрении морального закона и приводит к подчинению ему. Для таких существ, как мы, согласно Канту, требуется чувство удовольствия или восторга от выполнения своего долга. В более поздних работах (см., например, Кант 1797/2018) он даже указывал на обязанность культивировать такие моральные эмоции, как сострадание , любовь к людям и уважение к себе. В работе об эстетических суждениях Кант также отводил эмоциям центральную роль понимании красоты и возвышенного. И все же, в отличие от Юма, Кант считал подлинно моральными лишь действия, совершаемые исключительно из чувства долга.

Итак, видно, что интерес философов к эмоциям привел к появлению и развитию все более тонких и сложных их теорий. Но здесь имеет смысл задать вопрос: в чем, собственно, должно состоять философское исследование эмоций? Иначе говоря, какие основные вопросы предполагает эта область?

\end{document}
